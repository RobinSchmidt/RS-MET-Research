%\subsubsection{Differentiation}
Consider an arbitrary function $f(x)$ and imagine that we want to compute the slope of the graph at a given $x_0$. We could approximate it by considering the point $(x_0, f(x_0))$ and another point a small distance $h$ further to the right $(x_0+h, f(x_0+h))$ and compute the "rise over run" quotient $(f(x_0+h) - f(x_0)) / h$. For smaller and smaller $h$, the approximation would get better and better. Now we consider the limit as $h$ approaches zero. If that limit exists, it actually \emph{is} our desired slope.

