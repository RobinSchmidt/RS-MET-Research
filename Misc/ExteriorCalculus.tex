


\begin{comment}

Analogous objects in exterior calculus and classic vector calculus

0-form -> multivariate function, scalar field
1-form -> vector field
2-form -> bivector field, can be indentified with a vector field in 3D
3-form -> trivector field, can be identified with a scalar field in 3D

effect of the exterior derivative $\d$ on k-forms:

0-form -> d -> 1-form -> d -> 2-form -> d -> 3-form

this is analogous to:

scalar field -> grad -> vector field -> curl -> vector field -> div -> scalar field


todo: explain, how the exterior derivative would work in 4D where the curl doesn't work anymore

The $\d$ operator can be applied to a vector field (i.e. 1-form field) of any dimension, in which case it generalizes the curl operator from classic vector calculus. It also generalizes and unifies the other 2 first order differential operators from vector calculus (grad and div) when it is being applied to other kinds of fields, such as scalar (0-form) and bivector (2-form) fields.


the generalized Stokes theorem generalizes the curl theorem in two ways:

(1) it can now be applied to vector-fields in arbitrary dimensions
(2) it cannot only be applied to vector fields but also scalar fields, bivector fields, etc. One tpyically does not look at vector fields per se but at 1-form fields (i.e. covector fields?) instead (or, in general, at k-form fields). But they area isomorphic, so it's ok to identify them. however, in 3D, integrating a (co)bivector fields will yield a (co)vector fields, which is the vector potential of the original bivector field...wait ...does this co-bivector thing make sense? or *is* a bivector the same thing as a covector? yes, i think, in 3D a vector is identified with a 1-form and a bivector is identified with a 2-form which is also a covector

\end{comment} 