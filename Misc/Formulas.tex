%\documentclass[12pt, twocolumn]{article}
\documentclass[12pt]{article}  % todo: use book and for the top-level items use "chapter"
%\usepackage{fullpage}           % makes all margins 1 inch?
\topmargin=-1.0cm
\textheight=23cm
\evensidemargin=-1.0cm
\oddsidemargin=-1.0cm
\textwidth=19cm
\setcounter{secnumdepth}{-1}     % suppress numbering of sections
\usepackage{amsmath}
\usepackage{amssymb}             % for mathbb
\usepackage{hyperref}

\begin{document}
	
% formatting:
\parindent=0in
\parskip=0pt
\pagenumbering{roman}
		
% main text
\pagenumbering{arabic} \setcounter{page}{1}

\author{Robin Schmidt}

\title{Mathematical Recipes for Scientists, Engineers and Programmers}
\maketitle

%\section{}
%This document contains miscellaneous math formulas gathered from various sources that may or may not become useful at some point. A special focus is on recipe-like formulas that are suitable for being translated into code to become part of the library someday.

\tableofcontents

\section{Preface}
Mathematics is a broad subject and I would be way out of my depth to try to give a definition what it actually is. I like to think of it as the study of structures, patterns and equivalences and as a way to systematize, formalize and eventually automate thought processes. One big theme is to figure out, under which circumstances one thing is equal to another. Often, there are multiple ways to compute the same thing and a big sub-theme of finding equivalences is to find computational shortcuts that allow to do a computation more efficiently than was previously possible. The body of mathematical knowledge today is an impressive tower of known facts about abstract constructs of the human mind. The first of these abstract constructs that one usually encounters in elementary school is the idea of a natural number and one learns how to add, subtract, multiply and divide them. Building on that, one later encounters negative numbers, rational numbers, real numbers and complex numbers. I like to think of numbers as the ground floor of the tower. Numbers usually set the stage for doing mathematics on first encounter, but the foundations of mathematics can go down to even lower levels: If numbers are the ground floor, then it may be appropriate to think of set theory and mathematical logic as two basement floors. Having numbers in place, one can go up a stage and look at functions that map numbers to other numbers. One then may realize that addition, multiplication, etc. are actually functions, too: they take two inputs and map them to a single output. Such generalization with hindsight is also a common theme in math. A stage higher, you can look at mappings that map functions to other functions. And it goes ever higher up. Well, actually, it also kind of branches out while going up, so maybe a tree of knowledge might be better analogy than a tower - but then the roots (levels below ground) do not branch out as much as the upper levels. But there certainly is some sort of trunk that everybody needs to know. This trunk contains numbers themselves (from the natural to the complex ones), equations (and solution techniques for them when they contain unknowns), elementary functions (polynomial, rational, exponential, trigonometric and their inverses), linear algebra (vectors, matrices, linear systems of equations) and single variable calculus (derivatives, integrals, differential equations). With these basics in place, math branches out into various directions. Some of these are: multivariable calculus: the study of functions of several variables, abstract algebra: generalizes ideas such as addition, multiplication, etc. to other sorts of objects and identifies the common structure, number theory: studies natural numbers and especially prime numbers in depth, functional analysis: studies functions of functions, topology: studies qualitative properties of shapes, etc. 

\paragraph{}
This document is an attempt to give a condensed high level overview about what's going on in a particular subject and to give the required formulas and recipies to actually get the work done. There is little regard to derivation or justification and no regard whatsoever to proof or mathematical rigor. It's meant to be a collection of recipies for the practioner who needs to use math in applications. In focus are the questions: What is it? What can I do with it? How can I do it? The focus is deliberately not: Why does it work the way it does? That would fill volumes (and has done so) and thereby just distract from getting the work done.

\section{Basics}
\subsection{Foundations} 
 This book is about applied math, so we will not go deeply into the foundations of math. However, a brief and very superficial look into this topic is a good idea to set the stage for the material that follows. It also establishes the basics of the language which we will need to talk about mathematical concepts.

\subsubsection{Logic}
Math is the pursuit of finding truths, so it makes sense to have a framework, within which we can say that something is true or false. In mathematics, that framework is mathematical logic, more specifically propositional logic and predicate logic (a.k.a. first order logic). Propositional logic deals with propositions which are statements that can be either true or false. You also have ways of combining given propositions to make new, more complex propositions. For example, you can combine two propositions with a logical "and" (usually denoted as $\wedge$). The resulting new proposition is true, if and only if both of the input propositions are true, otherwise it's false. You also have a logical "or" (denoted as $\vee$), which in this context is taken to be an inclusive or: the combined proposition is true, if any one of the input propositions or both are true. You also have a logical "not" (denoted as $\neg$) which takes a single proposition as input and the result is true, if the the input is false and vice versa. To build up mathematics, propositional logic is not quite enough. Predicate logic builds up on propositional logic and lets you talk about objects and relations between them. There, you have so called quantors like the symbol for "there exists an object such that..." (denoted as $\exists$) or a symbol for "for all objects it is true that..." (denoted as $\forall$). Logic also provides the tools that are required to figure out whether a given proposition is true, given that some other propositions are true. That process of drawing conclusions from given (true) propositions is called deduction. There are yet other levels and kinds of logic, but these two are enough for the moment.

\subsubsection{Axioms}
One has to start somewhere. That starting point is typically a set of "axioms" together with the rules of logic. An axiom is a proposition that is just assumed to be true without further justification. Axioms should state things that are "obviously true". An example are the Peano axioms, some of which are: zero is a natural number, each natural number has a successor, any number is equal to itself, etc. If you really want to build up the whole tower of mathematics axiomatically, you have to \emph{choose} a set of axioms and from there, using only the rules of logic, i.e. deduction, find new propositions that are also true.

\subsubsection{Theorems and Proofs}
If you want to prove a proposition, the tools that you have in hand are all the propositions that are already known to be true together with the rules of logic. A proposition is known to be true if it is either an axiom or it has been previously proven by the same technique. A proof for a proposition is a sequence of true propositions in which each one follows from known or previous ones by applying the rules of logic and the last of which is the one you actually wanted to prove in the first place. If a proposition has been proven to be true, it becomes a "theorem". A theorem is a fundamentally important concept in math - math is all about finding theorems. You may have observed a pattern by looking at a bunch of examples and you may conjecture that the pattern is generally true. What you then have to do is to find a proof for your conjecture. If you have succeeded in this highly creative endeavour (i.e. your proof is determined to be correct by the mathematical community), your conjecture is elevated to the venerable status of a theorem. And if the theorem is important enough and you were the first to prove it, you will typically achieve immortality by having your name attached to the theorem for the rest of eternity. Thousands of years later, still everbody knows the name of Pythagoras today - although, it wasn't actually him who proved "Pythagoras' Theorem" - sometimes the world is a little unfair, too :'-(. Along the way of finding a proof, you may generate a whole bunch of proven propositions, some of which are only instrumental to your final goal, some of which are spinoffs, etc. There are some other terms for such "lesser theorems" such as "lemma", "corollary", etc. A theorem is usually a result with a certain level of importance, generality and usefulness. You wouldn't call something like $3+5=8$ a theorem, for example - although it manifestly is a true proposition (and can actually be proven).

\subsubsection{Definitions}
OK - this is kind of awkward. We now have to \emph{define} what \emph{a definition} is. A definition is actually just an agreement about certain conventions to be used in the following material, in particular about what a given term or symbol is supposed to mean. Definitions often stand at the beginning of the development of a new subject. Definitions cannot be right or wrong. They can just be more or less useful. For a definition to be useful, it should clearly encapsulate a concept that is important in the development of all the things further down the line. The so defined term or symbol shall be used a lot in the material to be developed and will be referred to often. It makes sense to pick definitions in such a way that theorems can be stated succinctly. What the most useful definitions for a particual (new) mathematical subject are is often not clear from the get go but instead crystallizes out over time as experience with the new subject grows and when a bit of hindsight is available. As users of math, we may take definitions for granted because smart mathematicians have already figured (and fought) them out for us (and for themselves, of course). But we should keep in mind that they are fundamentally just conventions to make it possible (and ideally convenient and easy) to talk about a given subject. They are not fundamental truths. They just establish the language that we will use. That's why sometimes different authors use different definitions. Sometimes there is just no universal consenus (yet or ever) about which definitions are the most useful ones. Which ones are more or less useful may also differ from field to field. So, care has to be taken when reading mathematical material from different sources - the definitions in use may not always agree.

\subsubsection{Set Theory}
Set theory is often said to be the foundation of all mathematics - even much more fundamental than the natural numbers. In fact, it is possible to "construct" the natural numbers from sets. We will not go down this road though, since this is not really relevant in applied math. The idea of a set was initially introduced by Georg Cantor in an intuitive way. His way of establishing set theory later turned out to have some flaws which is why it was later rebuilt more formally. The result of this rebuild is called "axiomatic set theory" and is very abstract and formal. Fortunately, Cantor's view, which is today sometimes called "naive set theory", is good enough for us. 

\paragraph{Sets}
In Cantor's definition "A set is a gathering together into a whole of definite, distinct objects of our perception or of our thought which are called elements of the set.". So, in essence, a set is just a bunch of things. Very general indeed. Sets are usually denoted in curly braces. For example, the set of the 3 letters a,b,c would be denoted as $\{a,b,c\}$. Two sets are considered equal, if and only if they contain the same elements. It does not matter in which order the elements are written down or if an element appears multiple times. So that means, for example, the sets $\{c,a,b\}$ or $\{a,c,a,a,b,c\}$ are in fact equal to the set $\{a,b,c\}$. By the way, the phrase "if and only if" appears sufficiently often in math texts that some authors use the abbreviation "iff" for that - yes, that's an "if" with a double-f. Sets can be given names, for example, we may call our set above $S$ and we may write this as $S = \{a,b,c\}$. Element membership is denoted by an $\in$ symbol, so to express the fact that $b$ is an element of the set $S$, we would write $b \in S$. In math, the sets we are dealing with are often sets of numbers and they may have many or even infinitely many elements. To denote very large or infinite sets compactly there are notations based on predicate logic. For example to denote the set of all numbers larger than 100 but less than 1000, we may write $\{x : x > 100 \wedge x < 1000\}$, but we are getting ahead of ourselves. To understand that notation, we actually first have to understand what $>$ and $<$ means. I'm sure, you do know what they mean, but in the context of set theory, these symbols first needs to be defined, too. To do so, we need some more tools...

\paragraph{Tuples}
Sometimes, we may want to model situations in which the order of elements actually does matter. Sets are per definition not suitable for this, so we need something else. That other thing is the tuple. A tuple is typically denoted by listing the elements in parentheses. The 3-tuple $(a,b,c)$ is not equal to $(c,a,b)$. ...tbc...

\paragraph{Relations}
A relation can formally be defined to be a set of tuples. Of particular importance are binary relations, i.e. sets of 2-tuples. Theses 2-tuples are also called ordered pairs. ...tbc...

% set algebra
%   Relations 
\subsection{Numbers and Arithmetic} 
 \subsubsection{Natural Numbers}

%Numbers (N,Z,Q,R,C), Polynomials, Rational Functions, Transcendental Functions


\section{Linear Algebra}
Linear algebra is the study of vector spaces and linear operations that can be applied to elements of these spaces. It provides tools for solving linear systems of equations and for doing geometric calculations. A vector space is a set of elements, called vectors, with which you can do two things: you can add together two vectors to get another vector and you can multiply a vector by scalar factor (i.e. a number) to get another vector. Scalars derive their name from the fact that they "scale" vectors, i.e. change their "size". These operations of addition and multiplication by a scalar must follow the usual associative, commutative and distributive laws. An operation is said to be linear, if two properties hold: First: applying the operation and then scaling the output by a factor should give the same result as scaling the input by the same factor and then applying the operation. Second: adding two inputs and then applying the operation to the sum must give the same result as applying the operation to both inputs seperately and then summing the results. If the space is $N$-dimensional where N is a finite natural number, vectors can be represented by an array of $N$ numbers and linear operations can be represented by N-by-N matrices (i.e. 2D arrays) of numbers. These numbers will depend on the choice of a basis which determines your coordinate axes, but the vector or operation (which is \emph{represented} by the array or matrix) does not depend on that choice. Vectors in 2D or 3D can be visualized as arrows and linear operations can be visualized as moving the tips of the arrows around in the space (in particular, constrained ways). In a more general setting, operations can also be defined to take a vector from one vector space as input and produce an output that lives in a different vector space. In this case, the representing matrix will be M-by-N (M rows, N columns) where M is the dimensionality of the output space and N the dimensionality of the input space and you have to make a choice for the basis for both of these spaces. Vector spaces can also be infinite dimensional. For example, you may consider the "space" of all functions defined on some interval. In such a case, the operations are usually called "operators". These operators take a function as input and produce another function as output. Examples of such operators are: take the derivative, compute the definite integral, multiply by a number, multiply by some other (fixed) function, take the square, etc. (the last one being actually a nonlinear operator).

\section{Calculus}
Calculus is the study of continuous functions and provides the tools for computing slopes of tangents, areas under curves and identifying important features such as extrema and inflection points. These tools are mainly derivatives and integrals. Functions can also be unknowns which in the context of calculus can be determined by differential- or integral equations. This is especially important in the physical sciences. Calculus also provides tools to approximate a given function by simpler functions that are well understood, most notably by Taylor- and Fourier series. This is important in numerical computations. Series themselves are also an important object of study in calculus. A series is an infinite sum of all the elements of an infinite sequence of numbers. Besides their role in the usual ("infinitesimal") calculus, sequences and series can also be studied by a discrete version of calculus which closely parallels the regular calculus. The discrete counterparts of derivatives, integrals and differential equations are differences, sums and difference equations. The discrete calculus may involve a step size $h$ (when absent, it's usually assumed to be unity) and the continuous calculus may be seen as the limiting case, when the stepsize approaches zero. To figure out what happens in this limiting process, calculus also studies the idea of a limit itself. This idea is actually the foundation of the usual calculus on which the ideas of derivatives and integrals are based.
\subsection{Continuous Functions}
 \subsubsection{Limits}
Consider the function $f(x) = \sin (x) / x$. As it stands, it is not defined at $x = 0$ because we would have a division by zero there. In fact, we have an indeterminate form of the form "zero divided by zero". We could just define the function to assume an arbitrary value at $x = 0$, say $f(0) = 0$. But it turns out that there is a more natural way to fill the gap. Imagine plugging ever smaller numbers into $f(x)$, like $0.1, 0.01, 0.001, $ etc. and observe what $f(x)$ does. It turns out that $f(x)$ will approach one. That may suggest to define $f(0) = 1$. But not so fast: What if we would approach zero from the left, using $-0.1, -0.01, -0.001,$ etc. as inputs instead? Good news: It turns out that $f(x)$ also approaches one in this case, so we are indeed justified to define $f(0)=1$ to fill the gap in a natural and meaningful way. The idea that a function approaches a certain value $f(x_0)$ when its input values approach a given value $x_0$ leads to the idea of a limit. We need to distinguish the limits approaching from the left and approaching from the right. Only when both of these are equal, we say that \emph{the} limit exists and we define it to be just that value. 
TODO: introduce mathematical notation for limits


\subsubsection{Continuity}
If the limit of a function $f$ exists at a given $x_0$ and agrees with the actual function value $f(x_0)$, we say that the function $f$ is continuous at $x_0$. With that definition in place, we can now be more specific about what we have meant with "filling the gap in a natural way". We have filled it in such a way as to make the resulting function continuous at $x_0 = 0$. At all other values of $x_0$, it already has been continuous all along. If a function is continuous at \emph{all} values $x_0$, no matter what we choose as $x_0$, we say that the function is continuous without the qualification "at $x_0$". Intuitively, continuity means that the graph of the function has no sudden jumps (a.k.a. discontinuities) and we can draw it without lifting the pencil or drawing vertical lines.


 
\subsection{Differentiation}
 %\subsubsection{Differentiation}
Consider an arbitrary function $f(x)$ and imagine that we want to compute the slope of the graph at a given $x_0$. We could approximate it by considering the point $(x_0, f(x_0))$ and another point a small distance $h$ further to the right $(x_0+h, f(x_0+h))$ and compute the "rise over run" quotient $(f(x_0+h) - f(x_0)) / h$. For smaller and smaller $h$, the approximation would get better and better. Now we consider the limit as $h$ approaches zero. If that limit exists, it actually \emph{is} our desired slope.


\subsection{Integration} 
 \subsubsection{Integration by Parts} 
\begin{align}
  \int_a^b u(x) v'(x) \, dx 
  & = \Big[u(x) v(x)\Big]_a^b - \int_a^b u'(x) v(x) \, dx\\[6pt]
  & = u(b) v(b) - u(a) v(a) - \int_a^b u'(x) v(x) \, dx.
\end{align}
With $u = u(x)$, $du = u'(x) \,dx$ such that $v = v(x)$ and $dv = v'(x) dx$:
\begin{align}
  \int u \, dv \ =\ uv - \int v \, du
\end{align}
 
\subsection{Differential Equations}
%\subsection{Series}
%\subsection{Integral Transforms} % Fourier- and Laplace trafo
%\subsection{Integral Equations}
%ToDo: finite differences, summation calculus, difference equations, z-trafo
% see https://www.youtube.com/watch?v=4AuV93LOPcE


\section{Multilinear Algebra}
Multilinear algebra is the study of operations that take multiple inputs and are linear in each of those inputs. For example, consider the sandwich product of two vectors and a matrix: $\mathbf{z = x^T A y}$. The matrix $\mathbf{A}$ can be thought of defining an operation that takes two vectors as inputs and produces a scalar as output. The so defined operation is indeed linear in both inputs in the sense that it makes no difference whether you add or scale at the input or output side.
 % covectors/k-forms, tensors, ..

\section{Multivariable Calculus}
Multivariable calculus is the calculus of continuous functions that may depend on several input variables and/or produce several output variables. As with single variable calculus, one may find minima and maxima of a function with multiple inputs using derivatives by requiring them to be zero (yes: derivatives in plural!). This is important in optimization. Because we now have several input variables, we may take derivatives with respect to any of them which is why they are called partial derivatives in this context. We can collect them in a vector which we then call the gradient. The tool to compute areas under functions - the integral - generalizes to compute other aggregate quantities such as lengths of curves and volumes and surface areas of 3D objects, the amount of a fluid flowing through a surface, the work done by a field on an object moving through that field, etc. The multivariable version of a differential equation is a partial differential equation, named like that because it involves partial derivatives. So, despite its name, a "partial" differential equation is actually a (much) more complicated thing than an ordinary differential equation. A subfield of multivariable calculus, called vector calculus, is mainly concerned with functions that live in the same 3-dimensional space that we do, making it suitable to describe many physical phenomena. The 3D vector calculus uses some special notations that are specific to 3D and do not readily generalize to an arbitrary number of dimensions. Most notably among them is the curl which is defined by a certain cross-product which itself makes only sense in 3D. Sometimes the equations are boiled down to just two dimensions, which you can easily embed in 3 dimensions, so with some hacks (the curl becomes a scalar) it kinda works there, too. Sometimes the equations also involve time which is then treated seperately from the spatial dimensions. Time being treated as "something else" than space coincides with our daily experience but makes it less than ideal in the context of relativity where time is treated on equal footing, which is why you often see different formalisms used there. Alternatives to vector calculus that do generalize to higher dimensional spaces are exterior calculus (a.k.a. calculus of differential forms), tensor calculus or geometric calculus. All these calculi are based on their respective algebras each of which defines a particular product and the problematic cross-product from 3D vector algebra is not a thing anymore in these algebras.

% scalar fields, vector fields, partial differentiation (grad, Jacobian, Hessian, Laplacian, divergence, curl), integral theorems (Green, Gauss, Stokes), differential forms, generalized Stokes theorem, Tensor fields


\section{Complex Calculus}
Complex calculus is the calculus of functions that map a complex number to another complex number. Since input and output are both 2-dimensional, it makes sense to look at complex calculus from the perspective of 2D vector fields. For a function $w = f(z) = u(x,y) + \i v(x,y)$ to be differentiable in the complex sense, the partial derivatives of the component functions $u,v$ with respect to $x,y$ must exist and satisfy the Cauchy-Riemann (CR) differential equations: $u_x = v_y, u_y = -v_x$. This condition has wide ranging implications which allow a lot of simplifications to be made in the computations of path integrals. This can also help to evaluate certain definite real integrals that would otherwise be hard to do. If the function $f$ is considered as a mapping that maps points from the $z$-plane to the $w$-plane, the CR equations imply that the images of curves in the $z-$plane that intersect at a given angle, will intersect at the same angle in the $w$-plane. Such an angle-preserving mapping is also called a conformal mapping. If a function is complex differentiable at a given point $z_0$, this implies that all higher order derivatives also exist and that the function can be locally expanded in a Taylor series that converges in a circular region whose radius extends to the nearest critical point [verify!]. A critical point is...
%todo: say something about multivariable complex calculus

%\section{Approximation}
%\paragraph{Trigonometric Interpolation} We have a bunch of datapoints $x_k, y_k$ and want to find 



\begin{thebibliography}{10}
\bibitem{MatheRezepte}
Christian Karpfinger. \newblock {\em Hoehere Mathematik in Rezepten, 3. Auflage}.
\end{thebibliography}	

\bibliography{Bibilography} 
\bibliographystyle{plain}  
	
\end{document}