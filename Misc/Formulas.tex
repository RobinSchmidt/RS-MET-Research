%\documentclass[12pt, twocolumn]{article}
\documentclass[12pt]{article}
%\usepackage{fullpage}           % makes all margins 1 inch?
\topmargin=-1.0cm
\textheight=23cm
\evensidemargin=-1.0cm
\oddsidemargin=-1.0cm
\textwidth=19cm
\setcounter{secnumdepth}{-1}     % suppress numbering of sections
\usepackage{amsmath}
\usepackage{amssymb}             % for mathbb
%\usepackage[latin1]{inputenc}    % for ä,ö,ü - doesn't work, we use ae,oe,ue

\begin{document}
	
% formatting:
\parindent=0in
\parskip=0pt
\pagenumbering{roman}
		
% main text
\pagenumbering{arabic} \setcounter{page}{1}

\title{Random Math Formulas}
\maketitle

\section{}
This document contains miscellaneous math formulas gathered from various sources that may or may not become useful at some point. A special focus is on recipe-like formulas that are suitable for being translated into code to become part of the library someday.

\tableofcontents





\section{Linear Algebra}

\section{Calculus}
\subsection{Differentiation}
\subsection{Integration} 
 \subsubsection{Integration by Parts} 
\begin{align}
  \int_a^b u(x) v'(x) \, dx 
  & = \Big[u(x) v(x)\Big]_a^b - \int_a^b u'(x) v(x) \, dx\\[6pt]
  & = u(b) v(b) - u(a) v(a) - \int_a^b u'(x) v(x) \, dx.
\end{align}
With $u = u(x)$, $du = u'(x) \,dx$ such that $v = v(x)$ and $dv = v'(x) dx$:
\begin{align}
  \int u \, dv \ =\ uv - \int v \, du
\end{align}
 
\subsection{Differential Equations}
%\subsection{Limits}
%\subsection{Series}
%ToDo: limits, series

\section{Approximation}
%\paragraph{Trigonometric Interpolation} We have a bunch of datapoints $x_k, y_k$ and want to find 



\begin{thebibliography}{10}
\bibitem{MatheRezepte}
Christian Karpfinger. \newblock {\em Hoehere Mathematik in Rezepten, 3. Auflage}.
\end{thebibliography}	

\bibliography{Bibilography} 
\bibliographystyle{plain}  
	
\end{document}