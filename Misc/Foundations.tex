This book is about applied math, so we will not go deeply into the foundations of math. However, a brief and very superficial look into this topic is a good idea to set the stage for the material that follows. It also establishes the basics of the language which we will need to talk about mathematical concepts.

\subsubsection{Logic}
Math is the pursuit of finding truths, so it makes sense to have a framework, within which we can say that something is true or false. In mathematics, that framework is mathematical logic, more specifically propositional logic and predicate logic (a.k.a. first order logic). Propositional logic deals with propositions which are statements that can be either true or false. You also have ways of combining given propositions to make new, more complex propositions. For example, you can combine two propositions with a logical "and" (usually denoted as $\wedge$). The resulting new proposition is true, if and only if both of the input propositions are true, otherwise it's false. You also have a logical "or" (denoted as $\vee$), which in this context is taken to be an inclusive or: the combined proposition is true, if any one of the input propositions or both are true. You also have a logical "not" (denoted as $\neg$) which takes a single proposition as input and the result is true, if the the input is false and vice versa. To build up mathematics, propositional logic is not quite enough. Predicate logic builds up on propositional logic and lets you talk about objects and relations between them. There, you have so called quantors like the symbol for "there exists an object such that..." (denoted as $\exists$) or a symbol for "for all objects it is true that..." (denoted as $\forall$). There are yet other levels and kinds of logic, but these two are enough for the moment.

\subsubsection{Axioms}
One has to start somewhere. That starting point is typically a set of "axioms" together with the rules of logic. An axiom is a proposition that is just assumed to be true without further justification. Axioms should state things that are "obviously true". An example are the Peano axioms, some of which are: zero is a natural number, each natural number has a successor, any number is equal to itself, etc. If you really want to build up the whole tower of mathematics axiomatically, you have to \emph{choose} a set of axioms and from there, using only the rules of logic, find new propositions that are also true.

\subsubsection{Theorems and Proofs}
If you want to prove a proposition, the tools that you have in hand are all the propositions that are already known to be true together with the rules of logic. A proposition is known to be true if it is either an axiom or it has been previously proven by the same technique. A proof for a proposition is a sequence of true propositions in which each one follows from known or previous ones by applying the rules of logic. If a proposition has been proven to be true, it becomes a "theorem". A theorem is a fundamentally important concept in math - math is all about finding theorems. You may have observed a pattern by looking at a bunch of examples and you may conjecture that the pattern is generally true. What you then have to do is to find a proof for you conjecture. If you have succeeded in this highly creative endeavour (i.e. your proof is determined to be correct by the mathematical community), your conjecture is elevated to the venerable status of a theorem. And if the theorem is important enough and you were the first to prove it, you will typically achieve immortality by having your name attached to the theorem for the rest of eternity. Thousands of years later, still everbody knows the name of Pythagoras today - although, it wasn't actually him who proved "Pythagoras' Theorem" - sometimes the world is a little unfair, too :'-(.

\subsubsection{Definitions}
OK - this is kind of awkward. We have to \emph{define} what \emph{a definition} is.

\subsubsection{Set Theory}


% Axioms, Theorems, Relations