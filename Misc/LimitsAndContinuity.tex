\subsubsection{Limits}
Consider the function $f(x) = \sin (x) / x$. As it stands, it is not defined at $x = 0$ because we would have a division by zero there. In fact, we have an indeterminate form of the form "zero divided by zero". We could just define the function to assume an arbitrary value at $x = 0$, say $f(0) = 0$. But it turns out that there is a more natural way to fill the gap. Imagine plugging ever smaller numbers into $f(x)$, like $0.1, 0.01, 0.001, $ etc. and observe what $f(x)$ does. It turns out that $f(x)$ will approach one. That may suggest to define $f(0) = 1$. But not so fast: What if we would approach zero from the left, using $-0.1, -0.01, -0.001,$ etc. as inputs instead? Good news: It turns out that $f(x)$ also approaches one in this case, so we are indeed jusitified to define $f(0)=1$. The idea that a function approaches a certain value $f(x_0)$ when its input values approach a given value $x_0$ leads to the idea of a limit. We need to distinguish the limits approaching from the left and approaching from the right. Only when both of these are equal, we say that \emph{the} limit exists and we define it to be just that value. 


\subsubsection{Continuity}
If the limit exists at a given $x_0$ and agrees with the actual function value at $x_0$, we say that the function $f$ is continuous at $x_0$. If a function is continuous at \emph{all} values $x_0$, no matter what we choose as $x_0$, we say that the function is continuous. Intuitively, continuity means that the graph of the function has no sudden jumps (a.k.a. discontinuities) and we can draw it without lifting the pencil or drawing vertical lines.


