Mathematics deals with various kinds of numbers, the most important of which are the natural, integer, rational, real and complex numbers. These sets of these numbers are denoted by  $\mathbb{N,Z,Q,R,C}$ respectively. In the given order, the sets further to the left are actually subsets of all the sets further to the right, so we have $\mathbb{N \subset Z \subset Q \subset R \subset C}$. The set $\mathbb{C}$ does not have to be the end of this progression (there are even larger sets such as quaternions, octonions, etc.), but in a very meaningful and satisfying sense, it can be justified to stop at $\mathbb{C}$.

\subsubsection{Natural Numbers}
Natural numbers are the counting numbers $0,1,2,3,4,5,\ldots$. They are indeed "natural" in the sense that we have an intuitive understanding what a number like $3$ means. It is an act of abstraction though, to recognize that a set of 3 apples and a set of 3 orang-utans have something in common, namely that they both have the "size" of 3. Whether or not zero is considered a natural number is a matter of convention. The historical fact that the "invention" (or discovery? TODO: elaborate in a footnote) of the number zero was a rather late development in western mathematics (it was recognized much earlier in indian mathematics) may suggest that the idea of zero is not so natural after all. However, I adopt the convention here that zero is a natural number because especially in the context of computers and programming, it is more often than not more convenient to have zero available in the set. There is no right or wrong here. This is a typical case of a definition for which there is no universal consensus in the math community. If zero is in the set $\mathbb{N}$ per definition but we nevertheless need to make a statement about all natural numbers except zero, the notation $\mathbb{N}^+$ is often used. Conversely, if some author defines zero to not be in the set $\mathbb{N}$ but wants to make a statement about all natural numbers including zero, the notation $\mathbb{N}_0$ is often used.

\paragraph{Elementary Operations}

\paragraph{Solving Equations}




\subsubsection{Integer Numbers}


\subsubsection{Rational Numbers}


\subsubsection{Real Numbers}

% intervals


\subsubsection{Complex Numbers}


\subsubsection{Other Numbers}
% make small paragraphs about: ordinal, cardinal, hyperreal, dual, p-adic, surreal, clifford numbers (aka multivectors), quaternions, octonions, sedenions, modular integers - what about matrices as representations?
