\subsubsection{Natural Numbers}
Natural numbers are the counting numbers $0,1,2,3,4,5,\ldots$. They are indeed "natural" in the sense that we have an intuitive understanding what a number like $3$ means. It is an act of abstraction though, to recognize that a set of 3 apples and a set of 3 orang-utans have something in common, namely that they both have the "size" of 3. Whether or not zero is considered a natural number is a matter of convention. The historical fact that the "invention" of the number zero was a rather late development in western mathematics (it was recognized much earlier in indian mathematics) may suggest that the idea of zero is not so natural after all. However, I adopt the convention here that zero is a natural number. Having zero in the set of natural numbers makes the statement and/or proof of some theorems more convenient and of some others less convenient, so which convention a given author adopts may depend on the author's preferences about which statements they want to be able to state more conveniently. There is no right or wrong here - it's a matter of definition.

\paragraph{Interlude} Above, I used the terms "set", "definition", "theorem" and "proof" assuming that the reader has an intuitive understanding what they mean. However, these terms are, in fact, also technical terms in mathematics and there are whole branches of mathematics that go into excruciating detail about what these mean. We ain't got no time for that. To get up to speed with math, we'll just assume that they are intuitively understood. That has to be good enough for the time being.
%maybe make a section "Foundations" at the end that defines these terms better, also define "axiom"

\paragraph{}

\subsubsection{Integer Numbers}


\subsubsection{Rational Numbers}


\subsubsection{Real Numbers}


\subsubsection{Complex Numbers}