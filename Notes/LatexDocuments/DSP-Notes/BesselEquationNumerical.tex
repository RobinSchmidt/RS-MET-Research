\title{Solving the Bessel Equation Numerically}
\author{Robin Schmidt (www.rs-met.com)}
\date{\today}
\maketitle

We consider a couple of technical details in implementing a numerical solution for the Bessel differential equation using an initial-value problem solver. The Bessel differential equation is given by:
\begin{equation}
 x^2 y'' + x y' + (x^2 - n^2)y = 0
\end{equation}
where $n$ is a free parameter.


\section{Transforming it into a 1st Order System}
The Bessel equation is a 2nd order differential equation. In order to make it suitable for a numerical initial-value-problem solver, we need to transform it into a 2-dimensional 1st order system. First, we solve for the highest derivative:
\begin{equation}
 y'' = - \frac{x y' + (x^2 - n^2) y}{x^2}
\end{equation}
Now we define:
\begin{equation}
 y_1 = y, \; y_2 = y' \; 
\end{equation}
which implies:
\begin{equation}
 \begin{aligned}
 y_1' &= &y'  &= y_2 \\
 y_2' &= &y'' &= - \frac{x y_2 + (x^2 - n^2) y_1}{x^2}
 \end{aligned}
\end{equation}
Collecting $y_1, y_2$ into a vector $\mathbf{y}$, we can write this as a vector-valued function:
\begin{equation}
 \mathbf{y}' =\mathbf{f}(x, \mathbf{y}) 
\end{equation}
which is in the form that we need for an initial value problem solver (for example, based on Runge-Kutta or whatever). A little complication arises when $x$ is close to zero because we have to divide by $x^2$ in the calculation of $y_2'$. This is treated by using the limit when $x$ approaches zero, whenever $x$ is close enough to zero. This limit is given by $-y_1$.







%\begin{thebibliography}{9}  % 9 indicates that there are no more than 9 entries
% %\bibitem{Gum} Charles Constantine Gumas. A century old, the fast Hadamard transform proves useful in digital communications
%\end{thebibliography}

