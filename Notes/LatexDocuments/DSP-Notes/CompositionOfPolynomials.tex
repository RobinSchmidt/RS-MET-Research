\title{Composition of Polynomials}
\author{Robin Schmidt (www.rs-met.com)}
\date{\today}
\maketitle

We consider the problem of finding the coefficients $c_0, \ldots c_{N_C}$ of an $N_C$th order polynomial $C(x)$ that is a composition of two polynomials $A(x)$ and $B(x)$ which have orders $N_A$ and $N_B$ respectively. 

\section{Notation}
We write the polynomials $A(x), B(x), C(x)$ as:
\begin{equation}
  A(x) = \sum_{m=0}^{N_A} a_m x^m, \qquad
  B(x) = \sum_{n=0}^{N_B} b_n x^n, \qquad
  C(x) = \sum_{k=0}^{N_C} c_k x^k
\end{equation}
where $C(x)$ is taken to be the composition of $A(x)$ and $B(x)$:
\begin{equation}
  C(x) = B \left( A(x) \right) = \sum_{n=0}^{N_B} b_n \left( \sum_{m=0}^{N_A} a_m x^m \right)^n
\end{equation}
The order of the resulting polynomial $C(x)$ is given by: 
\begin{equation}
 N_C = N_A N_B
\end{equation}

\section{Some Special Cases}

\subsection{Case $N_B = 0$}
\begin{equation}
  C(x) = \sum_{n=0}^{0} b_n \left( \sum_{m=0}^{N_A} a_m x^m \right)^n
       =                b_0 \left( \sum_{m=0}^{N_A} a_m x^m \right)^0
       = b_0
\end{equation}
from which we see:
\begin{equation}
  c_0 = b_0
\end{equation}

\subsection{Case $N_A = 0$}
\begin{equation}
  C(x) = \sum_{n=0}^{N_B} b_n \left( \sum_{m=0}^{0} a_m x^m \right)^n
       = \sum_{n=0}^{N_B} b_n \left( a_0 x^0 \right)^n
       = \sum_{n=0}^{N_B} b_n a_0^n
\end{equation}
from which we see:
\begin{equation}
  c_0 = \sum_{n=0}^{N_B} b_n a_0^n
\end{equation}

\subsection{Case $N_A = 1, N_B = 1$}
\begin{equation}
  C(x) = \sum_{n=0}^{1} b_n \left( \sum_{m=0}^{1} a_m x^m \right)^n
       = \sum_{n=0}^{1} b_n \left( a_0 x^0 + a_1 x^1 \right)^n
       = b_0 (a_0 x^0 + a_1 x^1)^0 + b_1 (a_0 x^0 + a_1 x^1)^1
\end{equation}
simplifying and rearranging the last line, we get:
\begin{equation}
 C(x) = (b_0 + b_1 a_0) x^0 + (b_1 a_1) x^1
\end{equation}
from which we see the coefficients:
\begin{equation}
  c_0 = b_0 + b_1 a_0, \qquad c_1 = b_1 a_1
\end{equation}





%\begin{thebibliography}{9}  % 9 indicates that there are no more than 9 entries
% %\bibitem{Gum} Charles Constantine Gumas. A century old, the fast Hadamard transform proves useful in digital communications
%\end{thebibliography}

