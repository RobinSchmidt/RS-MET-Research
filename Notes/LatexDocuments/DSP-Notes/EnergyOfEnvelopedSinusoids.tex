\title{Energy of Enveloped Sinusoids}
\author{Robin Schmidt (www.rs-met.com)}
\date{\today}
\maketitle

We consider the problem of finding the total energy of a sinusoidal function with an amplitude envelope. Our function is of the general form:
\begin{equation}
 f(t) = A \cdot e(t) \cdot \sin(\omega t + \phi)
\end{equation}
where $e(t)$ is our envelope function.

\section{Preliminaries}
In the following material, we need the following closed-form expressions for certain definite integrals:
\begin{equation}
\label{Eq:Integrals}
 \int_0^{\infty} e^{-\alpha t} dt = \frac{1}{\alpha}, \quad
 %\int_0^{\infty} e^{-\alpha t} \sin(\omega t + \phi) dt = \frac{\omega \cos \phi + \alpha \sin \phi}{\alpha^2 + \omega^2}, \quad 
 \int_0^{\infty} e^{-\alpha t} \cos(\omega t + \phi) dt = \frac{\alpha \cos \phi - \omega \sin \phi}{\alpha^2 + \omega^2}
\end{equation}

\section{Exponentially Decaying Sinusoid}
We consider the case, where the envelope is given by an exponentially decaying function:
\begin{equation}
 e(t) = e^{-\alpha t}
\end{equation}
where $\alpha$ determines the rate of decay. The function $f(t)$ is therefore given by:
\begin{equation}
\boxed
{
  f(t) = A \cdot  e^{-\alpha t} \cdot \sin(\omega t + \phi)
}
\end{equation}
The instantaneous power is defined as the function squared:
\begin{equation}
  f^2(t) = \left( A \cdot  e^{-\alpha t} \cdot \sin(\omega t + \phi) \right)^2
         = A^2 \cdot e^{-2\alpha t} \cdot \sin^2(\omega t + \phi)
\end{equation}
Using the trigonometric identity $\sin^2(x) = (1-\cos(2x))/2$ leads to:
\begin{equation}
  f^2(t) = \frac{A^2}{2} \left( e^{-2\alpha t} - e^{-2\alpha t} \cos(2\omega t + 2\phi) \right)
\end{equation}
The energy is defined as the definite integral of the instantaneous power over its entire range (which, in our problem setting, goes from zero to infinity):
\begin{equation}
  E = \int_0^{\infty} f^2(t) dt
\end{equation}
and so we have to evaluate:
\begin{equation}
\begin{aligned}
  E &= \int_0^{\infty} \frac{A^2}{2} \left( e^{-2\alpha t} - e^{-2\alpha t} \cos(2\omega t + 2\phi) \right) \\
    %&= \frac{A^2}{2} \int_0^{\infty} \left( e^{-2\alpha t} - e^{-2\alpha t} \cos(2\omega t + 2\phi) \right) \\
    &= \frac{A^2}{2} \left(\int_0^{\infty}  e^{-2\alpha t} - \int_0^{\infty}  e^{-2\alpha t} \cos(2\omega t + 2\phi) \right) \\
\end{aligned}  
\end{equation}
which we may evaluate using $\ref{Eq:Integrals}$, recognizing that $\alpha, \omega, \phi$ must be replaced by twice their values to match the formulas:
\begin{equation}
\boxed
{
  E = \frac{A^2}{2} \left( \frac{1}{2\alpha} - \frac{2\alpha\cos(2\phi)-2\omega\sin(2\phi)}{(2\alpha)^2+(2\omega)^2} \right) \\ 
}
\end{equation}

\section{Sinusoid with Attack/Decay Envelope}
Now, we consider an attack/decay  envelope, given by a difference of exponential decays with different decay rates:
\begin{equation}
 e(t) = e^{-\alpha_1 t} - e^{-\alpha_2 t}
\end{equation}
so, our function $f(t)$ is:
\begin{equation}
\boxed
{
  f(t) = A \cdot (e^{-\alpha_1 t} - e^{-\alpha_2 t}) \cdot \sin(\omega t + \phi)
}
\end{equation}
the instantaneous power is given by:
\begin{equation}
  f^2(t) = A^2 \cdot 
  \left(
      e^{-2\alpha_1 t}           \sin^2(\omega t+\phi) 
  - 2 e^{-(\alpha_1+\alpha_2) t} \sin^2(\omega t+\phi)  
  +   e^{-2\alpha_2 t}           \sin^2(\omega t+\phi)  
  \right)
\end{equation}
Now, for the evaluation of the energy integral, we proceed in a similar (but more tedious) way as above. We just have to do 3 integrals, each of which is of the same general form as the one above. The result is:
\begin{equation}
\boxed
{
  E = \frac{A^2}{2} 
  \left( 
   \frac{1}{2\alpha_1} + \frac{1}{2\alpha_2} - \frac{2}{\alpha_1 + \alpha_2}
    - \frac{2\alpha_1 c - 2\omega s}{(2\alpha_1)^2+(2\omega)^2}
    - \frac{2\alpha_2 c - 2\omega s}{(2\alpha_2)^2+(2\omega)^2} 
    + 2\frac{(\alpha_1+\alpha_2) c -2\omega s}{(\alpha_1+\alpha_2)^2+(2\omega)^2} 
   \right)
}
\end{equation}
where $s = \sin(2 \phi), c = \cos(2 \phi)$.
 

%TODO: allow different frequencies ($\omega_1, \omega_2$) and different start phases ($\phi_1, \phi_2$), maybe also different amplitudes

%TODO: compute indefinite integrals and subsequently definite integrals with finite bounds, see page 228/229 in 
%"Table of Integrals series and Products"' for expressions for sine and cosine and use a combination of both
%to realize an arbitrary start-phase




%\begin{thebibliography}{9}  % 9 indicates that there are no more than 9 entries
% %\bibitem{Gum} Charles Constantine Gumas. A century old, the fast Hadamard transform proves useful in digital communications
%\end{thebibliography}

