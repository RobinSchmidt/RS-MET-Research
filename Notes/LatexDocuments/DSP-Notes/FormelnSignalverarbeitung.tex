\documentclass[12pt]{report}

\topmargin=-0.8in
\evensidemargin=0in
\oddsidemargin=0in
\textheight=10in
\textwidth=6.5in

%\usepackage{fullpage}         %sorgt daf�r, dass alle R�nder 1 inch breit sind

\usepackage{amsmath}
\usepackage{amssymb}          % f�r mathbb
\usepackage[latin1]{inputenc} %%% damit man � tippen kann
\usepackage{german}

% Deklaration von einigen Operatoren:
\DeclareMathOperator{\grad}{grad}
\DeclareMathOperator{\dive}{div}
\DeclareMathOperator{\rot}{rot}
\DeclareMathOperator{\arccot}{arccot}
\DeclareMathOperator{\Res}{Res}

\begin{document}

%Formatierungseinstellungen:
\parindent=0in
\parskip=0pt

\title{Formelsammlung Signalverarbeitung (unfertige Arbeitsversion)}
\author{zusammengestellt von Robin Schmidt}
\date{\today}
\maketitle

%hier kommen ein paar Vorbemerkungen (Inhalsverzeichnis, etc.)
\pagenumbering{roman}
\tableofcontents{}

% hier werden Skalierungsfaktoren f�r die verschiedenen Arten von Abbildungen definiert:
\def\diagramScale{0.6}
\def\plotScale{0.66}


%ab hier beginnt der eigentliche Text:
\pagenumbering{arabic} \setcounter{page}{1}

% Hauptteil:
 \chapter{Zeitdiskrete Folgen}

\section{Definitionen}

\paragraph{Abtastung:} sei $x_c(t)$ eine zeitkontinuierliche Funktion und $T_s$ die Abtastperiode, dann ist die zu $x_c(t)$ geh�rende zeitdiskrete Folge:
\begin{equation}
 x[n] = x_c(n \cdot T_s)
\end{equation}

\paragraph{normierte Kreisfrequenz:} seien $f$ und $f_s$ eine Signalfrequenz und die Abtastrate (in $Hz$), dann ist die zu $f$ geh�rige normierte Kreisfrequenz:
\begin{equation}
 \Omega = \frac{2 \pi f}{f_s}
\end{equation}


\paragraph{Einheitsimpuls:}
\begin{equation}
 \delta [n] = 
  \begin{cases}
  0 & \text{f�r }  n \neq 0 \\ 
  1 & \text{f�r }  n = 0 
 \end{cases} 
\end{equation}

\paragraph{Einheitssprung:}
\begin{equation}
 u [n] = 
  \begin{cases}
  0 & \text{f�r }  n < 0    \\
  1 & \text{f�r }  n \geq 0
 \end{cases} 
\end{equation}

es gilt:
\begin{equation}
 \begin{aligned}
  \delta [n] &= u[n] - u[n-1] \\
  u [n]      &= \sum_{m=-\infty}^n \delta[m]
 \end{aligned}
\end{equation}

\paragraph{komplexe Exponentialfolge:} sei $z = r e^{j \Omega}$ eine komplexe Zahl
\begin{equation}
 x[n] = A \cdot z^n = A \cdot r^n e^{j \Omega n}
\end{equation}


 
 \chapter{Fourier-Transformation}

\paragraph{Diskrete Fourier-Transformation (DFT):}
\begin{equation}
 X[k] = \sum_{n=0}^{N-1} x[n] \; e^{-j 2 \pi n k / N} \qquad k=0,1, \ldots , N-1
\end{equation}

\paragraph{Inverse Diskrete Fourier-Transformation (IDFT):}
\begin{equation}
 x[n] = \frac{1}{N} \sum_{k=0}^{N-1} X[k] e^{j 2 \pi n k / N} \qquad n=0,1, \ldots , N-1
\end{equation}

\paragraph{Zeitdiskrete Fourier-Transformation (DTFT):}
\begin{equation}
 X(e^{j \Omega}) = \sum_{n=-\infty}^{\infty} x[n] \; e^{-j \Omega n}
\end{equation}

\paragraph{Inverse Zeitdiskrete Fourier-Transformation (IDTFT):}
\begin{equation}
 x[n] = \frac{1}{2 \pi} \int_{-\pi}^{\pi} X(e^{j \Omega}) 
\end{equation}
stimmt das?

\paragraph{Umrechnung Frequenz/Bin-Index:} sei $k$ der Bin-Index, f die Frequenz in $Hz$, $f_s$ die Sample-Rate,  $\Omega = 2 \pi f / f_s$ die zugeh�rige normierte Kreisfrequenz und $N$ die DFT-Blockl�nge:
\begin{equation}
 \begin{aligned}
  f       &= \frac{k f_s}{N}         \\
  \Omega  &= \frac{2 \pi k}{N}       \\
 \end{aligned}
\end{equation}


\paragraph{Fourier-Reihe:}

\paragraph{Fourier-Transformation:}

\paragraph{inverse Fourier-Transformation:}
 \chapter{IIR-Filter}

\paragraph{Differenzengleichung:}
\begin{equation}
 y[n] = \sum_{k=0}^N b_k x[n-k] - \sum_{k=1}^M a_k y[n-k]
\end{equation}

\paragraph{Transferfunktion:}
\begin{equation}
 H(z) = \frac{ \sum_{k=0}^N b_k z^{-k} } { \sum_{k=0}^M a_k z^{-k} }
      = \frac{ \sum_{k=0}^N b_k z^{-k} } { 1 + \sum_{k=1}^M a_k z^{-k} }; \qquad a_0 = 1
\end{equation}

\paragraph{Komplexer Frequenzgang:}
\begin{equation}
 H(e^{j \omega})
 = |H(e^{j \omega})| e^{\theta(j \omega)}
 = \frac{ \sum_{k=0}^N b_k \cos(k \omega)  
          - j \sum_{k=0}^N b_k \sin(k \omega)  
        } 
        { \sum_{k=0}^M a_k \cos(k \omega) 
          - j \sum_{k=0}^M a_k \sin(k \omega) 
        }
\end{equation}

\paragraph{Betragsfrequenzgang:}
\begin{equation}
 |H(e^{j \omega})| 
 =
 \left( \frac{   \left( \sum_{k=0}^N b_k \cos(k \omega) \right)^2 
              +  \left( \sum_{k=0}^N b_k \sin(k \omega) \right)^2 
             } 
             {   \left( \sum_{k=0}^M a_k \cos(k \omega) \right)^2
              +  \left( \sum_{k=0}^M a_k \sin(k \omega) \right)^2
             }
             \right)^{1/2}
\end{equation}

\paragraph{Phasenfrequenzgang:}
\begin{equation}
 \theta(j \omega) = - \tan^{-1} \frac{\sum_{k=0}^N b_k \sin(k \omega)}
                                     {\sum_{k=0}^N b_k \cos(k \omega)}
                    + \tan      \frac{\sum_{k=0}^M a_k \sin(k \omega) }
                                     {\sum_{k=0}^M a_k \cos(k \omega)}
\end{equation}
Frequenzg�nge nach dem Buch von Shenoi S.188 - aber er summiert im Nenner in der einen Summe bis $N$ in der anderen bis $M$ - das macht aber irgendwie keinen Sinn - was stimmt hier nun? (Achtung: ich habe $N$ in $M$ umbenannt und umgekehrt)

\paragraph{Gruppenlaufzeit:}
\begin{equation}
 \tau(\omega) = -\frac{d \theta(j \omega)}{d \omega} 
              = \frac{1}{1+u^2} \frac{du}{d \omega} - \frac{1}{1+v^2} \frac{dv}{d \omega}
\end{equation}
mit
\begin{equation}
 u = \frac{\sum_{k=0}^N b_k \sin(k \omega)}  {\sum_{k=0}^N b_k \cos(k \omega)}; \qquad
 v = \frac{\sum_{k=0}^M a_k \sin(k \omega)}  {\sum_{k=0}^M a_k \cos(k \omega)}
\end{equation}
 \chapter{Informations- und Codierungstheorie}

\section{Definitionen}

\paragraph{Einzelwahrscheinlichkeit:} sei $X$ eine diskrete Zufallsvariable mit m�glichen Realisierungen $x_k, \; k=1, \ldots,  K$. Die Wahrscheinlichkeit, dass $X$ den konkreten Wert $x_k$ annimmt ist:
\begin{equation}
 \label{eqn:Auftretenswahrscheinlichkeit}
 p_k = P(X = x_k)
\end{equation}
mit den Bedingungen
\begin{equation}
 0 \leq p_k \leq 1 \; \forall k \qquad \text{und} \qquad \sum_{k=1}^K p_k = 1
\end{equation}

\paragraph{Informationsgehalt} eines Ereignisses $x_k$:
\begin{equation}
 I(x_k) = \log \left( \frac{1}{p_k}  \right) = -\log(p_k)
\end{equation}
wenn der Logarithmus zur Basis 2 gew�hlt wird, ist die Einheit der Information $bits$, beim nat�rlichen Logarithmus $nats$. 1 $bit$ ist der maximale Informationsgehalt einer Ja/Nein Antwort.

\paragraph{Entropie} der Zufallsvariable $X$:
\begin{equation}
 H(X) = E \{ I(x_k) \} = \sum_{k=1}^K p_k I(x_k) = - \sum_{k=1}^K p_k \log p_k
\end{equation}
die Entropie ist also der mittlere Informationsgehalt.

\paragraph{maximale und relative Entropie:}
wenn alle $K$ Elementarereignisse gleichwahrscheinlich sind (d.h. $p_k = 1/K \; \forall k$), dann ist die Entropie maximal und nimmt den Wert: $H_{max} = \log_2 K$ an. Die relative Entropie (einer H�ufigkeitsverteilung) ist das Verh�ltnis der tats�chlichen Entropie und dieser maximal m�glichen Entropie:
\begin{equation}
 H_r = \frac{H}{H_{max}} = \frac{H}{\log_2 K}
\end{equation}
Achtung: der Begriff 'relative Entropie' wird auch f�r die Kullback-Leibler Divergenz benutzt, diese ist aber etwas anderes.

\paragraph{Redundanz einer H�ufigkeitsverteilung:}
\begin{equation}
 R = \frac{H_{max}-H}{H_{max}} = 1 - H_r = 1 - \frac{H}{\log_2 K}
\end{equation}



\paragraph{Kodierung} Vorgang, bei dem Elemente eines Quellalphabets eindeutig auf Elemente eines Kanalalphabets abgebildet werden.
 
 \chapter{Sonstiges}

\paragraph{Diskrete Faltung:}
\begin{equation}
  y[n] = x[n] * h[n] = \sum_{k=-\infty}^{\infty} x[k] \; h[n-k] 
\end{equation}

\paragraph{Volterra-System $N$-ter Ordnung:}
\begin{equation}
 \begin{aligned}
  y[n] = & \sum_{i=1}^N y_i[n] \\
       = & \sum_{\nu_1=0}^{\infty} h_1[\nu_1] \; x[n-\nu_1] \; +    \qquad & \text{(linearer Anteil)} \\
         & \sum_{\nu_1=0}^{\infty} \sum_{\nu_2=0}^{\infty}  
           h_2[\nu_1, \nu_2] \; x[n-\nu_1] \; x[n-\nu_2] \; +          \qquad & \text{(quadratischer Anteil)} \\  
         & \sum_{\nu_1=0}^{\infty} \sum_{\nu_2=0}^{\infty} \sum_{\nu_3=0}^{\infty} 
           h_3[\nu_1, \nu_2, \nu_3] \; x[n-\nu_1] \; x[n-\nu_2] \; x[n-\nu_3] \; +          
           \qquad & \text{(kubischer Anteil)} \\   
         & \qquad \vdots \\
         & \sum_{\nu_1=0}^{\infty} \cdots \sum_{\nu_N=0}^{\infty}  
           h_N[\nu_1, \ldots, \nu_N] \; x[n-\nu_1] \cdots x[n-\nu_N] \; \qquad & \text{(Anteil $N$-ter Ordnung)} 
 \end{aligned}
\end{equation}
wobei $h_i$ der Kernel $i$-ter Orndung ist.  
\end{document}
