\chapter{Fourier-Transformation}

\paragraph{Diskrete Fourier-Transformation (DFT):}
\begin{equation}
 X[k] = \sum_{n=0}^{N-1} x[n] \; e^{-j 2 \pi n k / N} \qquad k=0,1, \ldots , N-1
\end{equation}

\paragraph{Inverse Diskrete Fourier-Transformation (IDFT):}
\begin{equation}
 x[n] = \frac{1}{N} \sum_{k=0}^{N-1} X[k] e^{j 2 \pi n k / N} \qquad n=0,1, \ldots , N-1
\end{equation}

\paragraph{Zeitdiskrete Fourier-Transformation (DTFT):}
\begin{equation}
 X(e^{j \Omega}) = \sum_{n=-\infty}^{\infty} x[n] \; e^{-j \Omega n}
\end{equation}

\paragraph{Inverse Zeitdiskrete Fourier-Transformation (IDTFT):}
\begin{equation}
 x[n] = \frac{1}{2 \pi} \int_{-\pi}^{\pi} X(e^{j \Omega}) 
\end{equation}
stimmt das?

\paragraph{Umrechnung Frequenz/Bin-Index:} sei $k$ der Bin-Index, f die Frequenz in $Hz$, $f_s$ die Sample-Rate,  $\Omega = 2 \pi f / f_s$ die zugeh�rige normierte Kreisfrequenz und $N$ die DFT-Blockl�nge:
\begin{equation}
 \begin{aligned}
  f       &= \frac{k f_s}{N}         \\
  \Omega  &= \frac{2 \pi k}{N}       \\
 \end{aligned}
\end{equation}


\paragraph{Fourier-Reihe:}

\paragraph{Fourier-Transformation:}

\paragraph{inverse Fourier-Transformation:}