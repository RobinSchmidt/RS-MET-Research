\chapter{Informations- und Codierungstheorie}

\section{Definitionen}

\paragraph{Einzelwahrscheinlichkeit:} sei $X$ eine diskrete Zufallsvariable mit m�glichen Realisierungen $x_k, \; k=1, \ldots,  K$. Die Wahrscheinlichkeit, dass $X$ den konkreten Wert $x_k$ annimmt ist:
\begin{equation}
 \label{eqn:Auftretenswahrscheinlichkeit}
 p_k = P(X = x_k)
\end{equation}
mit den Bedingungen
\begin{equation}
 0 \leq p_k \leq 1 \; \forall k \qquad \text{und} \qquad \sum_{k=1}^K p_k = 1
\end{equation}

\paragraph{Informationsgehalt} eines Ereignisses $x_k$:
\begin{equation}
 I(x_k) = \log \left( \frac{1}{p_k}  \right) = -\log(p_k)
\end{equation}
wenn der Logarithmus zur Basis 2 gew�hlt wird, ist die Einheit der Information $bits$, beim nat�rlichen Logarithmus $nats$. 1 $bit$ ist der maximale Informationsgehalt einer Ja/Nein Antwort.

\paragraph{Entropie} der Zufallsvariable $X$:
\begin{equation}
 H(X) = E \{ I(x_k) \} = \sum_{k=1}^K p_k I(x_k) = - \sum_{k=1}^K p_k \log p_k
\end{equation}
die Entropie ist also der mittlere Informationsgehalt.

\paragraph{maximale und relative Entropie:}
wenn alle $K$ Elementarereignisse gleichwahrscheinlich sind (d.h. $p_k = 1/K \; \forall k$), dann ist die Entropie maximal und nimmt den Wert: $H_{max} = \log_2 K$ an. Die relative Entropie (einer H�ufigkeitsverteilung) ist das Verh�ltnis der tats�chlichen Entropie und dieser maximal m�glichen Entropie:
\begin{equation}
 H_r = \frac{H}{H_{max}} = \frac{H}{\log_2 K}
\end{equation}
Achtung: der Begriff 'relative Entropie' wird auch f�r die Kullback-Leibler Divergenz benutzt, diese ist aber etwas anderes.

\paragraph{Redundanz einer H�ufigkeitsverteilung:}
\begin{equation}
 R = \frac{H_{max}-H}{H_{max}} = 1 - H_r = 1 - \frac{H}{\log_2 K}
\end{equation}



\paragraph{Kodierung} Vorgang, bei dem Elemente eines Quellalphabets eindeutig auf Elemente eines Kanalalphabets abgebildet werden.
