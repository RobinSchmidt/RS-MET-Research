\section{Moog Filter}
A first order digital filter has the z-plane transfer function:
\begin{equation}
 G_1(z) = \frac{b_0 + b_1 z^{-1}}{1 + a_1 z^{-1}}
\end{equation}
When we model the Moog filter (excluding resonance), with a series connection of 4 such filters - this gives the transfer function:
\begin{equation}
 G_4(z) = \big( G_1(z) \big)^4 = \left( \frac{b_0 + b_1 z^{-1}}{1 + a_1 z^{-1}} \right)^4
\end{equation}
The resonance is introduced by feeding the filters (sign inverted) output back to the input. In the digital domain, we must include a unit delay into the feedback path corresponding to a factor of $z^{-1}$. With feedback gain $k$, we obtain the net transfer function:
\begin{equation}
 H(z) = \frac{G_4(z)} {1 + k \cdot z^{-1} \cdot G_4(z)}
\end{equation}
With such a feedback, we obtain a resonant peak at that frequency where the phase response of $z^{-1} G_4(z)$ goes through $-180$ degrees (or $-\pi$). The analog Moog filter runs into self oscillation with a feedback factor of $k=4$ because the magnitude response of the filter (without feedback) at the resonant frequency is $1/4$ - such that the net feedback gain becomes unity with $k=4$. In the analog domain, the resonant frequency (the -180 degree frequency) happens to be the same frequency as the cutoff-frequency (measured at the -3 dB point) of each one of the 4 first order filter stages. In the digital domain this is not true anymore because the unit delay in the feedback path adds to phase delay such that the resonant frequency detoriates from the individual 1st order cutoff frequencies. This is particularly true for high cutoff frequencies. Because the resonant frequency is now different from the cutoff frequency, it is also not true anymore that the $G_4$-filters gain (without feedback applied) is equal to $1/4$ at the resonant frequency. This has the effect that self-oscillation does not occur at $k=4$ anymore, but we must choose a higher $k$. To carry the filter over into the digital to domain, is seems to be desirable to preserve the following 3 properties:
\begin{enumerate}
	\item the filter should be lowpass in nature: $G_4(z=-1) = 0$
	\item the filters (normalized radian) cutoff frequency $\omega_c$ is defined as the frequency at which the resonant peak occurs: $\left. \angle \big( z^{-1} G_4(z) \big) \right|_{z=e^{j \omega_c}} = -\pi$
	\item the filter goes into self oscillation with $k=4$, so we want the magnitude response at the resonant frequency to be equal to 1/4: $\big. |G_4(z)| \big|_{z=e^{j \omega_c}}= \frac{1}{4}$
\end{enumerate}
Under the assumption that all 4 first order stages are equal, the first requirement implies that $G_1(-1)=0$, which is equivalent to $b_0 + b_1 \cdot (-1)^{-1} = b_0 - b_1 = 0$, and therefore we must have: $b_0 = b_1$. We will denote this value simply as $b$, so our transfer function $G_1(z)$ can be simplified to
\begin{equation}
\boxed
{
 G_1(z) = \frac{b + b z^{-1}}{1 + a_1 z^{-1}} = \frac{b (1 + z^{-1})}{1 + a_1 z^{-1}}
}
\end{equation}
We have now two degrees of freedom left, namely $b$ and $a_1$. Requirements 2 and 3 can be expressed a single requirement by means of complex numbers:
\begin{equation}
 \left. z^{-1} \cdot G_4(z) \right|_{z=e^{j \omega_c}} = \frac{1}{4} e^{-j \pi}
\end{equation}
In words: the delayed output of the filter $G_4(z)$ has magnitude $1/4$ and phase $-\pi$ at $z=e^{j \omega_c}$. With $e^{-j \pi} = -1$, the definition of $G_1(z)$ and $z=e^{j \omega_c}$ this becomes:
\begin{equation}
 e^{-j \omega_c} \cdot \left( \frac{b (1 + e^{-j \omega_c})}{1 + a_1 e^{-j \omega_c}} \right)^4  = - \frac{1}{4}
\end{equation}
This equation must now be solved for $b$ and $a_1$....

Requirement 3: $|G_4(e^{j \omega_c})|= \frac{1}{4}$ means that the magnitude of one stage at $\omega_c$ must be: $|G_1(e^{j \omega_c})| = \frac{1}{\sqrt{2}}$. The general expression for the magnitude response of a first order digital filter is:
\begin{equation}
 |H(e^{j \omega})| = \sqrt{\frac{b_0^2 + b_1^2 + 2 b_0 b_1 \cos (\omega)}{a_0^2 + a_1^2 + 2 a_0 a_1 \cos (\omega)}}
\end{equation}
with $a_0=1$ and $b_0=b_1=b$ this simplifies to (evaluated at $\omega=\omega_c$):
\begin{equation}
\boxed
{
 |G_1(e^{j \omega_c})| = \frac{1}{\sqrt{2}} 
 = \sqrt{\frac{2 b^2 + 2 b^2 \cos (\omega_c)}{1 + a_1^2 + 2 a_1 \cos (\omega_c)}}
 }
 \label{equation1}
\end{equation}
solving this for $b$ gives:
\begin{equation}
 b = \pm \frac{1}{2} \sqrt{\frac{1 + a_1^2 + 2a_1 \cos(\omega_c)}{1 + \cos(\omega_c)} } 
\end{equation}
likewise, solving for $a_1$ gives:
\begin{equation}
 a_1 = -\cos(\omega_c) \pm \sqrt{\cos^2(\omega_c) + 4 b^2 \cos(\omega_c) + 4 b^2 -1} 
\end{equation}
We can now substitute one of these expressions into requirement 2 (the requirement on the phase) and solve for the other variable. The general expression for the phase response of a first order filter is:
\begin{equation}
 \angle H(e^{j \omega}) = \arctan \left(- \frac{(a_0 b_1 - a_1 b_0) \sin (\omega)}
                                               {a_0 b_0 + a_1 b_1 + (a_0 b_1 + a_1 b_0) \cos (\omega)} \right)
\end{equation}
again, with $a_0=1$ and $b_0=b_1=b$ this simplifies to (evaluated at $\omega=\omega_c$):
\begin{equation}
 \angle G_1(e^{j \omega_c}) = \arctan \left(- \frac{(b - a_1 b) \sin (\omega_c)}
                                                   {b + a_1 b + (b + a_1 b) \cos (\omega_c)} \right)
\end{equation}
because phases add in a series connection, the phase response of $G_4$ is four times this value: $\angle G_4(e^{j \omega_c}) = 4 \cdot \angle G_1(e^{j \omega_c})$. Additionally, we must add the phase response from linear phase term $e^{-j \omega}$ from the unit delay (which is $-\omega_c$). The requirement $\left. \angle \big( z^{-1} G_4(z) \big) \right|_{z=e^{j \omega_c}} = -\pi$ becomes: 
\begin{equation}
4 \cdot \angle G_1(e^{j \omega_c}) - \omega_c = -\pi
\end{equation}
and so we must solve:
\begin{equation}
 \arctan \left(- \frac{(b - a_1 b) \sin (\omega_c)} {b + a_1 b + (b + a_1 b) \cos (\omega_c)} \right) 
 = \frac{-\pi + \omega_c}{4}
\end{equation}

\begin{equation}
\boxed
{
 \frac{(b - a_1 b) \sin (\omega_c)} {b + a_1 b + (b + a_1 b) \cos (\omega_c)}
 = - \tan \left( \frac{\omega_c -\pi}{4} \right)
}
 \label{equation2}
\end{equation}
Equation \ref{equation1} and \ref{equation2} represent a system of 2 equations in 2 unknowns which must be simultaneously solved for $a_1$ and $b$. Solving \ref{equation2} for $a_1$ yields:
\begin{equation}
 a_1 = - \frac{(\cos(\omega_c)+1) \sin(\alpha) + \sin(\omega_c) \cos(\alpha)}
              {(\cos(\omega_c)+1) \sin(\alpha) - \sin(\omega_c) \cos(\alpha)}
\end{equation}
with $\alpha = \frac{\omega_c - \pi}{4}$. This solution - surprisingly - does not depend on $b$ anymore so we can directly use it to obtain $b$ with the other of the two equations solved for $b$.






