\section{Prototype Poles and Zeros}

Notation for the design parameters: 
\begin{equation}
\begin{aligned}
 G:        \qquad & \text{peak/cut gain}                         \\
 G_0:      \qquad & \text{reference gain}                        \\
 G_B:      \qquad & \text{bandwidth gain}                        \\ 
 f_0:      \qquad & \text{peak/cut frequency}                    \\ 
 f_s:      \qquad & \text{sample-rate}                           \\
 \Delta f: \qquad & \text{bandwidth measured at the level $G_B$} \\  
\end{aligned}
\end{equation}
The ideal continuous time low-shelving filters magnitude-response be defined as:
\begin{equation}
 |H_a(\Omega)|^2 = \frac{G^2 + G_0^2 \varepsilon^2 F_N^2(w)}{1 + \varepsilon^2 F_N^2(w)}
\end{equation}
where $w=\Omega / \Omega_B$ is the normalized continuous time frequency variable $\varepsilon$ is a constant defined by:
\begin{equation}
 \varepsilon = \sqrt{ \frac{G^2-G_B^2}{G_B^2-G_0^2} }
\end{equation}
and the function $F_N(w)$ is given by:
\begin{equation}
 F_N(w) = 
 \begin{cases}
  w^N                               \qquad & \text{Butterworth}          \\
  C_N(w)                            \qquad & \text{Chebychev type 1}     \\
  1/C_N(w^{-1})                     \qquad & \text{Chebychev type 2}     \\ 
  cd(N u K_1, k_1), w = cd(uK, k)   \qquad & \text{Chebychev type 2}     \\  
 \end{cases}
\end{equation}
for $G_0=0$ and $G=1$, this reduces to the ordinary lowpass case, but this generalization allows the response to be raised on a pedestal in order to design shelving and peaking equalizers as well. The continuous time transfer function $H_a(s)$ is found by finding the left-hand $s$-plane poles and zeros. First, we distribute the overall Gain over all $N$ first order sections:
\begin{equation}
 g = G^{1/N} , \qquad g_0 = G_0^{1/N}
\end{equation}
and define the following variables:
\begin{equation}
 \phi_i = \frac{(2i - 1) \pi}{2N}, \qquad s_i = \sin(\phi_i), \qquad c_i = \cos(\phi_i)
 \qquad i = 1, \ldots, L
\end{equation}
Now, the poles and zeros come out as:

\paragraph{Butterworth poles and zeros}:
Let
\begin{equation}
 \beta = \varepsilon^{-1/N} \Omega_B =  \varepsilon^{-1/N} \tan \left( \frac{\Delta \omega}{2} \right)
\end{equation}
the $L$ complex conjugate left-hand poles and zeros $p_i, z_i \; i = 1,\ldots,L$ are given by:
\begin{equation}
 p_i = \beta (-s_i + j c_i) \qquad z_i = \frac{g \beta}{g_0} (-s_i + j c_i) 
\end{equation}
and the additional real poles and zeros for odd order filters are:
\begin{equation}
 p_0 = - \beta  \qquad z_0 = -\frac{g \beta}{g_0} \qquad
\end{equation}

\paragraph{Chebychev type 1 poles and zeros}:
Let
\begin{equation}
 u = \ln(g_0^{-1} \beta) \qquad v = \ln( (\varepsilon^{-1} + \sqrt{1 + \varepsilon^{-2}})^{1/N} ) \qquad
 \beta = (G \varepsilon^{-1} + G_B \sqrt{1 + \varepsilon^{-2}})^{1/N}
\end{equation}
Then:
\begin{equation}
 p_i = \Omega_B (-s_i \sinh(v) + j c_i \cosh(v)) \qquad z_i = \Omega_B (-s_i \sinh(u) + j c_i \cosh(u))
\end{equation}
and for $N$ odd:
\begin{equation}
 p_0 = -\Omega_B \sinh(v) \qquad z_0 = -\Omega_B \sinh(u)
\end{equation}

\paragraph{Chebychev type 2 poles and zeros}:
Let
\begin{equation}
 u = \ln(g^{-1} \beta) \qquad v = \ln( (\varepsilon + \sqrt{1 + \varepsilon^2 })^{1/N} ) \qquad
 \beta = (G \varepsilon + G_B \sqrt{1 + \varepsilon^2 })^{1/N}
\end{equation}
Then:
\begin{equation}
 p_i = \Omega_B (-s_i \sinh(v) + j c_i \cosh(v))^{-1} \qquad z_i = \Omega_B (-s_i \sinh(u) + j c_i \cosh(u))^{-1}
\end{equation}
and for $N$ odd:
\begin{equation}
 p_0 = -\Omega_B (\sinh(v))^{-1} \qquad z_0 = -\Omega_B (\sinh(u))^{-1}
\end{equation}

\paragraph{Elliptic poles and zeros}:
Assume $G \neq 0$ and $G_0 \neq 0$, let:
\begin{equation}
 u_i = \frac{(2i - 1) }{N}\qquad \zeta_i = \cd(u_i K, k) \qquad i = 1, \ldots, L 
\end{equation}
and $u_0, v_0$ are given by the solutions of the equations:
\begin{equation}
 \sn(j u_0 N K_1, k_1) = j \frac{G}{G_0 \varepsilon} \qquad 
 \sn(j v_0 N K_1, k_1) = j \frac{1}{\varepsilon}
\end{equation}
Then:
\begin{equation}
 p_i = j \Omega_B \cd((u_i - j v_0)K, k) \qquad z_i = \Omega_B \cd((u_i - j u_0)K, k)
\end{equation}
and for $N$ odd:
\begin{equation}
 p_0 = j \Omega_B \sn(j v_0 K, k) \qquad z_0 = j \Omega_B \sn(j u_0 K, k)
\end{equation}

Special cases for $G = 0$ or $G_0 = 0$: let
\begin{equation}
 \zeta_i = \cd(u_i K, k)
\end{equation}
Then
\begin{equation}
 z_i =
 \begin{cases}
 j \Omega_B (k \zeta_i)^{-1} \qquad & G_0 = 0,    G \neq 0 \; \text{(lowpass design)} \\
 j \Omega_B \zeta_i          \qquad & G_0 \neq 0, G = 0 \\
 \end{cases}
\end{equation}


Assume $G = 0$ and $G_0 \neq 0$:
\begin{equation}
 \zeta_i = \cd(u_i K, k) \qquad z_i = j \Omega_B (k \zeta_i)^{-1}
\end{equation}


Assume $G \neq 0$ and $G_0 = 0$ - this is the conventional elliptic textbook design















