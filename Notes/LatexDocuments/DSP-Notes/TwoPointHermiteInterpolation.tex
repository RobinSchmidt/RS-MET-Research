\section{Hermite Interpolation Between 2 Points}

\subsection{Problem Setting}
In general, the term "Hermite interpolation" refers to interpolation by means of a polynomial that passes through a given number of sample points $(x_i, y_i)$ and also satisfies constraints on some number of derivatives $y_i', y_i'', \ldots$ at these sample points. Here, we consider the problem of finding a polynomial that goes through the two points $(x_0 = 0, y_0)$ and $(x_1 = 1, y_1)$. In addition to prescribe the function values $y_0, y_1$, we also prescribe values for some number of derivatives $y_0', y_1'; y_0'', y_1''; etc.$. Our particular choice of the $x$ coordinates has been made to keep the formulas simple. However, if we want to have arbitrary $x$-coordinates for the endpoints, say $x_{min}, x_{max}$, we may simply transform the input value for the polynomial by $\tilde{x} = (x - x_{min}) / (x_{max} - x_{min})$. Our new variable $\tilde{x}$ will then pass through the range $0, \ldots, 1$ when the original $x$ passes through $x_{min}, \ldots, x_{max}$. The number of derivatives that we want to control dictates the order of the polynomial that we have to use. In order to be able to prescribe values for $M$ derivatives, we need a polynomial of order $N = 2 M + 1$.

\subsection{Derivation for the $7th$ Order Case}
To illustrate the procedure to compute the polynomial coefficients, we consider - as example - the case where we control $M = 3$ derivatives. This calls for a $7th$ order polynomial. In the following derivation, the framed equations are those that we actually need for the implementation. Our interpolating polynomial and its first 3 derivatives have the general form:
\begin{equation}
\begin{aligned}
 y    &=     a_7 x^7 +     a_6 x^6 +    a_5 x^5 +    a_4 x^4 +   a_3 x^3 +   a_2 x^2 + a_1 x + a_0 \\
 y'   &=   7 a_7 x^6 +   6 a_6 x^5 +  5 a_5 x^4 +  4 a_4 x^3 + 3 a_3 x^2 + 2 a_1 x   + a_1         \\
 y''  &=  42 a_7 x^5 +  30 a_6 x^4 + 20 a_5 x^3 + 12 a_4 x^2 + 6 a_3 x   + 2 a_1                   \\ 
 y''' &= 210 a_7 x^4 + 120 a_6 x^3 + 60 a_5 x^2 + 24 a_4 x   + 6 a_3                               \\  
\end{aligned} 
\end{equation}
To satisfy our constraints at the left endpoint $x_0 = 0$, we put in $x = 0$ on the right hand sides and $y_0, y_0', y_0'', y_0'''$ on the left hand sides, and we immediately obtain $a_0, a_1, a_2, a_3$:
\begin{equation}
\boxed
{
 y_0 = a_0, \quad y_0' = a_1, \quad y_0'' = 2 a_2, \quad y_0''' = 6 a_3
}
\end{equation}
...for the actual implementation, you need to solve them for the $a-$coefficients (this is left for the reader as exercise ;-). To satisfy our constraints at the right endpoint $x_1 = 1$, we put in $x = 1$ on the right hand sides and $y_1, y_1', y_1'', y_1'''$ on the left hand sides - we obtain 4 equations for the remaining 4 unknowns  $a_4, a_5, a_6, a_7$:
\begin{equation}
\begin{aligned}
 y_1    &=     a_7 +     a_6 +    a_5 +    a_4 +   a_3 +   a_2 + a_1 + a_0 \\
 y_1'   &=   7 a_7 +   6 a_6 +  5 a_5 +  4 a_4 + 3 a_3 + 2 a_2 + a_1       \\
 y_1''  &=  42 a_7 +  30 a_6 + 20 a_5 + 12 a_4 + 6 a_3 + 2 a_2             \\ 
 y_1''' &= 210 a_7 + 120 a_6 + 60 a_5 + 24 a_4 + 6 a_3                     \\  
\end{aligned} 
\end{equation}
bringing the already known $a_0, a_1, a_2, a_3$ to the left side:
\begin{equation}
\begin{aligned}
 y_1    -   a_3 -   a_2 - a_1 - a_0 &=     a_7 +     a_6 +    a_5 +    a_4  \\
 y_1'   - 3 a_3 - 2 a_2 - a_1       &=   7 a_7 +   6 a_6 +  5 a_5 +  4 a_4  \\
 y_1''  - 6 a_3 - 2 a_2             &=  42 a_7 +  30 a_6 + 20 a_5 + 12 a_4  \\ 
 y_1''' - 6 a_3                     &= 210 a_7 + 120 a_6 + 60 a_5 + 24 a_4  \\  
\end{aligned} 
\end{equation}
for convenience, we define constants $k_0, k_1, k_2, k_3$ for the 4 left hand sides of the equations:
\begin{equation}
\boxed
{
\begin{aligned}
 k_0 &= y_1    -   a_3  - a_2   - a_1 - a_0  \\
 k_1 &= y_1'   - 3 a_3  - y_0'' - a_1        \\
 k_2 &= y_1''  - y_0''' - y_0''              \\ 
 k_3 &= y_1''' - y_0'''                      \\  
\end{aligned} 
}
\end{equation}
where we have also used that $6 a_3 = y_0'''$ and $2 a_2 = y_0''$. Our system of equations now becomes:
\begin{equation}
\begin{aligned}
 k_0 &=     a_7 +     a_6 +    a_5 +    a_4  \\
 k_1 &=   7 a_7 +   6 a_6 +  5 a_5 +  4 a_4  \\
 k_2 &=  42 a_7 +  30 a_6 + 20 a_5 + 12 a_4  \\ 
 k_3 &= 210 a_7 + 120 a_6 + 60 a_5 + 24 a_4  \\  
\end{aligned} 
\end{equation}
finally, solving this system for the remaining 4 unknowns $a_4, a_5, a_6, a_7$ gives:
\begin{equation}
\boxed
{
\begin{aligned}
 a_4 &=   \frac{-k_3 + 15 k_2-90 k_1 +210 k_0}{6} \\
 a_5 &= - \frac{-k_3 + 14 k_2-78 k_1 +168 k_0}{2} \\
 a_6 &=   \frac{-k_3 + 13 k_2-68 k_1 +140 k_0}{2} \\ 
 a_7 &= - \frac{-k_3 + 12 k_2-60 k_1 +120 k_0}{6} \\  
\end{aligned} 
}
\end{equation}

\subsection{Results for Some Other Cases}
Having seen the derivation for the $7th$ order case, it shall suffice for other cases to just give the results. Here we go:

\subsubsection{1st order case}
\begin{equation}
 a_0 = y_0, \quad a_1 = y_1 - y_0
\end{equation}

\subsubsection{3rd Order Case}
\begin{equation}
 a_0 = y_0, \quad a_1 = y_0'
\end{equation}
\begin{equation}
 k_0 = y_1 - a_1 - a_0, \quad k_1 = y_1' - a_1
\end{equation}
\begin{equation}
 a_2 = 3 k_0 - k_1, \quad a_3 = k_1 - 2 k_0
\end{equation}

\subsubsection{5th Order Case}
\begin{equation}
 a_0 = y_0, \quad a_1 = y_0', \quad a_2 = \frac{y_0''}{2}
\end{equation}
\begin{equation}
 k_0 = y_1 - a_2 - a_1 - a_0, \quad k_1 = y_1' - y_0'' - a_1, \quad k_2 = y_1'' - y_0''
\end{equation}
\begin{equation}
 a_3 = \frac{k_2 - 8 k_1 + 20 k_0}{2}, \quad
 a_4 = -k_2 + 7 k_1 - 15 k_0,          \quad
 a_5 = \frac{k_2 - 6 k_1 + 12 k_0}{2}
\end{equation}

\subsection{The General Case}
For the general case, where we control $M$ derivatives by using a polynomial of order $N = 2M+1$, a general pattern emerges. The polynomial coefficients $a_n$ for powers up to $M$ can be computed straightforwardly via:
\begin{equation}
 a_n = \frac{y_0^{(n)}}{n!}, \quad n = 0, \ldots, M
\end{equation}
where $y^{(n)}$ denotes the $n-th$ derivative of $y$, the $0-th$ derivative is the function itself. Now, we establish a vector $\mathbf{k} = (k_0, \ldots, k_M)$ of $M+1$ $k$-values, whose element $k_n$ is given by:
\begin{equation}
 k_n = y_1^{(n)} - \sum_{i=n}^M  \alpha_{n,i} a_i \quad n = 0, \ldots, M
\end{equation}
where
\begin{equation}
 \alpha_{n,i} = \prod_{m=i-n+1}^i m
\end{equation}
Note that for this product to work in general, we must make use the definition of the empty product: $\prod_{i=n}^N a_i = 1, \quad \text{for } N < n$, i.e. when the end-index is lower than the start-index. We also establish a $(M+1)\times(M+1)$ matrix $\mathbf{A}$, whose element $A_{i,j}$ is given by:
\begin{equation}
 A_{i,j} = \prod_{m=M+j-i+2}^{M+j} m
\end{equation}
Now, we collect our remaining unknowns $a_{M+1}, \ldots, a_N$ into the vector $\mathbf{a}$, such that: $\mathbf{a} = (a_{M+1}, \ldots, a_M)$. The system of equations for the remaining unknowns may now be expressed as the matrix equation:
\begin{equation}
 \mathbf{k} = \mathbf{A} \mathbf{a}
\end{equation}
Numerically solving this equation for $\mathbf{a}$ (for example by Gaussian elimination) yields the remaining polynomial coefficients $a_{M+1}, \ldots, a_N$. [Question to self: can a simpler solution be derived that avoids the need for the general linear system solver?]







