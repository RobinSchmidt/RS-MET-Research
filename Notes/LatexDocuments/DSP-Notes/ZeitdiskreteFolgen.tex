\chapter{Zeitdiskrete Folgen}

\section{Definitionen}

\paragraph{Abtastung:} sei $x_c(t)$ eine zeitkontinuierliche Funktion und $T_s$ die Abtastperiode, dann ist die zu $x_c(t)$ geh�rende zeitdiskrete Folge:
\begin{equation}
 x[n] = x_c(n \cdot T_s)
\end{equation}

\paragraph{normierte Kreisfrequenz:} seien $f$ und $f_s$ eine Signalfrequenz und die Abtastrate (in $Hz$), dann ist die zu $f$ geh�rige normierte Kreisfrequenz:
\begin{equation}
 \Omega = \frac{2 \pi f}{f_s}
\end{equation}


\paragraph{Einheitsimpuls:}
\begin{equation}
 \delta [n] = 
  \begin{cases}
  0 & \text{f�r }  n \neq 0 \\ 
  1 & \text{f�r }  n = 0 
 \end{cases} 
\end{equation}

\paragraph{Einheitssprung:}
\begin{equation}
 u [n] = 
  \begin{cases}
  0 & \text{f�r }  n < 0    \\
  1 & \text{f�r }  n \geq 0
 \end{cases} 
\end{equation}

es gilt:
\begin{equation}
 \begin{aligned}
  \delta [n] &= u[n] - u[n-1] \\
  u [n]      &= \sum_{m=-\infty}^n \delta[m]
 \end{aligned}
\end{equation}

\paragraph{komplexe Exponentialfolge:} sei $z = r e^{j \Omega}$ eine komplexe Zahl
\begin{equation}
 x[n] = A \cdot z^n = A \cdot r^n e^{j \Omega n}
\end{equation}


