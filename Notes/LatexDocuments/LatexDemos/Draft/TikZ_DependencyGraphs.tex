\pgfplotsset{compat=1.12}


% Ambiguity With Partial ∂ Notation, and How to Resolve It
% https://www.youtube.com/watch?v=mICbKwwHziI



\begin{equation}
z = z(x,y), \quad
x = x(t),   \;
y = y(t)
\end{equation}

\vspace{-20pt} % Reduces vertical distance between equations
% But one could also use "gather", see here:
% https://www.quora.com/How-can-the-vertical-space-between-two-equations-in-LaTeX-be-decreased
% I think, using these: 
%\abovedisplayshortskip=0pt
%\belowdisplayshortskip=0pt
%\abovedisplayskip=0pt
%\belowdisplayskip=0pt
% would set up the spacing globally.

\begin{equation}
\frac{dz}{dt} =
\frac{\partial z}{\partial x} \frac{dx}{dt} +
\frac{\partial z}{\partial y} \frac{dy}{dt} 
\end{equation}


\begin{tikzpicture}

% Nodes:
\begin{scope}[every node/.style={rectangle, rounded corners, thick, draw, fill=yellow}]
  \node (z) at (0,1)  {$z$};
  \node (x) at (2,2)  {$x$};
  \node (y) at (2,0)  {$y$};
  \node (t) at (4,1)  {$t$};
\end{scope}

% Edges:
\begin{scope}[>={Stealth[black]}, every edge/.style={draw=black, thick}]
  \path [->] (z) edge node {} (x);
  \path [->] (z) edge node {} (y);
  \path [->] (x) edge node {} (t);
  \path [->] (y) edge node {} (t);
\end{scope}

% Equations:
\begin{scope}[every node/.style={rectangle}]
  \node (z_eq)  at (-1.5,   1.0) {$z = z(x,y)$ };
  \node (x_eq)  at ( 2.25,  2.5) {$x = x(t)$   };
  \node (y_eq)  at ( 2.25, -0.5) {$y = y(t)$   };
  \node (dz_dt) at ( 6.0,   1.0) {$\frac{dz}{dt} =
                                   \frac{\partial z}{\partial x} \frac{dx}{dt} +
                                   \frac{\partial z}{\partial y} \frac{dy}{dt}$ };
\end{scope}
% This section may be commented out if it turns out to be undesirable to render the formulas 
% redundantly next to the nodes.

\end{tikzpicture}


%---------------------------------------------------------------------------------------------------
\medskip \medskip \medskip 

\begin{equation}
f = f(x,y), \quad 
x = x(s,t),  \;
y = y(s,t)
\end{equation}

\vspace{-20pt}

\begin{equation}
\frac{\partial f}{\partial s} =
\frac{\partial f}{\partial x} \frac{dx}{ds} +
\frac{\partial f}{\partial y} \frac{dy}{ds} ,  \qquad
\frac{\partial f}{\partial t} =
\frac{\partial f}{\partial x} \frac{dx}{dt} +
\frac{\partial f}{\partial y} \frac{dy}{dt} ,  \qquad
\end{equation}


\begin{tikzpicture}

\begin{scope}[every node/.style={rectangle, rounded corners, thick, draw, fill=yellow}]

  \node (f) at (0,1)  {$f$};
  \node (x) at (2,2)  {$x$};
  \node (y) at (2,0)  {$y$};
  \node (s) at (4,2)  {$s$};
  \node (t) at (4,0)  {$t$};
 
\end{scope}

\begin{scope}[>={Stealth[black]},
              every edge/.style={draw=black, thick}]
              
  \path [->] (f) edge node {} (x);
  \path [->] (f) edge node {} (y);
  \path [->] (x) edge node {} (s);
  \path [->] (y) edge node {} (s);
  \path [->] (x) edge node {} (t);
  \path [->] (y) edge node {} (t);
      
\end{scope}

\end{tikzpicture}

%---------------------------------------------------------------------------------------------------
\medskip \medskip \medskip 




% ToDo:
% -mention also dependency trees and draw some examples of them , too
% -explain terminal (independent) variables
% -explain explicit partial derivative and total partial derivative
% -Maybe use z(x,y) instead of f(x,y). This fits better with the explanation in the video for 
%  using functional notation. Then f,g,h etc. are still free. We use s,t,u,v,w,x,y,z for variables,
%  f,g,h for functions