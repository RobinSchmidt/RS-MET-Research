% Needs:
% \usetikzlibrary{arrows.meta}  % for [>={Stealth[black]} I think

% A graph representation of the first few von Neumann ordinal numbers. A directed edge between two
% Neumann numbers means "is-element-of".

\pgfplotsset{compat=1.12}


\begin{tikzpicture}

\begin{scope}[every node/.style={rectangle, rounded corners, thick, draw, fill=yellow}]
  \node (0) at (0,0)  {0};
  \node (1) at (3,0)  {1};
  \node (2) at (6,0)  {2};
  \node (3) at (9,0)  {3};
  
  \node (w)  at (0,-3)  {$\omega$};
  \node (w1) at (3,-3)  {$\omega + 1$};
  \node (w2) at (6,-3)  {$\omega + 2$};
  \node (w3) at (9,-3)  {$\omega + 3$};
  
\end{scope}

\begin{scope}[>={Stealth[black]},
              every edge/.style={draw=black, thick}]
              
  % Edges between the naturals:              
  \path [->] (1) edge node {} (0);
  \path [->] (2) edge node {} (1);
  \path [->] (3) edge node {} (2);

  \path [->] (2) edge[bend right=20] node {} (0);
  \path [->] (3) edge[bend right=20] node {} (1);
  
  \path [->] (3) edge[bend right=40] node {} (0);
   
  % Edges between the omegas:
  \path [->] (w1) edge node {} (w);
  \path [->] (w2) edge node {} (w1);
  \path [->] (w3) edge node {} (w2);
  
  \path [->] (w2) edge[bend left=20] node {} (w);
  \path [->] (w3) edge[bend left=20] node {} (w1);
   
  \path [->] (w3) edge[bend left=40] node {} (w);
  
  % Edges between the naturals and the omegas: 
  \path [->] (w) edge node {} (0);
  \path [->] (w) edge node {} (1);
  \path [->] (w) edge node {} (2);   
  \path [->] (w) edge node {} (3); 

  \path [->] (w1) edge node {} (0);
  \path [->] (w1) edge node {} (1);
  \path [->] (w1) edge node {} (2);   
  \path [->] (w1) edge node {} (3);  
 
  \path [->] (w2) edge node {} (0);
  \path [->] (w2) edge node {} (1);
  \path [->] (w2) edge node {} (2);   
  \path [->] (w2) edge node {} (3); 
 
  \path [->] (w3) edge node {} (0);
  \path [->] (w3) edge node {} (1);
  \path [->] (w3) edge node {} (2);   
  \path [->] (w3) edge node {} (3);  
 
\end{scope}

\end{tikzpicture}
