\documentclass[12pt]{article}
\topmargin=-1.0cm
\textheight=23cm
\evensidemargin=-1.0cm
\oddsidemargin=-1.0cm
\textwidth=19cm

\usepackage{amsmath}
%\usepackage{amssymb} 

% PGF - Portable Graphics File:
\usepackage{pgfplots}
\usepgfplotslibrary{groupplots}
\usepgfplotslibrary{colormaps, external}
% https://www.overleaf.com/learn/latex/Pgfplots_package#The_document_preamble

% TikZ - TikZ ist kein Zeichenprogramm:
\usepackage{tikz}
%\usetikzlibrary{calc}                 % maybe later
\usetikzlibrary{positioning}
\usetikzlibrary{arrows,intersections}
\usetikzlibrary{arrows.meta}
\usetikzlibrary{patterns}
\usetikzlibrary{decorations.markings,shapes.arrows}


\usepackage{tikz-3dplot}



%\usepgfplotslibrary{external} 
%\usetikzlibrary{external}        % alternative to \usepgfplotslibrary{external} ?
%\tikzexternalize
% Reduces memory requirements and may speed up compilation time by rendering plots into individual
% external files (or something like that). See:
% https://tex.stackexchange.com/questions/7953/how-to-expand-texs-main-memory-size-pgfplots-memory-overload
% ...but this gives errors

% ...using this instead:
%\tikzexternalize[mode=graphics if exists,prefix=./TikzFigures/]
% does not produce any compilation errors but I don't see any pre-generated pdf figures anywhere so
% it's very doubtful that the extrernalization actually works as intended.

%---------------------------------------------------------------------------------------------------
% Colors:

% To define custom colors in plots - must appear after \usepackage{tikz}
\usepackage{color}
\definecolor{rsRed}   {rgb}{0.7,0.0,0.0}
\definecolor{rsYellow}{rgb}{0.5,0.4,0.0}
\definecolor{rsGreen} {rgb}{0.0,0.5,0.0}
\definecolor{rsCyan}  {rgb}{0.0,0.4,0.6}
\definecolor{rsBlue}  {rgb}{0.0,0.0,1.0}
\definecolor{rsPurple}{rgb}{0.5,0.0,0.8}

\definecolor{rsGray20}{rgb}{0.2,0.2,0.2}
\definecolor{rsGray30}{rgb}{0.3,0.3,0.3}
\definecolor{rsGray50}{rgb}{0.5,0.5,0.5}

% rename these colors to plotRed, plotYellow, etc. - maybe

%\definecolor{mygray}{rgb}{0.5,0.5,0.5}
%\definecolor{mymauve}{rgb}{0.58,0,0.82}



%---------------------------------------------------------------------------------------------------
% Colormaps:

% I try to define my own colormaps here. The idea is that the plots refer to the colormap by 
% names that are descriptive for the prupose/type of the map like "sequential", "diverging" or
% "qualitative". Then we can set up the colormap here in the setup file which serves then as
% a sort of style-sheet for teh colormap which can be set up once and for all - i.e. for all 
% plots in a document at once. It would be nice to be able to give alias names for predefined
% colormaps, e.g. somethin tlike "sequential = viridis" but I have not yet figured out if that's
% possible. Plots would then use "colormap name = rsSequential" insstead of 
% "colormap name = viridis". See the plot in TikZ_GaussTimesX_Heat.tex

%\pgfplotsset{/pgfplots/colormap={rsSequential}{rgb255=(255,255,255) rgb255=(0,0,0)}}
%\pgfplotsset{/pgfplots/colormap={rsSequential}{rgb255=(255,0,255) rgb255(0,0,0) rgb255=(255,255,255)}}
%\pgfplotsset
%{
%  /pgfplots/colormap={rsSequential}
%  {
%    color(0cm)=(blue); 
%    color(1cm)=(yellow); 
%    color(2cm)=(orange); 
%    color(3cm)=(red)
%  }
%}
%/pgfplots/colormap={hot}{color(0cm)=(blue); color(1cm)=(yellow); color(2cm)=(orange); color(3cm)=(red)}

% I try to give alias names to existing color maps - but neither of these attempts work:
%\pgfplotsset{/pgfplots/colormap={rsSequential}{colormap/viridis}}
%\pgfplotsset{/pgfplots/colormap={rsSequential}{viridis}}
% We get an "Illformed colormap specification" error.

% Use rsSequential, rsDiverging, rsQualitative


%\pgfplotsset{/pgfplots/colormap={rsBlackToWhite}{rgb255=(0,0,0) rgb255=(255,255,255)}}
%\pgfplotsset{/pgfplots/colormap={rsWhiteToBlack}{rgb255=(255,255,255) rgb255=(0,0,0)}}
%\pgfplotsset
%{
%  /pgfplots/colormap={rsEarth}
%  {
%    rgb255=(0,0,0) rgb255=(0,28,15) rgb255=(42,39,6) rgb255=(28,73,33) rgb255=(67,85,24) 
%    rgb255=(68,112,46) rgb255=(81,129,83) rgb255=(124,137,87) rgb255=(153,147,122) 
%    rgb255=(145,173,164) rgb255=(144,202,180) rgb255=(171,220,177) rgb255=(218,229,168) 
%    rgb255=(255,235,199) rgb255=(255,255,255)
%  }
%}



% See:
% https://tikz.dev/pgfplots/libs-colormaps
% https://tex.stackexchange.com/questions/359526/defining-custom-colormap
% https://tikz.dev/pgfplots/libs-colorbrewer

% https://tex.stackexchange.com/questions/359526/defining-custom-colormap
% it says that the relevant files for the built-in colormaps are:
% pgfplots.code.tex (line 36 ff) and pgfplotscolormap.code.tex (line 2372 ff)
% ...it doesn't say where these files are located, though


%---------------------------------------------------------------------------------------------------
% Formatted source code:

%\usepackage{listings}            
%\lstset{
%  backgroundcolor=\color{white},   
%  basicstyle=\footnotesize\ttfamily,  % the size of the fonts that are used for the code
%  captionpos=none,                    % no captions (and no empty space either)
%  commentstyle=\color{rsGreen},       % comment style
%  frame=single,	                      % adds a frame around the code
%  keywordstyle=\color{rsBlue},          % keyword style
%  stringstyle=\color{rsPurple},       % string literal style
%  columns=flexible,                   %
%  keepspaces=true,                    % keeps spaces in text
%  tabsize=2,
%}

% The idea is to give the latex code for the tikz pictures in the lstlisting environment like so:
%
%   \begin{lstlisting}[language=TeX]
%    ...
%   \end{lstlisting}
%
% But it doesn't really look good. The "begin" keyword seems to be not recognized. So, for now, I
% just use the verbatim environment instead.


