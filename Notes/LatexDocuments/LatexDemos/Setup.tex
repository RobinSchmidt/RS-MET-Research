\documentclass[12pt]{article}
\topmargin=-1.0cm
\textheight=23cm
\evensidemargin=-1.0cm
\oddsidemargin=-1.0cm
\textwidth=19cm

\usepackage{amsmath}
%\usepackage{amssymb} 

% PGF - Portable Graphics File:
\usepackage{pgfplots}
\usepgfplotslibrary{groupplots}
% https://www.overleaf.com/learn/latex/Pgfplots_package#The_document_preamble

% TikZ - TikZ ist kein Zeichenprogramm:
\usepackage{tikz}
%\usetikzlibrary{calc}                 % maybe later
\usetikzlibrary{positioning}
\usetikzlibrary{arrows,intersections}


%\usepgfplotslibrary{external} 
%\tikzexternalize
% Reduces memory requirements and may speed up compilation time by rendering plots into individual
% external files (or something like that). See:
% https://tex.stackexchange.com/questions/7953/how-to-expand-texs-main-memory-size-pgfplots-memory-overload
% ...but this gives errors


% To define custom colors in plots - must appear after \usepackage{tikz}
\usepackage{color}
\definecolor{rsRed}   {rgb}{0.7,0.0,0.0}
\definecolor{rsYellow}{rgb}{0.5,0.4,0.0}
\definecolor{rsGreen} {rgb}{0.0,0.5,0.0}
\definecolor{rsCyan}  {rgb}{0.0,0.4,0.6}
\definecolor{rsBlue}  {rgb}{0.0,0.0,1.0}
\definecolor{rsPurple}{rgb}{0.5,0.0,0.8}
% rename these colors to plotRed, plotYellow, etc. - maybe

%\definecolor{mygray}{rgb}{0.5,0.5,0.5}
%\definecolor{mymauve}{rgb}{0.58,0,0.82}



% For formatted source code:

%\usepackage{listings}            
%\lstset{
%  backgroundcolor=\color{white},   
%  basicstyle=\footnotesize\ttfamily,  % the size of the fonts that are used for the code
%  captionpos=none,                    % no captions (and no empty space either)
%  commentstyle=\color{rsGreen},       % comment style
%  frame=single,	                      % adds a frame around the code
%  keywordstyle=\color{rsBlue},          % keyword style
%  stringstyle=\color{rsPurple},       % string literal style
%  columns=flexible,                   %
%  keepspaces=true,                    % keeps spaces in text
%  tabsize=2,
%}

% The idea is to give the latex code for the tikz pictures in the lstlisting environment like so:
%
%   \begin{lstlisting}[language=TeX]
%    ...
%   \end{lstlisting}
%
% But it doesn't really look good. The "begin" keyword seems to be not recognized. So, for now, I
% just use the verbatim environment instead.


