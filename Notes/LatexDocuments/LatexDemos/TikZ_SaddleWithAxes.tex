\pgfplotsset{compat=1.12}

\begin{tikzpicture}[thick, >=stealth']
  \begin{axis}[xlabel = {$x$}, ylabel = {$y$}, view = {25}{40},
               xmin = -1.5, xmax = 1.5, ymin = -1.5, ymax = +1.5, 
               samples = 21, samples y = 21]
               
    % Origin and coordinate arrows: 
    \fill[black] (0,0,0) circle (5pt);
    \draw[->, ultra thick, rsRed]   (0,0,0) -- (1,0,0);
    \draw[->, ultra thick, rsGreen] (0,0,0) -- (0,1,0);
    \draw[->, ultra thick, rsBlue]  (0,0,0) -- (0,0,1);
    
    % The surface:
    \addplot3[surf, mesh, thick, color = black, draw opacity = 0.25, 
              domain = -1.2:1.2, y domain = -1.2:1.2] 
    {x^2 - y^2};
    
  \end{axis}
\end{tikzpicture}

% It's important to use compat=1.12 ...or maybe higher. With 1.9, it didn't work. I think, with 
% compatibility to earlier versions, the coordinates are interpreted differently, i.e. as 
% "screen-coodinates" rather than "world-coordinates" or something like that. I think, it can be 
% fixed with older versions by using "(axis cs: 0,0,0)" instead of "(0,0,0)". The fancy torus 
% example uses this method. But it's inconvenient.

%
% ToDo:
% -Figure out how we can a different number of samples for the 2nd parameter. Using "y samples = 21" 
%  didn't work. samples={11}{21} also didn't. Aha! It's "samples y = 21"
