\chapter{Abstract Algebra}
In the course of this book, we have encountered a great number of so called algebraic structures such as different number systems, vector spaces, etc.. So far, each of these structures more or less stood by itself. It's time to take a step back and recognize commonalities and differences among them and bring some order into this disorganized chaos. Doing so is the realm of abstract algebra. It abstracts out the commonalities between these disparate algebraic structures and establishes a common terminology and notation to talk about such things. Furthermore, it establishes some theorems that will hold for all structures that satisfy a certain set of axioms. This may potentially economize a lot of math: if you come up with a new algebraic structure by yourself and are able to verify that the axioms of a given known structure hold for your structure, you will immediately have a whole bunch of proven theorems in your hand without having to wade through all the tedious proofs for your concrete case again. Checking, if a smallish set of axioms holds true for your new structure will typically be much easier than trying to re-prove each and every theorem from some other known structure for the new structure. It also proves that certain structures that you may hope to discover (like, say, a 3D division algebra) cannot possibly exist, potentially saving you from some headaches in a futile search. The main concepts of abstract algebra are: group, ring, field and vector space. There's more but that would be my personal list of the most important algebraic structures. A simple algebraic structure consists of a set and one or more operations. These operations take as input a certain number of elements of the set and produce as output another element of the set. Especially important are the binary operations which take two elements as input. The different algebraic structures differ in their requirements for what sort of operations must be present and which rules the operations have to satisfy - things like associativity, commutativity, presence of neutral elements, etc.. In some more complex algebraic structures (e.g. vector spaces), there may be more than one set (vector spaces have two: scalars and vectors). Abstract algebra doesn't really fit into the usual dichotomy between discrete and continuous mathematics. It's more about overarching principles that apply to both of these big branches. What we will learn here is so universal that it will apply virtually everywhere. So let's get started.

% ToDo: point out relation to elementary algebra which is about solving equations for unknowns. 
% Abstract algebra is more concerned with questions about characterizing which equations are and 
% aren't solvable in the first place, I think - and in which structures they are or aren't solvable. 
% For example, x^2 = 2 has solutions in R but not in Q and x^2 = -1 has solutions in C but not in R.
% I think, the cancellation law in group theory is an important example for the connection between
% elementary and abstract algebra because it's an essential tool in solving equations.


\begin{comment}

Just like order theory abstracts from the concrete order relations less-than, greater-than, equivalent, etc., abstract algebra abstracts algebraic operations like addition, multiplications, etc.

Some history: Historically, algebra meant the science of solving (polynomial) equations. The linear
case is easy in the univariate case. For the quadratic case, we all have learned a formula at some
point. For the cubic and quartic case, there are also formulas but they are too complicated to be
tought in school. 

-Maybe mention also "algebra" as an algebraic structure. It's a vector space with a 
 multiplication between vectors.
 
-How does the boolean algebra with "and", "or", "not" fit in? We have two binary and one unary 
 operations, but the negations actually do not produce inverse elements of any sort (right?).
 But we do have two sorts of distributive laws:
  https://en.wikipedia.org/wiki/Boolean_algebra#Laws
 0 is the annihilator of \wedge and 1 the annihilator of \vee. In ordinary algebra, there is only
 an annihilator for multiplication which is the neutral element of addition
 ..maybe make a sectiion on other algebraic structures that don't fit in. There, we can also explain 
 monoids, magmas, semigroups, etc.
 See: 
 https://en.wikipedia.org/wiki/Universal_algebra#Other_examples
 https://en.wikipedia.org/wiki/Lattice_(order)
-Mention also Module, Lattice, Algebra
-Magma: https://en.wikipedia.org/wiki/Magma_(algebra) 
 
 
sections:
-Groups
 -Consists of a set and an operation that maps two members from the set to another member of the set.
 -If the operation is commutative, the group is called Abelian.
-Rings
 -Is a group in which each element has an inverse with respect to the group operation (verify!).
 -Defines a second operation that in itself follows similar rules as the first (associativity, maybe commutativity) and satisfies the distributive law when combined with the group operation.
-Fields
 -A field is a ring in which each element except 0 has a multiplicative inverse.
 -Explain algebraic closure (see Elliptic Tales, pg 38 ff). C is algebraically closed but
  Q and R are not. C is the algebraic closure of both. Algebraic closures also exist for
  finite fields ...how do they look like? They are actually infinite.
-Vector Spaces
 -Consists of 2 sets, scalars and vectors, with operations between them ...tbc...
-Algebras
 -Consists of a vector space and another operation (a "product") between vectors that yields another vector
-Lie Algebras:
-Vector space L with a bilinear L x L -> L, denoted as [x,y] that satisfies:
 [x,x] = 0
 [x,[y,z]] + [z,[x,y]] + [y,[z,x]] = 0 (Jacobi identity)
 -https://www.youtube.com/watch?v=tVElyYTdsPE 
-Representations
 -Each abstract group can be represented by an appropriately chosen set of matrices. The elements map to the matrices, the group operation maps to matrix multiplication.
 -The choice of the set of matrices need not be unique.
 -In certain cases, the structure may be upped to an algebra where a Lie bracket serves as the "product". A Lie bracket can be the commutator, the anticommutator, etc.
-Division algebra



Things to note:
-arity of operations: nullary (constants), unary (1 in, 1 out), binary (2 in, 1 out), ...
-associativity is what allows us to turn binary operations int n-ary operations

https://en.wikipedia.org/wiki/Algebraic_structure
https://en.wikipedia.org/wiki/Lattice_(order)
https://en.wikipedia.org/wiki/Congruence_relation

This chapter should perhaps go neither into the continuous math nor in the discrete math part but rather in some subsequent "overarching" part


https://www.youtube.com/watch?v=9aUsTlBjspE
The Insolvability of the Quintic
...says that this topic belongs to the end of an abstarct algebra course


https://www.youtube.com/watch?v=CxVCxe3-ciY&list=PLOWqYl7tM7vkDAccCyT_TwPGzMkey1rsG
Was ist Galois-Theorie? | Math Intuition
-How does this generation of roots work? Maybe take Q[w] where w = sqrt(2), i.e. the rationals 
 adjoin sqrt(2), then express elements as a + bw and take as group operation:
 (a + bw)(c + dw) = ab + bdw? ...not sure, if that works

Resources:
-Full pdf book on abstract algebra: http://abstract.ups.edu/download.html 
-Video series by "Mathemaniac" on group theory:
 https://www.youtube.com/watch?v=EsBn7G2yhB8&list=PLDcSwjT2BF_VuNbn8HiHZKKy59SgnIAeO
-Video series by "All Angles" on group theory: 
 https://www.youtube.com/playlist?list=PLffJUy1BnWj1vIbqT14uI1bJcoQV3smfo
-A classic example -- how the power set forms a ring. https://www.youtube.com/watch?v=fvMnVKq3UtU
 The operations are: +: symmetric difference, *: intersection

-What about physical units? I think, if we have a set of n base units U = { u_1, u_2,..., u_n }
 then we can view this set as a generating set of the group of all physical units. A physical
 quantity could then be viewed as an element of P = R x U where r in R is the numerical value and
 u in U is the phyiscal unit. We can freely multiply elements of P but we can only add them, when
 their U part matches. So, addition becomes a partial function on P.


Maybe make a section about:
https://en.wikipedia.org/wiki/Commutative_algebra



https://en.wikipedia.org/wiki/Homology_(mathematics)
https://en.wikipedia.org/wiki/Homological_algebra

https://en.wikipedia.org/wiki/-logy  "study of a subject"

-> homology - study of sameness? Studies when two things are really the same?

"homology is a general way of associating a sequence of algebraic objects, such as abelian groups or modules, with other mathematical objects such as topological spaces"

...so homology is the study of such associations?

"Homology was originally a rigorous mathematical method for defining and categorizing holes in a manifold

...so, homology studies when two types of holes are the same or not?


From Magmas to Fields: a trippy excursion through algebra - SoME2 3b1b
https://www.youtube.com/watch?v=y1b1GxZ-52s
Magma:              set with binary operation +
Semigroup:          the + operation is associative
Monoid:             there's a neutral element 0 for +
Group:              there is an inverse element -a for every element a
Abelian Group:      + is commutative
Ring:               there's a 2nd associtaive operation * which distributes over +
Unital Ring:        there's a neutral element 1 for *
Com. Ring:          * is commutative
Field:              there's an inverse element 1/a for every element a except 0
alg. closed Field:  every polynomial has a root in the field

...perhaps it could be refined? is there something in between rbelian group and ring, e.g. a structure where * is not associative or not distributive over +?


https://www.youtube.com/watch?v=U6Wv7SeoJdk
Was ist ein Homöomorphismus? Was bedeutet "homöomorph"?
We have a set ans something else tha structures the set. That "something else" is:
-Functions in algebra (structures: groups, rings, fields, ...)
-Relations in order theory  (structures: semi-order (?), total order)
-Subsets in topology (structures: )
-important are in all cases: maps that preserve the structure

https://en.wikipedia.org/wiki/Order_theory
https://en.wikipedia.org/wiki/Lattice_(order)#As_partially_ordered_set
https://en.wikipedia.org/wiki/Heyting_algebra
https://en.wikipedia.org/wiki/Boolean_algebra_(structure)
https://en.wikipedia.org/wiki/Algebraic_structure#Hybrid_structures

Universal Properties of Number Systems
https://www.youtube.com/watch?v=TdXKfcQ6OuA
Naturals:  initial semiring
Integers:  initial ring
Rationals: initial ordered field
Reals:     initial archimedian ordered field
Complex:   
https://en.wikipedia.org/wiki/Archimedean_property


https://math.stackexchange.com/questions/2477496/what-is-the-cauchy-completion-of-a-metric-space


Make a chapter about other algebras - Boolean, Wheel

About Wheel algebras:
https://www.youtube.com/watch?v=tXCovPlOUM0
https://www.youtube.com/watch?v=EhuyyKKQRWc


History of Mathematics: Classical algebra: 19th-century beginnings of modern algebra. 3rd Yr Lecture
https://www.youtube.com/watch?v=2wzbbwxif_M


Playlists on Bill Kinney's channel:

Abstract Algebra
https://www.youtube.com/playlist?list=PLmU0FIlJY-MmJ1EneBN-S3AYmc4CYo27V

Advanced Abstract Algebra
https://www.youtube.com/playlist?list=PLmU0FIlJY-Mm7BhXAfEchH0UZ4fK7GZv-

Abstract Algebra Problems with Solutions (including Proofs)
https://www.youtube.com/playlist?list=PLmU0FIlJY-MlqikmY6khGUZRueXwsRFQV

I think, this is the book: Contemporary Abstract Algebra (by Joseph A. Gallian) he often uses. It seems to be good. Might be worth to purchase.


Semigroup: (N+, +)
Monoid:    (N, +), (Z, *), (Q, *)
Group:     (Z, +), (Q, +), (Q\{0}, *)


\end{comment}