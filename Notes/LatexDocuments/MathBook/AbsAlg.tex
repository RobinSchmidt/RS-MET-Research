\chapter{Abstract Algebra}
In the course of this book, we have encountered a great number of so called algebraic structures such as different number systems, vector spaces, etc.. So far, each of these structures more or less stood by itself. It's time to take a step back and recognize commonalities and differences among them and bring some order into this disorganized chaos. Doing so is the realm of abstract algebra. It abstracts out the commonalities between these disparate algebraic structures and establishes a common terminology and notation to talk about such things. Furthermore, it establishes some theorems that will hold for all structures that sastisfy a certain set of axioms. This may potentially economize a lot of math: if you come up with a new algebraic structure by yourself and are able to verify that the axioms of a given known structure hold for your structure, you will immediately have a whole bunch of proven theorems in your hand without having to wade through all the tedious proofs for your concrete case again. Checking, if a smallish set of axioms holds true for your new structure will typically be much easier than trying to re-prove each and every theorem from some other known structure for the new structure. It also proves that certain structures that you may hope to discover (like, say, a 3D division algebra) cannot possibly exist, potentially saving you from some headaches in a futile search. The main concepts of abstract algebra are: group, ring, field and vector space. There's more but that would be my personal list of the most important algebraic structures. An algebraic structure consists of a set and one or more operations. These operations take as input a certain number of elements of the set and produce as output another element of the set. Especially important are the binary operations which take two elements as input. The different algebraic structures differ in their requirements for what sort of operations must be present and which rules the operations have to satisfy - things like associativity, commutativity, presence of neutral elements, etc.. Abstract algebra doesn't really fit into the usual dichotomy between discrete and continuous mathematics. It's more about overarching principles that apply to both of these big branches. What we will learn here is so universal that it will apply virtually everywhere. So let's get started.


\begin{comment}

sections:
-Groups
 -Consists of a set and an operation that maps two members from the set to another member of the set.
 -If the operation is commutative, the group is called Abelian.
-Rings
 -Is a group in which each element has an inverse with respect to the group operation (verify!).
 -Defines a second operation that in itself follows similar rules as the first (associativity, maybe commutativity) and satisfies the distributive law when combined with the group operation.
-Fields
 -Is a ring in which each element except 0 has a multiplicative inverse.
-Vector Spaces
 -Consists of 2 sets, scalars and vectors, with operations between them ...tbc...
-Algebras
 -Consists of a vector space and another operation (a "product") between vectors that yields another vector
-Representations
 -Each abstract group can be represented by an appropriately chosen set of matrices. The elements map to the matrices, the group operation maps to matrix multiplication.
 -The choice of the set of matrices need not be unique.
 -In certain cases, the structure may be upped to an algebra where a Lie bracket serves as the "product". A Lie bracket can be the commutator, the anticommutator, etc.
-Division algebra

Things to note:
-arity of operations: nullary (constants), unary (1 in, 1 out), binary (2 in, 1 out), ...
-associativity is what allows us to turn binary operations int n-ary operations

https://en.wikipedia.org/wiki/Algebraic_structure

This chapter should perhaps go neither into the continuous math nor in the discrete math part but rather in some subsequent "overarching" part

Resources:
-Full pdf book on abstract algebra: http://abstract.ups.edu/download.html 
-Video series by "Mathemaniac" on group theory:
 https://www.youtube.com/watch?v=EsBn7G2yhB8&list=PLDcSwjT2BF_VuNbn8HiHZKKy59SgnIAeO
-Video series by "All Angles" on group theory: 
 https://www.youtube.com/playlist?list=PLffJUy1BnWj1vIbqT14uI1bJcoQV3smfo
-A classic example -- how the power set forms a ring. https://www.youtube.com/watch?v=fvMnVKq3UtU
 The operations are: +: symmetric difference, *: intersection

\end{comment}