\section{Fields}
In the realm of abstract algebra, a \emph{field} is a ring with some additional requirements. It must be a unital ring, which means that a multiplicative identity (i.e. neutral element) must exist. We will also require that all elements except the additive identity must have multiplicative inverses. The exception is motivated by the idea that division by zero is impossible. It turns out that this will case for any field if we want $a \cdot 0 = a \cdot (0 + 0) = a \cdot 0 + a \cdot 0$ to be true for any $a$ that is an element of our field. Furthermore, we require that the multiplication is also commutative. The prototypical examples of fields are the rational numbers $\mathbb{Q}$ and the real numbers $\mathbb{R}$. Whenever we have a field, we can do computations with the elements of the field in the same way that we are used to from elementary school. That means, we can add, subtract, multiply and divide to our hearts content with the sole exception of division by zero. As with rings, we denote a field as a triple $(\mathbb{F}, +, \cdot)$ made from an underlying set $\mathbb{F}$ and two operations which we call addition and multiplication. Formally, the requirements for a field are:
\begin{itemize}
\item $(\mathbb{F},+,\cdot)$ forms a unital ring, i.e. a ring with a multiplicative neutral element.
\item Every element except the additive neutral element has a multiplicative inverse element.
\item Multiplication is commutative.
\end{itemize}
Because these properties are stated at a pretty high level, let's state them boiled down to a lower level:

\medskip
\begin{tabular}{l l}
Additive closure: 
& $\forall a,b \in \mathbb{F}: \; a + b \in \mathbb{F}$  \\	
Multiplicative closure: 
& $\forall a,b \in \mathbb{F}: \; a \cdot b \in \mathbb{F}$  \\	
Associativity of addition: 
& $\forall a,b,c \in \mathbb{F}: \;  (a + b) + c = a + (b + c)$   \\
Associativity of multiplication: 
& $\forall a,b,c \in \mathbb{F}: \;  (a \cdot b) \cdot c = a \cdot (b \cdot c)$   \\
Commutativity of addition: 
& $\forall a,b \in \mathbb{F}: \;  a + b = b + a$   \\
Commutativity of multiplication: 
& $\forall a,b \in \mathbb{F}: \;  a \cdot b = b \cdot a$   \\
Existence of additive identity: 
& $\exists 0 \in \mathbb{F}: \; (\forall a \in \mathbb{F}: 0 + a = a)$ \\
Existence of multiplicative identity: 
& $\exists 1 \in \mathbb{F}: \; (\forall a \in \mathbb{F}: 1 \cdot a = a)$ \\
Existence of additive inverses: 
& $\forall a \in \mathbb{F}: \; (\exists (-a) \in \mathbb{F}: (-a) + a = 0 )$ \\
Existence of multiplicative inverses: 
& $\forall a \in \mathbb{F} \setminus \{0\}: \; (\exists a^{-1} \in \mathbb{F}: a^{-1} \cdot a = 1 )$ \\
Multiplication distributes over addition: 
& $\forall a,b,c \in R: \;  
a \cdot (b + c) = a \cdot b + a \cdot c, \; 
(b + c) \cdot a = b \cdot a + c \cdot a$
\end{tabular}
\medskip
[...todo: verify that the list is complete]

Any mathematical structure that satisfies these requirements is a structure in which can do calculations and solve equations in the same way as we are used to from the real numbers.
 
%As usual, $\mathbb{F}^*$ is a shorthand for $\mathbb{F} \setminus \{0\}$.

\begin{comment}

Explain the edge case where additive neutral elements have a multiplicative inverse. I think Michael Penn has a video about that called somethign like "when division by zero isn't ..." IIRC	

 We have used the symbol $1$ to denote the multiplicative neutral element. 

The notation $a^{-1}$ is now used for the multiplicative inverse of $a$, so we need a new notation for the additive inverse. We will use $-a$ to denote the additive inverse of $a$. In the axiom about multiplicative inverses, we have excluded the $0$ element. The additive neutral element (i.e. zero) is the only element for which we do not require to have a multiplicative inverse. It can in fact be shown that - some uninteresting trivial edge cases aside - the additive neutral element cannot possibly have a multiplicative inverse in any ring. 

https://en.wikipedia.org/wiki/Finite_field
https://en.wikipedia.org/wiki/Cyclotomic_field

Warum kann man nicht durch null teilen? Oder: Was sind Zahlen eigentlich?
https://www.youtube.com/watch?v=v05rI0lGk6A

https://en.wikipedia.org/wiki/Quadratic_field


https://www.youtube.com/watch?v=
4BfCmZgOKP8  Was sind Galoiskörper? [Weitz]


https://en.wikipedia.org/wiki/Splitting_field

-In groups, we can always solve equations like a*x = b
-In fields, we can always solve equations like a*x + b = c (in rings only sometimes)
-In algebraically closed fields, we can always solve polynomial equations - but only in principle, 
 i.e. in the sense that solutions exist and we may find them numerically. Up to degree 4, there are
 formulas in terms of the coeffs. ...do these formulas also apply to finite fields?
-Field extensions can be writte like R/Q or just with the subset symbol
-Degree of field extension: if L is a field extesnion over K. We may interpret L as a vector space 
 over K. For example. C is a field extension over R has degree 2. We write this as 
 [C:R] = dim_R(C) = 2, dim_F(F) = 1 always, [Q(sqrt(2)) : Q] dim_Q(Q(sqrt(2))) = 2,
 [Q(cbrt(2)) : Q] = 3  bcs  Q(cbrt(2))  = { a + b sqrt(2) + c cbrt(2^2) :  a,b,c in Q }
 [R : Q] = inf, R = { a + b r_1 + c * r_2 + d * r_3 + ...   where r_in in R }
 [F(a) : F] = deg(m_a(x)) where m_a(x) is the minimal polynomial for a (a is an algebraic number)
 For a chain of field extensions F \subset G \subset H we have:
 [H :  F] = [H : G] * [G : F]



	
\end{comment}