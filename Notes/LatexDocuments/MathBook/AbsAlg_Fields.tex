\section{Field Theory}
In the realm of abstract algebra, a \emph{field} is a ring with some additional requirements. It must be a unital ring, which means that a multiplicative identity (i.e. neutral element) must exist. We will also require that all elements except the additive identity must have multiplicative inverses. The exception is motivated by the idea that division by zero is impossible. It turns out that this will case for any field if we want $a \cdot 0 = a \cdot (0 + 0) = a \cdot 0 + a \cdot 0$ to be true for any $a$ that is an element of our field. Furthermore, we require that the multiplication is also commutative. The prototypical examples of fields are the rational numbers $\mathbb{Q}$ and the real numbers $\mathbb{R}$. Whenever we have a field, we can do computations with the elements of the field in the same way that we are used to from elementary school. That means, we can add, subtract, multiply and divide to our hearts content with the sole exception of division by zero. This also implies that divisibility is not an interesting concept in fields because every element is divisible by every other element except zero. There is no divisibility structure that could be studied, as we did in ring theory - irreducibility, primes, etc. are not a thing anymore. As with rings, we denote a field as a triple $(\mathbb{F}, +, \cdot)$ made from an underlying set $\mathbb{F}$ and two operations which we call addition and multiplication. Formally, the requirements for a field are:
\begin{itemize}
\item $(\mathbb{F},+,\cdot)$ forms a unital ring, i.e. a ring with a multiplicative neutral element.
\item Every element except the additive neutral element has a multiplicative inverse element.
\item Multiplication is commutative.
\end{itemize}
Because these properties are stated at a pretty high level, let's state them boiled down to a lower level:

\medskip
\begin{tabular}{l l}
Additive closure: 
& $\forall a,b \in \mathbb{F}: \; a + b \in \mathbb{F}$  \\	
Multiplicative closure: 
& $\forall a,b \in \mathbb{F}: \; a \cdot b \in \mathbb{F}$  \\	
Associativity of addition: 
& $\forall a,b,c \in \mathbb{F}: \;  (a + b) + c = a + (b + c)$   \\
Associativity of multiplication: 
& $\forall a,b,c \in \mathbb{F}: \;  (a \cdot b) \cdot c = a \cdot (b \cdot c)$   \\
Commutativity of addition: 
& $\forall a,b \in \mathbb{F}: \;  a + b = b + a$   \\
Commutativity of multiplication: 
& $\forall a,b \in \mathbb{F}: \;  a \cdot b = b \cdot a$   \\
Existence of additive identity: 
& $\exists 0 \in \mathbb{F}: \; (\forall a \in \mathbb{F}: 0 + a = a)$ \\
Existence of multiplicative identity: 
& $\exists 1 \in \mathbb{F}: \; (\forall a \in \mathbb{F}: 1 \cdot a = a)$ \\
Existence of additive inverses: 
& $\forall a \in \mathbb{F}: \; (\exists (-a) \in \mathbb{F}: (-a) + a = 0 )$ \\
Existence of multiplicative inverses: 
& $\forall a \in \mathbb{F} \setminus \{0\}: \; (\exists a^{-1} \in \mathbb{F}: a^{-1} \cdot a = 1 )$ \\
Multiplication distributes over addition: 
& $\forall a,b,c \in R: \;  a \cdot (b + c) = a \cdot b + a \cdot c$
\end{tabular}

\medskip
[TODO: verify that the list is complete]

Any mathematical structure that satisfies these requirements is a structure in which can do calculations and solve equations in the same way as we are used to from the rational or real numbers. The rational numbers can be seen as the prototypical example of a field after which the abstraction was modeled. We don't need a right-distributive law because we have required a commutative multiplication. As usual, we will use $\mathbb{F}^*$ as a shorthand for $\mathbb{F} \setminus \{0\}$. With this definition, we could also say that a field $\mathbb{F}$ is a set with two operations $+$ and $\cdot$ such that $(\mathbb{F},+,\cdot)$ forms a commutative unital ring and  $(\mathbb{F}^*,\cdot)$ forms a commutative group. [VERIFY!] 

%So, a field is a ring in which we also have multiplicative inverses for all elements except for the additive neutral element.

% If F has these porperties except commutativity for multiplication, then it's called a skew field (ex: Quaternions) ...does it need to be anticommutative? No - I don't think so - it's also called a division ring:
% https://en.wikipedia.org/wiki/Division_ring
% ...but "skew" seems to suggest anticommutativity

% in fields, there are no primes and composites - just units and zero-divisors (I think)


%===================================================================================================
\subsection{Construction of Fields}

%---------------------------------------------------------------------------------------------------
\subsubsection{Fields of Fractions}
Given an integral domain $R$ (i.e. a commutative unital ring without zero divisors), we can construct a field from it in a way that generalizes the construction of the rational numbers from the ring of integers. It works as follows: Take all pairs $(a,b) \in R^2$ with $b \neq 0$ and identify two pairs $(a,b)$ and $(c,d)$ if they should represent the same fraction. That is, we say $(a,b) = (c,d)$ iff $a d = b c$. Let $F$ be the set of the resulting equivalence classes and let $[a,b]$ denote the equivalence class of $(a,b)$. On the set of these equivalence classes, we define addition as $[a,b] + [c,d] = [a d + bc, bd]$ and multiplication as  $[a,b] \cdot [c,d] = [ac, bd]$. As neutral elements, we use $0_F = [0_R, 1_R]$ and $1_F = [1_R, 1_R]$ where the subscript $F$ means that this is an element of our new, to be constructed field $F$ and the subscript $R$ means that it is an element from our original ring that we started with. The additive and multiplicative inverses in $F$ are given by $-[a,b] = [-a, b]$ and $[a,b]^{-1} = [b,a]$ respectively. We find an isomorphic copy of our original ring $R$ within our newly constructed field $F$ by using the function $f: R \rightarrow F$ with $f(a) = [a,1_R]$ as isomorphism. The so constructed structure $F$ is indeed a field and it is called the \emph{field of fractions} aka the \emph{field of quotients} of $R$. If $R = \mathbb{Z}$, then we will arrive at the rational numbers $F = \mathbb{Q}$. This is the smallest field containing the integers and, in fact, even the smallest field containing the naturals. Of course, the construction is more general in that it can be applied to any integral domain $R$. I think, the so constructed field $F$ is the always smallest possible field containing the original ring $R$ in the sense that every other field containing $R$ is a strict superset of the field of fractions [VERIFY!]. 

Question: Can we use a similar construction to construct a group from a monoid? Like how the integers were constructed from the naturals in the chapter on axiomatic set theory? Figure out! If yes, explain how in the group theory section - if not, explain why not. What about constructing a ring from a group?

% see ACRS pg 34, ABoAA pg 203
% https://en.wikipedia.org/wiki/Field_of_fractions

% https://de.wikipedia.org/wiki/Monoid

% Q is the smallest field containing N

% how to make a field
% https://www.youtube.com/watch?v=YSdidFMwn14



%---------------------------------------------------------------------------------------------------
\subsubsection{Finite Fields}
A \emph{finite field} is a field that has only finitely many elements. It turns out that the ring $\mathbb{Z}_m$ of integers modulo $m$ is, in fact, not just a ring but a field whenever the modulus $m$ is a prime number in which case we usually use the letter $p$ for "prime" instead of the general $m$ for "modulus". That means that for any $a \in \mathbb{Z}_p$, we will find some number $b \in \mathbb{Z}_p$ such that $a b = 1$. That is: we have multiplicative inverses for all elements. When we consider $\mathbb{Z}_m$ with a composite number $m$, we will find that multiplicative inverses exist only for those elements of $\mathbb{Z}_m$ which are coprime with $m$ [VERIFY!]. When the number of elements $n$ of a finite set is not a prime but a power of some prime $p$ such that $n = p^k$ for some natural number $k > 1$, then modular arithmetic will not lead to a finite field either - but it is still possible to construct a finite field of size $n$. In such a field, the implementation of the operations of addition and multiplication are more complicated than simple modular addition and multiplication in $\mathbb{Z}_n$ (see below). These two types\footnote{The case for prime $m$ is actually just a special case of the general case.} of finite fields are the only finite fields that exist and for each size $n$ the finite field of size $n$ is unique (up to isomorphism). These finite fields are also called \emph{Galois fields} and the Galois field of size $n$ is denoted as $GF(n)$. 

\paragraph{Representation of Galois Fields}
The elements of these Galois fields can be concretely represented by polynomials of degree up to $k-1$ over $\mathbb{Z}_p$. That is, a concrete representation of the abstract idea of $GF(n)$ with $n = p^k$ where $p$ is prime is given by $\mathbb{Z}_p[x] / (p(x))$ where $p(x)$ is some fixed irreducible polynomial of degree $k$ over $\mathbb{Z}_p$. The modulo operation ensures that we are dealing only with polynomials of degree less than $k$, i.e. up to $k-1$. To multiply two elements of $GF(n)$ in this representation, we multiply the two corresponding polynomials from $\mathbb{Z}_p[x]$ (i.e. convolve their coefficient arrays) and then take the remainder after doing polynomial division by our chosen $p(x)$. Of course, all these computations are done in $\mathbb{Z}_p$. Addition in this representation is just polynomial addition in $\mathbb{Z}_p$, i.e. element-wise (modular) addition of the respective coefficients. Our polynomials will always have $k$ coefficients and for each coefficient, there are $p$ possible values such that all in all, we are dealing with $n = p^k$ different polynomials that may occur as remainder when dividing by our $p(x)$ of degree $k$. These $n$ different polynomials represent the elements of our Galois field. You may have multiple choices for $p(x)$ to choose from. It doesn't matter which one you choose because the result will always be isomorphic to the result of another choice. Modular arithmetic in $\mathbb{Z}_p$ for prime $p$ is the edge case of $n = p^k$ for $k = 1$ where we are dealing only with the constant (i.e. degree zero) polynomials. VERIFY...TBC...TODO: Give $GF(4)$ as example - with multiplication table. Maybe also $GF(8), GF(9)$. Explain construction of the multiplication operation in more detail or at least give a reference that shows how this can be done. There is some example code in Experiments.cpp

% https://en.wikipedia.org/wiki/Finite_field
% https://mathworld.wolfram.com/FiniteField.html
% https://e.math.cornell.edu/people/belk/numbertheory/ClassificationFiniteFields.pdf

% The elements of these Galois fields can be concretely represented  
% The modular artithmetic is the edge case for k = 1 where we are dealing only with the constant polynomials, I think.


% Two Simple Finite Field Examples: Z_2 and Galois Field GF(2^2) = GF(4)
% https://www.youtube.com/watch?v=IBtghmj1yx8

% https://en.wikipedia.org/wiki/Modular_arithmetic#Integers_modulo_m

%modular integers 

% For each prime 


\paragraph{Applications}
These finite fields have a lot of applications in coding theory...TBC...mention Reed-Solomon codes

% https://en.wikipedia.org/wiki/Finite_field#Applications

% Fehlerkorrektur mit Reed-Solomon-Codes (CD, DVD, Blu-ray, DSL, DVB, RAID, QR-Codes, etc.)
% https://www.youtube.com/watch?v=uOLW43OIZJ0
% 13:45 - All finite fields have primitive roots of unity

%---------------------------------------------------------------------------------------------------
\subsubsection{Subfields and Extension Fields}
By now, you know the drill already: if a subset of the underlying set of a given field forms itself a field (under the same operations), then we say that this subset forms a \emph{subfield} of our given field (under these operations). We already know examples of such subfield relations. $\mathbb{R}$ is a subfield of $\mathbb{C}$ and $\mathbb{Q}$ is a subfield of $\mathbb{R}$ and therefore also of $\mathbb{C}$ (the relation is transitive). From the perspective of $\mathbb{Q}$, the bigger fields $\mathbb{R}$ and $\mathbb{C}$ are called \emph{extension fields} or \emph{field extensions}. We will denote a general field by $\mathbb{F}$ and its extension field typically by $\mathbb{E}$. The theory of field extensions can be seen as a theory of solutions (aka \emph{roots}) of polynomial equations. It turns out that when we are given a polynomial equation $p(x) = 0$ of degree $n$ with coefficients from some field $\mathbb{F}$, then we will always find $n$ (not necessarily distinct) solutions to that equation in a suitable extension $\mathbb{E}$ of the field $\mathbb{F}$. If our given field $\mathbb{F}$ does not already contain the roots, we can always adjoin the roots to obtain a bigger field that does contain the roots. We then say that the field is generated by the roots of our polynomial. Of course, the trivial extension where $\mathbb{E} = \mathbb{F}$ is also possible as a special case. We will typically have to deal with two algebraic structures: the \emph{field} $\mathbb{F}$ of coefficients and the \emph{ring} $\mathbb{F}[x]$ of polynomials over the given field $\mathbb{F}$.

VERIFY! What about the field of rational functions over $\mathbb{F}$ - is this denoted as $\mathbb{F}(x)$, i.e. with parentheses rather than brackets?

% What is F called in this context? base field?
% Note tha we are typically dealing with tow algrbaic structures: the field of coefficients and the ring of polynomials over the given field.

% R is transcendental extension of Q and C is algebraic extension of R?

% R is subfield of C,  Q is subfield of R (and therefore also of C)

% Notation is similar to adjoining elements to a ring - just using () instead of []

% https://mathworld.wolfram.com/ExtensionField.html
% https://en.wikipedia.org/wiki/Field_extension

% The theory of field extensions is concerned with the solutions of polynomial equations. It turns out that when we are given a polynomial equation of degree n with coeffs from some field, then we will always find n solutions in a suitable extension of the field [VERIFY, se ABoAA, pg 311]


\paragraph{Algebraic Extensions of $\mathbb{Q}$}
The field $\mathbb{Q}$ is something like the prototypical example of all fields. It is the smallest field containing the natural numbers [VERIFY!]. For some things, rational numbers are not enough. It is well known that the square root of two is not a rational number. If we just add $\sqrt{2}$ to the field of rational numbers and then make sure to also add everything else that is needed to continue to satisfy the field axioms, we obtain an extended field which we denote by $\mathbb{Q}(\sqrt{2})$ and call "Q adjoined square-root of two". This is totally analogous to what we did with rings when adjoining a new element, just that for fields we use parentheses like $(\ldots)$ instead of brackets like $[\ldots]$. It turns out that the elements of our extended field can always be written in the form $a + b \sqrt{2}$ where $a,b \in \mathbb{Q}$. That means, we can view $\mathbb{Q}(\sqrt{2})$ as a 2D vector space\footnote{Abstract vector spaces will be the subject of the next chapter. But you already know the term from linear algebra and for the moment, it's ok to think about vector spaces in this way.} over $\mathbb{Q}$ with a basis of $\{1, \sqrt{2}\}$, for example - there are other bases for this space, too. The number $\sqrt{2}$ is the root of the polynomial $x^2 - 2$, i.e. a solution to the polynomial equation $x^2 - 2 = 0$. Generally, we call an extension of a field \emph{algebraic}, when the adjoined elements are roots of polynomials. ...TBC...

% What about Q(x) ...I think, this is the field of rational functions over Q?
% Explain root field aka splitting field of a polynomial

% https://math.libretexts.org/Bookshelves/Abstract_and_Geometric_Algebra/Abstract_Algebra%3A_Theory_and_Applications_(Judson)/18%3A_Integral_Domains/18.01%3A_Fields_of_Fractions

% -ToDo: maybe show that sum, difference, product and quotient of (a + b*sqrt(n)) and (c + d*sqrt(n)) is again of the same form (for any integer n that is not a square number)

% Galois extension: is obtained by adjoining all the roots of a polynomial
% https://www.youtube.com/watch?v=zCU9tZ2VkWc at 9:47
% https://en.wikipedia.org/wiki/Galois_extension
% ..I think, it's just another name for splitting field / root field?



\paragraph{Algebraic Closure of a Field}
In an algebraically closed field, we can solve any polynomial equation. In fact, any polynomial of degree $n$ will always factor into precisely $n$ linear factors [VERIFY]. Algebraically closed fields are as nice as it gets when the goal is to solve algebraic equations. The field of complex numbers $\mathbb{C}$ is algebraically closed. When we start with any given field $\mathbb{F}$ and add to it precisely all the objects that are needed to turn it into algebraically closed field, then that so obtained new field is called the algebraic closure of $\mathbb{F}$. In this sense, $\mathbb{C}$ is the algebraic closure of $\mathbb{R}$. The algebraic closure of the field of rational numbers $\mathbb{Q}$ is the set of algebraic numbers $\mathbb{A}$. 

%It is at the same time the algebraic closure of $\mathbb{Z}$ because any polynomial root finding problem with for a polynomial with rational coefficients can be expressed using integer coefficients only by simply multiplying both sides by the least common multiple of all the denominators of the coefficients [VERIFY!].

% For algebraic closures, we usually start with a field rather than a ring

% https://en.wikipedia.org/wiki/Algebraic_closure


\paragraph{Basic theorem of Field Extensions}
Let $\mathbb{F}$ be a field and $p = p(x)$ be a nonconstant polynomial in $\mathbb{F}[x]$. Then there exists an extension field $\mathbb{E} \supseteq \mathbb{F}$ and an element $r \in \mathbb{E}$ such that $r$ is a root of of $p$, i.e. $p(r) = 0$.
% ABoAA pg 275

\medskip
As an immediate consequence, by recursively reducing $p(x)$ by dividing out the linear factors of the form $(x-r)$ from $p$, it follows that there exists a (possibly even larger) extension field that contains all the $n$ roots of $p$ where $n$ is the degree of $p$. [VERIFY!]
% ABoAA pg 276

\medskip
This sounds a lot like the fundamental theorem of algebra which is, in fact, a special case where  $\mathbb{E} = \mathbb{F} = \mathbb{C}$. The complex numbers have the nice property of being algebraically closed such that we do not need to extend the field any further in order to find the roots of any polynomial whatsoever with coefficients from $\mathbb{C}$. [VERIFY]

\paragraph{Splitting Fields}
Given a base field $\mathbb{F}$ and a polynomial with coefficients from $\mathbb{F}$, we can adjoin all the roots of this polynomial $p(x)$ to the field to obtain an extension field $\mathbb{E}$. In this extended field, we are able to split the polynomial into linear factors. That's why this field is also called the splitting field or root field of $p$. ...TBC.. 

%I think it's also called Galois extension?
% don't confuse it with Galois fields - these are finite fields

% Cyclotomic extension: adjoin n-th root of unity
% Kummer extension: andjoin n-th root of something while the n-nth root of unity are already there


\paragraph{Transcendental Extension of $\mathbb{Q}$}
If we adjoin one or more transcendental numbers, i.e. numbers like the semicircle constant $\pi$ or Euler's number $\e$, to our base field $\mathbb{Q}$, we obtain what is called a transcendental extension. ...TBC...

% -explain (Cauchy?) completeness
% https://en.wikipedia.org/wiki/Complete_metric_space
% https://en.wikipedia.org/wiki/Complete_metric_space#Completion


%\paragraph{Extensions of $\mathbb{R}$}





\paragraph{Algebraic Extensions of $\mathbb{F}_p$} We can also extend finite fields by adjoining new elements to them. Examples for finite fields are the integers modulo some prime number $p$ using modular addition and multiplication as operations. These fields contain precisely the $p$ elements $\{0,1,2,\ldots,p-1\}$\footnote{More pedantically, one should say their elements are the equivalence classes of $0,1,2,\ldots,p-1$}. There are also finite fields of sizes of powers of prime numbers, i.e. of sizes $p^n$ for some natural number $n \geq 1$. But the addition and multiplication operations on them are more complicated that just doing modular arithmetic. It can be shown that these types are the only types of finite fields. That means: any finite field has as its size (cardinality) a power of a prime number. If that power is the first power, we can get away with simple modular arithmetic to implement those fields. ...TBC...ToDo: explain how can we envision/implement the fields with size $p^n$ for $n \neq 1$, explain their applications for example in coding theory.

% -explain algebraic closures of finite fields. These are complicated, I think. See elliptic tales.



%---------------------------------------------------------------------------------------------------
\subsubsection{Division by Zero}
We learn already in elementary school that division by zero is "impossible". The history of mathematics is full of examples of things that were at one point deemed to be impossible which became possible later by extending the number system suitably. Subtracting 5 from 3 is impossible in the natural numbers but no problem in the integers. Dividing 3 by 5 is impossible in the integers but doable in the rationals. Taking the square root of 2 is impossible in the rationals but totally fine in the reals. Finally, taking the square root of a negative number is impossible in the reals but possible in the complex numbers. So, one might hope that with a suitable definition that extends our number system even further, also the problem of division by zero might be solved. Unfortunately, it turns out that doing so will necessarily invalidate one of our most basic calculation rules ...TBC...

\paragraph{The Extended Real Line}...TBC..explain projective real line 

% Field Extensions and the Hyperreals
% https://www.youtube.com/watch?v=A7vRtMyR9gE&list=PLHW2zF4VTA-cidobAyHGzugw-XXh_H5tL&index=3
% -R(i) is a *simple* extension of R with generator i. Q(sqrt(2)) is another example of a simple
%  extension
% -The hyperreals *R are the reals with w adjoined. w is a number that is greater than any real
%  number....I think, it's the infinite ordinal omega? espilon = 1/w is an ininitesimal

% https://en.wikipedia.org/wiki/Extended_real_number_line


\paragraph{Hyperreal Numbers} Let's again start with the field of real numbers $\mathbb{R}$ and adjoin a special number $\omega$ with the property that $\omega > r$ for any real number $r$. For such a number to be larger than any real number, it must somehow be infinite. To make the set a field, we must also add $1/\omega$ which we will denote as $\epsilon$. That epsilon is called an infinitesimal and it has the property that it is greater than zero but less than any positive real number.  ...TBC...explain standard part, dual numbers

\paragraph{Wheel Theory}...

%Was passiert, wenn man durch 0 dividiert
%https://www.youtube.com/watch?v=_b51R7eSgxw&list=PL542920k_cOq1gzagygodiAQg5WA6h2K4&index=4




\begin{comment}

Explain the edge case where additive neutral elements have a multiplicative inverse. I think Michael Penn has a video about that called somethign like "when division by zero isn't ..." IIRC	

 We have used the symbol $1$ to denote the multiplicative neutral element. 

The notation $a^{-1}$ is now used for the multiplicative inverse of $a$, so we need a new notation for the additive inverse. We will use $-a$ to denote the additive inverse of $a$. In the axiom about multiplicative inverses, we have excluded the $0$ element. The additive neutral element (i.e. zero) is the only element for which we do not require to have a multiplicative inverse. It can in fact be shown that - some uninteresting trivial edge cases aside - the additive neutral element cannot possibly have a multiplicative inverse in any ring. 

https://en.wikipedia.org/wiki/Finite_field
https://en.wikipedia.org/wiki/Cyclotomic_field

Warum kann man nicht durch null teilen? Oder: Was sind Zahlen eigentlich?
https://www.youtube.com/watch?v=v05rI0lGk6A

https://en.wikipedia.org/wiki/Quadratic_field


https://www.youtube.com/watch?v=
4BfCmZgOKP8  Was sind Galoiskörper? [Weitz]


https://en.wikipedia.org/wiki/Splitting_field

-In groups, we can always solve equations like a*x = b
-In fields, we can always solve equations like a*x + b = c (in rings only sometimes)
-In algebraically closed fields, we can always solve polynomial equations - but only in principle, 
 i.e. in the sense that solutions exist and we may find them numerically. Up to degree 4, there are
 formulas in terms of the coeffs. ...do these formulas also apply to finite fields?
-Field extensions can be writte like R/Q or just with the subset symbol
-Degree of field extension: if L is a field extesnion over K. We may interpret L as a vector space 
 over K. For example. C is a field extension over R has degree 2. We write this as 
 [C:R] = dim_R(C) = 2, dim_F(F) = 1 always, [Q(sqrt(2)) : Q] dim_Q(Q(sqrt(2))) = 2,
 [Q(cbrt(2)) : Q] = 3  bcs  Q(cbrt(2))  = { a + b sqrt(2) + c cbrt(2^2) :  a,b,c in Q }
 [R : Q] = inf, R = { a + b r_1 + c * r_2 + d * r_3 + ...   where r_in in R }
 [F(a) : F] = deg(m_a(x)) where m_a(x) is the minimal polynomial for a (a is an algebraic number)
 For a chain of field extensions F \subset G \subset H we have:
 [H :  F] = [H : G] * [G : F]


https://www.youtube.com/watch?v=Ct2fyigNgPY
-10:10: an equation with an abelian Galois group is solvabel by radicals
-11:40: Galois group of a quinitc is S_2 * S_5 which is non-abelian (?...not sure)


https://en.wikipedia.org/wiki/Field_of_fractions

-Maybe to explain agebraic extensions and root fields, use Z as base-ring and extend by 
 fractions like 1/2, 1/3 etc. Use p(x) = 0 = (x + 1/2) * (x + 1/3) = x^2 + x/2 + x/3 + 1/6
 leading to 0 = 6 x^2 + 6x/2 + 6x/3 + 1 = 6 x^2 + 5 x + 1. The root ring is Z[1/6]
 ...hmm...don't know, if the analogy to root fields really works out
-I think, in general, a polynomial (x + 1/a)*(x + 1/b)*(x + 1/c)*... results in a root ring of
 Z[1/lcm(a,b,c,...)], so we can always adjoin all roots of p by adjoining a single fraction. With
 root fields over Q, we really need to adjoin multiple elements - one for each root, I think. That
 also means that this permutation business with the Galois group is much simpler. We have only one
 element adjoined, so a general element of Z[1/n] will look like a + b/n and the only two 
 automorphisms Z[1/n]  ->  Z[1/n] that keep Z fixed are  a + b/n  ->  a + b/n  and
 a + b/n  ->   a - b/n. So, the "Galois group" (automorphism group that keep base ring fixed) will
 always only have two elements: the identity and the "fractional conjugation" (i.e. the one that
 negates the fractional part)
-But: ABoAA (pg 313) says that Q(sqrt(2), sqrt(3)) can be obtained simpler as 
 Q(sqrt(2) + sqrt(3)), i.e. as a simple extension. That implies that we should be able to express
 sqrt(2) as a polynomial a0 + a1*r + a2*r^2 + a3*r^3 + .... where r = sqrt(2) + sqrt(3). But how?
 Figure out the coeffs of the polynomial! They should be rational. Maybe try an ansatz:
   sqrt(2) = a0 + a1*r + a2*r^2
 square both sides:
   2 = a0^2 + a0*a1*r + a0*a2*r^2 + ...
 and solve for a0,a1,a2. Maybe we need a higher degree ansatz or maybe a lower degree is already
 sufficient? Figure it out!
-That seems to be analogous the case of obatining  Z(1/2, 1/3) as Z(1/2 + 1/3) = Z(3/6 + 2/6) 
 = Z(5/6). Z(5/6) = Z(1/6) because we can produce 1/6 as 1 - 5/6.

https://mathworld.wolfram.com/SplittingField.html
I think, this is the same as the root-field? yes:
https://www.collinsdictionary.com/dictionary/english/root-field
I think, it's called "splitting field" because the polynomial "splits" into linear factors


https://en.wikipedia.org/wiki/Galois_group
https://de.wikipedia.org/wiki/Galoisgruppe
https://mathworld.wolfram.com/GaloisGroup.html
The Galois group of a polynomial of degree n (or rather, the polynomial's splitting field) is always
a subgroup of S_n (the symmetric group of order n, i.e. the group of all possible permutations
of n elements). In the "worst case", it is S_n itself. Up to n = 4, S_n itself is "solvable" (and
therefore also all its subgroups are solvable). A group G is solvabel, if it has a sequence of
normal subgroups 
  ${e} = G_0 \subset G_1 \ldots \subset G_m = G$
such that each $G_k$ is a normal subgroup of $G_{k+1}$ and $G_{k+1} / G_k$ is abelian. An equivalent
definition is that g has a normal series in which all quotient groups are cyclic of prime order.

see ABoAA, pg 338, 342, 343

Field Theory - an Introduction
https://www.youtube.com/watch?v=BcF6APQYU1w&list=PLHW2zF4VTA-cidobAyHGzugw-XXh_H5tL&index=2


https://en.wikipedia.org/wiki/Valuation_ring

Surreal Numbers
https://www.youtube.com/watch?v=QaK81xaIb-U

https://en.wikipedia.org/wiki/Ordered_field#Definitions
	
https://en.wikipedia.org/wiki/Levi-Civita_field
https://en.wikipedia.org/wiki/Rational_function	

https://en.wikipedia.org/wiki/Valuation_(algebra)
	
ABoAA:
pg 202: 
-in any integral domain of characteristic p: (a+b)^p = a^p + b^p
-every finite integral domain is a field.

	
	
\end{comment}