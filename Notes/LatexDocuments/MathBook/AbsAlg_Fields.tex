\section{Fields}
In the realm of abstract algebra, a \emph{field} is a ring with some additional requirements. It must be a unital ring, which means that a multiplicative identity (i.e. neutral element) must exist. We will also require that all elements except the additive identity must have multiplicative inverses. The exception is motivated by the idea that division by zero is impossible. It turns out that this will case for any field if we want $a \cdot 0 = a \cdot (0 + 0) = a \cdot 0 + a \cdot 0$ to be true for any $a$ that is an element of our field. Furthermore, we require that the multiplication is also commutative. The prototypical examples of fields are the rational numbers $\mathbb{Q}$ and the real numbers $\mathbb{R}$. Whenever we have a field, we can do computations with the elements of the field in the same way that we are used to from elementary school. That means, we can add, subtract, multiply and divide to our hearts content with the sole exception of division by zero. As with rings, we denote a field as a triple $(\mathbb{F}, +, \cdot)$ made from an underlying set $\mathbb{F}$ and two operations which we call addition and multiplication. Formally, the requirements for a field are:
\begin{itemize}
\item $(\mathbb{F},+,\cdot)$ forms a unital ring, i.e. a ring with a multiplicative neutral element.
\item Every element except the additive neutral element has a multiplicative inverse element.
\item Multiplication is commutative.
\end{itemize}
Because these properties are stated at a pretty high level, let's state them boiled down to a lower level:

\medskip
\begin{tabular}{l l}
Additive closure: 
& $\forall a,b \in \mathbb{F}: \; a + b \in \mathbb{F}$  \\	
Multiplicative closure: 
& $\forall a,b \in \mathbb{F}: \; a \cdot b \in \mathbb{F}$  \\	
Associativity of addition: 
& $\forall a,b,c \in \mathbb{F}: \;  (a + b) + c = a + (b + c)$   \\
Associativity of multiplication: 
& $\forall a,b,c \in \mathbb{F}: \;  (a \cdot b) \cdot c = a \cdot (b \cdot c)$   \\
Commutativity of addition: 
& $\forall a,b \in \mathbb{F}: \;  a + b = b + a$   \\
Commutativity of multiplication: 
& $\forall a,b \in \mathbb{F}: \;  a \cdot b = b \cdot a$   \\
Existence of additive identity: 
& $\exists 0 \in \mathbb{F}: \; (\forall a \in \mathbb{F}: 0 + a = a)$ \\
Existence of multiplicative identity: 
& $\exists 1 \in \mathbb{F}: \; (\forall a \in \mathbb{F}: 1 \cdot a = a)$ \\
Existence of additive inverses: 
& $\forall a \in \mathbb{F}: \; (\exists (-a) \in \mathbb{F}: (-a) + a = 0 )$ \\
Existence of multiplicative inverses: 
& $\forall a \in \mathbb{F} \setminus \{0\}: \; (\exists a^{-1} \in \mathbb{F}: a^{-1} \cdot a = 1 )$ \\
Multiplication distributes over addition: 
& $\forall a,b,c \in R: \;  a \cdot (b + c) = a \cdot b + a \cdot c$
\end{tabular}

\medskip
[...todo: verify that the list is complete]

Any mathematical structure that satisfies these requirements is a structure in which can do calculations and solve equations in the same way as we are used to from the rational or real numbers. The rational numbers can be seen as the prototypical example of a field after which the abstraction was modeled. We don't need a right-distributive law because we have required a commutative multiplication. As usual, we will use $\mathbb{F}^*$ as a shorthand for $\mathbb{F} \setminus \{0\}$.

%===================================================================================================
\subsection{Construction of Fields}


\subsubsection{Fields of Fractions}
Given an integral domain $R$ (i.e. a commutative unital ring without zero divisors), we can construct a field from it in a way that generalizes the construction of the rational numbers from the ring of integers. It works as follows: Take all pairs $(a,b) \in R^2$ with $b \neq 0$ and identify two pairs $(a,b)$ and $(d,c)$ if they should represent the same fraction. That is, we say $(a,b) = (c,d)$ iff $a d = b c$. Let $F$ be the set of the resulting equivalence classes and let $[a,b]$ denote the equivalence class of $(a,b)$. On the set of these equivalence classes, we define addition as $[a,b] + [c,d] = [a d + bc, bd]$ and multiplication as  $[a,b] \cdot [c,d] = [ac, bd]$. As neutral elements, we use $0_F = [0_R, 1_R]$ and $1_F = [1_R, 1_R]$ where the subscript $F$ means that this is an element of our new, to be constructed field $F$ and the subscript $R$ means that it is an element from our original ring that we started with. The so constructed structure $F$ is indeed a field and it is called the \emph{field of fractions} aka the \emph{field of quotients} of $R$. If $R = \mathbb{Z}$, then we will arrive at the rational numbers $F = \mathbb{Q}$. This is the smallest field containing the integers and, in fact, even the smallest field containing the naturals. Of course, the construction is more general in that it can be applied to any integral domain $R$. I think, the so constructed field $F$ is the always smallest possible field containing the original ring $R$ in the sense that every other field containing $R$ is a strict superset of the field of fractions [VERIFY!]. 

% see ACRS, pg 34
% https://en.wikipedia.org/wiki/Field_of_fractions

% Q is the smallest field containing N

%---------------------------------------------------------------------------------------------------
\subsubsection{Subfields and Extension Fields}
By now, you know the drill already: if a subset of the underlying set of a given field forms itself a field (under the same operations), then we say that this subset forms a subfield of our given field (under these operations). We already know examples of such subfield relations. $\mathbb{R}$ is a subfield of $\mathbb{C}$ and $\mathbb{Q}$ is a subfield of $\mathbb{R}$ and therefore also of $\mathbb{C}$ (the relation is transitive). From the perspective of $\mathbb{Q}$, the bigger fields $\mathbb{R}$ and $\mathbb{C}$ are called extension fields [VERIFY]. 

% R is transcendental extension of Q and C is algebraic extension of R?

% R is subfield of C,  Q is subfield of R (and therefore also of C)

% Notation is similar to adjoining elements to a ring - just using () instead of []

\paragraph{Algebraic Extensions of $\mathbb{Q}$}
The field $\mathbb{Q}$ is something like the prototypical example of all fields. It is the smallest field containing the natural numbers [VERIFY!]. For some things, rational numbers are not enough. It is well known that the square root of two is not a rational number. If we just add $\sqrt{2}$ to the field of rational numbers and then make sure to also add everything else that is needed to continue to satisfy the field axioms, we obtain an extended field which we denote by $\mathbb{Q}(\sqrt{2})$ and call "Q adjoined square-root of two". This is totally analogous to what we did with rings when adjoining a new element, just that for fields we use parentheses like $(\ldots)$ instead of brackets like $[\ldots]$. It turns out that the elements of our extended field can always be written in the form $a + b \sqrt{2}$ where $a,b \in \mathbb{Q}$.

% What about Q(x) ...I think, this is the field of rational functions over Q?
% Explain root field aka splitting field of a polynomial

% https://math.libretexts.org/Bookshelves/Abstract_and_Geometric_Algebra/Abstract_Algebra%3A_Theory_and_Applications_(Judson)/18%3A_Integral_Domains/18.01%3A_Fields_of_Fractions

\paragraph{Algebraic Closure of a Field}
In an algebraically closed field, we can solve any polynomial equation. In fact, any polynomial of degree $n$ will always factor into precisely $n$ linear factors [VERIFY]. Algebraically closed fields are as nice as it gets when the goal is to solve algebraic equations. The field of complex numbers $\mathbb{C}$ is algebraically closed. When we start with any given field $\mathbb{F}$ and add to it precisely all the objects that are needed to turn it into algebraically closed field, then that so obtained new field is called the algebraic closure of $\mathbb{F}$. In this sense, $\mathbb{C}$ is the algebraic closure of $\mathbb{R}$. The algebraic closure of the field of rational numbers $\mathbb{Q}$ is the set of algebraic numbers $\mathbb{A}$. 

%It is at the same time the algebraic closure of $\mathbb{Z}$ because any polynomial root finding problem with for a polynomial with rational coefficients can be expressed using integer coefficients only by simply multiplying both sides by the least common multiple of all the denominators of the coefficients [VERIFY!].

% For algebraic closures, we usually start with a field rather than a ring

% https://en.wikipedia.org/wiki/Algebraic_closure

\paragraph{Transcendental Extension of $\mathbb{Q}$}
If we adjoin one or more transcendental numbers, i.e. numbers like the semicircle constant $\pi$ or Euler's number $\e$, to our base field $\mathbb{Q}$, we obtain what is called a transcendental extension. ...TBC...

% -explain (Cauchy?) completeness
% https://en.wikipedia.org/wiki/Complete_metric_space
% https://en.wikipedia.org/wiki/Complete_metric_space#Completion


\paragraph{Algebraic Extensions of $\mathbb{F}_p$}

% -explain algebraic closures of finite fields. These are complicated, I think. See elliptic tales.



%---------------------------------------------------------------------------------------------------
\subsubsection{Division by Zero}
We learn already in elementary school that division by zero is "impossible". The history of mathematics is riddled with examples of things that were at one point deemed to be impossible which became possible later by extending the number system suitably. Subtracting 5 from 3 is impossible in the natural numbers but no problem in the integers. Dividing 3 by 5 is impossible in the integers but doable in the rationals. Taking the square root of 2 is impossible in the rationals but totally fine in the reals. Finally, taking the square root of a negative number is impossible in the reals but possible in the complex numbers. So, one might hope that with a suitable definition that extends our number system even further, also the problem of division by zero might be solved. Unfortunately, it turns out that doing so will necessarily invalidate one of our most basic calculation rules ...TBC...

\paragraph{The Extended Real Line}...

\paragraph{Hyperreal Numbers}...

\paragraph{Wheel Theory}...

%Was passiert, wenn man durch 0 dividiert
%https://www.youtube.com/watch?v=_b51R7eSgxw&list=PL542920k_cOq1gzagygodiAQg5WA6h2K4&index=4




\begin{comment}

Explain the edge case where additive neutral elements have a multiplicative inverse. I think Michael Penn has a video about that called somethign like "when division by zero isn't ..." IIRC	

 We have used the symbol $1$ to denote the multiplicative neutral element. 

The notation $a^{-1}$ is now used for the multiplicative inverse of $a$, so we need a new notation for the additive inverse. We will use $-a$ to denote the additive inverse of $a$. In the axiom about multiplicative inverses, we have excluded the $0$ element. The additive neutral element (i.e. zero) is the only element for which we do not require to have a multiplicative inverse. It can in fact be shown that - some uninteresting trivial edge cases aside - the additive neutral element cannot possibly have a multiplicative inverse in any ring. 

https://en.wikipedia.org/wiki/Finite_field
https://en.wikipedia.org/wiki/Cyclotomic_field

Warum kann man nicht durch null teilen? Oder: Was sind Zahlen eigentlich?
https://www.youtube.com/watch?v=v05rI0lGk6A

https://en.wikipedia.org/wiki/Quadratic_field


https://www.youtube.com/watch?v=
4BfCmZgOKP8  Was sind Galoiskörper? [Weitz]


https://en.wikipedia.org/wiki/Splitting_field

-In groups, we can always solve equations like a*x = b
-In fields, we can always solve equations like a*x + b = c (in rings only sometimes)
-In algebraically closed fields, we can always solve polynomial equations - but only in principle, 
 i.e. in the sense that solutions exist and we may find them numerically. Up to degree 4, there are
 formulas in terms of the coeffs. ...do these formulas also apply to finite fields?
-Field extensions can be writte like R/Q or just with the subset symbol
-Degree of field extension: if L is a field extesnion over K. We may interpret L as a vector space 
 over K. For example. C is a field extension over R has degree 2. We write this as 
 [C:R] = dim_R(C) = 2, dim_F(F) = 1 always, [Q(sqrt(2)) : Q] dim_Q(Q(sqrt(2))) = 2,
 [Q(cbrt(2)) : Q] = 3  bcs  Q(cbrt(2))  = { a + b sqrt(2) + c cbrt(2^2) :  a,b,c in Q }
 [R : Q] = inf, R = { a + b r_1 + c * r_2 + d * r_3 + ...   where r_in in R }
 [F(a) : F] = deg(m_a(x)) where m_a(x) is the minimal polynomial for a (a is an algebraic number)
 For a chain of field extensions F \subset G \subset H we have:
 [H :  F] = [H : G] * [G : F]


https://www.youtube.com/watch?v=Ct2fyigNgPY
-10:10: an equation with an abelian Galois group is solvabel by radicals
-11:40: Galois group of a quinitc is S_2 * S_5 which is non-abelian (?...not sure)


https://en.wikipedia.org/wiki/Field_of_fractions

-Maybe to explain agebraic extensions and root fields, use Z as base-ring and extend by 
 fractions like 1/2, 1/3 etc. Use p(x) = 0 = (x + 1/2) * (x + 1/3) = x^2 + x/2 + x/3 + 1/6
 leading to 0 = 6 x^2 + 6x/2 + 6x/3 + 1 = 6 x^2 + 5 x + 1. The root ring is Z[1/6]
 ...hmm...don't know, if the analogy to root fields really works out
-I think, in general, a polynomial (x + 1/a)*(x + 1/b)*(x + 1/c)*... results in a root ring of
 Z[1/lcm(a,b,c,...)], so we can always adjoin all roots of p by adjoining a single fraction. With
 root fields over Q, we really need to adjoin multiple elements - one for each root, I think. That
 also means that this permutation business with the Galois group is much simpler. We have only one
 element adjoined, so a general element of Z[1/n] will look like a + b/n and the only two 
 automorphisms Z[1/n]  ->  Z[1/n] that keep Z fixed are  a + b/n  ->  a + b/n  and
 a + b/n  ->   a - b/n. So, the "Galois group" (automorphism group that keep base ring fixed) will
 always only have two elements: the identity and the "fractional conjugation" (i.e. the one that
 negates the fractional part)
-But: ABoAA (pg 313) says that Q(sqrt(2), sqrt(3)) can be obtained simpler as 
 Q(sqrt(2) + sqrt(3)), i.e. as a simple extension. That implies that we should be able to express
 sqrt(2) as a polynomial a0 + a1*r + a2*r^2 + a3*r^3 + .... where r = sqrt(2) + sqrt(3). But how?
 Figure out the coeffs of the polynomial! They should be rational. Maybe try an ansatz:
   sqrt(2) = a0 + a1*r + a2*r^2
 square both sides:
   2 = a0^2 + a0*a1*r + a0*a2*r^2 + ...
 and solve for a0,a1,a2. Maybe we need a higher degree ansatz or maybe a lower degree is already
 sufficient? Figure it out!
-That seems to be analogous the case of obatining  Z(1/2, 1/3) as Z(1/2 + 1/3) = Z(3/6 + 2/6) 
 = Z(5/6). Z(5/6) = Z(1/6) because we can produce 1/6 as 1 - 5/6.

https://mathworld.wolfram.com/SplittingField.html
I think, this is the same as the root-field? yes:
https://www.collinsdictionary.com/dictionary/english/root-field
I think, it's called "splitting field" because the polynomial "splits" into linear factors


https://en.wikipedia.org/wiki/Galois_group
https://de.wikipedia.org/wiki/Galoisgruppe
https://mathworld.wolfram.com/GaloisGroup.html
The Galois group of a polynomial of degree n (or rather, the polynomial's splitting field) is always
a subgroup of S_n (the symmetric group of order n, i.e. the group of all possible permutations
of n elements). In the "worst case", it is S_n itself. Up to n = 4, S_n itself is "solvable" (and
therefore also all its subgroups are solvable). A group G is solvabel, if it has a sequence of
normal subgroups 
  ${e} = G_0 \subset G_1 \ldots \subset G_m = G$
such that each $G_k$ is a normal subgroup of $G_{k+1}$ and $G_{k+1} / G_k$ is abelian. An equivalent
definition is that g has a normal series in which all quotient groups are cyclic of prime order.

see ABoAA, pg 338, 342, 343


	
\end{comment}