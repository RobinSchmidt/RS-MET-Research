\section{Group Theory} 
One of the most fundamental algebraic structures is called a group. A group is defined by an underlying set and one binary operation on elements of this set. An example of a group would be the integer numbers with addition as the operation. One typically denotes a group as a pair consisting the set and the operation, so the group of integers with addition is denoted by $(\mathbb{Z},+)$. Let's abstract from this concrete example and consider an arbitrary set $G$ on which an arbitrary binary operation, denoted by $\circ$, is defined and consider the pair $(G,\circ)$. For such a pair to qualify as a group, the operation $\circ$ has to satisfy the following rules:

\medskip
\begin{tabular}{c l}
Associativity: 
& $\forall a,b,c \in G: \;  (a \circ b) \circ c = a \circ (b \circ c)$   \\
Existence of neutral element: 
& $\exists e \in G: \; (\forall a \in G: e \circ a = a)$ \\
Existence of inverse elements: 
& $\forall a \in G: \; (\exists a^{-1} \in G: a^{-1} \circ a = e )$ \\
\end{tabular}
\medskip

Let's translate that foreign mathspeak to plain english: Associativity requires that when we combine 3 elements $a,b,c$ via our operation by firstly combining $a$ and $b$ and then secondly combining the result of that with $c$ yields the same final result as combining $a$ with the result of the combination of $b$ and $c$. This is an abstraction of the idea that $(2+3)+4 = 2+(3+4)$ because $5+4 = 2+7$. The second rule says that there must exist some special element $e$ within the set which, when combined with any element $a$ from the set, just gives back $a$ itself. This special element $e$ is called the neutral element of the operation because it changes nothing when it's being combined with any element from the set. It's an abstraction of the idea that $0+x = x$ for any $x$ whatsoever. The number zero is the neutral element of addition in the integers. At the outset, it may seem that we should actually be more precise and call it the left-neutral element because we require it to behave neutrally only from the left. However, group theory gives us the theorem that any left-neutral element is automatically always also right-neutral, so it's justified to just call it the neutral element without any left or right qualifier. Group theory also gives us the theorem that such a neutral element is always unique so it's indeed justified to call it \emph{the} neutral element instead of \emph{a} neutral element. By the way, it is common to refer by the term "group" also to the underlying set rather than the pair if it's clear from the context about which operation we talk. The last rule requires that for any element $a$ from the group there must exist some element $a^{-1}$ also from the group, such that when we form the combination $ a^{-1} \circ a$ via our group operation, then the neutral element will come out as result. This element $a^{-1}$ is called the inverse element to $a$. So, the rules says that such inverse elements must exist for all group elements. In the integers with addition, the inverse element to $3$ is $-3$ because $-3 + 3 = 0$ and zero is the neutral element. Again, we need to require only the existence of left-inverses and group theory assures us that any left-inverse element is proven to always automatically also be right-inverse and is unique so we are justified to call it the inverse element. We may begin to see the value of group theory. If we want to verify that right-inverses exist in some brand new mathematical structure that we came up with, we actually just need to check these three simple conditions and if they hold true, we are good and can in fact assert a lot more properties of our structure without needing to verify them all separately.

\paragraph{Notation}
So far, we have used the symbol $\circ$ in infix notation like in $c = a \circ b$ to denote the combination of elements $a,b$ by our group operation. This was done to indicate that we don't not necessarily mean a concrete well known operation such as addition or multiplication. Depending on the context, the group operation may indeed be addition. Or it may be multiplication. Or something else entirely. It may also mean the composition of certain \emph{actions} in the sense of performing one action after another. We may be used to the circle being used for function composition and indeed, if we consider only invertible (i.e. strictly monotonic) functions, they do indeed form a group under function composition (where "under" here means: "with the group operation being"). This usage of the circle is not universal. You will also often encounter the dot being used: $c = a \cdot b$ and sometimes the operation symbol is scrapped entirely: $c = a b$ like we often do with multiplication when dealing with numbers, matrices, functions and so on. From now on, we will also adopt these notations. We will also denote the combination of an element $a$ with itself as $a^2$, i.e. $a^2 = a a = a \cdot a$. In general we will denote by $a^n$ the idea to combine $a$ with itself $n-1$ times. That is, we have $n$ "factors" all of them being $a$ and we have $n-1$ "dots" in between them: $a^4 = a \cdot a \cdot a \cdot a$. 

\subsection{Some Basic Theorems}
Let's axiomatically assume that our rules stated above are true for some given set and operation (let's say, we have verified them). What can we deduce from these axioms alone without knowing anything more about our pair? Without proof, I'll hand you down the following useful theorems, some of which were already mentioned:


% -uniqueness of inverses and neutral element
% -power rules
% In this context, it is natural to define $a^0 = e$.


\subsection{Cayley Tables}


% use the Klein Four group as example...or maybe the group of symmetries of a triangle

%\subsection{Some Examples}
%Now that we have defined what a group is in an abstract sense, let's look at some concrete examples of very different groups to get an appreciation for how surprisingly universally applicable these concepts are.
%
%
%\paragraph{Integers with Addition}
%
%\paragraph{Nonzero Rationals with Multiplication}
%
%\paragraph{Modular Integers with Modular Addition}
%
%\paragraph{Rotations in the Plane}
%
%\paragraph{Permutations}
%%bring this last
%%\paragraph{The Modular Integers with Prime Modulus with Modular Multiplication}
%\paragraph{Invertible Functions with Composition}


\begin{comment}

https://en.wikipedia.org/wiki/Group_(mathematics)
https://en.wikipedia.org/wiki/Cayley_table
https://en.wikipedia.org/wiki/Dihedral_group

\end{comment}