\section{Group Theory} 
One of the most fundamental algebraic structures is called a group. A group is defined by an underlying set and one binary operation on elements of this set. An example of a group would be the integer numbers with addition as the operation. One typically denotes a group as a pair consisting the set and the operation, so the group of integers with addition is denoted by $(\mathbb{Z},+)$. Let's abstract from this concrete example and consider an arbitrary set $G$ on which an arbitrary binary operation, denoted by $\circ$, is defined and consider the pair $(G,\circ)$. For such a pair to qualify as a group, the operation $\circ$ has to satisfy the following rules:

\medskip
\begin{tabular}{c l}
Associativity: 
& $\forall a,b,c \in G: \;  (a \circ b) \circ c = a \circ (b \circ c)$   \\
Existence of neutral element: 
& $\exists e \in G: \; (\forall a \in G: e \circ a = a)$ \\
Existence of inverse elements: 
& $\forall a \in G: \; (\exists a^{-1} \in G: a^{-1} \circ a = e )$ \\
\end{tabular}
\medskip

Let's translate that foreign mathspeak to plain english:








\begin{comment}

https://en.wikipedia.org/wiki/Group_(mathematics)

\end{comment}