\section{Polynomial Algebra}
Linear algebra is, to a large part, about finding solution vectors $\mathbf{x}$ to equations of the general form $\mathbf{A x} = \mathbf{b}$ which are called linear equations. More generally, algebra is about the solution of algebraic equations which are defined to be equations that can be formed by using the basic arithmetic operations and roots. The roots are included because they are thought of as being inversions of the operation of repeatedly multiplying an object by itself so they are in some sense induced by the multiplication operation. Polynomial equations are an important subset of algebraic equations. In one dimension, these are equations of the general form:
\begin{equation}
 \sum_{n=0}^N a_n x^n = 0
\end{equation}
where the $a_n$ are known quantities, called the coefficients and $x$ is an unknown \emph{variable} or \emph{indeterminate} that we want to find. If on the right hand side, we have some other constant than $0$, we can just move it over to the left hand side and absorb it in the $a_0$ coefficient. The left hand side is called a polynomial in $x$. We have already met polynomials as a certain type of functions: we can plug in a value $x$ and it spits out a value $f(x)$. When polynomials are viewed as functions, it makes sense to ask: for which values of $x$ does the function produce zero as output. That question is what solving the equation $f(x) = 0$ encodes. Besides the view of polynomials as functions that produce an output for a given input and as equations that we want to solve, we may also view polynomials as formal expressions. Viewed as such an expression, a polynomial is determined by its sequence of coefficients and we forget about the functional aspect - we just think of it as a finite sequence of numbers. These sequences of polynomial coefficients admit an algebra among themselves: we can add, subtract, multiply and (with caveats) divide such sequences of polynomial coefficients. We may even consider infinite sequences which leads us to the idea of formal power series. With such power series, the formal point of view really yields a generalized view because power series can be interpreted as functions only when the series is convergent - but when seen just as formal expressions, we may work with them even when the series does not converge. With polynomials, we don't have that convergence problem, so we can always view a polynomial as a function - but in some contexts, we may voluntarily de-emphasize that point of view. It should also be noted that there are settings in which two polynomials with different coefficients encode the same function. That happens when we do not work over our usual real or complex numbers but over a finite ring. Over finite sets, there is only a finite number of possible functions from the set to itself but there is still an infinite number of different coefficient sequences so some of them must encode the same function. From this point of view, the formal and functional view of polynomials differ in what it means for two polynomials to be equal to one another. The formal sense of equality of polynomials is more restrictive than the functional sense. In the formal view, the sequences of coefficients must match whereas in the functional view, only the input/output relation must match which is, in some contexts, a weaker requirement as we have seen. Over infinite rings though, the two requirements are the same [VERIFY!].

...TBC...

% Example for two polynomials that are the same a functions but different as formal polynomials:
% a(x) = x^5 + 1, b(x) = x - 4  in Z_5[x]. In infinite fields, this cannot happen - in this case,
% the notions of equality of polynomials seen as functions as as formal expressions are tzhe same
% see ABoAA, pg 260

%===================================================================================================
\subsection{Operations on Polynomials}

%---------------------------------------------------------------------------------------------------
\subsubsection{Arithmetic Operations}

\paragraph{Addition and Subtraction}

\paragraph{Multiplication}

\paragraph{Division}

\paragraph{Greatest Common Divisor}

% -Euclidean algorithm for polynomials
% -Extended euclidean algorithm
% -least common multiple



\paragraph{Composition}

%---------------------------------------------------------------------------------------------------
\subsubsection{Calculus Operations}

\paragraph{Derivative}

\paragraph{Antiderivative}

\paragraph{Definite Integrals}


%---------------------------------------------------------------------------------------------------
\subsubsection{Miscellaneous Operations}

\paragraph{Evaluation}

\paragraph{Base Change}



%===================================================================================================
\subsection{Roots and Factorization}
Just like integers can be factored into primes, polynomials can be \emph{factored} into \emph{irreducible} polynomials.

% All the ideals of F[x] are principal ideals.

% For a given fixed polynomial a(x), the principal ideal generated by a polynomial a(x) consists of all the products a(x) s(x) where s(x) ranges over all the polynomials in F[x].

% see ABoAA pg 251

%---------------------------------------------------------------------------------------------------
\subsubsection{Factors} 
Suppose a polynomial $a(x)$ can be written as the product of two other polynomials $b(x)$ and $c(x)$ such that $a(x) = b(x) c(x)$. Then, we say: (1) $b$ and $c$ are \emph{factors} of $a$. (2) $a$ is a \emph{multiple} of $b$ as well as a multiple of $c$. (3) $b$ divides $a$ and $c$ divides $a$. To express the third thing, we also write $b|a$ and $c|a$ just as we did when one integer divides another.

%---------------------------------------------------------------------------------------------------
\subsubsection{Reducibility}
A polynomial $a(x)$ is said to be reducible over a given field $\mathbb{F}$ if there are polynomials $b(x), c(x)$ in $\mathbb{F}[x]$ such that $a(x) = b(x) c(x)$. Note that it is important to consider the field over which we want to reduce. For example $(x+\sqrt{2})(x-\sqrt{2}) = x^2 - 2$ is reducible over $\mathbb{R}$ but irreducible over $\mathbb{Q}$. Likewise, $(x+\i)(x-\i) = x^2 + 1$ is reducible over $\mathbb{C}$ but irreducible over $\mathbb{R}$. Sometimes, the field about which we talk must be inferred from the context.

% -Factorization

\paragraph{Factorization Theorem}
Every polynomial can be written as a constant times a product of monic irreducible polynomials. As formula: $P(x) = k p_1(x) p_2(x) \ldots p_3(x)$.  ...TBC...

\paragraph{Euclid's Lemma for Polynomials}
Let $p$ be irreducible. If $p|ab$ then $p|a$ or $p|b$. Here $p,a,b$ are all polynomials, i.e. $p = p(x), \ldots$.

%ABoAA, pg 254 - there's more there - theorem 3

% ToDo: aboaa p 262 - theo 5
% Any polynomial with integer coeffs that is reducible over Q is already reducible over z

\paragraph{Eisenstein Irreducibility Criterion}
Let $a(x) = a_0 + a_1 x + \ldots + a_n x^n$ be a polynomial with integer coefficients. Suppose there is a prime number $p$ which divides every coefficient $a_k$ except the leading coefficient $a_n$. Suppose that $p$ does not divide $a_n$ and $p^2$ does not divide $a_0$. Then $a(x)$ is irreducible over $\mathbb{Q}$.


% ABoAA pg 263
% Let  a(x) = a_0 + a_1 x + ... + a_n x^n  be apolynomial with integer coeffs. Suppose there is
% a prime number $p$ which divides every coeff of a(x) exept the leading coeff a_n. Suppose p does 
% not divide a_n and p^2 does not divide a_0. Then a(x) is irreducible over Q


%---------------------------------------------------------------------------------------------------
\subsubsection{Quotient Rings}
We will now consider rings of equivalence classes of polynomials where two polynomials are considered equivalent, when they produce the same remainder when we apply polynomial division to them with a given fixed polynomial $q(x)$. These rings are denoted in the form $\mathbb{F}[x] / q(x)$. For example, the notation $\mathbb{R}[x] / (x^2 + 1)$ means the following: We start with the ring of polynomial $\mathbb{R}[x]$ with real coefficients. We "mod" (apply a "modulo" operation) this ring by the fixed polynomial $q(x) = x^2 + 1$. This means we consider two elements of $\mathbb{R}[x]$ to be equivalent, we they produce the same polynomial $r(x)$ as remainder after a polynomial division by $q(x) = x^2 + 1$. The possible remainders are polynomials of degree at most $1$, that is, they can be written in the form $a_0 + a_1 x$. It will turn out that this specific example is actually isomorphic to the ring of complex numbers of the form $a + b \i$, which is indeed more than just a ring but even a field. If we would have taken $q(x) = x^2 - 1$ as modulus instead, we would get something isomorphic to the ring of hyperbolic (aka split-complex) numbers, which is not a field. If we would have taken $q(x) = x^2$, we would arrive at a ring that is isomorphic to the dual numbers which is not a field either.  VERIFY!...TBC...


% Try to implement it using rsModularInteger<rsPolynomial<rsFraction<int>>>

% hyperbolic numbers are not a field bcs (1+j) * (1-j) = 0
% dual numbers are not a field because eps * eps = 0


% R[x] / (x^2 + 1)
% Maybe use m(x)  to denote the modulus polynomial - or maybe use q(x)

% Number Systems Invented to Solve the Hardest Problem - History of Rings | Ring Theory E0
% https://www.youtube.com/watch?v=M-9_rZfVQVE

%===================================================================================================
\subsection{Galois Theory}
Galois theory is a sort of culmination point of abstract algebra where all the important algebraic structures that we have seen so far - groups, rings, fields, vector spaces - will be used. We will talk about \emph{rings} of polynomials whose coefficients come from a given \emph{field}. We will talk about field extensions and regard them as \emph{vector spaces} over their base field. We will talk about permutation \emph{groups} that operate on elements of these vector spaces by permuting the components of the vectors. ...TBC...VERIFY



%---------------------------------------------------------------------------------------------------
\subsubsection{The Abel-Ruffini Theorem} 

% https://en.wikipedia.org/wiki/Abel%E2%80%93Ruffini_theorem
% https://de.wikipedia.org/wiki/Satz_von_Abel-Ruffini

%===================================================================================================
\subsection{Multivariate Polynomials}


%---------------------------------------------------------------------------------------------------
\subsubsection{Symmetric Polynomials}
A symmetric polynomial is a multivariate polynomial whose value doesn't change when we permute the values of the input variables. For example $x^2 + y^2 - xy$ is symmetric\footnote{I assume that $xy = yx$, i.e. we deal with a commutative ring} but  $x^2 - y^2 + x^3 y$ is not. For the latter, you may find some special assignments for $x$ and $y$ which allows us to swap $x$ and $y$ without changing the value of the polynomial, but the non-change of value property will not hold in general for any assignment. In the former case, it doesn't matter at all what we assign to $x$ and $y$ - the output of the polynomial will always be insensitive to a swap. ...TBC...

% I think to find such assignments, we would have to solve the nonlinear system of equations
%  $x^2 - y^2 + x^3 y = c$,  $y^2 - x^2 + y^3 x = c$?

\paragraph{Elementary Symmetric Polynomials}
%When we have a polynomial given in product form $p(x) = (x-x_1)(x-x_2)(x-x_3)$ and expand this, we get $x^3 - (x_1 + x_2 + x_3) x^2 + (x_1 x_2 + x_1 x_3 + x_2 x_3) x^1 - x_1 x_2 x_3 x^0$. VERIFY

% The coefficients of p, expressed in terms of its roots, are given by the elementary symmetric polynomials in the roots

%Prelude to Galois Theory: Exploring Symmetric Polynomials
%https://www.youtube.com/watch?v=3imeTgGBaLc  by Martin Trifonov

%They arise

% x_1 + x_2 + x_3 + x_4, x_1 x_2 + x_1 x_3 + x_1 x_4 + x_2 x_3 + x_2 x_4 + x_3 x_4,
% x_1 x_2 x_3 + x_1 x_2 x_4 + x_1 x_3 x_4 + x_2 x_3 x_4, x_1 x_2 x_3 x_4

%\begin{eqnarray}
%e_1(x_1,x_2,\ldots,x_n) &=& \sum_{i=1}^n x_i \\
%e_2(x_1,x_2,\ldots,x_n) &=& \sum_{i=1}^n \sum_{j=i}^n  x_i x_j \\
%e_3(x_1,x_2,\ldots,x_n) &=& \sum_{i=1}^n \sum_{j=i}^n \sum_{k=j}^n  x_i x_j x_k
%\end{eqnarray}
% No - that's wrong!



%===================================================================================================
\subsection{Rational Functions}




\begin{comment}

-maybe put the section between Calculus and Multilinear Algebra

ToDo:
-Move some of the content from the section about functions over to here
-Base change (maybe under some sort of transformatiuons) where we can also list scaling or shifting
 the argument, raising the argument to a power
-Root finding, Viete's theorem, formulas for quadratic, cubic and quartic case, unsolvability of
 quintic case, companion matrix
-Construction of (sets of) polynomials with certain properties
 -Interpolation polynomials
 -Approximation polynomials
 -Orthogonal polynomials
-Multivariate polynomials
 -elementary symmetric polynomials 
-Explain principal and primitive roots
 https://en.wikipedia.org/wiki/Principal_root_of_unity
 https://en.wikipedia.org/wiki/Root_of_unity#primitive
 https://en.wiktionary.org/wiki/principal_root

% Interpretation of polynomials as:
% -functions
% -equations to be solved
% -formal expressions
%  -this view generalizes to power series even when they do not converge

-Two polynomials seen as formal expressions are equal iff all their coeffs are equal. 
 When viewed as function, it may happen that two polynomials with different coefficients
 give rise to the same function. That happens when the domain and range of the polynomial
 is a finite ring. For such finite rings, there is only a finite number of functions from
 the ring to itself but there is an infinite number of different coefficient sequences, so
 some of these sequences must encode the same function.
 
-As formal expressions, we can define an algebra on the set of polynomials itself: 
 polynomials can be added and multiplied by one another. There is even a notion of 
 polynomial division with remainder
 
ACRS, pg 17 ff 
 
 ACRS, pg 34: field of fractions ..done in the fields section
 
-Make subsections for:
 -Unsolvability of the Quintic (theorem of Abel-Ruffini)
 -Sylvester matrix, resultant, elementary symmetric polynomials
 -discriminant, which is essentially the resultant of a polynomial and its derivative, see
  https://en.wikipedia.org/wiki/Resultant


ABoAA
-pg 261: if s/t is a root of a(x) in Z[x] the s|a_0 and t|a_n, i.e. a_0 divides s and a_n divides t

https://en.wikipedia.org/wiki/Resultant
https://en.wikipedia.org/wiki/Elimination_theory
https://en.wikipedia.org/wiki/Gr%C3%B6bner_basis
https://en.wikipedia.org/wiki/System_of_polynomial_equations
https://en.wikipedia.org/wiki/B%C3%A9zout%27s_theorem
https://en.wikipedia.org/wiki/Multi-homogeneous_B%C3%A9zout_theorem

https://en.wikipedia.org/wiki/Reciprocal_polynomial

https://en.wikipedia.org/wiki/Polynomial_greatest_common_divisor#Euclid's_algorithm
https://en.wikipedia.org/wiki/Polynomial_greatest_common_divisor#Subresultant_pseudo-remainder_sequence
 
https://en.wikipedia.org/wiki/Sturm%27s_theorem 
https://en.wikipedia.org/wiki/Sturm%27s_theorem#Generalized_Sturm_chains




Der Fundamentalsatz über symmetrische Polynome:
https://www.youtube.com/watch?v=_qnj3Iz6vL4

Every symmetric polynomial can be expressed as algebraic combination of the elementary symmetric polynomials in a unique way. An "algebraic combination" is basically a polynomial. So we can construct *every* symmetric polynomial as polynomial of polynomials, namely the *elementary* symmetric polynomials. In Mathematica, the function SymmetricReduction can do this - takes as input a symmetric polynomial and expresses it as polynomial of elementary symmetric polynomials.

Viete's theorem: the coefficient a_k of any polynomial p(x) in x of degree n with n roots x_1,...,x_n is given by the elementary symmetric polynomial e_{n,k} (or is it e_{n,n-k} ?) in the roots x_1, ..., x_n. But there's also a sign alternation, so the polynomial must be prepended by (-1)^k or (-1)^(n-k) or something. So, the theorem tells us how to express the coeffs of p(x) in terms of elementary symmetric polynomials in x_1...x_n - the coeffs are themselves expressed as multivariate polynomials in the roots.

...could this theorem help to figure out what happens to the roots when we add two polynomials? The sum polynomial will be obtained by summing the coeffs - therefore, we sum these polynomials in the roots. Maybe try it with a sum of two quadratic polynomials for a simple example.


Resultante und Sylvester-Matrix: Die algebraischen Zahlen bilden einen Körper.
https://www.youtube.com/watch?v=dC6dxFhzKoc

https://www.youtube.com/watch?v=6MgN4B_PpUo&list=PLb0zKSynM2PCrgebQsfrzEsUIuA0I_wdG&index=35
"Klausurlemma". If a polynomial with integer coefficients has a rational root p/q, then p divides the constant coefficient and q divides the leading coefficient. It follows that if a monic polynomial has a rational root, then that root must be an integer. Or: Roots of monic polynomials are either integer or irrational. Or: If a polynomial is irreducible over Z, then it is also irreducible over Q. If f = g*h and all coeffs of f are divisible by some prime p, then all coeffs of g are divisible by p or all coeffs of h are divisible by p.


https://mathworld.wolfram.com/PolynomialDiscriminant.html
https://math.stackexchange.com/questions/2303465/why-discriminant-of-a-polynomial-is-so-special

 
-make a similar section on 
 -Differential Algebra 
 -Lie Algebras
 ...but maybe the algebra of polynomials should not be a section but rather a chapter in its own
 right - maybe after the Linear Algebra chapter. The topic might be too big for a section and
 Abstract Algebra might be too late to cover that material. 
 
Quintic equations:
https://www.youtube.com/watch?v=t4u_lwONAdc    I can solve any quintic equation!!  https://de.wikipedia.org/wiki/Bringsches_Radikal
https://en.wikipedia.org/wiki/Bring_radical
 
How to Get to Galois Theory Naturally
https://www.youtube.com/watch?v=lkvkUT3Qdw4 
-Every polynomial has a symmetry group
-The structure of this group determines, if the polynomial is solvable by radicals

For Galois theory we need groups, rings and fields 
Notation:

F              FIELD of coefficients (ex: real numbers R)
F[x]           RING of polynomials over F
p(x)           element of F[x], polynomial in x with coeffs from F, ex: p(x) = x^2 + 1
r1,r2,...      roots of a polynomial p
F(c1,c2,..)    F with the roots of p adjoined: root-field/splitting field of p(x) over F
               ex: complex numbers C. (ABoAA pg 313). It is a VECTOR SPACE over F
F[x] / p(x)    Quotient ring of F[x] wrt p(x) ??? see below      
Gal(K:F)       Galois GROUP of K over F where: F: field of coeffs, K: root-field of p(x), p(x):
               a given polynomial (suppressed in the notation - must be clear from context). It's 
               the group of all automorphisms of K which fix F. (ABoAA pg 325)
[K:F] = n      "the degree of the field extension K over F is n"


ex 1:
F = Q, p(x) = x^2 - 2, r1 = sqrt(2), r2 = -sqrt(2), K = Q(sqrt(2))
Gal(K:F) = ...

ex 2:
F = R, p(x) = x^2 + 1, r1 = i, r2 = -1, K = R(sqrt(-1)) = C
Gal(K:F) = identity + complex cojugation? 

Note to F[x] / p(x): Normally, we build quotient sets wrt to an equivalence relation - how is a
polynomial an quivalence relation? Maybe it defines one by saying: all polynomials that have the 
same roots as p are equivalent to p? Nah - that just filters out a subset of R[x]. I think, it's
about the remainders from polynomail division: all polynomials if F[x] that have the same remainder
after dividing out p(x) go into one equivalence class - or something

Note to Gal(K:F): I think the Galois group is a group of permutations of the root 

ABoAA: the important stuff starts at pg 270
274:  F(c) ~= F[x] / <p(x)>. F-adjoin-c is isomorphic to the quotient ring of polynomials over F
      by <p(x)> which is the set of multiples o p(x) (see pg 273). c is a root of p and p is the
      minimal polynomial of c...I guess. I think, by "multiples" it is meant that the other factor
      can also be a polynomial - see pg 251.

https://math.stackexchange.com/questions/395028/how-to-deal-with-polynomial-quotient-rings
https://sites.millersville.edu/bikenaga/abstract-algebra-1/quotient-rings-of-polynomial-rings/quotient-rings-of-polynomial-rings.html


It Took 2137 Years to Solve This
https://www.youtube.com/watch?v=EX7U0DGBmbM
-about the impossibility of certain geometric constructions

Welche Polynome Computer am liebsten mögen (Algebra mit Polynomen)
https://www.youtube.com/watch?v=1PLzxn1Tfb0

Fehlerkorrektur mit Reed-Solomon-Codes (CD, DVD, Blu-ray, DSL, DVB, RAID, QR-Codes, etc.)
https://www.youtube.com/watch?v=uOLW43OIZJ0




Cyclotomic Polynomials
https://www.youtube.com/playlist?list=PLRcOL8MUr-pdt34AEueXl7aCoE7v-Mya_
https://www.youtube.com/watch?v=b3G5Gi4suHs  Visualizing Cyclotomic Polynomials
https://www.youtube.com/watch?v=D3KYA8wVWw0  Visualizing Cyclotomics Part 2 - Any Two Primes
...
-They have the primitive(!) n-th roots of unity as roots


Prime Polynomials
https://www.youtube.com/watch?v=XGQ_GoV71RE


New Maths Discovery! (All polynomial equations solved fast)
https://www.youtube.com/watch?v=9cnVW1sy-G4

Paper with Dean Rubine on Solving Polynomial Equations and the Geode (I) | N J Wildberger
https://www.youtube.com/watch?v=oIHd3zDDDCE
https://www.tandfonline.com/doi/full/10.1080/00029890.2025.2460966


History of Mathematics: Classical algebra: 19th-century beginnings of modern algebra. 3rd Yr Lecture
https://www.youtube.com/watch?v=2wzbbwxif_M

\end{comment}