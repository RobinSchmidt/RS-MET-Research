\section{Ring Theory}
After having seen groups, the next idea to explore is that of a ring. We assume to have a commutative group $(G, +)$ as a starting point. Note that here we use $+$ to denote our commutative group operation instead of the previous usage of $\cdot$ or juxtaposition within group theory. That is because in ring theory, the dot or juxtaposition will be needed for a second operation that we now want to introduce. We have seen that the integers form a group under addition. But we can also multiply integers. We now want to capture the behavior of integers under addition and multiplication in an abstract way. To this end, we assume that we start with a commutative group and we will call the group operation "addition" from now on. We assume furthermore that we have defined a second operation between elements of the group which we intend to be our abstract multiplication. Abstractly, a ring can be defined as a triple $(R,+,\cdot)$ consisting of a set $R$ with two binary operations between elements of the set such that the following rules hold true, which we shall henceforth call ring axioms:
\begin{itemize}
\item $(R,+)$ forms a commutative group
\item $(R,\cdot)$ forms a semigroup (term is explained below)
\item Multiplication distributes over addition: 
$\forall a,b,c \in R: \;  
a \cdot (b + c) = a \cdot b + a \cdot c, \; 
(b + c) \cdot a = b \cdot a + c \cdot a$
\end{itemize}
where the most interesting thing is the last law. This law is called the distributivity law and it establishes a connection between the two operations $+$ and $\cdot$. Note that we need two versions (left and right) of the distributive law because we do not in general require multiplication to be commutative. We do require commutativity for addition, though. In writing down the distributive laws, I also implicitly assumed that $\cdot$ takes precedence over $+$ such that no parentheses are needed for the right hand sides.

\paragraph{Notation and Terminology}
Here, a \emph{semigroup} is a structure like a group but we do not necessarily require the existence of inverse elements and not even a neutral element. We now have not one but two operations, so we need to adapt our notation and terminology a bit to accommodate for that. We have called the operations "addition" and "multiplication" but it should be understood that we mean abstractions of these operations. We will denote the additive neutral element, formerly known as $e$, from now on as $0$ and we will also call it the \emph{additive identity}. If multiplication happens to commutative as well, we call our ring a commutative ring. The study of such commutative rings is called commutative algebra. If the ring is commutative, one of the distributivity laws is sufficient - the other one then follows from it and commutativity. For multiplication, we do not require the existence of a neutral element because we require our structure to be only a semigroup with respect to multiplication. By the way, a semigroup in which a neutral element exists is called a monoid. A ring in which a multiplicative neutral element exists, i.e. one in which the "semigroup" requirement is upgraded to a "monoid" requirement, is called a ring with unity or \emph{unital ring}. The multiplicative neutral element is then denoted by $1$ and also called the \emph{multiplicative identity}. If we just say "identity" without a qualifier, we mean the multiplicative identity. We may also call the additive inverses \emph{negatives} and the multiplicative inverses \emph{reciprocals}. We don't require existence of reciprocals, though. If, for some element $a$, a reciprocal does exist, we call $a$ a \emph{unit}. So, the units of a ring are its (multiplicatively) invertible elements. A ring in which no (multiplicative) identity exists is sometimes called a \emph{rng}. This is no typo - it's a "ring" without the "i" and pronounced as "rung".  A ring without negatives is called a \emph{semiring} or sometimes \emph{rig}. Looks like they just leave out the first letters i and n of the name of the missing elements from the word ring. A ring without i has no identity, a ring without n has no negatives. This is just an aside - we won't use that terminology here.

\subsection{Motivation}
As said, in a general ring, we do not require the existence of multiplicative inverses for all of its elements. The absence of reciprocals at least for some elements introduces ideas like divisibility, prime numbers, factorizations, etc. Such concepts are the heart and soul of number theory. A number theoretic question of great interest was the question whether there exist integer solutions $(a,b,c)$ to the equation $a^n + b^n = c^n$ for some integer power $n$. For $n=2$, infinitely many such integer solutions exist. They are called Pythagorean triples because for $n=2$, the equation becomes $a^2 + b^2 = c^2$ such that the triples that satisfy the equation can be interpreted geometrically as side lengths of a right triangle. We can find all Pythagorean triples by the following procedure: Pick 3 integers $k,n,m$ such that $k \geq 1$, $n > m \geq 1$, $\gcd(n,m) = 1$, $n + m$ odd, which means that $n$ and $m$ shall not be both even or both odd. Then, all the triples are given by $(a,b,c) = k (n^2 - m^2, 2 n m, n^2 + m^2)$. This generates all the Pythagorean triples without duplicates. You may swap $a$ and $b$ in the solutions to get yet more solutions which are not really considered to be different. With $k=1$, you get the so called simple solutions which are not just a multiple of another solution. For $n > 2$, the answer turns out that there are no integer solutions to $a^n + b^n = c^n$. This is called "Fermat's Last Theorem" even though Fermat himself did not provide a proof. He famously scribbled into the margin of some math book that he has a proof but the margin is too small to contain it. As history has it, the quest to come up with some proof took mathematicians centuries and it is likely that Fermat's supposed proof wasn't valid because the techniques used in modern proofs were not known to Fermat. The first proof accepted by the mathematical community was provided in 1994 by Andrew Wiles, so it should now actually be called "Wiles Theorem" but this story is such a famous bit of math history that the old name will probably stick. In order to find a general proof, many more recently developed mathematical tools were needed and ring theory is one of them [VERIFY] and in particular the theory of cyclotomic integers [VERIFY] .

% https://en.wikipedia.org/wiki/Fermat%27s_Last_Theorem
% https://en.wikipedia.org/wiki/Proof_of_Fermat%27s_Last_Theorem_for_specific_exponents
% https://en.wikipedia.org/wiki/Wiles%27s_proof_of_Fermat%27s_Last_Theorem

% units are neither prime nor composite? In Z, the units are +1 and -1. In Z[i], we have 1,-1,i,-i. Maybe the motivation form this term is the imaginary unit? we could call -1 the "negative unit". We have 4 kinds of numbers: primes, composites, units and zero.

% numbers that result from another number by multiplication by a unit are asscociates of each other

% explain change of notation for additive inverses as -a

% https://en.wikipedia.org/wiki/Unit_(ring_theory)
% https://en.wikipedia.org/wiki/Rng_(algebra)

% https://en.wikipedia.org/wiki/Prime_element
% https://en.wikipedia.org/wiki/Prime_ideal
% https://en.wikipedia.org/wiki/Primary_ideal
% https://en.wikipedia.org/wiki/Unique_factorization_domain


\subsection{Adjoining Elements}
An important idea in ring theory is the idea of starting with a given ring and \emph{adjoining} some new element to it. Assume that we start with $\mathcal{Z} = (\mathbb{Z},+,\cdot)$. We now want to add some new object that is not yet an element of the underlying set $\mathbb{Z}$ to that underlying set, i.e. we want to augment the underlying set by some new element. That process is called adjoining. 

\subsubsection{The Polynomial Ring $\mathbb{Z}[x]$}
Let's call the new element to be adjoined $x$. We don't say anything about what that mysterious new object $x$ is supposed to be. We treat it like a symbolic variable. Of course, just adding some single element to the underlying set destroys the fulfillment of the ring requirements. To maintain closure under addition, we must also add $x+1$, $x-1$, $x+2$, $x-2$, $x+3$ and so on. To maintain closure under multiplication, we need to add $2 x$, $-2 x$, $3 x$, etc. But that's not yet all that have to add - not even close. We also need to add $x^2, x^3, x^4, \ldots$ and of course also $3 x^5 - 23 x^2 + 42$ and many more elements. By adjoining a single new object $x$, we are forced by the closure requirements to also add all elements of the general form $a_0 + a_1 x + a_2 x^2 + a_3 x^3 + \ldots$ where $a_0, a_1, a_2, a_3, \ldots \in \mathbb{Z}$. We obtain the set of all polynomials in some symbolic variable $x$ with coefficients from $\mathbb{Z}$. The result is called the ring "$\mathbb{Z}$-adjoin-$x$" and denoted by $\mathbb{Z}[x]$.

\subsubsection{The Gaussian Integers $\mathbb{Z}[\i]$}
If instead of adjoining an unspecified symbolic variable $x$, we adjoin the symbol $\i$ for the imaginary unit with the property $\i^2 = -1$, all polynomials of degree higher than 1 will boil down to a linear polynomial of the general form $a_0 + a_1 \i$ where $a_0,a_1 \in \mathbb{Z}$. This happens because $\i^2$ and higher powers of $\i$ can be simplified into $\pm 1$ and $\pm \i$. Consider the polynomial $2 - 3 \i + 4 \i^2 - 5 \i^3 + 6 \i^4 - 7 \i^5$. Wouldn't we have any rule, we would just keep it as a polynomial in $\i$ and we would be working with a polynomial ring just that we would now call our variable $\i$ instead of $x$. But we do have the rule $\i^2 = -1$ and when we apply it, the polynomial simplifies to $4 - 5 \i$. This will always happen. Every element of our new ring can be expressed in the form $a + b \i$ where $a,b \in \mathbb{Z}$. We denote that ring by $\mathbb{Z}[\i]$ and call it "$\mathbb{Z}$-adjoin-i" or the "Gaussian integers" after Carl Friedrich Gauss who worked a lot with numbers of that kind in his number-theoretic investigations. The picture we should have in mind for these Gaussian integers is the lattice of points of complex numbers whose real and imaginary part are integers, i.e. a lattice of points in the complex plane.


\subsection{Structure of Rings}

\subsubsection{Subrings}
We have produced bigger rings from given ones by adjoining elements. We may also take a given ring $R$ and extract a smaller ring from it which we call a \emph{subring} $S$. A subring is defined analogously to a subgroup: we take a subset of the underlying set. If that subset together with the two operations forms itself a ring, then we have a subring. Multiplying any element $a \in S$ from the subring with any element $b \in S$ from the subring must yield again an element $c \in S$ from the subring. For example, the set of even integers $2 \mathbb{Z}$ forms a subring of $\mathbb{Z}$. Some authors include the existence of a multiplicative identity into the requirements for a ring. With such a definition, $2 \mathbb{Z}$ would not qualify as subring because it has no multiplicative identity. Then, one would have to resort to call it a subrng - a subring without the i [VERIFY!]. Fortunately, I have used a definition that lets us avoid that awkward language.

% https://en.wikipedia.org/wiki/Subring

\subsubsection{Ideals}
The notion of an \emph{ideal} is more restrictive than that of a subring. Among other requirements, a subring $S$ must be closed under multiplication. For a subset of the underlying set of a ring $R$ to be an ideal $I$ of $R$, the closure under multiplication requirement is tightened: It must be the case that the products of elements $a \in I$ from the ideal $I$ with elements $b \in R$ from the original ring $R$ must again produce elements $c \in I$ from the ideal. If the ring is not commutative, we must distinguish between left, right and two-sided ideals. For these, we require:

\medskip
\begin{tabular}{l l}
Left ideal:      & $\forall a \in I, r \in R: \; r a \in I$  \\
Right ideal:     & $\forall a \in I, r \in R: \; a r \in I$  \\
Two-sided ideal: & $\forall a \in I, r \in R: \; r a \in I \wedge a r \in I$  \\
\end{tabular}
\medskip

The "left" and "right" in this terminology refers to where the factor $r$ from the ring occurs in the product. As an example, consider again the set of even numbers $2 \mathbb{Z}$. We already know that it is a subring of $\mathbb{Z}$ because any product of any pair of even numbers is again an even number. But is it also an ideal? Yes, because it also satisfies the stronger requirement that the product of any even number with any integer number whatsoever gives again an even number. 
%...TBC...

% https://sites.millersville.edu/bikenaga/abstract-algebra-1/ideals-and-subrings/ideals-and-subrings.pdf
% https://en.wikipedia.org/wiki/Ideal_(ring_theory)
% https://www.math.uci.edu/~ndonalds/math120b/3ideals.pdf

% under which conditions is a subring an ideal?

\subsubsection{Product Rings}


\subsubsection{Quotient Rings}
In ring theory, ideals play a role analogous to that of the normal subgroups in group theory. There, we used normal subgroups to create quotient groups which were sets of ...TBC...

% google: factor ring
% The quotient of a ring R, by a two-sided ideal I, is denoted by R / I . The result is called a quotient or factor ring. It is the set of additive cosets { a + I | a ∈ I } , with addition and multiplication defined as ( a + I ) + ( b + I ) = a + b + I , and ( a + I ) ( b + I ) = a b + I .


\subsubsection{Units and Association}
Every natural number greater than one has a unique factorization in terms of prime numbers. Unique up to an arbitrary ordering of the factors, that is. The number one has be deliberately left out from the definition of what a prime number is because if we would include it the factorization in terms of primes wouldn't be unique anymore because we could multiply our number by $1^n$ for any $n$. For example $6 = 2 \cdot 3 \cdot 1^n$ for any $n$. We could repair this by stating the theorem as "in terms of primes other than 1" but that would be ugly. The number 1 is considered to be in a class of its own - it's neither a prime nor a composite number. The same is true for zero, by the way - it's another kind of very special number. Within the natural numbers, 1 is the only number that has a multiplicative inverse. This inverse is 1 itself, i.e. 1 is its own inverse. In the integers, we have two numbers that have an inverse - namely $1$ and $-1$. They also happen to be their own inverses, but that's not the important point here. Important is only that they have inverses. If we want to say something similar about unique factorizations for integers, we should exclude numbers with inverses from those factorizations because we could always multiply by an arbitrary power of such a number provided that we also include the same power of its inverse. For example $6 = 2 \cdot 3 \cdot (-1)^n$ for any even $n$. 


% bring example of inverses and associates in the gaussian integers

%It turns out that existence of inverses destroy any hope of a unique factorization of some number because

%If we want to say something similar about unique factorizations for integers, we should 

%and more generally rings, 


%In a ring, some elements may have multiplicative inverses, aka reciprocal, and some don't. Those which do have reciprocals are called units.
% maybe name it Units and Associates
% give examples

% we must exclude units from the factorization because we could multiply by a unit and its inverse as often was we wish, so there is no 

% we'll get a factorization unique up to pulling out units

% https://en.wikipedia.org/wiki/Unit_(ring_theory)
% https://en.wikipedia.org/wiki/Unit_(ring_theory)#Associatedness

\subsubsection{Prime Elements and Irreducibility}


% maybe name it primes and irreducible elements or prime elements and irreducibility



% in general, there is a distinctions between primality and irreducibility



\subsection{More Examples of Rings}
We have seen some basic rings like $\mathbb{Z}$ which can be regarded as the prototypical example for a ring that was used as model on which the more abstract notion of a ring is based and we have seen the rings $\mathbb{Z}[x]$ and $\mathbb{Z}[i]$ which where obtained by adjoining certain kinds of new elements. Here, we will briefly mention some more examples of rings.

\subsubsection{Numbers}
ToDo: integers $\mathbb{Z}$, Gaussian Integers $\mathbb{Z}[i]$, modular integers $\mathbb{Z}_p$

\subsubsection{Polynomial Rings}
ToDo: polynomial rings over various base fields, maybe also multivariate polynomials: $\mathbb{Q}[x]$, $\mathbb{R}[x]$, $\mathbb{C}[z]$, $\mathbb{Z}_p[x]$, $\mathbb{R}[x,y]$, rings of power series like
$\mathbb{R} \llbracket x \rrbracket$

% https://math.stackexchange.com/questions/190073/what-is-mathbb-zt-what-are-the-double-brackets
%$\mathbb{R} \llbracket x \rrbracket$
%\llbracket
% Z[\sqrt(2)], Z[\sqrt{-3}] (historic reasons)

\subsubsection{The Dyadic Rationals $\mathbb{Z}[1/2]$}

\subsubsection{The Cyclotomic Integers $\mathbb{Z}[\zeta]$}
Like we did when constructing $\mathbb{Z}[x]$ and $\mathbb{Z}[\i]$, we again start with the ring $\mathbb{Z}$. Now we want to adjoin a special number $\zeta = e^{2 \i \pi / p}$ for some fixed number $p$ that we choose once and for all. For $p=2$, we get $\zeta = e^{\i \pi} = -1$ which satisfies $\zeta^2 = 1$. In general, we will have $\zeta^p = 1$ which is called the \emph{cyclotomic equation}. When we adjoin our $\zeta$ to $\mathbb{Z}$, we will obtain the \emph{cyclotomic integers} [VERIFY!!]. 

%The Gaussian integers can be considered a sort of special case for $p=4$. ...but in general, we are interested in cases where $p$ is a prime number (maybe p should be either 4 or prime)

...TBC...

% In general, we are interested in fields 

% If it is not clear what prime p we refer to, the zeta may be indexed by p
%that has the property

% https://mathworld.wolfram.com/CyclotomicInteger.html
% https://mathworld.wolfram.com/CyclotomicEquation.html
% https://mathworld.wolfram.com/deMoivreNumber.html
% http://fermatslasttheorem.blogspot.com/2006/02/cyclotomic-integers.html

% what if we use $\zeta = e^{\pi \i / n}$? Then we obtain the Gaussian integers for n = 2. It seem like the Gaussian integers fit into the framework of cyclotomic integers for p=4, but that is not allowed because 4 is not a prime

%, we must also add a whole bunch of other new elements, for example $x+1$, $x-1$, $x+2$, $x-2$, $2 x$, $-2 x$, $42 x - 71$, $23 x^7 + 5 x^4 - 2$, etc.

%postulate that some other object that is not element, let's call it $x$,

%In the example rings of the Gaussian integers $\mathbb{Z}[i]$ and the polynomial rings like $\mathbb{R}[x]$, we added a single 

% https://en.wikipedia.org/wiki/Cyclotomic_field

\subsubsection{Classification of Finite Rings}
Finite rings are more complicated objects than finite groups and have more structure due to the presence of two binary operations that interact with each other. One might be inclined to expect that the task of classification of finite rings is more complicated than in the case of finite groups. But the opposite is true. This is plausible because when we have more constraints, we will actually find less mathematical structures that satisfy all these constraints. It's actually quite simple to characterize the set of all possible finite rings. Every finite \emph{simple} ring is isomorphic to $\mathbb{F}_q^{n \times n}$ which is the ring of all $n \times n$ matrices over a given finite field $\mathbb{F}_q$ of order $q$. Finite fields will be introduced in the next section. The term "simple" deserves some attention ...TBC...what is a simple ring and how can we construct the non-simple ones?

%https://en.wikipedia.org/wiki/Finite_ring
%https://en.wikipedia.org/wiki/Simple_ring


\begin{comment}


Examples of rings that are not fields in order of importance:
-Integers
-Matrices with matrix addition and multiplication. The subset of invertible matrices 
 forms a field (i think)
-Polynomials over a field
-Modular integers with modular addition and multiplication. If the modulus isn't prime, it's
 only a ring. If it is prime, it is a field.
-Gaussian integers - obtained by adjoining i to Z

ToDo:
-Ideals: 
 -generalize the notion of "multiple of"
 -is a set that splits the ring into equivalence classes (like in modular integers)
 -is itself a subring of the ring, i.e. ring (without identity - we don't call it rng bcas we 
  don't demand rings to have an identity)
-Quotient rings, factor rings, etc.

https://en.wikipedia.org/wiki/Quotient_ring
https://mathworld.wolfram.com/QuotientRing.html


What about geometric examples? For group theory, we have a lot of geometric examples - reflectiosn, translations, rotations. I think, for geometric examples of rings, we need to look into algebraic geometry?

What is algebraic geometry? [by Aleph 0]
https://www.youtube.com/watch?v=MflpyJwhMhQ

Was sind Ringe? Was sind Körper? (Algebra) [Weitz]
https://www.youtube.com/watch?v=MAcdesa9RqA

Number Systems Invented to Solve the Hardest Problem - History of Rings | Ring Theory E0
https://www.youtube.com/watch?v=M-9_rZfVQVE&list=PL6VQBwZazp7YSRz9jluchQJSDRv_yQKeZ


https://en.wikipedia.org/wiki/Ring_(mathematics)
https://de.wikipedia.org/wiki/Ring_(Algebra)
https://en.wikipedia.org/wiki/Unit_(ring_theory)




\end{comment}