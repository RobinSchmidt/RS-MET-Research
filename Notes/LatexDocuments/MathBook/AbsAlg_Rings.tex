\section{Ring Theory}
After having seen groups, the next idea to explore is that of a ring. We assume to have a commutative group $(G, +)$ as a starting point. Note that here we use $+$ to denote our commutative group operation instead of the previous usage of $\cdot$ or juxtaposition within group theory. That is because in ring theory, the dot or juxtaposition will be needed for a second operation that we now want to introduce. We have seen that the integers form a group under addition. But we can also multiply integers. We now want to capture the behavior of integers under addition and multiplication in an abstract way. To this end, we assume that we start with a commutative group and we will call the group operation "addition" from now on. We assume furthermore that we have defined a second operation between elements of the group which we intend to be our abstract "multiplication". Abstractly, a ring can be defined as a triple $(R,+,\cdot)$ consisting of a set $R$ with two binary operations between elements of the set such that the following rules hold true, which we shall henceforth call ring axioms:
\begin{itemize}
\item $(R,+)$ forms a commutative group
\item $(R,\cdot)$ forms a semigroup (term is explained below)
\item Multiplication distributes over addition: 
$\forall a,b,c \in R: \;  
a \cdot (b + c) = a \cdot b + a \cdot c, \; 
(b + c) \cdot a = b \cdot a + c \cdot a$
\end{itemize}
where the most interesting thing is the last law. This law is called the distributivity law and it establishes a connection between the two operations $+$ and $\cdot$. Note that we need two versions (left and right) of the distributive law because we do not in general require multiplication to be commutative. We do require commutativity for addition, though. In writing down the distributive laws, I also implicitly assumed that $\cdot$ takes precedence over $+$ such that no parentheses are needed for the right hand sides.

% Z is the prototype of a ring. The other important example is the ring of polynomials over a given field.

\paragraph{Notation and Terminology}
Here, a \emph{semigroup} is a structure like a group but we do not necessarily require the existence of inverse elements and not even a neutral element. We now have not one but two operations, so we need to adapt our notation and terminology a bit to accommodate for that. We have called the operations "addition" and "multiplication" but it should be understood that we mean abstractions of these operations. We will denote the additive neutral element, formerly known as $e$, from now on as $0$ and we will also call it the \emph{additive identity}. If multiplication happens to commutative as well, we call our ring a commutative ring. The study of such commutative rings is called commutative algebra. If the ring is commutative, one of the distributivity laws is sufficient - the other one then follows from it and commutativity. For multiplication, we do not require the existence of a neutral element because we require our structure to be only a semigroup with respect to multiplication. By the way, a semigroup in which a neutral element exists is called a monoid. A ring in which a multiplicative neutral element exists, i.e. one in which the "semigroup" requirement is upgraded to a "monoid" requirement, is called a ring with unity or \emph{unital ring}. The multiplicative neutral element is then denoted by $1$ and also called the \emph{multiplicative identity}. If we just say "identity" without a qualifier, we mean the multiplicative identity. We may also call the additive inverses \emph{negatives} and the multiplicative inverses \emph{reciprocals}. We don't require existence of reciprocals, though. If, for some element $a$, a reciprocal does exist, we call $a$ a \emph{unit}. So, the units of a ring are its (multiplicatively) invertible elements. A ring in which no (multiplicative) identity exists is sometimes called a \emph{rng}. This is no typo - it's a "ring" without the "i" and pronounced as "rung".  A ring without negatives is called a \emph{semiring} or sometimes \emph{rig}. Looks like they just leave out the first letters i and n of the name of the missing elements from the word ring. A ring without i has no identity, a ring without n has no negatives. This is just an aside - we won't use that terminology here.

% Absorbing element: an element a for which a b = b a = a is alled an absorbing element. Example: 0 is the absorbing element of multiplication. In some contexts, infinity can alse be an absorbing element wrt to addition and multiplication ...but that's a bit more complicated when we also consider -infinity and zero...

%===================================================================================================
\subsection{Motivation}
As said, in a general ring, we do not require the existence of multiplicative inverses for all of its elements. The absence of reciprocals at least for some elements introduces ideas like divisibility, factorizations, prime numbers, etc. Such concepts are the heart and soul of number theory. A number theoretic question of great interest was the question whether there exist integer solutions $(a,b,c)$ to the equation $a^n + b^n = c^n$ for some integer power $n$. For $n=2$, infinitely many such integer solutions exist. They are called Pythagorean triples because for $n=2$, the equation becomes $a^2 + b^2 = c^2$ such that the triples that satisfy the equation can be interpreted geometrically as side lengths of a right triangle. We can find all Pythagorean triples by the following procedure: Pick 3 integers $k,n,m$ such that $k \geq 1$, $n > m \geq 1$, $\gcd(n,m) = 1$, $n + m$ odd, which means that $n$ and $m$ shall not be both even or both odd. Then, all the triples are given by $(a,b,c) = k (n^2 - m^2, 2 n m, n^2 + m^2)$. This generates all the Pythagorean triples without duplicates. You may swap $a$ and $b$ in the solutions to get yet more solutions which are not really considered to be different. With $k=1$, you get the so called simple solutions which are not just a multiple of another solution. For $n > 2$, the answer turns out that there are no integer solutions to $a^n + b^n = c^n$. This is called "Fermat's Last Theorem" even though Fermat himself did not provide a proof. He famously scribbled into the margin of some math book that he has a proof but the margin is too small to contain it. As history has it, the quest to come up with some proof took mathematicians centuries and it is likely that Fermat's supposed proof wasn't valid because the techniques used in modern proofs were not known to Fermat. The first proof accepted by the mathematical community was provided in 1994 by Andrew Wiles, so it should now actually be called "Wiles Theorem" but this story is such a famous bit of math history that the old name will probably stick. In order to find a general proof, many more recently developed mathematical tools were needed and ring theory is one of them. 

% Ring theory is also very important in the context of polynomials. Polynomial rings are the second prototypical example of rings.

% Maybe one could say that ring theory is mostly concerned with questions about divisibility and factorizations and that's why it features prominently in number theory?

% https://en.wikipedia.org/wiki/Fermat%27s_Last_Theorem
% https://en.wikipedia.org/wiki/Proof_of_Fermat%27s_Last_Theorem_for_specific_exponents
% https://en.wikipedia.org/wiki/Wiles%27s_proof_of_Fermat%27s_Last_Theorem

% units are neither prime nor composite? In Z, the units are +1 and -1. In Z[i], we have 1,-1,i,-i. Maybe the motivation form this term is the imaginary unit? we could call -1 the "negative unit". We have 4 kinds of numbers: primes, composites, units and zero.

% numbers that result from another number by multiplication by a unit are asscociates of each other

% explain change of notation for additive inverses as -a

% https://en.wikipedia.org/wiki/Unit_(ring_theory)
% https://en.wikipedia.org/wiki/Rng_(algebra)

% https://en.wikipedia.org/wiki/Prime_element
% https://en.wikipedia.org/wiki/Prime_ideal
% https://en.wikipedia.org/wiki/Primary_ideal
% https://en.wikipedia.org/wiki/Unique_factorization_domain


%\subsection{Basic Theorems for Rings}
% 0 \cdot a = 0
% http://weitz.de/files/skript.pdf
% give a similar list of elementary rules like for group theory
% ...but maybe that would be redundant because we alraedy know that rings form a group or
% semigroup wrt both operations. The only interesting new thing could be general rules for
% interactions of + and *. We have 0 * a = 0 for all rings. But just one theorem does not justify
% a whole subsection

%===================================================================================================
\subsection{Construction of Rings}
We will now look into ways for how we can construct new rings from existing rings. 

%https://en.wikipedia.org/wiki/Ring_(mathematics)#Constructions

%---------------------------------------------------------------------------------------------------
\subsubsection{Product Rings}
Given two rings $R$ and $S$, we can from a product ring by forming the set product of the underlying sets and applying the ring operations element-wise to the resulting pairs of elements. This results in the so called direct product of $R$ and $S$. For example $\mathbb{Z}_{12} = \mathbb{Z}_4 \times \mathbb{Z}_3 = \mathbb{Z}_2 \times \mathbb{Z}_2 \times \mathbb{Z}_3 = \mathbb{Z}_2^2 \times \mathbb{Z}_3$. In general, for any $\mathbb{Z}_n$, we can factor the ring into smaller rings according to the prime factorization of $n$.
% This follows from the Chinese Remainder Theorem.

% This works like with th direct product of two groups

% https://en.wikipedia.org/wiki/Product_of_rings
% https://en.wikipedia.org/wiki/Chinese_remainder_theorem

%---------------------------------------------------------------------------------------------------
\subsubsection{Rings of Functions}
We can take this a step further. Rather than taking the set product of the underlying sets of the two rings, we could take the set $R^S$, i.e. the set of all functions from $S$ to $R$. This will also give a ring. In this case, $S$ doesn't even have to be itself a ring - it's enough if $S$ is a set. As an example, let $S = \mathbb{N}$ and $R = \mathbb{Z}$. Then we are working with the set $\mathbb{Z}^\mathbb{N}$ which is the set of all functions from $\mathbb{N}$ to $\mathbb{Z}$, i.e. the set of all integer valued sequences. With element-wise addition and multiplication, this set forms indeed a ring.

% not sure, if it is called like that, maybe it's calle ring of functions

% https://www.youtube.com/watch?v=MAcdesa9RqA at 17:27
% Let R be a ring and S be a set, then we can consider R^S which is the set of all maps from S to R. 
% Then R^S is also a ring.
% Ex: S = N, R = Z. Elements of our ring are functions from n to Z, i.e. sequences of integer numbers.
% We get a ring of such sequences by allowing to add and multiply sequences element-wise.

% Maybe it has to do with this:
% https://en.wikipedia.org/wiki/Endomorphism_ring

% https://math.stackexchange.com/questions/625351/rings-of-functions

%---------------------------------------------------------------------------------------------------
\subsubsection{Matrix Rings}
Given a Ring $R$, we can construct a new ring by taking square matrices with elements from our ring. Such matrix rings may be denoted by $M_n(R)$, Mat$_n(R)$ or $R^{n \times n}$. We will use the latter notation. This matrix ring has \emph{nilpotent elements} and \emph{zero divisors} (to be defined below). Its center (i.e. the set of commuting elements) consists of the scalar multiples of the identity matrix. The sets of upper or lower triangular matrices and the set of diagonal matrices over $R$ do also form rings which are, in fact, \emph{subrings} (defined later) of the ring of all square matrices over $R$. The ring of diagonal $n \times n$ matrices is isomorphic to the product ring made from $n$ copies of $R$.

% https://en.wikipedia.org/wiki/Ring_(mathematics)#Matrix_ring_and_endomorphism_ring
% https://en.wikipedia.org/wiki/Matrix_ring

%---------------------------------------------------------------------------------------------------
\subsubsection{Adjoining Elements}
An important idea in ring theory is the idea of starting with a given ring and \emph{adjoining} some new element to it. Assume that we start with $\mathcal{Z} = (\mathbb{Z},+,\cdot)$. We now want to add some new object that is not yet an element of the underlying set $\mathbb{Z}$ to that underlying set, i.e. we want to augment the underlying set by some new element. That process is called adjoining. 

\paragraph{The Polynomial Ring $\mathbb{Z}[x]$}
Let's call the new element to be adjoined $x$. We don't say anything about what that mysterious new object $x$ is supposed to be. We treat it like a symbolic variable. Of course, just adding some single element to the underlying set destroys the fulfillment of the ring requirements. To maintain closure under addition, we must also add $x+1$, $x-1$, $x+2$, $x-2$, $x+3$ and so on. To maintain closure under multiplication, we need to add $2 x$, $-2 x$, $3 x$, etc. But that's not yet all that we have to add - not even close. We also need to add $x^2, x^3, x^4, \ldots$ and of course also $3 x^5 - 23 x^2 + 42$ and many more elements. By adjoining a single new object $x$, we are forced by the closure requirements to also add all elements of the general form $a_0 + a_1 x + a_2 x^2 + a_3 x^3 + \ldots$ where $a_0, a_1, a_2, a_3, \ldots \in \mathbb{Z}$. We obtain the set of all polynomials in some symbolic variable $x$ with coefficients from $\mathbb{Z}$. The result is called the ring "$\mathbb{Z}$-adjoin-$x$" and denoted by $\mathbb{Z}[x]$.

% https://www.youtube.com/watch?v=_RTHvweHlhE&list=PLi01XoE8jYoi3SgnnGorR_XOW3IcK-TP6&index=25
% Picking the coeffs and x from Z was just for concreteness. In general, they may come from any ring
% including more exotic ones like the aforementioned product rings, matrix rings, etc. Well,
% we didin't actually say that wpick x from Z ...hmm - but we need to pick it from some set
% such that the multiplications between the coeffs and powers of x are defined. 
% R(x) with parenthese means the set (field) of rational functions of x


% https://en.wikipedia.org/wiki/Polynomial_ring

% https://www.youtube.com/watch?v=MAcdesa9RqA   at around 43
% -takes different approach: defines polynomial ring over R as subring of the ring of all 
%  functions R -> R

% The ring of polynomials over a field is actually one of the most important examples of a ring in abstract algebra

\paragraph{The Gaussian Integers $\mathbb{Z}[\i]$}
If instead of adjoining an unspecified symbolic variable $x$, we adjoin the symbol $\i$ for the imaginary unit with the property $\i^2 = -1$, all polynomials of degree higher than 1 will boil down to a linear polynomial of the general form $a_0 + a_1 \i$ where $a_0,a_1 \in \mathbb{Z}$. This happens because $\i^2$ and higher powers of $\i$ can be simplified into $\pm 1$ and $\pm \i$. Consider the polynomial $2 - 3 \i + 4 \i^2 - 5 \i^3 + 6 \i^4 - 7 \i^5$. Wouldn't we have any rule, we would just keep it as a polynomial in $\i$ and we would be working with a polynomial ring just that we would now call our variable $\i$ instead of $x$. But we do have the rule $\i^2 = -1$ and when we apply it, the polynomial simplifies to $4 - 5 \i$. This will always happen. Every element of our new ring can be expressed in the form $a + b \i$ where $a,b \in \mathbb{Z}$. We denote that ring by $\mathbb{Z}[\i]$ and call it "$\mathbb{Z}$-adjoin-i" or the "Gaussian integers" after Carl Friedrich Gauss who worked a lot with numbers of that kind in his number-theoretic investigations. The picture we should have in mind for these Gaussian integers is the lattice of points of complex numbers whose real and imaginary part are integers, i.e. a lattice of points in the complex plane.

%===================================================================================================
\subsection{Structure of Rings}
After having seen some methods to create new, larger rings from existing rings, we'll now look into dissecting rings into their important constituents. Some of these constituents that we will identify, such as \emph{subrings}, will themselves also be rings. That gives us another way to "construct" new rings from existing rings although it might be more appropriate to call it an "extraction" rather than a "construction".

% Maybe rename this subsection into Special Ring Elements and move the subsubsection about ideals
% into a section in its own right. Or maybe have subsections "Special Elements" and "Special 
% Subsets" or "Special Subrings" or: "Taxonomy of Elements" and "Special Subrings" or maybe name
% the section Ideals and just mention subrings at the beginning. 

%---------------------------------------------------------------------------------------------------
\subsubsection{Divisibility}
An important relation that can be defined between two elements $a,b$ of a ring $R$ is the \emph{divisibility} relation. A ring element $a$ is called a \emph{left divisor} a ring element $b$, if there is a ring element $x$ such that $a x = b$. If $a$ is a left divisor of $b$, then $b$ is a \emph{right multiple} of $a$ and vice versa. Right divisors and left multiples are defined analogously. In a commutative ring, the two notions coincide and we just say that "$a$ is a \emph{divisor} of $b$" or that "$a$ divides $b$" and we write this as $a|b$. In a non-commutative ring, we say that $a$ is a two-sided divisor of $b$ if it is a left and right divisor at the same time, i.e. there exists an $x \in R$ such that $a x = b$ and there exists a $y \in R$ such that $y a = b$ where $x$ may be different from $y$ in general. For this divisibility relation, it matters which ring we consider. Therefore, we may qualify the relation by saying "$a$ divides $b$ in the ring $R$" if the ring $R$ isn't clear from the context. For example, $2$ does not divide $5$ in $\mathbb{Z}$ but $2$ does divide $5$ in $\mathbb{Q}$ because $2 \cdot \frac{5}{2} = 5$. As we will see later, $\mathbb{Q}$ is not only a ring but even a \emph{field} and in such fields, the divisibility relation is rather uninteresting because there, every nonzero element divides every element. The divisibility relation is reflexive: $a|a$ and transitive: if $a|b$ and $b|c$ then $a|c$. These two features put divisibility into the category of a so called \emph{preorder} [VERIFY]. 

% https://en.wikipedia.org/wiki/Divisibility_(ring_theory)
% https://en.wikipedia.org/wiki/Preorder

% https://math.stackexchange.com/questions/1992277/preorders-vs-partial-orders-clarification



%---------------------------------------------------------------------------------------------------
\subsubsection{Zero Divisors}
A \emph{zero divisor} is any element $z$ of a ring $R$ with the property that there exists at least one nonzero element $x$ in $R$ whose product with $z$ is zero. Zero divisors are called like that because they \emph{divide zero} in the sense of the aforementioned divisibility relation [VERIFY!]. In non-commutative rings, we must distinguish between left, right and two-sided zero divisors:

\medskip
\begin{tabular}{l l}
$z$ is left zero divisor:       & $\exists x \neq 0 \in R: \; z x = 0$  \\
$z$ is right zero divisor:      & $\exists x \neq 0 \in R: \; x z = 0$  \\
$z$ is two-sided zero divisor:  & $(\exists x \neq 0 \in R: \; z x = 0) 
                              \vee (\exists y \neq 0 \in R: \; y z = 0)$  \\
\end{tabular}
\medskip

A two-sided zero divisor is an element that is both a left and right zero divisor. Note that we do \emph{not} require $x z = 0 \wedge z x = 0$ for some $x$ from a two-sided zero divisor, i.e. we do not require that $z$ annihilates the same element $x$ from the left and from the right. Rather, we require that for each type of product with $z$ being the left or right factor, there exists some element that gets annihilated by $z$ in that product. Zero itself counts also as zero divisor and is called the trivial zero divisor. Some authors exclude zero itself from the definition of zero divisors. Doing it one or the other way is sometimes more and sometimes less convenient. I opted for this convention because it seems more consistent to me. We will later also define units which are a generalization of the idea of 1 and I view the zero divisors as a similar generalization of 0. Elements that are not zero divisors are also called \emph{regular} or \emph{cancellable}. In non-commutative rings, these may also be qualified as left regular etc. An example for nontrivial zero divisors are the non-invertible matrices in matrix rings. A ring that has no nontrivial zero divisors is called a \emph{domain}. If that ring is commutative, it's sometimes also called an \emph{integral domain}. A product ring has always nontrivial zero divisors. \emph{Nilpotent} and \emph{idempotent} elements are always two-sided zero-divisors. These are defined as elements $x$ for which there exists some natural number $n$ such that $x^n = 0$ (nilpotent) or $x^n = x$ (idempotent). That means, some power of a nilpotent element gives zero and some power of an idempotent element gives back the element itself.

% Is the also a term for elements for which  $x^n = 1$. Is that even possible? Perhaps unipotent?

% https://en.wikipedia.org/wiki/Zero_divisor
% https://en.wikipedia.org/wiki/Nilpotent
% https://en.wikipedia.org/wiki/Unipotent
% https://en.wikipedia.org/wiki/Idempotent_(ring_theory)
% https://en.wikipedia.org/wiki/Zero_divisor
% https://en.wikipedia.org/wiki/Domain_(ring_theory)
% https://en.wikipedia.org/wiki/Zero-product_property


% https://www.youtube.com/watch?v=S8bATrOPgWU
% -mentions unipotent and anti-unipotent at around 2:30 - square to +1 or to -1

%---------------------------------------------------------------------------------------------------
\subsubsection{Units}
Every natural number greater than one has a unique factorization in terms of prime numbers. Unique up to an arbitrary ordering of the factors, that is. The number one has been deliberately excluded from the definition of what a prime number is because if we would include it, the factorization in terms of primes wouldn't be unique anymore because we could multiply any number by $1^n$ without changing it for any $n$. For example $6 = 2 \cdot 3 \cdot 1^n$ for any $n$. We could repair this by stating the theorem as "in terms of primes other than 1" but that would be ugly. The number 1 is considered to be in a class of its own - it's neither a prime nor a composite number. The same is true for zero, by the way - it's another kind of very special number. Within the natural numbers, 1 is the only number that has a multiplicative inverse. This inverse is 1 itself, i.e. 1 is its own inverse. In the integers, we have two numbers that have an inverse - namely $1$ and $-1$. They also happen to be their own inverses, but that's not the important point here. Important is only that they have inverses. If we want to say something similar about unique factorizations for other number systems, we should exclude numbers with inverses from those factorizations because we could always multiply by an arbitrary power of such a number provided that we also include the same power of its inverse. For example $6 = 2 \cdot 3 \cdot (-1)^n$ for any even $n$ in the integers. Because numbers that have inverses would thwart any attempt of a unique factorization (perhaps unique only up to some equivalence relation, as we will see), we consider these numbers with inverses as special enough to deserve a name: they are called \emph{units}. The set of units of a ring $R$ forms a group under the ring multiplication. This group is called the unit group and denoted by $R^*$ or $U(R)$ or $E(R)$ (from the German term Einheit). In polynomial rings, the units are the nonzero constants. 

%That means, two polynomials are asso

%

% https://en.wikipedia.org/wiki/Unit_(ring_theory)
% https://en.wikipedia.org/wiki/Unit_(ring_theory)#Group_of_units

% https://www.youtube.com/watch?v=HHpODrFZ4YQ&list=PLi01XoE8jYoi3SgnnGorR_XOW3IcK-TP6&index=26
% -set of all units also denoted by $R^{\times}$
% -set of units 1,5,7,11 of Z_12 is given as example

%---------------------------------------------------------------------------------------------------
\subsubsection{Association}
In the integers, we have the two units $1$ and $-1$. Any two numbers that differ only by a factor given by one the units are considered to be \emph{associated} with each other. For example $-5$ is associated with $5$ because they differ only by a factor of $-1$ which is a unit. In any ring, pairs of additive inverses are always associated with each other. So in the Gaussian integers, any $a + b \i$ will be associated with $-a - b \i$. But there are more associates. In the Gaussian integers, we have 4 units, namely $1,-1,\i,-\i$ where 1 and $-1$ are still their own inverses and $\i$ and $-\i$ are inverses of each other. Multiplication by $\i$ rotates by 90 degrees counterclockwise and multiplication by $-\i$ by the same angle clockwise. We call $\i$ the \emph{imaginary unit}. For consistency, it could make sense to call $1$ and $-1$ the positive and negative unit respectively (and $-\i$ perhaps negative imaginary unit) but that doesn't seem to be standard terminology [VERIFY!]. In the Gaussian integers, any number $a + b \i$ will also be associated with its complex conjugate and the negative of the complex conjugate, so any Gaussian integer is a member of a quartet of 4 associated numbers $(a + b\i, a - b \i, -a + b \i, -a - b \i)$ which are related geometrically by rotations of multiples of 90 degrees. Association is an equivalence relation on a ring $R$ and in this sense, these 4 numbers are considered to be equivalent. If $x$ is associated with $y$, we will write $x \sim y$. We call the resulting equivalence classes \emph{association classes} and denote the association class of some $a \in R$ by $[a]_{\sim} = \{r \in R: r \sim a\}$. It turns out that this equivalence relation is needed to identify certain numbers with one another to obtain a unique factorization. It will be unique only up to this equivalence - and, of course, up to the ordering of the factors, as always. For example, in $\mathbb{Z}$, the number $-6$ can be factored as $-6 = (-2) \cdot 3$ or as $-6 = 2 \cdot (-3)$. We don't want to consider these two factorizations as different so we need to identify the positives with their negative counterparts via this $\sim$ relation. In polynomial rings, two polynomials are associated with one another if one can be obtained from the other by scaling it by a nonzero constant. In a commutative ring, the association relation can be constructed from the divisibility relation as mutual divisibility, i.e. by saying that two elements $a,b$ are associated if $a|b$ and $b|a$. [TODO: what about non-commutative rings? Maybe we need a mutual two-sided divisor relation?]. 

% https://en.wikipedia.org/wiki/Unit_(ring_theory)#Associatedness
% ACRS, pg 26: 
% -a formal power series a0 + a1 x + a2 x^2 +... is a unit iff a0 is a unit
% -divisibilty induces a partial order (reflexive, transitive, antisymmetric) on the collection of
%  association classes 
% -Notation: [a]_{~} = \{ r \in R: r ~ a  \}. This is called the "association class" of a

%---------------------------------------------------------------------------------------------------
\subsubsection{Prime Elements}
Prime elements are a generalization of the well known prime numbers in $\mathbb{N}$ that is suitable for general rings. ...TBC...

%In a ring, we have the additive inverses so 
% https://en.wikipedia.org/wiki/Prime_element

%\subsubsection{Irreducible Elements}
% In general, there is a distinctions between primality and irreducibility but these
% concepts coincide in a unique factorizaion domain and also in the more general GCD domains

% https://en.wikipedia.org/wiki/Irreducible_element

%---------------------------------------------------------------------------------------------------
\subsubsection{Common Divisors}
It may happen that two ring elements $a,b$ have a common divisor $d$. That means, for given $a,b$ there may exist a $d \in R$ with the property $d | a$ and $d | b$. ...TBC...

% -greatest common divisor (needs notion of "greatness", i.e. size, i.e. norm)
% -lowest common multiple
% -number of divisors, compositeness (quantitatively)

%---------------------------------------------------------------------------------------------------
\subsubsection{Subrings}
We have produced bigger rings from given ones by adjoining elements or forming products. We may also take a given ring $R$ and extract a smaller ring from it which we call a \emph{subring} $S$. A subring is defined analogously to a subgroup: we take a subset of the underlying set. If that subset together with the two operations forms itself a ring, then we have a subring. Multiplying any element $a \in S$ from the subring with any element $b \in S$ from the subring must yield again an element $c \in S$ from the subring. For example, the set of even integers $2 \mathbb{Z}$ forms a subring of $\mathbb{Z}$. Some authors include the existence of a multiplicative identity into the requirements for a ring. With such a definition, $2 \mathbb{Z}$ would not qualify as subring because it has no multiplicative identity. Then, one would have to resort to call it a subrng - a subring without the i [VERIFY!]. Fortunately, I have used a definition that lets us avoid that awkward language.

% toDo: explain also "extension ring" or "ring extenstion"
% https://en.wikipedia.org/wiki/Subring

%---------------------------------------------------------------------------------------------------
\subsubsection{Ideals}
The notion of an \emph{ideal} is more restrictive than that of a subring. Among other requirements, a subring $S$ must be closed under multiplication. For an ideal $I$ of $R$, the closure under multiplication requirement is tightened: It must be the case that the products of elements $i \in I$ from the ideal $I$ with elements $r \in R$ from the original ring $R$ must again produce elements inside the ideal. The subset is not only closed in the sense that you \emph{cannot leave it} but also acts as an "attractor" in the sense that you immediately \emph{get sucked into it} as soon as you multiply \emph{anything} by an element of it, so to speak. Another was to characterize an ideal is to say, that for every element that is inside the ideal, all of its multiples (as in: products with other ring elements) are also inside the ideal. If the ring is not commutative, we must distinguish between left, right and two-sided ideals. For these, we require:

\medskip
\begin{tabular}{l l}
Left ideal:      & $\forall i \in I, r \in R: \; r i \in I$  \\
Right ideal:     & $\forall i \in I, r \in R: \; i r \in I$  \\
Two-sided ideal: & $\forall i \in I, r \in R: \; r i \in I \wedge i r \in I$  \\
\end{tabular}
\medskip

The "left" and "right" in this terminology refers to where the factor $r$ from the ring occurs in the product. A two sided ideal is a left and right ideal at the same time and sometimes just called an ideal without qualifier. In commutative rings, all three notions coincide. The ideal generated by and element $i$ is denoted by $(i)$. As an example, consider again the set of even numbers $2 \mathbb{Z}$. We already know that it is a subring of $\mathbb{Z}$ because any product of any pair of even numbers is again an even number. But is it also an ideal? Yes, because it also satisfies the stronger requirement that the product of any even number with any integer number whatsoever gives again an even number. By similar arguments, any set of the form $n \mathbb{Z}$ is also an ideal in $\mathbb{Z}$ for any $n \in \mathbb{Z}$ [VERIFY]. As an example of a subring that is not an ideal, consider $\mathbb{Z}$ as a subring of $\mathbb{Z}[\i]$. An ideal in $\mathbb{Z}[\i]$ would be $2\mathbb{Z}[\i]$, the Gaussian integers with even real and imaginary parts.

% Explain etymololy of the term "ideal"
% $(a)$ is the smallest possible subset of R that has the desired properties

% https://de.wikipedia.org/wiki/Ideal_(Ringtheorie)
% https://en.wikipedia.org/wiki/Ideal_(ring_theory)
% https://mathworld.wolfram.com/Ideal.html

% https://math.stackexchange.com/questions/1243296/are-ideals-also-rings
%   -> if a ring is defined to have unity (as we do), then yes

% pretty concise and simple:
% https://sites.millersville.edu/bikenaga/abstract-algebra-1/ideals-and-subrings/ideals-and-subrings.pdf

% ideals generated from multiple elements
% (2,5) = 2 m + 5 m, m,n in Z
% but: 2,5, are ring elements and m,n are integers, so how is the product to be interpreted? In terms 
% of repeated addition? It can't be the ring multipliation because that isn't defined on R x Z. 
% Maybe look at (10,14) . they have a gcd of 2 so we should get (2)? or (15,21) or (10, 15)


% https://www.youtube.com/watch?v=rB4qn3qX-fo    Michael Penn
% Abstract Algebra | The motivation for the definition of an ideal.

\paragraph{Operations on Ideals} We can apply operations to two ideals $I$ and $J$ to obtain new ideals of $R$.

\medskip
\begin{tabular}{l l l}
Intersection:  & $I \cap J$& $ = \;\; \{k : k \in I, k \in J\}$  \\
Sum:           & $I + J$   & $ = \;\; \{i + j : i \in I, j \in J\}$  \\
Product:       & $I J$     & $ = \;\; \{\sum_n i_n j_n : i_n \in I, j_n \in J\}$  \\
Quotient:      & $I:J$     & $ = \;\;  \{r \in R : r J \subseteq I\}$  \\
Radical:       & $\sqrt{I}$& $ = \;\; \{ r \in R : (\exists n \in \mathbb{N} :  r^n \in I) \}$  \\
\end{tabular}
\medskip

In words: The intersection of two ideals consists of all their common elements. The sum of two ideals consists of all sums that can be formed from the elements of the ideals $I,J$. The product consists of all finite sums of products $i j$ where $i \in I$ and $j \in J$. The quotient ... [TODO: explain in words]. The radical on an ideal consists of all elements of the ring, some power of which lies in the ideal. You may ask yourself, if we have sum, product and quotient, then what about the difference? I think, the answer is that if we would define a difference as $I - J = \{i - j : i \in I, j \in J\}$, which may seem natural, then we would obtain the exact same set as for the sum $I + J$, so it wouldn't give us anything new. If, on the other hand, we would attempt to define the difference as $I - J = \{i \in I : i \notin J\}$, then the resulting set would not be an ideal, so such a definition would not seem to be useful (my guesses). The union of two ideal does also not give an ideal.

% But what about defining the difference as $I - J : \{a : a \in I, a \notin J\}$. But maybe that would not give an ideal, so it's not interesting?

% https://math.stackexchange.com/questions/1737282/is-the-difference-of-ideals-an-ideal
% https://math.stackexchange.com/questions/1524077/union-and-sum-of-ideals-is-not-ideal


\medskip
As example, take $I = \mathbb{Z}_6, J = \mathbb{Z}_8$ which are both ideals of $\mathbb{Z}$. Their intersection is given by $I \cap J = \mathbb{Z}_{24}$. That is because $24$ is the lowest common multiple of $6$ and $8$ [VERFIY claim]. ...TBC...

% 


% Maybe take as examples for the ring operations Z_4 and Z_6. The intersection should be Z_12, the
% product should be Z_2, the sum: { 4, 10, 16, 22, 28, ..., 8, 14, 20, } = Z_2 as well? 
% Ah - no - the product is not Z_2 but: { 4*6=24, 4*12=48, 4*18=72, ... 8*6=48, 12*6=72 } = Z_24.
% In this view, the idea of a "prime ideal" makes sense, I think. These are then ideal that can
% not be expressed as products of other ideals? I deliberately do not say "products of smaller
% ideals" because, in a sense, these factor ideals have more elements when seen as sets (not more
% in term of cardinality but in the sense of gaving some elements that the product doesn't have).
% The products contain less elements than the factors - not in the sense of cardinality but in the
% sense of a subset relation. we have $I \cdot J \subseteq I$ and $I \cdot J \subseteq J$ [VERIFY]
%
% Definitions for sum, product, intersection, radical, quotient taken from:
% https://agag-gathmann.math.rptu.de/class/commalg-2013/commalg-2013-c1.pdf
% ToDo: verify if for the quotient thst \subseteq is the right interpretation of the 
% \subset symbol in the pdf
% yes: https://en.wikipedia.org/wiki/Ideal_quotient
%
% Radical: all elements of the ring, some power of which lies in the ideal. So the radical has more
% elements than the original ideal. The originla ideal is included because for n=1, the condition is
% is satisfied. Mayb there may be mor elements. For example sqrt(Z_8) will contain 2 because 
% 2^3 = 8 and it will contain 4 because 4^2 = 16
%
% https://en.wikipedia.org/wiki/Ideal_(ring_theory)#Ideal_operations
% https://commalg.subwiki.org/wiki/Product_of_ideals
% https://math.stackexchange.com/questions/290229/explaining-the-product-of-two-ideals
% https://en.wikipedia.org/wiki/Ideal_quotient
% https://en.wikipedia.org/wiki/Radical_of_an_ideal
% https://de.wikipedia.org/wiki/Radikal_(Mathematik)
%
% factor rings, ideal structure of z and Z_n
% https://www.math.uci.edu/~ndonalds/math120b/3ideals.pdf

\paragraph{Prime Ideals} Now that we have defined what the product of two ideals $I$ and $J$ is supposed to be, it is natural to define a \emph{prime ideal} as an ideal that cannot be expressed as a product of two other ideals [!!!VERIFY!!! I've literally made that up myself bcs it seems so natural].

% https://en.wikipedia.org/wiki/Prime_ideal
% https://en.wikipedia.org/wiki/Minimal_prime_ideal
%
% Not to be conused with:
% https://en.wikipedia.org/wiki/Primary_ideal

\paragraph{Principle Ideals} A principle ideal is an ideal that is generated by multiplying every element of the ring $R$ by some single, specific ring element $a$. Unless the ring is commutative, we must distinguish between left, right and two-sided versions:

\medskip
\begin{tabular}{l l}
Left principle ideal:      & $I = R a = \{ r a : r \in R \}$ \\
Right principle ideal:     & $I = a R = \{ a r : r \in R \}$ \\
Two-sided principle ideal: & $I = RaR = \{ \sum_n r_n a s_n : r_n,s_n \in R \} $ \\
\end{tabular}
\medskip

The definition of the two-sided version is the set of all finite sums of products of the form $r a s$ with $r,s \in R$. It ensures that the the closure property under addition is satisfied in the ideal.% It's taken from wikipedia and it says that a citation is needed, so take it with a grain of salt.

% german: Hauptideal
% https://en.wikipedia.org/wiki/Principal_ideal


%\paragraph{Prime Ideals}

% https://en.wikipedia.org/wiki/Primary_decomposition


%\medskip
% https://en.wikipedia.org/wiki/Primitive_ideal
% -principle ideal: an ideal generated by a single element via multiplying all ring elements by that
%  element. Note that this is not the same as generating it via combining it with itself recursively
%  as we did in group theory. This wouldn't work here. 2 would generate the powers of two, not the
%  evens  because we look ath the multiplicative operation. But multiplying every integer by 2 does
%  also generate the evens. Don't confuse the notions of an *additive* generator with the generation
%  process here
% -prime ideals: ideals generated by prime elements (recall that generators generate subgroups) but
%  that's not the definition.  A prime P ideal has the property that any composite element c that is
%  found in it must have a factor in P (VERIFY). On wikipedia, it says: form any product ab in P, at
%  least one of the factors must be in in P. It cannot have any products in it unless it has at
%  least also one of the factors in it
% -under which conditions is a subring an ideal?
% -proper ideal: ideal other that R, nontrivial ideal: other than {0}

% https://mathworld.wolfram.com/PrimeIdeal.html


%---------------------------------------------------------------------------------------------------
\subsubsection{Quotient Rings}
In ring theory, ideals play a role analogous to that of the normal subgroups in group theory. There, we used normal subgroups to create quotient groups which were sets of ...TBC...

% maybe this should go to somewhere after ideals have been defined

% google: factor ring
% The quotient of a ring R, by a two-sided ideal I, is denoted by R / I . The result is called a quotient or factor ring. It is the set of additive cosets { a + I | a ∈ I } , with addition and multiplication defined as ( a + I ) + ( b + I ) = a + b + I , and ( a + I ) ( b + I ) = a b + I .




%\subsection{Special Kinds of Rings}

%https://en.wikipedia.org/wiki/Ring_(mathematics)#Special_kinds_of_rings
% see ScratchPad.txt 
% Maybe integrate this into the section below by making it a section about special kinds and
% concrete examples.


%===================================================================================================
\subsection{Taxonomy of Rings}

%---------------------------------------------------------------------------------------------------
\subsubsection{Rings with Special Properties}
The following table gives rings with increasingly more features in each line. Each entry has the same features as all of its predecessors (i.e. all lines above it) plus some additional feature listed in the right column. So, when going down the table from top to bottom, the rings get more and more specific and all entries above a given line are supersets (i.e. generalizations with less features) of the structure on the given line.

\medskip
\begin{tabular}{l l}
Ring                        & Satisfies ring axioms \\
Unital Ring                 & Has multiplicative identity \\
Commutative Ring            & Multiplication is commutative \\
Integral Domain             & Has no zero divisors \\
Integrally Closed Domain    & Is complicated [TODO: figure out!] \\
GCD Domain                  & Any pair of elements has greatest common divisor \\
Unique Factorization Domain & Unique factorization is possible \\
Principal Ideal Domain      & Every ideal is principal \\
Euclidean Domain            & Division with remainder is possible \\
Field                       & Multiplicative inverses exist \\
Algebraically Closed Field  & Every non-constant polynomial has a root
\end{tabular}
\medskip

You will find this chain of inclusions on many pages on Wikipedia but they use a ring definition that requires a multiplicative identity. I didn't include this requirement in the ring axioms such that their "rng" is our "ring" and their "ring" is our "unital ring". There is no universal consensus about this. There are some other definitions that do not fit into such an inclusion chain. A \emph{domain} without the qualifier "integral" is a unital ring without zero divisors which is not necessarily commutative. There is also the notion of a \emph{division ring} which is basically a field minus the commutativity requirement. A ring is called a \emph{boolean ring} if every element squares to itself. 

% This chain of inclusions can be found on wikipedia on any of these pages:
%
% https://en.wikipedia.org/wiki/Integral_domain
%
% But wikipedia uses a ring definition that reuqires a multiplicative identity. I didn't include
% this requirement such that their "rng" is our "ring" and their "ring" is our "unital ring".

% domain: like integral domain but not commutative
% division ring:

% https://en.wikipedia.org/wiki/Division_ring
% https://en.wikipedia.org/wiki/Domain_(ring_theory)

% https://en.wikipedia.org/wiki/B%C3%A9zout_domain
% https://en.wikipedia.org/wiki/Valuation_ring
% 
% What is above? Semirings?

% https://en.wikipedia.org/wiki/Boolean_ring

%---------------------------------------------------------------------------------------------------
\subsubsection{The Characteristic of a Ring}
In some rings $R$ it may happen that if we repeatedly add the multiplicative identity $1$ to itself, we may at some point hit the additive identity $0$. If that happens, the number of copies of $1$ that is needed for this to occur for the first time is called the characteristic of the ring $R$. If it doesn't ever happen, we say that the characteristic of $R$ is zero. ...TBC...


% https://en.wikipedia.org/wiki/Characteristic_(algebra)


%---------------------------------------------------------------------------------------------------
\subsubsection{More Examples of Rings}
We have seen some basic rings like $\mathbb{Z}$ which can be regarded as the prototypical example for a ring that was used as model on which the more abstract notion of a ring is based and we have seen the rings $\mathbb{Z}[x]$ and $\mathbb{Z}[i]$ which where obtained by adjoining certain kinds of new elements. Here, we will briefly mention some more examples of rings.

%---------------------------------------------------------------------------------------------------
\paragraph{Number Rings}
ToDo: integers $\mathbb{Z}$, Gaussian Integers $\mathbb{Z}[i]$, modular integers $\mathbb{Z}_p$

%---------------------------------------------------------------------------------------------------
\paragraph{Polynomial Rings}
ToDo: polynomial rings over various base fields, maybe also multivariate polynomials: $\mathbb{Q}[x]$, $\mathbb{R}[x]$, $\mathbb{C}[z]$, $\mathbb{Z}_p[x]$, $\mathbb{R}[x,y]$, rings of power series like
$\mathbb{R} \llbracket x \rrbracket$

% https://math.stackexchange.com/questions/190073/what-is-mathbb-zt-what-are-the-double-brackets
%$\mathbb{R} \llbracket x \rrbracket$
%\llbracket
% Z[\sqrt(2)], Z[\sqrt{-3}] (historic reasons)

%---------------------------------------------------------------------------------------------------
\paragraph{The Dyadic Rationals $\mathbb{Z}[\frac{1}{2}]$}
If we adjoin the number $\frac{1}{2}$ to $\mathbb{Z}$, we obtain a ring that contains all fractions that have a power of $2$ as denominator. A finite approximation of this ring is used to represent floating point numbers in a computer. If, instead of adjoining just $\frac{1}{2}$, we would adjoin also $\frac{1}{3}$, $\frac{1}{5}$, $\frac{1}{7}$, $\frac{1}{11}$, etc. for all prime denominators, we would obtain $\mathbb{Q}$.
% Of course, that "etc." here means to adjoin all the infinitely many reciprocals of all the prime
% numbers. I think, that is called in infinite extension or extension of infinite degree? Or does
% "degree" only apply to field extensions where the degree is interpreted as degree of a minimal
% polynomial of the number to be adjoined?

%---------------------------------------------------------------------------------------------------
\paragraph{The Cyclotomic Integers $\mathbb{Z}[\zeta_n]$}
Consider the so called cyclotomic equation $x^n = 1$. Its solutions are the $n$th roots of unity which are sometimes also called de Moivre numbers. Cylcotomic means circle-splitting which is a fitting name because they indeed split the unit circle into equally sized pizza segments. We'll denote them by $\zeta_n = e^{2 \i \pi / n}$.  If we adjoin $\zeta_n$ to $\mathbb{Z}$ for some given $n$, we'll obtain the ring of the cyclotomic integers for that given $n$. For $n=1$ we just adjoin $\zeta_1 = e^{2 \i \pi / 1} = 1$ which was already there, so nothing changes. For $n=2$, we get $\zeta_2 = e^{\i \pi} = -1$, we adjoin $-1$ which is also already there. For $n=4$, we have $\zeta_4 = e^{\i \pi / 2} = \i$, so in this case, we'll get the Gaussian integers because we adjoin $\i$. They define a square lattice in the complex plane and have the 4 units $1,-1,\i,-\i$. This can be generalized. For $n=3$, we get the so called Eisenstein integers which define a lattice of equilateral triangles in the complex plane. The ring of Eisenstein integers has the 6 units $1, -1, \zeta_3, -\zeta_3, \zeta_3^2, -\zeta_3^2$. A general cyclotomic integer for given $n$ looks like $a_0 + a_1 \zeta_n + a_2 \zeta_n^2 + \ldots + a_{n-1} \zeta_n^{n-1}$. Powers of $\zeta_n$ higher than $n-1$ cannot occur because $\zeta_n^n$ wraps back to $1$ and then the cycle starts again pretty much like the powers of $\i$ cycle back to 1 at $\i^4$ in the complex numbers. In the complex numbers, we will also have $\zeta_4^2 = \i^2 = -1$ and $\zeta_4^3 = \i^3 = -\i$ which is why two coefficients $a_0, a_1$ are sufficient to encode a cyclotomic number with $n=4$, i.e. a complex number. Due to symmetries, the terms $a_0 \zeta_4^0 + a_2 \zeta_4^2$ and $a_1 \zeta_4^1 + a_3 \zeta_4^3$ get combined into single terms because $\zeta_4^0 = -\zeta_4^2$ and $\zeta_4^1 = -\zeta_4^3$. In number theory, one is mostly interested in the cases where $n$ is a prime number. The cyclotomic integers with prime $n$ have played a crucial role in the proof of Fermat's Last Theorem.

% I guess, only for n=3, n=4 and n=6 we get nice regular lattices (triangular, square, hexagonal) and for other n, the lattices look more complex? Try to draw the lattice for n=5!

% x^n - 1 = (x-1) (x^{n-1} + x^{n-2} + ... + 1) because we can factor out a linear factor (x-1) from x^n - 1 because x_0 = 1 trivially solves x^n - 1 = 0 so 1 is a root of x^n - 1

% https://mathworld.wolfram.com/CyclotomicInteger.html
% https://mathworld.wolfram.com/CyclotomicEquation.html
% https://mathworld.wolfram.com/deMoivreNumber.html
% https://en.wikipedia.org/wiki/Cyclotomic_polynomial
% https://fermatslasttheorem.blogspot.com/2006/02/cyclotomic-integers.html
% https://fermatslasttheorem.blogspot.com/2006/05/basic-properties-of-cyclotomic.html
% https://crypto.stanford.edu/pbc/notes/numbertheory/cyclo.html
% https://de.wikipedia.org/wiki/Eisenstein-Zahl
% https://en.wikipedia.org/wiki/Eisenstein_integer
% https://mathworld.wolfram.com/EisensteinInteger.html
% https://en.wikipedia.org/wiki/Cyclotomic_field

\paragraph{Ring of Power Set} Let $S$ be any set and $\mathcal{P}(S)$ be its power set, i.e. the set of all subsets of $S$. We can turn $\mathcal{P}(S)$ into a ring by defining a suitable addition and multiplication operation among subsets. Let $A, B \in \mathcal{P}(S)$. Multiplication is defined as set intersection $A \cdot B = A \cap B$ and addition is defined as the symmetric difference: $A + B = (A \cup B) \setminus (A \cap B) = (A \setminus B) \cup (B \setminus A)$. Together with the so defined operations, $\mathcal{P}(S)$ indeed becomes a ring. The multiplicative identity is given by the whole set: $1 = S$, the additive identity is the empty set: $0 = \emptyset$ and the additive inverse of $A$ is given by $A$ itself: $-A = A$. Our ring is a boolean ring of characteristic 2. See \cite{YT_PowerSetRing} for more details.

% ACRS pg 48
% https://www.youtube.com/watch?v=fvMnVKq3UtU

%---------------------------------------------------------------------------------------------------
\subsubsection{Classification of Finite Rings}
Finite rings are more complicated objects than finite groups and have more structure due to the presence of two binary operations that interact with each other. One might be inclined to expect that the task of classification of finite rings is more complicated than in the case of finite groups. But the opposite is true. This is plausible because when we have more constraints, we will actually find less mathematical structures that satisfy all these constraints. It's actually quite simple to characterize the set of all possible finite rings. Every finite \emph{simple} ring is isomorphic to $\mathbb{F}_q^{n \times n}$ which is the ring of all $n \times n$ matrices over a given finite field $\mathbb{F}_q$ of order $q$. Finite fields will be introduced in the next section. The qualification "simple" deserves some attention, though.

...TBC...what is a simple ring and how can we construct the non-simple ones?

%https://en.wikipedia.org/wiki/Finite_ring
%https://en.wikipedia.org/wiki/Simple_ring


\begin{comment}


Examples of rings that are not fields in order of importance:
-Integers
-Matrices with matrix addition and multiplication. The subset of invertible matrices 
 forms a field (i think)
-Polynomials over a field
-Modular integers with modular addition and multiplication. If the modulus isn't prime, it's
 only a ring. If it is prime, it is a field.
-Gaussian integers - obtained by adjoining i to Z

ToDo:
-Ideals: 
 -generalize the notion of "multiple of"
 -is a set that splits the ring into equivalence classes (like in modular integers)
 -is itself a subring of the ring, i.e. ring (without identity - we don't call it rng bcas we 
  don't demand rings to have an identity)
-Quotient rings, factor rings, etc.


https://en.wikipedia.org/wiki/Quotient_ring
https://mathworld.wolfram.com/QuotientRing.html


What about geometric examples? For group theory, we have a lot of geometric examples - reflectiosn, translations, rotations. I think, for geometric examples of rings, we need to look into algebraic geometry?

What is algebraic geometry? [by Aleph 0]
https://www.youtube.com/watch?v=MflpyJwhMhQ

Was sind Ringe? Was sind Körper? (Algebra) [Weitz]
https://www.youtube.com/watch?v=MAcdesa9RqA

Number Systems Invented to Solve the Hardest Problem - History of Rings | Ring Theory E0
https://www.youtube.com/watch?v=M-9_rZfVQVE&list=PL6VQBwZazp7YSRz9jluchQJSDRv_yQKeZ


https://en.wikipedia.org/wiki/Ring_(mathematics)
https://de.wikipedia.org/wiki/Ring_(Algebra)
https://en.wikipedia.org/wiki/Unit_(ring_theory)
https://en.wikipedia.org/wiki/Annihilator_(ring_theory)
https://en.wikipedia.org/wiki/Ascending_chain_condition
https://en.wikipedia.org/wiki/Noetherian_ring
https://en.wikipedia.org/wiki/Artinian_ring
https://en.wikipedia.org/wiki/Valuation_ring
https://en.wikipedia.org/wiki/Dimension_theory_(algebra)

https://en.wikipedia.org/wiki/Integral_domain
Has the nice set inclusion diagram

https://en.wikipedia.org/wiki/Derivation_(differential_algebra)

Reducibility:

https://www.youtube.com/watch?v=e5d6A_JAGpE  Irreduzible Polynome (Teil 1/2) | Math Intuition
-Primality and irreducibility mean the same thing in Z
-x, x+1 are irreducible, x^2 = x*x is reducible
-Constant polynomials are units (neither educible nor irreducible)
 ..except zero - that is another special thing
-f is irredicible, iff from f = g*h follows: g or h is a unit, i.e. if every possible 
 factorization of f is trivial 
-Irreducibility depends on the base field: x^2 - 2 is irreducible over Q[x] but is reducible
 to (x - sqrt(2)) (x + sqrt(2)) over R[x] or Q[sqrt(2)][x]. 
 
https://www.youtube.com/watch?v=U4YUSep4a2M  Irreduzible Polynome (Teil 2/2) | Math Intuition
-x^2 + 1 is irreducible in R[x]  but reducible in C[x]
-A polynomial in R[x] is reducible if it has a root in R. That's a sufficient but not a necessary
 condition.

https://www.youtube.com/watch?v=VwU3uNxKzCo  
Was ist ein Ideal? - Teil 1/2 (Beispiele, Definition erklärt)


Abstract algebra reminds me a bit of reductionism in physics. One breaks down a system into its
elementary constituents. Here, the "system" is polynomial equations over R or C. As elementary
ingredients, we have identified the base *field* C and the polynomial *ring* over it. So we are
already working in an algebraic structure that is built from two more elementary ones. Reducing
polynomials, factoring groups, etc. are other reductions/decompositions that occur in algebra.

https://www.youtube.com/watch?v=87QXdOxeVmU&list=PLi01XoE8jYoi3SgnnGorR_XOW3IcK-TP6&index=27
Integral domains
-> has no zero divisors
-> solving equations by factoring and setting individual factors to zero will find all solutions
   ...otherwise, the technique may miss some solutions
-> solving equations by cancelling factors on both sides works. Otherwise, cancelling my produce
   extra solutions.
-> is defined to be also commutative, but that is actually not so important. Non-commutative version
   is just called domain


https://www.youtube.com/watch?v=ofy2Kw2sIZg  The Missing Operation | Epic Math Time
-Has an interesting ring isomorphism based on e^x. (R, +, *) gets mapped to (R_+, *, a^(ln(b))). 
 I think, it extends our example of the group isomorphism (R, +) to (R_+, *) given in the group
 theory chapter
 
0^0: A 200+ Years Debate 
https://www.youtube.com/watch?v=O8aKKKdQmxY 
-Talks abouth the evaluation operation as a homomorphism from a polynomial ring into its base
 ring


What was Fermat’s “Marvelous" Proof? | Infinite Series
https://www.youtube.com/watch?v=SsVl7_R2MvI
-When you have a ring, you can talk about primality


\end{comment}