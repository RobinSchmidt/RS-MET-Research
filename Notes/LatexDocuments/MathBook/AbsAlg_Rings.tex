\section{Ring Theory}
After having seen groups, the next idea to explore is that of a ring. We assume to have a commutative group $(G, +)$ as a starting point. Note that here we use $+$ to denote our commutative group operation instead of the previous usage of $\cdot$ or juxtaposition within group theory. That is because in ring theory, the dot or juxtaposition will be needed for a second operation that we now want to introduce. We have seen that the integers form a group under addition. But we can also multiply integers. We now want to capture the behavior of integers under addition and multiplication in an abstract way. To this end, we assume that we start with a commutative group and we will call the group operation "addition" from now on. We assume furthermore that we have defined a second operation between elements of the group which we intend to be our abstract multiplication. Abstractly, a ring can be defined as a triple $(R,+,\cdot)$ consisting of a set $R$ with two binary operations between elements of the set such that the following rules hold true, which we shall henceforth call ring axioms:
\begin{itemize}
\item $(R,+)$ forms a commutative group
\item $(R,\cdot)$ forms a semigroup (term is explained below)
\item Multiplication distributes over addition: 
$\forall a,b,c \in R: \;  
a \cdot (b + c) = a \cdot b + a \cdot c, \; 
(b + c) \cdot a = b \cdot a + c \cdot a$
\end{itemize}
where the most interesting thing is the last law. This law is called the distributivity law and it establishes a connection between the two operations $+$ and $\cdot$. Note that we need two versions (left and right) of the distributive law because we do not in general require multiplication to be commutative. We do require commutativity for addition, though. In writing down the distributive laws, I also implicitly assumed that $\cdot$ takes precedence over $+$ such that no parentheses are needed for the right hand sides.

\paragraph{Notation and Terminology}
Here, a \emph{semigroup} is a structure like a group but we do not necessarily require the existence of inverse elements and not even a neutral element. We now have not one but two operations, so we need to adapt our notation and terminology a bit to accommodate for that. We have called the operations "addition" and "multiplication" but it should be understood that we mean abstractions of these operations. We will denote the additive neutral element, formerly known as $e$, from now on as $0$ and we will also call it the \emph{additive identity}. If multiplication happens to commutative as well, we call our ring a commutative ring. The study of such commutative rings is called commutative algebra. If the ring is commutative, one of the distributivity laws is sufficient - the other one then follows from it and commutativity. For multiplication, we do not require the existence of a neutral element because we require our structure to be only a semigroup with respect to multiplication. By the way, a semigroup in which a neutral element exists is called a monoid. A ring in which a multiplicative neutral element exists, i.e. one in which the "semigroup" requirement is upgraded to a "monoid" requirement, is called a ring with unity or \emph{unital ring}. The multiplicative neutral element is then denoted by $1$ and also called the \emph{multiplicative identity}. If we just say "identity" without a qualifier, we mean the multiplicative identity. We may also call the additive inverses \emph{negatives} and the multiplicative inverses \emph{reciprocals}. We don't require existence of reciprocals, though. If, for some element $a$, a reciprocal does exist, we call $a$ a \emph{unit}. So, the units of a ring are its (multiplicatively) invertible elements. A ring in which no (multiplicative) identity exists is sometimes called a \emph{rng}. This is no typo - it's a "ring" without the "i" and pronounced as "rung".  A ring without negatives is sometimes called a \emph{rig}. Looks like they just leave out the first letters i and n of the name of the missing elements from the word ring. A ring without i has no identity, a ring without n has no negatives. This is just an aside - we won't use that terminology here.

% units are neither prime nor composite? In Z, the units are +1 and -1. In Z[i], we have 1,-1,i,-i. Maybe the motivation form this term is the imaginary unit? we could call -1 the "negative unit". We have 4 kinds of numbers: primes, composites, units and zero.

% numbers that result from another number by multiplication by a unit are asscociates of each other

% explain change of notation for additive inverses as -a



% https://en.wikipedia.org/wiki/Unit_(ring_theory)
% https://en.wikipedia.org/wiki/Rng_(algebra)

% https://en.wikipedia.org/wiki/Prime_element
% https://en.wikipedia.org/wiki/Prime_ideal
% https://en.wikipedia.org/wiki/Primary_ideal
% https://en.wikipedia.org/wiki/Unique_factorization_domain

\subsection{Adjoining Elements}
Assume that we start with a given ring like $\mathcal{Z} = (\mathbb{Z},+,\cdot)$. We now want to add some new object that is not yet an element of the underlying set $\mathbb{Z}$ to that underlying set, i.e. we want to augment the underlying set by some new element. That process is called adjoining. 

\subsubsection{The Polynomial Ring $\mathbb{Z}[x]$}
Let's call the new element to be adjoined $x$. We don't say anything about what that mysterious new object $x$ is supposed to be. We treat it like a symbolic variable. Of course, just adding some single element to the underlying set destroys the fulfillment of the ring requirements. To maintain closure under addition, we must also add $x+1$, $x-1$, $x+2$, $x-2$, $x+3$ and so on. To maintain closure under multiplication, we need to add $2 x$, $-2 x$, $3 x$, etc. But that's not yet all that have to add - not even close. We also need to add $x^2, x^3, x^4, \ldots$ and of course also $3 x^5 - 23 x^2 + 42$ and many more elements. By adjoining a single new object $x$, we are forced by the closure requirements to also add all elements of the general form $a_0 + a_1 x + a_2 x^2 + a_3 x^3 + \ldots$ where $a_0, a_1, a_2, a_3, \ldots \in \mathbb{Z}$. We obtain the set of all polynomials in some symbolic variable $x$ with coefficients from $\mathbb{Z}$. The result is called the ring "$\mathbb{Z}$-adjoin-$x$" and denoted by $\mathbb{Z}[x]$.

%what we get when we also add everything else that is needed to maintain the required closure properties, is actually the set of all polynomials in some symbolic variable $x$ with coefficients from $\mathbb{Z}$. The result is called the ring "$\mathbb{Z}$-adjoin-$x$" and denoted by $\mathbb{Z}[x]$. ...TBC...

\subsubsection{The Gaussian Integers $\mathbb{Z}[i]$}
If instead of adjoining an unspecified symbolic variable $x$, we adjoin the symbol $\i$ for the imaginary unit with the property $\i^2 = -1$, all polynomials of degree higher than 1 will boil down to a linear polynomial of the general form $a + b \i$ where $a,b \in \mathbb{Z}$. This happens because $\i^2$ and higher powers of $\i$ can be simplified into $\pm 1$ and $\pm \i$. Consider the polynomial $2 + 3 \i + 4 \i^2 - 5 \i^3 + 6 \i^4 - 7 \i^5$. Wouldn't we have any rule, we would just keep it as a polynomial in $\i$ and we would be working with a polynomial ring just that we would now call our variable $\i$ instead of $x$. But we do have the rule $\i^2 = -1$ and when we apply it, the polynomial simplifies to ...TBC...

\subsubsection{The Cyclotomic Integers $\mathbb{Z}[\zeta]$}
Now we want to adjoin a special number $\zeta = e^{2 \pi \i / p}$ for some prime number $p$ that we choose once and for all. For the smallest prime $p=2$, we get $\zeta = e^{\pi \i} = -1$ which satisfies $\zeta^2 = 1$. In general, we will have $\zeta^p = 1$ which is called the cyclotomic equation. When we adjoin our $\zeta$ to $\mathbb{Z}$, we will obtain the cyclotomic integers [VERIFY!!] ...TBC...

%that has the property

% https://mathworld.wolfram.com/CyclotomicInteger.html
% https://mathworld.wolfram.com/CyclotomicEquation.html
% https://mathworld.wolfram.com/deMoivreNumber.html
% http://fermatslasttheorem.blogspot.com/2006/02/cyclotomic-integers.html

% what if we use $\zeta = e^{\pi \i / n}$? Then we obtain the Gaussian integers for n = 2. It seem like the Gaussian integers fit into the framework of cyclotomic integers for p=4, but that is not allowed because 4 is not a prime

%, we must also add a whole bunch of other new elements, for example $x+1$, $x-1$, $x+2$, $x-2$, $2 x$, $-2 x$, $42 x - 71$, $23 x^7 + 5 x^4 - 2$, etc.

%postulate that some other object that is not element, let's call it $x$,

%In the example rings of the Gaussian integers $\mathbb{Z}[i]$ and the polynomial rings like $\mathbb{R}[x]$, we added a single 

% https://en.wikipedia.org/wiki/Cyclotomic_field

\subsection{More Examples of Rings}
We have seen some basic rings like $\mathbb{Z}$ which can be regarded as the prototypical example for a ring that was used as model on which the more abstract notion of a ring is based and we have seen the rings $\mathbb{Z}[x]$, $\mathbb{Z}[i]$ and $\mathbb{Z}[\zeta]$ which where obtained by adjoining certain kinds of new elements. Here, we will briefly mention some more examples of rings.

\subsubsection{Numbers}
ToDo: integers $\mathbb{Z}$, Gaussian Integers $\mathbb{Z}[i]$, modular integers $\mathbb{Z}_p$

\subsubsection{Polynomial Rings}
ToDo: polynomial rings over various base fields, maybe also multivariate polynomials: $\mathbb{Q}[x]$, $\mathbb{R}[x]$, $\mathbb{C}[z]$, $\mathbb{Z}_p[x]$, $\mathbb{R}[x,y]$, rings of power series like
$\mathbb{R} \llbracket x \rrbracket$

% https://math.stackexchange.com/questions/190073/what-is-mathbb-zt-what-are-the-double-brackets
%$\mathbb{R} \llbracket x \rrbracket$
%\llbracket





\begin{comment}


Examples of rings that are not fields in order of importance:
-Integers
-Matrices with matrix addition and multiplication. The subset of invertible matrices 
 forms a field (i think)
-Polynomials over a field
-Modular integers with modular addition and multiplication. If the modulus isn't prime, it's
 only a ring. If it is prime, it is a field.
-Gaussian integers - obtained by adjoining i to Z




What about geometric examples? For group theory, we have a lot of geometric examples - reflectiosn, translations, rotations. I think, for geometric examples of rings, we need to look into algebraic geometry?

What is algebraic geometry? [by Aleph 0]
https://www.youtube.com/watch?v=MflpyJwhMhQ

Was sind Ringe? Was sind Körper? (Algebra) [Weitz]
https://www.youtube.com/watch?v=MAcdesa9RqA

Number Systems Invented to Solve the Hardest Problem - History of Rings | Ring Theory E0
https://www.youtube.com/watch?v=M-9_rZfVQVE&list=PL6VQBwZazp7YSRz9jluchQJSDRv_yQKeZ


https://en.wikipedia.org/wiki/Ring_(mathematics)
https://de.wikipedia.org/wiki/Ring_(Algebra)
https://en.wikipedia.org/wiki/Unit_(ring_theory)




\end{comment}