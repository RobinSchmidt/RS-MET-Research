\section{Ring Theory}
After having seen groups, the next idea to explore is that of a ring. We assume to have a commutative group $(G, +)$ as a starting point. Note that here we use $+$ to denote our commutative group operation instead of the previous usage of $\cdot$ or juxtaposition within group theory. That is because in ring theory, the dot or juxtaposition will be needed for a second operation that we now want to introduce. We have seen that the integers form a group under addition. But we can also multiply integers. We now want to capture the behavior of integers under addition and multiplication in an abstract way. To this end, we assume that we start with a commutative group and we will call the group operation "addition" from now on. We assume furthermore that we have defined a second operation between elements of the group which we intend to be our abstract multiplication. Abstractly, a ring can be defined as a triple $(R,+,\cdot)$ consisting of a set $R$ with two binary operations between elements of the set such that the following rules hold true, which we shall henceforth call ring axioms:

\medskip
\begin{tabular}{l l}
$(R,+)$ forms a commutative group \\
$(R,\cdot)$ forms a semigroup \\
Multiplication distributes over addition: 
& $\forall a,b,c \in R: \;  
 a \cdot (b + c) = a \cdot b + a \cdot c, \; 
 (b + c) \cdot a = b \cdot a + c \cdot a$
\end{tabular}
\medskip

\paragraph{Notation and Terminology}
Here, a semigroup is a structure like a group but we do not necessarily require the existence of inverse elements and not even a neutral element. We now have not one but two operations, so we need to adapt our notation and terminology a bit to accommodate for that. We have called the operations "addition" and "multiplication" but it should be understood that we mean abstractions of these operations. We will denote the additive neutral element, formerly known as $e$, from now on as $0$. Note that we need two versions (left and right) of the distributive law because we do not in general require multiplication to be commutative. We do require commutativity for addition, though. If multiplication happens to commutative as well, we call our ring a commutative ring. The study of such commutative rings is called commutative algebra. If the ring is commutative, one of the distributivity laws is sufficient - the other one then follows from it and commutativity. In writing down the distributive laws, I also implicitly assumed that $\cdot$ takes precedence over $+$ such that no parentheses are needed for the right hand sides. For multiplication, we do not require the existence of a neutral element because we require our structure to be only a semigroup with respect to multiplication. By the way, a semigroup in which a neutral element exists is called a monoid. A ring in which a multiplicative neutral element exists, i.e. one in which the "semigroup" requirement is upgraded to a "monoid" requirement, is called a ring with unity or \emph{unital ring}. The multiplicative neutral element is then denoted by $1$. We still don't require multiplicative inverse elements. If, for some element $a$, a multiplicative inverse inverse does exist, we call $a$ a unit [VERIFY!]. So, the units of a ring are its (multiplicatively) invertible elements.


% $(R,\cdot)$ forms a semigroup, i.e. a group in which not all elements have inverses. It's actually even a monoid

% The additive neutral element
%  the additive inverse of $a$ is denoted as $-a$
%The nonzero rationals (or reals) also form a group under multiplication.



\begin{comment}
Explain the edge case where additive neutral elements have a multiplicative inverse. I think Michael Penn has a video about that called somethign like "when division by zero isn't ..." IIRC

Examples:
-Integers
-Modular integers with modular addition and multiplication
-Matrices with matrix addition and multiplication
-Polynomials

What about geometric examples? For group theory, we have a lot of geometric examples - reflectiosn, translations, rotations. I think, for geometric examples of rings, we need to look into algebraic geometry?

What is algebraic geometry? [by Aleph 0]
https://www.youtube.com/watch?v=MflpyJwhMhQ

Was sind Ringe? Was sind Körper? (Algebra) [Weitz]
https://www.youtube.com/watch?v=MAcdesa9RqA

https://en.wikipedia.org/wiki/Ring_(mathematics)
https://de.wikipedia.org/wiki/Ring_(Algebra)
https://en.wikipedia.org/wiki/Unit_(ring_theory)



 We have used the symbol $1$ to denote the multiplicative neutral element. 

The notation $a^{-1}$ is now used for the multiplicative inverse of $a$, so we need a new notation for the additive inverse. We will use $-a$ to denote the additive inverse of $a$. In the axiom about multiplicative inverses, we have excluded the $0$ element. The additive neutral element (i.e. zero) is the only element for which we do not require to have a multiplicative inverse. It can in fact be shown that - some uninteresting trivial edge cases aside - the additive neutral element cannot possibly have a multiplicative inverse in any ring. 


Closure under multiplication: 
& $\forall a,b \in R: \; a \cdot b \in R$  \\	
Multiplication is associative: 
& $\forall a,b,c \in R: \;  (a \cdot b) \cdot c = a \cdot (b \cdot c)$   \\

% Nope - that's an additional axiom for rings:
%Multiplicative inverse elements exist: 
%& $\forall a \in R \setminus \{0\} : \; (\exists a^{-1} \in R: a^{-1} \cdot a = 1 )$ \\

% Sometimes required:
%A multiplicative neutral element exists: 
%& $\exists 1 \in R: \; (\forall a \in R: 1 \cdot a = a)$ \\
% but sometimes, these are also called monoids (or ring with unity?)


\end{comment}