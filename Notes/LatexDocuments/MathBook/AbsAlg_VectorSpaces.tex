\section{Vector Spaces}
For a \emph{vector space}, we need two ingredients: \emph{scalars} and \emph{vectors}. The scalars are elements of a field $\mathbb{F}$, called the \emph{base field}, and the vectors can, for intuition building, be thought of as elements of $\mathbb{F}^n$ for some positive natural number $n$. It is possible to generalize this to things like $\mathbb{F}^\mathbb{F}$, the space of all functions from $\mathbb{F}$ to $\mathbb{F}$ but thinking about $\mathbb{F}^n$ is fine. To be even more concrete, you may imagine $\mathbb{R}^3$. But we are in the abstract algebra chapter here, so let's keep it general. Actually, even $\mathbb{F}^n$ is too concrete. We will instead just use $V$ to denote the set of vectors. ...TBC...


%===================================================================================================
\subsection{Subspaces} 
We have seen subgroups, subrings and subfields. It may not be surprising that we define a subspace of a given vector space in an entirely analoguous way. ...TBC...


%===================================================================================================
\subsection{Operations}


% Intersection
% Union - does not give a vector space, I think

\subsubsection{Direct Sum}

%https://en.wikipedia.org/wiki/Direct_sum


\subsubsection{Orthogonal Complement}

% https://en.wikipedia.org/wiki/Orthogonal_complement




\subsubsection{Quotient Space}
Let $V$ be a vector space and $U$ be a subspace of $V$. We partition the vector space $V$ into equivalence classes by saying that two vectors belong to the same equivalence class, when their difference is an element of a certain subspace $U$ of our original vectors space $V$. ...TBC...

% https://www.youtube.com/watch?v=d6ixmHDlrNo&list=PLdTL21qNWp2Z2iZOktAgucHyWMnJPA6eu&index=23&t=4s



\subsubsection{Implementation}
The operations on or between vector spaces, like taking an orthogonal complement or forming a direct sum, are very abstract ideas indeed. To get a more hands-on feeling for what is going on, I'll suggest a possible concrete (maybe naive) implementation of these operations. 

\paragraph{The Data Structure}
Firstly, we need a way to represent a vector space as a data structure. Every vector space has a basis, so it seems to make sense to represent a vector space by its basis. A basis is just a set of vectors. We will assume here that we represent a basis of a vector space as an $m \times n$ matrix $\mathbf{B}$ whose columns are considered to be our basis vectors. The matrix has $n$ columns, so we represent an $n$-dimensional space. The number $m$ - the number of rows of the matrix - can be any number $m \geq n$ but for the operations to be nontrivial, we'll want $m > n$. We may imagine $m$ to be the dimensionality of some embedding space. That is: we want to view our vector space of interest as a subspace of some possibly higher dimensional embedding space. This is because we want to implement operations like the orthogonal complement which is defined on a subspace with respect to a higher dimensional embedding space. 

%over the field $\mathbb{R}^n$ by an $n \times m$ matrix.

\paragraph{The Algorithms}
Given a vector space represented by its basis vectors stored in the rows of an $m \times n$ matrix $\mathbf{B}$, we now want compute a matrix $\mathbf{C}$ that represents a basis for the orthogonal complement of $\mathbf{B}$ with respect to the embedding space. That is, we want to compute a basis for $\mathbf{C} = \mathbf{B}^\perp$. The shape of $\mathbf{C}$ has to be $m \times k$ where $k = m - n$ because $\mathbf{B}$ represents a basis for an $n$-dimensional subspace of the $m$-dimensional embedding space, so the number of remaining dimensions is $m-n$. ...TBC...

% Give an algorithm that checks if some subspace V is a subspace of another subspace W. For example
% the embedding space may be U = R^3 (U for "universal space" or "universe"), W may be a plane 
% through the origina and V be a line thorugh the origin. It may or may not lie in the plane. The
% algorithm should figure that out.


\begin{comment}

-discuss operations like orthogonal complement, direct sum of vector spaces etc.
-maybe let Orthogonal complement and Direct sum be subsections, i.e. scrap the level "Operations"
-Implementation could also be a subsection
-Moduln: like a vector space but made from an underlying ring rather than a field

-Matrix Lie groups are (?):
 -Vector space under matrix addition (just like more general sets of matrices)
 -Group under matrix multiplication (that's the special property?)
	
\end{comment}