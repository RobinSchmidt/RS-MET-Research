\section{Physics}

%===================================================================================================
\subsection{Classical Mechanics}

%---------------------------------------------------------------------------------------------------
\subsubsection{Newtonian Mechanics}
Newtonian mechanics is the classical approach to classical mechanics discovered by the famous British physicist and mathematician Isaac Newton himself. Newtonian mechanics predicts, how a particle with mass $m$ will respond to the force $F$ exerted on it. It will respond by changing its state of motion. The response will be a \emph{change} of its \emph{velocity} $v$ according to the applied force. This is encapsulated in the fundamental law $F = m a$ where $a$ is the acceleration. Acceleration is literally defined as a change of velocity $v$: $a = \frac{dv}{dt}$. These scalar equations apply to motion in a single dimension. This may be good enough in some simple cases but in general, force, velocity and acceleration are generally treated as vector valued quantities (mass remains scalar, though). The appropriate mathematical framework that is used nowadays is that of \emph{vector calculus} although that field was not yet developed at Newton's time. He had to use different, more elementary formulations, i.e. he had to write out all the sums of partial derivatives explicitly where we today use more convenient differential operators which encapsulate such commonly occurring sums like in div, curl, etc. [VERIFY]. In the modern formulation of Newtonian physics, you will see a lot of such \emph{differential operators} to compute one vector valued quantity from another at a point. You will also see a lot of \emph{line integrals}, for example to compute the energy that a particle gains or loses when it moves through a force field. You will use the theorem for \emph{potential fields} when the force field happens to be conservative, which it often is. You will see \emph{volume integrals}, for example to compute a total mass and center of mass of an object given its density as function of $x,y,z$. You will also see a lot of systems of \emph{ordinary differential equations}. That's the sort of math that is used in Newtonian mechanics. Mostly calculus.

...TBC...

\paragraph{Energy and Momentum}

\paragraph{Linear Motion}
% energy and momentum

\paragraph{Rotational Motion}

\paragraph{Gravitation}

\paragraph{Particle Systems}
% chaotic for numObjects >= 3, applicable to solar systems

% maybe use also vector 
% give vector analysis equations for rotation

% math: vector analysis: differential operators, line integrals (for energy accrued on trajectory in force field), volume integrals (for total mass wehn density is given)


%---------------------------------------------------------------------------------------------------
\subsubsection{Lagrangian Mechanics}
Newtonian mechanics revolves around the equation $F = m a$ or $\mathbf{F} = m \mathbf{a}$ in multiple dimensions. In order to be applicable to predict the evolution of a given system, we typically need to be able to calculate what the the total force vector $\mathbf{F}$ is. In some problem settings, it is hard to come by with an explicit expression for the force $\mathbf{F}$. [VERIFY] In such cases, it might be easier to choose an approach that involves the kinetic and potential energies rather than the forces. Kinetic energy is the energy stored in the motion of an object whereas potential energy is energy that is stored in the position of an object. ...TBC...

% a higher level approach from which the Newtonian equations emerge. is it really a "higher level" or just a different approach?
% especially applicable to probelms with constraint forces?



%
% kinetic: energy:  motion energy
% potential energy: positional energy

% math: calculus of variations ...was that already developed at Lagrange's time?

%---------------------------------------------------------------------------------------------------
\subsubsection{Hamiltonian Mechanics} [VERIFY ALL]
In Lagrangian mechanics, the central quantity of interest was the so called Lagrangian of a system which is the kinetic energy \emph{minus} the potential energy. In Hamiltonian mechanics, the quantity of interest is the kinetic energy \emph{plus} the potential energy, i.e. the total energy. This total energy is also called the Hamiltonian in this context. ...TBC...

\paragraph{The Legendre Transformation}
The recipe to convert back and forth between the Lagrangian and the Hamiltonian formulation of a given problem is called the Legendre transformation (which should not be confused with the Legendre transform - a certain type of integral transform). ...TBC...

% https://en.wikipedia.org/wiki/Legendre_transformation
% https://de.wikipedia.org/wiki/Legendre-Transformation

% https://en.wikipedia.org/wiki/Legendre_transform_(integral_transform)




%===================================================================================================
\subsection{Fluid Dynamics}
% fluid dynamics: complex calculus


%===================================================================================================
\subsection{Electrodynamics}

% math: vector analysis: integral theorems (surface integrals in particular)
% complex analysis ()for circuit theory)


%===================================================================================================
\subsection{Quantum Mechanics}
Quantum mechanics deals with the world of atomic and subatomic particles. It turns out that their behavior is not adequately described by classical mechanics. Classical mechanics describes the behavior of point like particles in a deterministic way. This is not how such small particles behave. Quantum mechanics describes the behavior of spatially smeared out particles in a probabilistic way. The main tool is the so called \emph{quantum wavefunction}. This is a complex valued (!) function of space and time and typically denoted by the greek letter psi as $\psi = \psi(t,x,y,z)$. There is a law called the \emph{Born rule} that allows us to convert the complex values of the wavefunction into probabilities of observing the particle in a particular \emph{state}, e.g. at a particular position. The Schroedinger equation is a partial differential equation that determines the time evolution of the wavefunction $\psi$. And "determines" is to be taken literally here - the time evolution of the wavefunction is indeed entirely deterministic. It is just the \emph{observation} that is probabilistic, i.e. the probabilistic aspect kicks in only at the point of making an observation, i.e. at the moment when we "apply" the Born rule. There are some very deep philosophical questions with regard to what constitutes such a moment, but I won't get into these here (search for "measurement problem" if you want to read more about this). The "quantum" in quantum mechanics comes from the fact that the states in which a quantum system can be observed do not lie on a continuum. In quantum mechanics, any given physical system has a finite or countably infinite set of states that it can be found in. The observable states do not lie on a continuum as in classical mechanics but are discrete or \emph{quantized}. The main mathematical toolkit for quantum mechanics is the machinery of linear algebra - but it will often be the infinite dimensional version of it, i.e. the theory of linear operators that we have met in the section about functional analysis. Our \emph{state vectors} will often be continuous functions and our matrices will be operators that act on on these functions. That, together with the fact that we must deal with complex valued vectors or functions and that there is a very special notation in use, makes it all a bit intimidating at first - but essentially it's mostly just linear algebra.
...TBC...

% have the matrices also a name? transition matrcies? evolution matrcies ? tiem evolution operator?
% https://en.wikipedia.org/wiki/Time_evolution

\subsubsection{The Schroedinger Equation}
The  Schrodinger Equation is to quantum mechanics what $F = m a$ is to Newtonian mechanics. It is the main governing equation that determines how the system will evolve over time. It can be stated as:
\begin{equation}
 \psi_{t} = \frac{1}{\i \hbar} H \psi 
 \qquad \text{where} \qquad
 H = ...
\end{equation}
This is an uncommon way to write it down. I like to put the time derivative unadorned to the left hand side such that the right hand side is an explicit prescription for "what comes next". This is how we would "implement" the equation in a computer program for a numerical solver and this is the way I think about things. ...TBC...

% https://en.wikipedia.org/wiki/Schr%C3%B6dinger_equation
% https://en.wikipedia.org/wiki/Probability_current

% maybe use vector notation for psi and matrix for H
% 

\paragraph{Bras and Kets}
The form of the Schroedinger equation uses the notation for partial differential equations that we have introduced in the section about them using subscript $t$ for a derivative with respect to $t$. This looks different from what you will typically find in physics textbooks. There, the preferred notation for quantum mechanics is the so called bra-ket notation invented by Paul Dirac. In this notation, normal vectors are denoted as so called "ket" vectors

% write it in bra-ket notation

% https://en.wikipedia.org/wiki/Bra%E2%80%93ket_notation

\subsubsection{The Born Rule}


% time independent schroedinger equation


%The do not behave in a deterministic way like classical particles do but rather in a probabilistic way. This behavior is captured in a mathematical object called the quantum wavefunction which is the primary tool to describe the particle

% they are also not point-like but rather spread out in space

% https://en.wikipedia.org/wiki/Wave_function

% https://en.wikipedia.org/wiki/Measurement_problem

% Schroedinger eqaution descirbes time evolution of psi
% Born rule describes how probabilities are derived from the wavefunction
% https://en.wikipedia.org/wiki/Born_rule

% The function can be observed only in one of the eigenstates. I think, we must form the scalar product of the state with the eigenstate to get the probability to find the system in that eigenstate?

% math: Theory of linear operators, probability, Hamiltonian is also important.
% wave-particle duality

%===================================================================================================
\subsection{Statistical Mechanics and Thermodynamics}
This subfield of physics deals with the application of Newtonian (and sometimes quantum) mechanics to particle systems with a number of particles that is so huge, that it's impossible and also not meaningful to try to describe what individual particles do. Instead, one considers the properties of the whole ensemble of particles as a whole. Notions like density, pressure and temperature are developed as \emph{emergent} properties of the whole ensemble. The mathematical tools that are required for this are the ideas from probability theory and statistics. ...TBC...

% emergent laws
% math: probability, statistics?

% https://en.wikipedia.org/wiki/Thermodynamics
% https://en.wikipedia.org/wiki/Laws_of_thermodynamics

% https://en.wikipedia.org/wiki/Statistical_mechanics
% https://www.britannica.com/science/statistical-mechanics

%===================================================================================================
\subsection{Continuum Mechanics}

% describes how continuous solid (elastic) media respond to forces by deformation

% math: tensor calculus, partial differential equations




%===================================================================================================
\subsection{Relativity}

% math: general: tensor calculus, differential geometry; special: hyperbolic numbers/geometry


%\subsection{Elementary Particle Physics}
% String theor, M-theory 
% math: group theory? symmetry? maybe even number theory? the zeta(-1) = -1/12 occurs somewhere in string theory, I think.





\begin{comment}
List branchs of physics and which areas of math are especially important in those:
-Classical Mechanics:
 -Differential Equations (ordinary and partial)
-Lagrangian and Hamiltonian Mechanics:
 -Calculus of Variations
-Electrodynamics:
 -Vector Calculus
-Fluid Dynamics: 
 -Vector calculus
 -Complex Analysis (Rimeann mapping, conformal mapping)
-Continuum Mechanics:
 -Tensor Calculus
-Quantum Mechanics:
 -Linear Algebra
  -Unitary and Hermitian matrices
  -Eigenvalues and -vectors
 -Group Theory:
  -Matrix Lie-groups (Pauli, Dirac, etc.)
 -Operator Theory
-Relativity:
 -Tensor calculus
 -Differential geometry
 -Group theory (Lorentz group, Poincare group, etc.)
-Thermodynamics / Statistical Mechanics: Probability theory (verify!)
-Cosmology: 
 -Differential geometry
 -Information theory (verify!), 
 -Topology (verify!)
-String Theory / M-Theory:
 -Manifolds (diff. geo.)
 -Knot Theory (?)
 -Number Theory
 
 
-Data Science:
 -Multivariable Calculus, Optimization
 -Tensor Algebra
 
 Lagrange Mechanik verstehen! - Lagrange Funktion, Euler-Lagrange Gleichung (Physik)
https://www.youtube.com/watch?v=Uk24tKZUq1M&list=PLdTL21qNWp2YiZaBF9xMb82kSpBc3YnxQ&index=71 
 
----------------------------------------------------------------------------------------------------
General Relativity:

Famous quote by physicist John Archibald Wheeler about the essence of general relativity:

"Spacetime tells matter how to move, matter tells spacetime how to curve."

Analogy in Newton's world:
Forces tell masses how to move (accelerate): F = m a, a = F/m. Masses tell (gravitational) forces how to pull: F_g = G m_1 m_2 r / |r|^3. Maybe Newton's law should be rephrased in terms of momentum to make the analogy even closer: d p / d t = F

In relativity: The change of the energy-mmomentum tensor is some function of the metric tensor - in a sloppy symbolic way: T_{\mu \nu} = f( g_{\mu \nu} ) and likewise g_{\mu \nu} = f^{-1} (T_{\mu \nu}). But actually the function $f$ also depend on spatial and temporal derivatives of the (tensor valued) quantity $g$, not just on $g$ itself - although, if we assume $g = g(t,x,y,z)$ then that function actually contains all the info abouts its (partial) derivatives anyway. But usually, when we write down a differential equation, we make such dependencies on derivatives epxlicit by writing things like: $f(g, g_t, g_x, g_y, g_z, g_{tt}, g_{tx}, g_{ty}, \ldots)$

https://en.wikipedia.org/wiki/Einstein_field_equations


https://www.youtube.com/watch?v=KW4yBSV4U38
at 7:50, she says, "they are differential equations for the metric tensor". So, the focus seems to be on the time evolution of g, not T. The equations by themselves describe both on equal footing (I think)

 
 

\end{comment}