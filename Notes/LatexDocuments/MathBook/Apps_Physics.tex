\section{Physics}

%===================================================================================================
\subsection{Classical Mechanics}

%---------------------------------------------------------------------------------------------------
\subsubsection{Newtonian Mechanics}
Newtonian mechanics is the classical approach to classical mechanics discovered by the famous British physicist and mathematician Isaac Newton himself. Newtonian mechanics predicts, how a particle with mass $m$ will respond to the force $F$ exerted on it. It will respond by changing its state of motion. The response will be a \emph{change} of its \emph{velocity} $v$ according to the applied force. This is encapsulated in the fundamental law $F = m a$ where $a$ is the acceleration. Acceleration is literally defined as a change of velocity $v$: $a = \frac{dv}{dt}$. These scalar equations apply to motion in a single dimension. This may be good enough in some simple cases but in general, force, velocity and acceleration are generally treated as vector valued quantities (mass remains scalar, though). The appropriate mathematical framework that is used nowadays is that of vector analysis although that field was not yet developed at Newton's time. He had to use different, more elementary formulations, i.e. he had to write out all the sums of partial derivatives explicitly where we today use more convenient differential operators which encapsulate such commonly occurring sums like in div, curl, etc. [VERIFY].

...TBC...

\paragraph{Linear Motion}

\paragraph{Rotational Motion}

% maybe use also vector 
% give vector analysis equations for rotation

% math: vector analysis: differential operators, line integrals (for energy accrued on trajectory in force field), volume integrals (for total mass wehn density is given)


%---------------------------------------------------------------------------------------------------
\subsubsection{Lagrangian Mechanics}
Newtonian mechanics revolves around the equation $F = m a$ or $\mathbf{F} = m \mathbf{a}$ in multiple dimensions. In order to be applicable to predict the evolution of a given system, we typically need to be able to calculate what the the total force vector $\mathbf{F}$ is. In some problem settings, it is hard to come by with an explicit expression for the force $\mathbf{F}$. [VERIFY] In such cases, it might be easier to choose an approach that involves the kinetic and potential energies rather than the forces. Kinetic energy is the energy stored in the motion of an object whereas potential energy is energy that is stored in the position of an object. ...TBC...

% a higher level approach from which the Newtonian equations emerge. is it really a "higher level" or just a different approach?
% especially applicable to probelms with constraint forces?



%
% kinetic: energy:  motion energy
% potential energy: positional energy

% math: calculus of variations ...was that already developed at Lagrange's time?

%---------------------------------------------------------------------------------------------------
\subsubsection{Hamiltonian Mechanics} [VERIFY ALL]
In Lagrangian mechanics, the central quantity of interest was the so called Lagrangian of a system which is the kinetic energy \emph{minus} the potential energy. In Hamiltonian mechanics, the quantity of interest is the kinetic energy \emph{plus} the potential energy, i.e. the total energy. This total energy is also called the Hamiltonian in this context. ...TBC...

\paragraph{The Legendre Transformation}
The recipe to convert back and forth between the Lagrangian and the Hamiltonian formulation of a given problem is called the Legendre transformation (which should not be confused with the Legendre transform - a certain type of integral transform). ...TBC...

% https://en.wikipedia.org/wiki/Legendre_transformation
% https://de.wikipedia.org/wiki/Legendre-Transformation

% https://en.wikipedia.org/wiki/Legendre_transform_(integral_transform)


%===================================================================================================
\subsection{Electrodynamics}

% math: vector analysis: integral theorems (surface integrals in particular)


%===================================================================================================
\subsection{Quantum Mechanics}

% math: Theory of linear operators, probability, Hamiltonian is also important.

%===================================================================================================
\subsection{Thermodynamics}

% math: probability, statistics?

%===================================================================================================
\subsection{Relativity}

% math: tensor analysis, differential geometry


%\subsection{Elementary Particle Physics}
% String theor, M-theory 
% math: group theory? symmetry? maybe even number theory? the zeta(-1) = -1/12 occurs somewhere in string theory, I think.





\begin{comment}
List branchs of physics and which areas of math are especially important in those:
-Classical Mechanics:
 -Differential Equations (ordinary and partial)
-Lagrangian and Hamiltonian Mechanics:
 -Calculus of Variations
-Electrodynamics:
 -Vector Calculus
-Fluid Dynamics: 
 -Vector calculus
 -Complex Analysis (Rimeann mapping, conformal mapping)
-Continuum Mechanics:
 -Tensor Calculus
-Quantum Mechanics:
 -Linear Algebra
  -Unitary and Hermitian matrices
  -Eigenvalues and -vectors
 -Group Theory:
  -Matrix Lie-groups (Pauli, Dirac, etc.)
 -Operator Theory
-Relativity:
 -Tensor calculus
 -Differential geometry
 -Group theory (Lorentz group, Poincare group, etc.)
-Thermodynamics / Statistical Mechanics: Probability theory (verify!)
-Cosmology: 
 -Differential geometry
 -Information theory (verify!), 
 -Topology (verify!)
-String Theory / M-Theory:
 -Manifolds (diff. geo.)
 -Knot Theory (?)
 -Number Theory
 
 
-Data Science:
 -Multivariable Calculus, Optimization
 -Tensor Algebra
 
 Lagrange Mechanik verstehen! - Lagrange Funktion, Euler-Lagrange Gleichung (Physik)
https://www.youtube.com/watch?v=Uk24tKZUq1M&list=PLdTL21qNWp2YiZaBF9xMb82kSpBc3YnxQ&index=71 
 
----------------------------------------------------------------------------------------------------
General Relativity:

Famous quote by physicist John Archibald Wheeler about the essence of general relativity:

"Spacetime tells matter how to move, matter tells spacetime how to curve."

Analogy in Newton's world:
Forces tell masses how to move (accelerate): F = m a, a = F/m. Masses tell (gravitational) forces how to pull: F_g = G m_1 m_2 r / |r|^3. Maybe Newton's law should be rephrased in terms of momentum to make the analogy even closer: d p / d t = F

In relativity: The change of the energy-mmomentum tensor is some function of the metric tensor - in a sloppy symbolic way: T_{\mu \nu} = f( g_{\mu \nu} ) and likewise g_{\mu \nu} = f^{-1} (T_{\mu \nu}). But actually the function $f$ also depend on spatial and temporal derivatives of the (tensor valued) quantity $g$, not just on $g$ itself - although, if we assume $g = g(t,x,y,z)$ then that function actually contains all the info abouts its (partial) derivatives anyway. But usually, when we write down a differential equation, we make such dependencies on derivatives epxlicit by writing things like: $f(g, g_t, g_x, g_y, g_z, g_{tt}, g_{tx}, g_{ty}, \ldots)$

https://en.wikipedia.org/wiki/Einstein_field_equations


https://www.youtube.com/watch?v=KW4yBSV4U38
at 7:50, she says, "they are differential equations for the metric tensor". So, the focus seems to be on the time evolution of g, not T. The equations by themselves describe both on equal footing (I think)

 
 

\end{comment}