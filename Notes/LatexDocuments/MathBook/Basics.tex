\chapter{Basics}
Mathematics is a broad subject and I would be way out of my depth to try to give a definition what it actually is. I like to think of it as the study of structures, patterns and equivalences and as a way to systematize, formalize and eventually automate thought processes. One big theme is to figure out, under which circumstances one thing is equal to another. Often, there are multiple ways to compute the same thing and a big sub-theme of finding equivalences is to find computational shortcuts that allow to do a computation more efficiently than was previously possible. The body of mathematical knowledge today is an impressive tower of known facts about abstract constructs of the human mind. The first of these abstract constructs that one usually encounters in elementary school is the idea of a natural number and one learns how to add, subtract, multiply and divide them. Building on that, one later encounters negative numbers, rational numbers, real numbers and complex numbers. I like to think of numbers as the ground floor of the tower. Numbers usually set the stage for doing mathematics on first encounter, but the foundations of mathematics can go down to even lower levels: If numbers are the ground floor, then it may be appropriate to think of set theory and mathematical logic as two basement floors. Having numbers in place, one can go up a stage and look at functions that map numbers to other numbers. One then may realize that addition, multiplication, etc. are actually functions, too: they take two inputs and map them to a single output. Such generalization with hindsight is also a common theme in math. A stage higher, you can look at mappings that map functions to other functions. And it goes ever higher up. Well, actually, it also kind of branches out while going up, so maybe a tree of knowledge might be better analogy than a tower - but then the roots (levels below ground) do not branch out as much as the upper levels. But there certainly is some sort of trunk that everybody needs to know. This trunk contains numbers themselves (from the natural to the complex ones), equations (and solution techniques for them when they contain unknowns), elementary functions (polynomial, rational, exponential, trigonometric and their inverses), linear algebra (vectors, matrices, linear systems of equations) and single variable calculus (derivatives, integrals, differential equations). With these basics in place, math branches out into various directions. Some of these are: multivariable calculus: the study of functions of several variables, abstract algebra: generalizes ideas such as addition, multiplication, etc. to other sorts of objects and identifies the common structure, number theory: studies natural numbers and especially prime numbers in depth, functional analysis: studies functions of functions, topology: studies qualitative properties of shapes, etc. 

\paragraph{Goals and Audience}
This document is an attempt to give a condensed high level overview about what's going on in a particular subject and to give a comprehensive catalog of formulas, recipies and algorithms to actually get the work done. There is little regard to derivation or justification and no regard whatsoever to proof or mathematical rigor. It's meant to be a collection of recipies for the practitioner who needs to use math in applications. In focus are the questions: What is it? What can I do with it? How can I do it? The focus is deliberately not: Why does it work the way it does? That would fill volumes (and has done so) and thereby just distract from getting the work done. The scope is broad but shallow. I don't want to drown the reader in details of derivations. If you are looking for a detailed, in depth understanding, you will need to consult actual math textbooks. This book here should serve more as a launchpad and give you the right keywords to serach for.  Despite being shallow with regard to derivations (and therefore understanding), I strive to be comprehensive with regard to listing potentially important formulas and recipes. It's more like a cheat-sheet. I try to minimize using forward references, i.e. references to material that is only covered in later chapters. But in some cases, these are inavoidable. The body of math knowledge is actually even more complicated than a tree. Past the bottom layers, it's more like a vast interconnected network where everything hangs together, so putting it in a strict linear order from bottom to top is difficult. I'll try nonetheless. The material is organized as follows: Part 1 deals with basics, linear algebra, calculus and geometry - roughly speaking, the world of continuous mathematics. Part 2 is devoted to discrete mathematics... Part 3 is devoted to applications.... tbc...



% Latex output is a bit ugly - only one line on a full page!
