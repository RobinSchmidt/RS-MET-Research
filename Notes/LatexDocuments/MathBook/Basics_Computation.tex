\section{Computation}
No that we have numbers, we want to do something with them. That something is computation ...tbc...

\subsection{Arithmetic}
Arithmetic is about performing the elementary operations of addition, subtraction, multiplication and division and on a higher level also exponentiation and extraction of roots. I would argue that when taking roots is included, taking logarithms should be included as well for consistency (both operations are ways of inverting the exponentiation), but that's usually not listed among the arithmetic operations. Be that as it may, we will look at logarithms in this section, too. We will also learn sum and product notation, the factorial and binomial coefficients...tbc...



% Maybe mention also, how the operatioons succesion, addition, multiplication, and exponentiation
% build up on each other and how this could continue with tetration etc.
% https://www.youtube.com/watch?v=u1x_FJZX6Vw
%
% Explaing infinte sums, products and continued fractions
%
% In a section about mathematical constants, explain how formulas for the they can be generated in
% a systematic way using continued fractions and the conservative matrix field. this should be a
% an optional section with asterisk



\begin{comment}

ToDo:

-introduce positional number systems, in particular decimal and maybe binary as alternative
-give algorithms for long addition, subtraction, multiplication, division
-maybe also for numbers in scientific notation, i.e. floating point numbers
-introduce sum and product notation
 -maybe with a spoiler to infinite sums, use $\sum_{k=1}^{\infty} (1/10)^k = 0.1111... = 1/9$ as example
-introduce sums, products, factorials and binomial coefficients
-but maybe that stuff should go into the "Elementary Algebra" section because it involves 
 variables

References:

https://en.wikipedia.org/wiki/Arithmetic
https://www.britannica.com/science/arithmetic
https://en.wikipedia.org/wiki/Positional_notation
https://en.wikipedia.org/wiki/Mixed_radix


\end{comment}


%\subsection{Elementary Algebra}
%Elementary algebra is about solving an equation for an unknown variable, typically denoted as $x$. This is done by isolating $x$ on one side of the equation and moving all known quantities to the other side. On this other side, no unknown quantity should occur anymore such that it can be directly evaluated. ...tbc...
% example 2 x + 3 = 9 -> 2 x = 6 -> x = 3
% Really ...is it? Or is elementary algebra more generally "computation with letters"?
% ...yep...wikipedia says so


\begin{comment}

ToDo:
-make a (sub)section "Algebraic Reasoning" analogous to the "Geometric Reasoning" section in the Geometry chapter, so we can refer from there to here for analogies

References:

https://en.wikipedia.org/wiki/Elementary_algebra
https://en.wikipedia.org/wiki/Equation

Factoring a sum of two cubic terms into a linear and quadratic factor: 
  a^3 + b^3 = (a + b) (a^2 - a b + b^2)
Can be put near the binomial formulas

Maybe this whole section should be called Elementary Algebra.
This video says (at 5:25):
https://www.youtube.com/watch?v=FQYOpD7tv30
(elementary) algebra is the study of abstraction applied to numbers. 

One could perhaps also say: algebra is computing with letters ("Rechnen mit Buchstaben"), 
deriving general formulas with placeholders into which we can plug in actual values (numbers).
It's also about algorithms to compute an unknown quantity from known quantities (think 
polynomial division, Gaussian elimination, partial fraction expansion). In this context, a 
formula can be seen as a very simple algorithm. A formula usually translates to a single 
assignment operation in a computer program. ..well, some formulas are a bit more complicated.
The quadratic formula spits out two values due to the +- sqrt..., so it would translate to
two assignments (or more when we use intermediate vars). But something like Pythagoras' 
theorem just translates to the assignment: c = sqrt(a*a + b*b) when we assume that c is the
unknown. A formula like lhs = rhs can often be directly translated into an assignment 
operation which, in the context of algorithms, can be seen as a simple mini "algorithm" made
from just a single assignment. If a is unknown, we would translate Pythagoras to 
a = sqrt(c*c - b*b), so one formula may potentially translate to multiple different of such
mini-algorithms. More complicated algorithms may contain branches and loops. In math notation
branches can be expressed in the form of piecewise defined right hand sides and loops in the
form of sequences. These sequences may actually be infinite in which case we step into
calculus territory. An example would be the Babylonian algorithm to compute a square-root. 
It's defined via an infinite sequence. In this specific case, the sequence is defined
recursively. In general sequences may also be defined in other ways as well (for example,
explicitly...i.e...by a formula for the elements).


What other types of "Computation" could we have? Maybe I should explain the basic algorithms 
for hand-calculations with pen and paper?

\end{comment}


