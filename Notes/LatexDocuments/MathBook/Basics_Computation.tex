\section{Computation}
No that we have numbers, we want to do something with them. That something is computation ...tbc...

\subsection{Arithmetic}
Arithmetic is about performing the elementary operations of addition, subtraction, multiplication and division and on a higher level also exponentiation and extraction of roots. I would argue that when taking roots is included, taking logarithms should be included as well for consistency (both operations are ways of inverting the exponentiation), but that's usually not listed among the arithmetic operations. Be that as it may, we will look at logarithms in this section, too. We will also learn sum and product notation, the factorial and binomial coefficients...tbc...


\begin{comment}

ToDo:

-introduce positional number systems, in particular decimal and maybe binary as alternative
-give algorithms for long addition, subtraction, multiplication, division
-maybe also for numbers in scientific notation, i.e. floating point numbers
-introduce sum and product notation
 -mayby with a spoiler to infinite sums, use $\sum_{k=1}^{\infty} (1/10)^k = 0.1111... = 1/9$ as example
-introduce sums, products, factorials and binomial coefficients

References:

https://en.wikipedia.org/wiki/Arithmetic
https://www.britannica.com/science/arithmetic
https://en.wikipedia.org/wiki/Positional_notation
https://en.wikipedia.org/wiki/Mixed_radix


\end{comment}


\subsection{Elementary Algebra}
Elementary algebra is about solving an equation for an unknown variable, typically denoted as $x$. This is done by isolating $x$ on one side of the equation and moving all known quantities to the other side. On this other side, no unknown quantity should occur anymore such that it can be directly evaluated. ...tbc...
% example 2 x + 3 = 9 -> 2 x = 6 -> x = 3


\begin{comment}

ToDo:
-make a (sub)section "Algebraic Reasoning" analogous to the "Geometric Reasoning" section in the Geometry chapter, so we can refer from there to here for analogies

References:

https://en.wikipedia.org/wiki/Elementary_algebra
https://en.wikipedia.org/wiki/Equation

Factoring a sum of two cubic terms into a linear and quadratic factor: 
  a^3 + b^3 = (a + b) (a^2 - a b + b^2)
Can be put near the binomial formulas

\end{comment}


