\section{Foundations} 

This book is about applied math, so we will not go deeply into the foundations of math. However, a brief and very superficial look into this topic is a good idea to set the stage for the material that follows. It also establishes the basics of the language which we will need to talk about mathematical concepts.

\subsubsection{Logic}
Math is the pursuit of finding truths, so it makes sense to have a framework, within which we can say that something is true or false. In mathematics, that framework is mathematical logic, more specifically propositional logic and predicate logic (a.k.a. first order logic). Propositional logic deals with propositions which are statements that can be either true or false. You also have ways of combining given propositions to make new, more complex propositions. For example, you can combine two propositions with a logical "and" (usually denoted as $\wedge$). The resulting new proposition is true, if and only if both of the input propositions are true, otherwise it's false. You also have a logical "or" (denoted as $\vee$), which in this context is taken to be an inclusive or: the combined proposition is true, if any one of the input propositions or both are true. You also have a logical "not" (denoted as $\neg$) which takes a single proposition as input and the result is true, if the the input is false and vice versa. To build up mathematics, propositional logic is not quite enough. Predicate logic builds up on propositional logic and lets you talk about objects and relations between them. There, you have so called quantors like the symbol for "there exists an object such that..." (denoted as $\exists$) or a symbol for "for all objects it is true that..." (denoted as $\forall$). Logic also provides the tools that are required to figure out whether a given proposition is true, given that some other propositions are true. That process of drawing conclusions from given (true) propositions is called deduction. There are yet other levels and kinds of logic, but these two are enough for the moment.

\subsubsection{Axioms}
One has to start somewhere. That starting point is typically a set of "axioms" together with the rules of logic. An axiom is a proposition that is just assumed to be true without further justification. Axioms should state things that are "obviously true". An example are the Peano axioms, some of which are: zero is a natural number, each natural number has a successor, any number is equal to itself, etc. If you really want to build up the whole tower of mathematics axiomatically, you have to \emph{choose} a set of axioms and from there, using only the rules of logic, i.e. deduction, find new propositions that are also true.

\subsubsection{Theorems and Proofs}
If you want to prove a proposition, the tools that you have in hand are all the propositions that are already known to be true together with the rules of logic. A proposition is known to be true if it is either an axiom or it has been previously proven by the same technique. A proof for a proposition is a sequence of true propositions in which each one follows from known or previous ones by applying the rules of logic and the last of which is the one you actually wanted to prove in the first place. If a proposition has been proven to be true, it becomes a "theorem". Thie idea of a theorem is a fundamentally important concept in math - math is all about finding theorems. You may have observed a pattern by looking at a bunch of examples and you may conjecture that the pattern is generally true. What you then have to do is to find a proof for your conjecture. If you have succeeded in this highly creative endeavour (i.e. your proof is determined to be correct by the mathematical community), your conjecture is elevated to the venerable status of a theorem. And if the theorem is important enough and you were the first to prove it, you will typically achieve immortality by having your name attached to the theorem for the rest of eternity. Thousands of years later, still everbody knows the name of Pythagoras today - although, it wasn't actually him who proved "Pythagoras' Theorem" - sometimes the world is a little unfair, too :'-(. Along the way of finding a proof, you may generate a whole bunch of proven propositions, some of which are only instrumental to your final goal, some of which are spinoffs, etc. There are some other terms for such "lesser theorems" such as "lemma", "corollary", etc. A theorem is usually a result with a certain level of importance, generality and usefulness. You wouldn't call something like $3+5=8$ a theorem, for example - although it manifestly is a true proposition (and can actually be proven).

\subsubsection{Definitions}
OK - this is kind of awkward. We now have to \emph{define} what \emph{a definition} is. A definition is actually just an agreement about certain conventions to be used in the following material, in particular about what a given term or symbol is supposed to mean. Definitions often stand at the beginning of the development of a new subject. Definitions cannot be right or wrong. They can just be more or less useful. For a definition to be useful, it should clearly encapsulate a concept that is important in the development of all the things further down the line. The so defined term or symbol shall be used a lot in the material to be developed and will be referred to often. It makes sense to pick definitions in such a way that theorems can be stated succinctly. What the most useful definitions for a particual (new) mathematical subject are is often not clear from the get go but instead crystallizes out over time as experience with the new subject grows and when a bit of hindsight is available. As users of math, we may take definitions for granted because smart mathematicians have already figured (and fought) them out for us (and for themselves, of course). But we should keep in mind that they are fundamentally just conventions to make it possible (and ideally convenient and easy) to talk about a given subject. They are not fundamental truths. They just establish the language that we will use. That's why sometimes different authors use different definitions. Sometimes there is just no universal consenus (yet or ever) about which definitions are the most useful ones. Which ones are more or less useful may also differ from field to field. So, care has to be taken when reading mathematical material from different sources - the definitions in use may not always agree.

\subsubsection{Set Theory}
Set theory is often said to be the foundation of all mathematics - even much more fundamental than the natural numbers. In fact, it is possible to "construct" the natural numbers from sets. We will not go down this road though, since this is not really relevant in applied math. The idea of a set was initially introduced by Georg Cantor in an intuitive way. His way of establishing set theory later turned out to have some flaws which is why it was later rebuilt more formally. The result of this rebuild is called "axiomatic set theory" and is very abstract and formal. Fortunately, Cantor's view, which is today sometimes called "naive set theory", is good enough for us. 

\paragraph{Sets}
In Cantor's definition "A set is a gathering together into a whole of definite, distinct objects of our perception or of our thought which are called elements of the set.". So, in essence, a set is just a bunch of things. A very general concept indeed. Sets are usually denoted in curly braces. For example, the set of the 3 letters a,b,c would be denoted as $\{a,b,c\}$. Two sets are considered equal, if and only if they contain the same elements. It does not matter in which order the elements are written down or if an element appears multiple times. So that means, for example, the sets $\{c,a,b\}$ or $\{a,c,a,a,b,c\}$ are in fact equal to the set $\{a,b,c\}$. By the way, the phrase "if and only if" appears sufficiently often in math texts that some authors use the abbreviation "iff" for that - yes, that's an "if" with a double-f. Sets can be given names, for example, we may call our set above $S$ and we may write this as $S = \{a,b,c\}$. Element membership is denoted by an $\in$ symbol, so to express the fact that $b$ is an element of the set $S$, we would write $b \in S$. 

\medskip 
Sets can have other sets as elements and that nesting capability can be used recursively to build arbitrarily complex structures purely from sets. These structures also include the number systems that are used in math. For example, the number zero can be represented by the empty set: $0 = \{\}$, which is also denoted by $\emptyset$, the number one by the set that contains the empty set (i.e. zero): $1 = \{ 0 \} =  \{ \emptyset \}$, the number two by the set that contains zero and one: $2 = \{ 0, 1 \} = \{ \emptyset, \{ \emptyset \} \}$ and so on. Of course, that's super tedious and nobody actually thinks about numbers this way - but in principle, it can be done.

\medskip 
In math, the sets we are dealing with are often sets of numbers and they may have many or even infinitely many elements. To denote very large or infinite sets compactly there are notations based on predicate logic. For example to denote the set of all numbers larger than 100 but less than 1000, we may write $\{x : x > 100 \wedge x < 1000\}$, but now we are getting ahead of ourselves. To understand that notation, we actually first have to understand what $>$ and $<$ means. I'm pretty sure, you already do know what they mean, but in the context of set theory, these symbols first needs to be defined, too. To do so, we need some more tools...

\paragraph{Tuples}
Sometimes, we may want to model situations in which the order of elements actually does matter. Sets are per definition not suitable for this (at least not directly), so we need something else. That other thing is the tuple. A tuple is typically denoted by listing the elements in parentheses. The 3-tuple $(a,b,c)$ is not equal to $(c,a,b)$. Tuples with two elements are also called ordered pairs, 3-tuples triples, 4-tuples quadruples and 5-tuples quintuples.

\medskip 
Sidenote: If you really want to be puristic and build \emph{everything} from sets, you can use sets to model tuples, too: you could just pack the tuple members into another set with another object that serves as index, so $(a,b,c)$ would become $\{\{a,1\},\{b,2\},\{c,3\}\}$ and $(c,a,b)$ would become  $\{\{c,1\},\{a,2\},\{c,3\}\}$. These sets of sets would indeed be uniquely indentified purely by their elements regardless of order because the correct order could be reconstructed due to the 2nd element in the inner sets wich serve as tags. There are other ways to model tuples purely via sets, too.

% https://en.wikipedia.org/wiki/Ordered_pair#Defining_the_ordered_pair_using_set_theory

\paragraph{Relations}
A relation can formally be defined to be a set of tuples. Of particular importance are binary relations, i.e. sets of 2-tuples, aka ordered pairs. ...tbc...

% set algebra
%   Relations, Functions, cartesian product, intersection, union, difference, subset, etc.