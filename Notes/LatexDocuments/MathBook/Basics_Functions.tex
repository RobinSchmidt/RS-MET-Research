\section{Functions}
Functions are like machines that take some input and produce some output. In the case of mathematical functions, these inputs and outputs are usually numbers. There are more general notions, taking other types of inputs and producing other types of output, for example sets, vectors, matrices, etc. but here, we will mainly look at functions that take one number as input and produce another number as output and those two numbers will typically assumed to be real numbers.

\subsection{Domain and Range}
The two sets (of numbers) from which inputs can be taken and outputs will be produced are called the domain and range of the function respectively. Functions are usually denoted by lowercase letters such as $f,g,h$ etc. If a function takes real numbers as inputs and also produces real numbers as output, we write this as: $f: \mathbb{R} \rightarrow \mathbb{R}$. The argument is usually denoted by $x$ and of the function and is written in parentheses and the output is often denoted by $y$. When we write something like $y = f(x)$, we mean that the function $f$ produces the number $y$ as output when we give it the number $x$ as input. An example could be $y = f(x) = x^2$. The function woul just square the input and return that as result. 

\medskip
\subsubsection{Codomain and Image}
There is a little ambiguity with respect to the notion of range. It may mean either one these two things: sometimes it may mean the set of numbers from which outputs can \emph{potentially} be drawn and sometimes the (possibly smaller) set of numbers, that can \emph{actually} be the result. For example, if the domain of $f(x) = x^2$, is taken to be the set real numbers $\mathbb{R}$, we could think of $f$ as a mapping between real numbers and real numbers, i.e. $f: \mathbb{R} \rightarrow \mathbb{R}$. However, not all possible real numbers are reached because the result of $x^2$ will never be a negative number. We could also consider $f$ as a function from the real numbers to the \emph{nonnegative} real numbers, i.e. $f: \mathbb{R} \rightarrow \mathbb{R}^+_0$. To distinguish these two notions, sometimes the term "codomain" is used for the set that outputs could potentially be drawn from and "image" is used for the set from which numbers are actually produced by the function for a given input domain.

% https://en.wikipedia.org/wiki/Domain_of_a_function
% https://en.wikipedia.org/wiki/Range_of_a_function

\subsection{Taxonomy}
There are certain features that a function may or may not have which turn out to important for classification. A function is said to be \emph{monotonically increasing}, if from $a \geq b$ follows $f(a) \geq f(b)$. It means that the function either goes upward or stays constant but never goes downward. If the stronger condition  $a > b \Rightarrow f(a) > f(b)$ holds true, then $f$ is \emph{strictly monotonically increasing}. Such a function really goes upward everywhere. Not even plateaus are allowed. A \emph{(strictly) monotonically decreasing} function is defined analogously with $\leq, <$ rather than $\geq, >$. If a function is either (strictly) monotonically increasing or decreasing, it is called \emph{(strictly) monotonic}. A function $f$ is said to be \emph{bounded from above} when there is some finite number $U$ such that $f(x) \leq U$ for all $x$. The number $U$ is called an \emph{upper bound} for $f$. Likewise, a function $f$ is called bounded from below, when there's some number $L$ such that $f(x) \geq L$ for all $x$. In this case, $L$ is called a \emph{lower bound} for $f$. A function that is bounded from below and above is called \emph{bounded}. A function for which we have $f(x) = f(-x)$ has \emph{even symmetry} and a function with $f(x) = -f(-x)$ has \emph{odd symmetry}. Every function can be decomposed into an even part and odd part like so: 
\begin{equation}
\label{Eq:EvenOddFuncDecomp}
f_e(x) = \frac{f(x) + f(-x)}{2}, \;\;
f_o(x) = \frac{f(x) - f(-x)}{2}, \qquad
f(x) = f_e(x) + f_o(x)
\end{equation}
\emph{Injectivity}, \emph{surjectivity} and \emph{bijectivity} are also important features which have already been discussed in the section about set theory. Functions can also be \emph{periodic}. That means that there exists some number $p$, called the period, such that $f(x) = f(x + k p)$ for all inputs $x$ and all integers $k$. Such a function repeats itself over and over. Functions may have \emph{asymptotes} which are straight lines which the function approaches as the argument goes to plus or minus infinity. Asymptotes can also be vertical lines. In this case, the line is approached as the argument approaches a finite value and at that value itself, the function is usually undefined. When a function $f(x)$ has a vertical asymptote at some input value $x_0$, we call $x_0$ a \emph{pole} of the function. A function is said to be \emph{compactly supported}, if it is nonzero only in some finite interval and zero outside that interval. A function is called \emph{convex} if you can pick any two points $(a,f(a)),(b,f(b))$ on its graph, connect them with a straight line and the function will be below that line everywhere. The function behaves like a chain hanging down, i.e. it is curved downward like a bowl. The opposite of convex is \emph{concave}. Such functions are everywhere above the straight line. They look like an arch. A \emph{continuous} function is one that can be drawn without lifting the pencil or drawing vertical lines. Functions can also be classified according to their degree of \emph{smoothness} and/or according to whether or not they have a finite area under their graph. To define exactly what that means, we will need a few tools from calculus (differentiability, integrability), so we will defer further discussion about these features to later chapters.

% surjective: codomain coincides with image?

\subsection{Defining a Function}

\paragraph{Explicit Rules}
One way to define a function is to prescribe a simple \emph{calculation rule} such as $f(x) = x^2 + 3 x - 5$ that applies equally to all possible inputs $x$. Up to until a few hundred years, this was the only way that was acceptable for mathematicians to define a function. Today, we have become a lot more liberal in that regard and allow functions to be defined in more esoteric ways. Some of these ways will involve concepts that are discussed only later in the book. If you don't understand these definitions, you can skip them on a first reading. In a first pass, you need to understand only this and the next way.

\paragraph{Piecewise Definitions}
A slightly more complicated way to define a function would be to have different calculation rules for different subsets of the domain. If these subsets are succesive intervals of the real number line, this way of defining a function is called a \emph{piecewise definition}. They are written down like this:
\begin{equation}
f(x) = 
\begin{cases} 
 0 \quad& x < 0 \\
 -x     & 0   \leq x < 1 \\
 x^2    & x \geq 1
\end{cases}
\end{equation}
This means, the function will output $0$ for $x < 0$ and $-x$ when $x$ is between zero and one (including zero, excluding one) and $x^2$ when $x \geq 1$. The example function is discontinuous at $x=1$. It jumps from $-1$ to $+1$ there. At zero, it has a corner - that is: a discontinuity in its slope.

\paragraph{Conditional Definitions}
The piecewise definition is a specific kind of a conditional definition where the conditions are of the form "if $x$ is in interval ... then ...". A function can also be defined by more general conditions such as:
\begin{equation}
f(x) = 
\begin{cases} 
 1 \quad& x \in    \mathbb{Q} \\
 0      & x \notin \mathbb{Q}
\end{cases}
\end{equation}
The function given above is called the Dirichlet function or indicator function of the rational numbers. It outputs one for rational inputs and zero for irrational inputs.

\paragraph{Definition by Data}
A function can also be defined by recorded data points of some measurement. To turn these discrete data points into a continuous function, one would also have to prescribe some sort of \emph{interpolation} rule such as "connect the dots by straight lines". The data together with the interpolation rule constitute a bona fide function: we can give it an input $x$ and it will give back an output $y$. 

\paragraph{Algorithmic}
Functions can be defined by the limit of some infinite recursive process or algorithm. An example of that is Bolzano's function. Such functions will often show self-similar, "fractal" behavior: you can zoom in as much as you like and it will always look similar. 
...TBC...ToDo: explain construction rule
% https://de.wikipedia.org/wiki/Bolzanofunktion
% https://demonstrations.wolfram.com/BolzanosFunction/

\paragraph{Infinite Sums}
The infinite process is sometimes defined to be an infinite sum. The individual terms are defined in terms of already known, simpler functions. An example of such a definition is the Weierstrass function defined as:
\begin{equation}
f(x) = \sum_{n=0}^\infty a^n \cos(b^n \pi x), 
\qquad 0 < a < 1, \; b \in \mathbb{N}, odd, b \geq 7
\end{equation}
It has two tweakable parameters $a,b$. It is defined as an infinite sum of exponentially weighted sinusoidal functions. This function has also a self-similar nature and a couple of strange properties that were unheard of in mathematics up to the 19th century such as being continuous everywhere but differentiable nowhere.
% https://en.wikipedia.org/wiki/Weierstrass_function

\paragraph{Power Series}
Often we will encounter infinite sums of weighted powers of $x$. A function definition based on such a sum is called a \emph{power series} definition. Most of our common functions can be expressed as such a power series even though they may have originally have been constructed by other ways. Exponential and sinusoidal functions are usually first defined by means of limiting processes or geometric constructions respectively. However, they do also have power series based definitions. They can be found from the original definitions by means of a Taylor expansion. This leads to:
\begin{equation}
e^x = \sum_{k=0}^\infty \frac{x^k}{k!}, \quad
\sin(x) = ...
\end{equation}
Power series are often used to expand the domain of a function - for example, from the real to the complex numbers. The thing is that all we need to evaluate a power series is to know, how to add or multiply two objects together and how to scale an object by a factor. When we have objects on which we have these 3 operations defined, we can evaluate, for example, the exponential function of that object via the above power series. The "object" can be a real or complex number - or it can be something else such as a matrix or an operator. We'll see later what these things are. Of course, such a power series does not need to originate from already known functions. It can be made up in any way you like - you just need to prescribe some rule to compute the weights. How you came up with that rule doesn't matter. What matters is the question, whether or not the series actually converges. More on that in the section about series in the calculus chapter.

\paragraph{Analytic Continuation}
Speaking of convergence, consider the following infinite sum:
\begin{equation}
	f(x) = \sum_{n=1}^\infty \frac{1}{n^x}
\end{equation}
This sum converges only for $x > 1$. If we allow complex arguments, the condition is $\Re(x) > 1$. So in the complex plane, this sum above defines a function for those complex numbers whose real part is greater than one. For other values, the function is as of yet undefined. However, as we will see later, complex analysis provides a tool known as \emph{analytic continuation} by means of which we can uniquely expand the domain of the function to the whole complex plane except for the single point $x=1$. The so defined function is known as the Riemann Zeta function and an important object in number theory due to its unexpected connection to prime numbers. The point that I want to make here is that the process of analytic continuation gives us yet another way to define a function.

\paragraph{Continued Fractions}
Another infinite process that is sometimes useful to define a function is called a \emph{continued fraction expansion}...TBC...

\paragraph{Inverse Functions}
A function can be defined by its inverse function. In general, you can take any given function $y = g(x)$ and then define another function $f$ by asking: "for a given $y$, what $x$ would I have to plug into my function $g$ to get that $y$?". For example, the logarithm can be defined to be the inverse of the exponential function. For a slightly more interesting example, consider the function $g(x) = x e^x$. Its inverse can not be expressed in terms of common functions. Yet, it does have an inverse (with some caveats). The inverse function of $x e^x$ is called the "product logarithm" or "Lambert-W" function and has applications in the solution of equations where an unknown $x$ appears as base and as exponent or inside and outside of a logarithm.
% https://en.wikipedia.org/wiki/Lambert_W_function
% https://en.wikipedia.org/wiki/Lambert_W_function#Applications
% y = x e^x  <->  log(y) = log(x) + x
% https://www.desmos.com/calculator/s3gxvjbovr

% $f(y) = g^{-1}(y)$
%ToDo: arcsin, ...
%$ actually, the Lambert-W function is a case of an inverse function - in this case, the inverse of x e^x. maybe for implicit functions use y e^y = x + sin(x)

% W solves the ODE: y' = y / (x*(1+y)), see:
% https://www.youtube.com/watch?v=cMZ_blqKKZU

\paragraph{Implicit Equations}
Defining a function via its inverse is a special case of an implicit function definition. In a more general setting, we could have a bivariate function of the form $g(x,y) = 0$. In the case of the product logarithm function, we would have $x = y e^y$, so $g(x,y) = 0 = y e^y - x$.  This is an equation that can be solved for $x$ on one side of the equation. That's why we can define a function in terms of the inverse of the other side. But that doesn't always have to be the case. Consider modifying the equation $x = y e^y$ to $x + x^5 = y e^y$ or, equivalently $g(x,y) = 0 = y e^y - x - x^5$. Now, we can't  solve for $x$ on one side. And as before, we can't solve for y either. Still, we can ask: "for a given $y$, what value of $x$ can I plug into $g(x,y)$ such that it evaluates to zero?". There doesn't have to be a solution and if there is one, it doesn't need to be unique. But if there is a unique solution, then that equation defines a function. If there are multiple solutions, we may pick one of them to be our function of interest. You can play with the function defined by $ y e^y = x + b \sin(a x)$ a bit here on Desmos: \href{https://www.desmos.com/calculator/jfxwuhot8k}{jfxwuhot8k}. Tweak the parameters $a,b$. When any of them is zero, then the function is the regular product-log functions. When both are nonzero, we get a wiggly version of it. ...TBC...


% \href{https://www.desmos.com/calculator/t3t92nihws}{t3t92nihws}

% https://en.wikipedia.org/wiki/Implicit_function

% y e^y = x + x^5
% https://www.desmos.com/calculator/ekao9wnrun
% https://www.desmos.com/calculator/t3t92nihws

% y e^y = x + b sin(a x)
% https://www.desmos.com/calculator/izay7xuihf
% https://www.desmos.com/calculator/jfxwuhot8k

% y + sin(y)   = x e^x      a wiggly version of Lambert-W
% y + sin(a y) = x e^x      wiggles get stronger for a > 1, it becomes multi-valued
% https://www.desmos.com/calculator/mowoqggdpj

% x tanh(x) = y atan(y)     has two graphs
% https://www.desmos.com/calculator/9cfsbbm0d9


%Now we want to cosider functions defined by so called \emph{implicit equations}. These are equations of the form $g(x,y) = 0$ for some given function bivariate $g$. They define a set of points in the $xy$-plane. Now, we can define a function $y(x)$ by requiring that it should satisfy our implicit equation, i.e. $g(x,y(x)) = 0$ is satisfied for all $x$ in some domain. That's an implicit definition for $y(x)$. For example, we could require the following equation to hold:
%\begin{equation}
% y e^y = x \;  \Leftrightarrow \; x - y e^y = 0 
%\end{equation}
%In this case $g(x,y) = x - y e^y$. I did not make up this example. The so defined function $y(x)$ is called the "Lambert-W" function or "product-logarithm" function. 

% Move that into the Interlude:


\paragraph{Interlude - Evaluation}
Of course, an implicit definition tells us nothing at all about how we would go about evaluating $y$ for a given $x$. Nonetheless, it is a valid definition. Definitions do not necessarily need to prescribe an explicit evaluation algorithm. They are supposed to tell us \emph{what} it is, not \emph{how} we may compute it. If they do tell us how to compute it, that's a bonus. There are algorithms to solve such implicit equations though. But that's a topic for a later chapter. You may also wonder how we are supposed to evaluate functions that are defined by some infinite process. The answer is: on a computer, we cannot really represent real numbers exactly anyway. Our floating point numbers are always approximations. So, from a practical point of view, it is good enough to be able to evaluate functions up to some precision. For the infinite processes, that usually means that we can obtain a sufficiently precise approximation of the function by only doing a finite number of steps until we deem the result to be close enough. The best we can expect is to be precise within one ulp ("unit in the last place"). Usually, we'll accept even much higher errors.





\paragraph{Functional Equations}
Now let's consider the factorial function $n! = 1 \cdot 2 \cdot 3 \cdot \ldots \cdot n$. As it stands, it's defined only for natural numbers. There is a way to define it for all real and even complex numbers. The key lies in the observation of the factorial's functional equation which is given by $n! = n \cdot (n-1)!$. In our expanded definition, we want that equation to be satisfied for all real inputs. So we search for a continuous function $f(x)$ for which the functional equation $f(x) = x \cdot f(x-1)$ holds true. It can be shown that the following integral based definitions:
\begin{equation}
\Pi(x)    = \int_0^\infty t^x e^{-t} \, dt, \quad
\Gamma(x) = \Pi(x-1) = \int_0^\infty t^{x-1} e^{-t} \, dt,
\end{equation}
do indeed lead to functions that satisfy the desired functional equation. The first definition agrees with the factorials at the naturals, i.e. $\Pi(x) = x!$ for $x \in \mathbb{N}$. That means, it interpolates (i.e. makes continuous) the formerly discrete factorial function. The second definition is called the \emph{gamma function} and it agrees with the shifted factorial function. There is some discussion on \href{https://math.stackexchange.com/questions/1362523/why-is-the-gamma-function-off-by-1-from-the-factorial}{stackexchange} about why the gamma function might have been defined with that strange shift. Supposedly, it will make some higher level stuff more convenient but at first, it seems more unnatural and inconvenient than the $\Pi$ function. Nevertheles, it's the gamma function that is used in math all over the place. Anyway, my point here is that a function definition can be based on a functional equation. 
[VERIFY all of this!]

\paragraph{Parameter Integrals}
In this case of the factorial function, the solution of the functional equation led to a definition in terms of a \emph{parameter integral} which is an integral in which the integrand depends on a parameter - here $x$.
I think, the general way to define a function $f$ in terms of a parameter integral is to have some function $g(x,t)$ and then use:
\begin{equation}
 f(x)  = \int_{a(x)}^{b(x)} g(t,x) \, dt
\end{equation}
For $\Pi(x)$ we have $g(x,t) = t^x e^{-x}, a(x) = 0, b(x) = \infty$.
[VERIFY all of this!]
...TBC...
% https://en.wikipedia.org/wiki/Gamma_function
% https://math.stackexchange.com/questions/1362523/why-is-the-gamma-function-off-by-1-from-the-factorial
% https://mathoverflow.net/questions/20960/why-is-the-gamma-function-shifted-from-the-factorial-by-1
% https://de.wikipedia.org/wiki/Parameterintegral
% Arens pg 409 ff, Bärwolff pg 166 (1 example),

\paragraph{Antiderivatives}
Speaking of integrals - we sometimes have a known function $f$ and want to define another function $F$ as an antiderivative of $f$. An important example is the area under a Gaussian bell curve from minus infinity up to some value $x$:
\begin{equation}
 F(x)  = \int_{-\infty}^{x} e^{-t^2}  \, dt
\end{equation}
I think, this can also be seen as another special case of a "parameter integral" definition with $g(x,t)=e^{-t^2}, a(x)= -\infty, b(x)=x$? ...figure out! Explain the erf-function

\paragraph{Differential Equations}
Finding an antiderivative is a problem of the form: for a given $g(x)$, find a function $f(x)$ such that $f'(x) = g(x)$. That equation is actually a special case of a more general class of equations called differential equations. These are equations in which the unknown function $f$ and derivatives of it may appear. A function $f$ that satisfies the differential equation is called a solution of the differential equation. Of course, saying something like: "let $f$ be the solution to the following differential equation..." is also a way of defining a function. There are some caveats, though: differential equations typically have many solutions so we need some additional constraints (typically boundary conditions or initial conditions) to pick out one of those solutions uniquely ...TBC...give examples for how exp and sin can be defined by an ODE, mention Bessel functions
%Having seen how function can be defined via integrals, the next step is consider functions that are defined by differential equations...TBC...

% differential equations - start with simple examples like exp, sin, cos, the give more complex examples like Bessel

\paragraph{Integral Equations}
An integral equation is an equation where an unknown function $f$ appears under an integral sign. Like with differential equations, solving the integral equation amounts to finding $f$. For example, imagine that we have two known functions $g, h$ and we want to find $f$ such that:
\begin{equation}
 \int_a^b f(u) h(x-u) \, du = g(x)
\end{equation}
for some constants $a,b$ (which are often minus and plus infinity). We could interpret $g$ as a recorded signal, $h$ as the impulse response or "kernel" of a filter and we want to reconstruct $f$ as it was before it went through the filter. The desired function $f$ appears under a convolution integral. This sort of equation is a particular kind of integral equation and does indeed define a function $f$.

%...TBC...
%ToDo: bring an example involving deconvolution - we are given a signal $g(x)$ which is supposed to be our desired $f(x)$ convolved with some known kernel $h(x)$. Then $f(x)$ appears under the convolution integral. Finding it involves deconvolution...can perhaps be done in the Foruier domain - but that's not the problem of the definition




%\paragraph{Implicit Differential Equations}
% well - so far we have said nothing in the paragraph about diffeqs that restricts us to explicit ones so we may consider the implicit ones to be already subsumed

%What about an implicit differential equation

\paragraph{A Non Function}
Consider the following nonsensical looking attempt of defining a function:
\begin{equation}
f(x) = 
\begin{cases} 
0       \quad& x \neq 0 \\
\infty       & x = 0
\end{cases} \qquad \text{with} \qquad
\int_{-\infty}^{\infty} f(x) \, dx = 1
\end{equation}
We are trying to define a "function" that is zero everywhere except at zero. There, it should have an infinitely tall and infinitely thin spike. The (infinite) height of the spike should be adjusted in such a way that the total area under the function is unity. This "function" is called the Dirac delta function - and it isn't really a function because the definition does not really make sense formally. But when this strange function-like object appears under an integral, it suddenly starts to make a lot of sense and is very useful. We'll meet this bizarre Dirac spike again later in the chapter about functional analysis where it will be more rigorously defined as a \emph{distribution} which is a certain kind of \emph{functional}. Especially in physics and signal processing, we often work with this thing informally as if it were a function.


%ToDo: integral equation
% maybe make these subsubsection of an "Elementary functions"  section

\paragraph{Whoa!} That was a tour de force! I have presented quite a lot of ways how a mathematical function can be defined. Some of them required some rather high level math which doesn't really belong into such an early chapter. I did this to point out that the idea of "defining a function" can, when intepreted liberally as we do in the modern world, include much more diverse and creative ways rather than just a simple and explicit calculation rule. However - most functions that we encounter in everyday life are not of such a complicated nature. There's a set of functions that tends to pop up everywhere and constitutes a sort of basic mathematical vocabulary that everybody needs to be familiar with. I'm talking about the so called...

\subsection{Elementary Functions}
In general, a function can be any mapping. But there is a certain set of functions that permeates mathematics so thoroughly that it has been given a special name: the elementary functions. These are all functions that can be constructed by a finite formula involving only the 4 basic arithmetic operations, roots, exponentials, logarithms, trigonometric functions and the inverses of all these functions.

% https://www.youtube.com/watch?v=l6w868U8C-M

\subsubsection{Polynomials}
A polynomial can be written in various forms. Some important canonical forms are: (1) a weighted sum of integer powers of $x$, (2) a (scaled) product of so called linear factors, (3) a nested expression. So, a polynomial is a function of one of the forms:
\begin{equation}
 f(x) = \sum_{k=0}^n a_k x^k 
      = a_n \prod_{k=1}^{n} (x - r_k)
      = a_0 + x(a_1 + x(a_2 + \ldots + x(a_{n-1} + x a_n)))
\end{equation}
where the weights $a_k$ in the sum form are called the coefficients of the polynomial and the highest power $n$ for which there is a nonzero coefficient is called the degree of the polynomial. The $r_k$ in the product form are called the roots of the polynomial. These are the values of $x$ where the output of the function $y = f(x)$ is zero. This is immediately obvious from the fact that a product is zero as soon as any of its factors is zero, which will clearly be the case if $x = r_k$ for one of the $r_k$. The roots $r_k$ may be real or complex and are not necessarily distinct. The number of times by which a root appears in the product is called the multiplicity of the root. An important theorem, the fundamental theorem of algebra, states that every polynomial of degree $n$ has exactly $n$ such roots (counted with multiplicity, i.e. a double root counts twice, etc.). A polynomial of degree 1, i.e. a function of the form $f(x) = a_0 + a_1  x$, is a very simple special case which is of practical importance and is sometimes - somewhat inconsistently - called a linear function. This terminology is inconsistent because such a function satisfies the usual requirements for linearity - homogeneity and additivity - only when the additive constant is zero, i.e. when $a_0 = 0$. Nevertheless, you'll find that terminology used often. The practical importance of such linear functions is that we will often use them to approximate more complicated functions - a process known as linearization. 

\paragraph{Evaluation}
The evaluation of a polynomial is best done in nested form. This is the most efficient way and also one of the most stable ways numerically. We assume that the polynomial is given as a list or array of $n$ coefficients $a_0, \ldots, a_n$.
\begin{lstlisting}
def polyEval(p, x):
	"""Evaluates the polynomial p at the given x using Horner's algorithm"""
	k = len(p)-1           # last valid index
	if(k < 0):
		return 0
	y = p[k]
	while(k > 0):
		k -= 1
		y = y*x + p[k]
	return y
\end{lstlisting}
This way of evaluating a polynomial is also known as Horner's rule.
...hmm...not sure, if code should appear in the normal text. maybe put it into an appendix?

\paragraph{Addition and Subtraction}
Polynomials, given as arrays of coefficients, can be added and subtracted element wise. If the input arrays are of different length, i.e. the polynomials are of a different degree, the lower degree coefficient array has to be padded with zeros to match the length of the larger one.

...python code is buggy

\paragraph{Multiplication}
When we multiply two polynomials $f(x), g(x)$ with degrees $m,n$ together, we will get a new polynomial of degree $m+n$. The coefficients of this new polynomial can be found by an algorithm called \emph{convolution}. Let's see how this would look like:
\begin{eqnarray}
&  (a_0 + a_1 x + a_2 x^2 + a_3 x^3 + \ldots)
   (b_0 + b_1 x + b_2 x^2 + b_3 x^3 + \ldots)  \\
=&  c_0 + c_1 x + c_2 x^2 + c_3 x^3 + c_4 x^4 + c_5 x^5 + c_6 x^6 + \ldots \\
=&  a_0 b_0 \\
&+ (a_0 b_1 + a_1 b_0) x \\
&+ (a_0 b_2 + a_1 b_1 + a_2 b_0) x^2 \\
&+ (a_0 b_3 + a_1 b_2 + a_2 b_1 + a_3 b_0) x^3  \\
&+ (a_0 b_4 + a_1 b_3 + a_2 b_2 + a_3 b_1 + a_4 b_0) x^4 \\
&+ (a_0 b_5 + a_1 b_4 + a_2 b_3 + a_3 b_2 + a_4 b_1 + a_5 b_0) x^5 \\
&+ (a_0 b_6 + a_1 b_5 + a_2 b_4 + a_3 b_3 + a_4 b_2 + a_5 b_1 + a_6 b_0) x^6 + \ldots
\end{eqnarray}
The coefficient $c_n$ in the resulting polynomial is the sum of all products of input coefficients $a_i, b_j$ such that $i+j = n$. Let's introduce another index $k$ to replace $i$ and then use $n-k$ to replace $j$. This ensures that $k + (n-k) = n$. We now have:
\begin{equation}
  c_n = \sum_{k=0}^n a_k b_{n-k}
\end{equation}
That formula to compute the new list of $c$-coefficients from the $a$- and $b$-coefficients is called convolution. Multiplication of two polynomials is achieved by convolving their coefficient lists. Convolution is an important mathematical operation that crops up in different variations everywhere. Polynomial multiplication is just one of the many places. Signal processing is full of it. Its continuous variant plays a role in the solution of certain differential equations which are all over the place in physics.

...TBC... give Python code for it

\paragraph{Division}
Let's now see, if we can reverse the process. Let's assume that we know the coefficient arrays $c_n$ and $b_n$ and want to figure out all the $a_n$. We need to make sure that the coefficients for like powers of $x$ match. For example, at the top of the triangle, we have the product $a_0 b_0$. That's how we computed the coeff for the constant term in the product polynomial $c_0$. So we have $c_0 = a_0 b_0$. That actually already tells us $a_0 = c_0 / b_0$. Now look at the line immediately below the tip of the triangle. The sum in brackets must equal $c_1$. But $c_1 = a_0 b_1 + a_1 b_0$ means $a_1 = (c_1 -a_0 b_1)/b_0$. We have found our next coefficient! Everything on the right is already known, The $b$-coeffs are assumed to be known anyway and $a_0$ has just been computed in the previous step. One line more to establish the pattern: we must have $c_2 = a_0 b_2 + a_1 b_1 + a_2 b_0$ so we have $a_2 = (c_2 - a_0 b_2 - a_1 b_1)/b_0$. In general:
\begin{equation}
a_n = \frac{1}{b_0} (  c_n - \sum_{k=0}^{n-1} a_k b_{n-k} ) 
\end{equation}
...but there's a caveat: what, if $b_0 = 0$? Maybe we should find $m$ as the index of the first nonzero $b$-coeff and use $b_m$ as divisor? I think, the upper limit of the sum can then be reduced to $n-m-1$. -> figure out! What about polynomial division with remainder?

%Oh my dog! Polynomial division! That pesky algorithm that we all once had to learn in school and then have forgotten for good! Worry not - you won't have to relearn it here and now. Executing algorithms with pencil and paper is not something we humans like to do and I personally find it as pointless as the next guy. Instead, I'll give you computer code for it, so we'll never ever have to do it again by hand. ....

% what if c = 1, b = 1 + x^2 - what a would we get by deconv? maybe some sort of polynomial approximation of 1 / (1 + x^2)? We get a0 = 1, a2 = -1, a4 = +1. I guess all odd coeffs are zero and the +-1 pattern for the even coeffs continues? Figure out! But it does indeed look like taking more such terms improves the approximation, see:
% https://www.desmos.com/calculator/7rekjnfmtw
% move this comment to the cpp code where we implement or test this deconv algo - make an experiment! Compare coeffs to Taylor coeffs and the Approximation to the Taylor approximation!

\paragraph{Composition}

%You will sometimes find a weaker form of this statement called fundamental theorem of algebra, namely a statement that each polynomial has a root. But the stronger statement immediately follows from the possibility of splitting off a linear factor.

%..tbc..
% product form, fundamental theorem of algebra, linear functions as special case
% factor out a linear factor, polynomial addition, multiplication, division - maybe with (pseudo) code or better: sage code, maybe provide als sage code for basic integer arithmetic - maybe in the section numbers in 

\subsubsection{Rational Functions}
A rational function is a ratio or quotient of two polynomials, i.e. function of the form:
\begin{equation}
 f(x) = \frac{\sum_{k=0}^n a_k x^k}{\sum_{k=0}^m b_k x^k}
\end{equation}

\paragraph{Partial Fraction Expansion}



\subsubsection{Algebraic Functions}
The set of algebraic functions contains all the functions that can be constructed from using the elementary operations and extracting roots....
% partial fraction expansion

\subsubsection{Trigonometric Functions}
The trigonometric functions $\sin(x), \cos(x)$ originally arise from certain geometric considerations about triangles (more on that later in the geometry chapter) but can also be defined in terms of a power series:






\subsubsection{Exponential Functions}
A general exponential function is a function of the form $y = f(x) = b^x$ for some constant $b$ which is called the basis in this context. Choosing the special number $e$ as basis, we obtain \emph{the} exponential function $y = f(x) = e^x = \exp(x)$. It can be defined via a power series, a limit or as inverse of a parameter integral [VERIFY the last]:
\begin{equation}
e^x 
= \exp(x)
= \sum_{k=0}^{\infty} \frac{x^k}{k!}
= \lim_{n \rightarrow \infty} \left( 1 + \frac{x}{n} \right)^n, \quad
x = \int_1^y \frac{1}{t} \, dt
\end{equation}
% continuous fraction
It satisfies the following differential equation and functional equation:
\begin{equation}
\frac{d}{dx} e^x = e^x, \quad
e^{x+y} = e^x e^y
\end{equation}

% give differential equation and functional equation that exp satisfies - maybe some other types of equation, too?

% https://de.wikipedia.org/wiki/Exponentialfunktion
% https://en.wikipedia.org/wiki/Exponential_function

\subsubsection{Hyperbolic Functions}
When we decompose the exponential function into its odd and even parts, we obtain the so called hyperbolic sine and cosine functions $\sinh(x), \cosh(x)$ and when we form their quotient, we obtain the hyperbolic tangent $\tanh(x)$:
\begin{equation}
 \sinh(x) =	\frac{e^x - e^{-x}}{2}, \quad	
 \cosh(x) =	\frac{e^x + e^{-x}}{2}, \quad
 \tanh(x) =	\frac{\sinh(x)}{\cosh(x)} 
          = \frac{e^x - e^{-x}}{e^x + e^{-x}}
\end{equation}
They can all also be directly be defined by a power series:

\subsection{Special Functions}
% gamma, bessel, elliptic integrals


\subsubsection{Multifunctions}
In the set-theoretic framework, functions are required to be right-unique relations. We sometimes need to work with function-like objects that violate this condition. Multifunctions are multi valued functions in the sense that for a given input $x$ they produce not a unique output value $y$ but multiple values. For example, consider the equation $x^2 = 4$. It has the two solutions $+2$ and $-2$. \emph{The} square root on $\mathbb{R}$ is usually \emph{defined} to be $+2$ - but that's a matter of definition. It would also be viable to define the square root to be multi-valued. Then, instead of giving the solution $2$, it would give the solution $\{2,-2\}$. The output would not be a single number but a whole set of numbers. This is indeed how we usually interpret the square-root function on $\mathbb{C}$. [VERIFY!] In $\mathbb{R}$, we would say that $\sqrt[4]{16} = 2$. In $\mathbb{C}$, we would say that $\sqrt[4]{16} = \{2,2\i,-2,-2\i\}$ and in general, the $n$th root of any nonzero complex number would produce a set of $n$ solutions. For logarithms on $\mathbb{C}$, thing get even crazier - due to the fact that $e^z = e^{z + 2 \pi k \i}$ for any integer $k$, the logarithm of any (nonzero?) complex number is the infinite set:...
...TBC...

%the square-root function on the real numbers. 
% what about fractional roots like the (7/5)th root? ...or in general, how about z^w for any complex exponent w?

%If you go back to the diagram showing the rela  Fig. ..., 

% https://en.wikipedia.org/wiki/Multivalued_function
% https://en.wikipedia.org/wiki/Set-valued_function
% https://math.stackexchange.com/questions/3726882/square-root-as-a-multi-valued-function

\begin{comment}

-polynomials, rational functions, power series, trig-functions, exp/log, floor/ceil/round, atan2
-maybe explain functions also as unary operators? abs, negation


How to define functions:

done:
-explicitly, e.g. y = x^2
-piecewise (may also be explicit but that's actually an independent question)
-by data (togther with an interpolation rule)
-Dirichlet function: "infinitely piecewise"? "pointwise"? "condition-based"? How about:
 f(x) = numerator, if x rational otherwise 0
-by an algorithm (see fractal functions (Bolzano function) Weitz video "Monster der Analysis")
 -> limit of an infinite recursive process
-power series - this is the usual way to expand definitions of function to e.g. complex 
 numbers, matrices, etc.
-other types of series , example: Weierstrass function - also in "Monster der Analysis"
-series in some domain, analytic continuation where the series diverges, example:
 Riemann Zeta function.
-functional equations, examples: 
 f(x + y) = f(x) * f(y)  ->  f(x) = exp(x)
 f(a x) = b f(x)  ->  f(x) = x^(b/a) or x^(a/b)
 f(x+1) = x * f(x) -> Gamma-function, leads to a definition via an integral
-(parameter) integrals...?
-implicit functions ...maybe something like 
 y + exp(y) = x + sin(x), y - exp(y) = x + sin(x), y - exp(y) = x - sin(x), y + exp(y) = sqrt(x^x)
 is this an example?: https://en.wikipedia.org/wiki/Lambert_W_function
 if not, then maybe it should be considered as yet another way to define a function
 y e^y = x -> g(x,y) = x - y e^y

todo:
-definite or indefinite integral of some given function, examples:
 erf (int of Gauss), jacobi-elliptic (int of 1/sqrt(...))
-as the solution of a given problem (let f(x) be the solution of ..blabla)
-differential equations, examples: Bessel
-integral equation, example: assume that a kernel h(x) is given (for example Gauss-func) and
 some function g(x) is given. We seek f(x) such that g = convolve(f,h). Useful in signal reconstruction (compensate for undesired filtering)
-delay differential? what about something like f(x+1/2) - f(x-1/2) = f'(x) or equivalently
 f(x+1) - f(x) = f'(x + 1/2). Idea: we want a function whose derivative equals the central
 difference approximation of it. "differentio-functional"?
-maybe it makes sense to have a section about equations - what type of equations we see in math...but for some types (e.g. differntial or integral equations), we need some calculus
-what about the Dirac delta function? It's not really a function, though

\end{comment}

%Numbers (N,Z,Q,R,C), Polynomials, Rational Functions, Transcendental Functions