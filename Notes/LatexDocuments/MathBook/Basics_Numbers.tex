\section{Numbers and Arithmetic} 

Mathematics deals with various kinds of numbers, the most important of which are the natural, integer, rational, real and complex numbers. These sets of these numbers are denoted by  $\mathbb{N,Z,Q,R,C}$ respectively. In the given order, the sets further to the left are actually subsets of all the sets further to the right, so we have $\mathbb{N \subset Z \subset Q \subset R \subset C}$. The set $\mathbb{C}$ does not have to be the end of this progression (there are even larger sets such as quaternions, octonions, etc.), but in a very meaningful and satisfying sense, it can be justified to stop at $\mathbb{C}$.

\subsection{Natural Numbers}
Natural numbers are the counting numbers $0,1,2,3,4,5,\ldots$. The set of natural numbers is denoted as $\mathbb{N}$. They are indeed "natural" in the sense that we have an intuitive understanding what a number like $3$ means. It is an act of abstraction though, to recognize that a set of 3 apples and a set of 3 orang-utans have something in common, namely that they both have the "size" of 3. Whether or not zero is considered a natural number is a matter of convention. The historical fact that the invention (or discovery?)\footnote{There is a philosophical debate about whether mathematics is invented or discovered. Personally, I think, the question is stated improperly. There are different aspects of math some of which are invented (mainly definitions and axioms) and other aspects that are discovered (mainly theorems, proofs and conjectures).} of the number zero was a rather late development in western mathematics (it was recognized much earlier in indian mathematics) may suggest that the idea of zero is not so natural after all. However, I adopt the convention here that zero is a natural number because especially in the context of computers and programming, it is more often than not more convenient to have zero available in the set. There is no right or wrong here. This is a typical case of a definition for which there is no universal consensus in the math community. If zero is in the set $\mathbb{N}$ per definition but we nevertheless need to make a statement about all natural numbers except zero, the notation $\mathbb{N}^+$ is often used. Conversely, if some author defines zero to not be in the set $\mathbb{N}$ but wants to make a statement about all natural numbers including zero, the notation $\mathbb{N}_0$ is often used.

\subsubsection{Arithmetic Operations}
There are some elementary operations that take two natural numbers as input and produce another natural number as output. Operations that take two operands are called binary operations, those that take only one operand are called unary operations. The basic binary operations between numbers are usually introduced in elementary school as addition, subtraction, multiplication and division denoted by the symbols $+,-,\cdot,/$ respectively, where the operator symbol is surrrounded by the two operands (computer scientists call this "infix notation"). The dot for the multiplication is sometimes omitted. Let's recall the laws that additon and multiplication obey. For any natural numbers $a,b,c$, we have:

\medskip
\begin{tabular}{c l}
  $a + b = b + a$                             & Commutativity of addition \\
  $a \cdot b = b \cdot a$                     & Commutativity of multiplication \\
  $(a + b) + c = a + (b + c)$                 & Associativity of addition \\
  $(a \cdot b) \cdot c = a \cdot (b \cdot c)$ & Associativity of multiplication \\
  $a \cdot (b + c) = a \cdot b + a \cdot c$   & Distributivity of multiplication over addition
\end{tabular}
\medskip

We could write down some laws for subtraction and division, too. However, it turns out to be more economic to think of subtraction and division not as elementary operations in their own right but rather to consider them in terms of addition and multiplication of so called inverse elements. We will later define the meaning of $-a$ and $1/a$ and with those definitions in place, the laws above will be sufficient and we don't need extra laws for subtraction and division. But we first need to talk about...
% what about precedence and parentheses?

\subsubsection{Divisibility}
A natural number $p$ is said to be divisible by some other natural number $d$, if there exists a natural number $q$, such that $p = q \cdot d$ and $q = p/d$. In this case, we call $p$ the product, $q$ the quotient and $d$ the divisor and we express the fact that $p$ is divisible by $d$ by the notation $d | p$ which we read as "d divides p". For example, $15$ is divisible by $3$, because there exists a natural number, namely $5$, such that $15 = 5 \cdot 3$.
% ToDo: introduce division with remainder: p = q*d + r, introduce symbols for division /, talk about the gcd and lcm
Every number is divisible by itself and by one.

\subsubsection{Prime Numbers}
Some numbers have the special property that they are \emph{only} divisible by themselves and by one but have no other divisors. These numbers are called prime numbers and can in a certain sense be seen as the atomic building blocks from which all natural numbers except 0 and 1 are made up. This is meant in the following sense: for any natural number greater than 1, there always exists one and only one way to construct it as a product of prime numbers. We say that the factorization of a number into its prime factors is uniqely determined. However, that uniquness holds only up to ordering of the factors - the order cannot possibly matter because multiplication is commutative. Consider the number 60: We can write it as a product as follows $60 = 2 \cdot 2 \cdot 3 \cdot 5$. It has four prime factors where the factor 2 appears twice. 2,3 and 5 are indeed prime numbers - they cannot be further divided into smaller factors. Prime numbers such as 5 have only one single factor in their prime factorization and that factor is the number itself. The word "prime" can be used as noun or adjective: we can say that 5 is prime and also 5 is \emph{a} prime. In this context, the number 1 is not considered a prime by definition. This is again a matter of definition just as the question whether or not 0 should be considered to be a natural number. In this case, however, there is a consensus in the math community that 1 should be defined to not be a prime (that consensus, however, is actually not that old). Numbers that are not prime are called composite numbers because they are made up by more than one (prime) factor. The fact that any natural number greater than one can be constructed as a product of prime numbers and that this factorization is uniquely determined (up to order of the factors) is actually an important theorem: the fundamental theorem of arithmetic. It was known already to Euclid and appears in his "Elements".
% factorization, fundamental theorem of arithmetic

\subsubsection{Powers}
Multiplication can be viewed as iterated addition: In $15 = 3 \cdot 5$, the right hand side can be interpreted as $5 + 5 + 5$, i.e. we add together $3$ copies of the number $5$. In analogy, we can define a new operation as iterated multiplication. This will be useful. For example, in the prime factorization of 60, the number 2 appeared twice and in general for any number, a prime factor of that number may appear multiple times. We want a convenient notation to express something like $2 \cdot 2 \cdot 2$. The notation for this is $2^3$: we write the factor as usual and the "number of times" in superscript. The factor is called the "base" and the "number of times" is called the exponent or, depending on context, also the "power". This operation is called "exponentiation" or "raising to a power". With this new notation, we can write the prime factorization of 60 as $60 = 2^2 \cdot 3 \cdot 5$.  We read something like $2^5$ as: "two to the power of five" or "two to the fifth power" or shorter "two to the five" or "two to the fifth". If the exponent is 2 or 3, as in $5^2$ or $5^3$, we may also say "five squared" or "five cubed" respectively. That comes from the geometric fact that $5^2$ is the area of a square with side length $5$ and $5^3$ is the volume of cube with side length 5.

\medskip
Consider $2^3 = 2 \cdot 2 \cdot 2 = 8$. We have already said that in this context, 2 is called the base and 3 is called the exponent. What if we consider the act of raising some arbitrary number $b$ (like 2 in this example) to the power of 3 as a unary operation (i.e. an operation with only one input where we assume the exponent to be fixed) that we apply to $b$ and imagine we are given only the result of the operation (8, in this case). Can we undo the operation, i.e. figure out, that the base must have been 2? That ought to be possible because no other natural number than 2 will produce 8 when raised to the power of 3. The idea of undoing an exponentiation leads to the idea of extracting the "n-th root". Here $n = 3$, so we want to extract the 3rd root from 8. We write this as $2 =\sqrt[3]{8}$ and say "two is the third root of eight" or  "two is the cube root of eight".

\medskip
Consider $2^3 = 8$ again but now let's assume that the base 2 is fixed and we consider the unary operation of raising 2 to any arbitrary power $p$. As before, we are given the result of the operation - 8 - and we want to figure out that the power $p$ that 2 was raised to, to produce that number 8. Again, this should be possible because no other exponent than 3 will produce 8 when 2 is raised to the power of it. ...

% -list the laws for powers
% -explain how each operation can be though of iterating a lower-level operation
%  -> add is iterated increment/successor
%  -> mul is iterated addition
%  -> pow is iterated multiplication
% -mention tetration (briefly), see https://www.youtube.com/watch?v=BR_2bKV1XDk, 
% -maybe introduce roots and logarithms


% euclidean algo, gcd, lcm, coprime
% factorial, binomial coeffs

\subsubsection{Number Systems}
todo: decimal, binary, octal, hexadecimal, sexagesimal, etc. - base p


\subsubsection{Solving Equations}
In the definition of divisibility, we said something like "if there exists a natural number such that...". Such a definition is rather implicit and doesn't really give us any clue how we would go about figuring out whether or not such a number exists. Moreover, if such a number does indeed exist, we may want to know, what that number is. We need an algorithm to find that unknown number! But let's first consider a simpler but similar question: Does a number, let's call it $x$, exist such that $3 + x = 5$ and if so, what is it? The question is simple enough that we can solve it by the "method of looking closely" (as my high school math teacher used to say). We immediately "see" that the answer is: "yes, it's two!". But what did we actually do in our minds to figure this out? Maybe we imagined a number line and counted the number of steps that we would have to go to the right from 3 to reach 5? Or maybe we just brute forced the solution by trial and error? Can we do something more systematic and more efficient that we can cast into an algorithm that can then also be applied to more complicated problems for which just "looking closely" doesn't really cut it? 

\medskip
The general theme here is that we want to solve an equation for an unknown variable, which is typically called $x$, just as we did above. The goal is to isolate $x$ on one side of the equation such that it stands there alone. On the other side of the equation we want to see only known quantities, such that we can actually compute that side of the equation directly. Because both sides of the equation are equal (because it's an equation), this computation will give us our formerly unknown number $x$. The strategy to reach this goal is to apply certain allowed manipulations to the equation. So, what kinds of manipulations are "allowed" and why?  We can certainly transform both sides of the equation individually according to the algebraic laws (associativity, commutativity, etc.). But to reach our goal of isolating the unknown $x$ on one side, we will also need some operations in which the two sides interact. These operations should not change the fact that the left hand side (LHS) is equal to the right hand side (RHS). This requirement will be satisfied, iff we do the \emph{same thing} to both sides. If we subtract $3$ from both sides of the equation $3 + x = 5$, we will get $3 + x - 3 = 5 - 3$ which simplifies to $x = 2$ and we are done. This was a very simple example. In general, we may have to combine many manipulation steps, one after another and in just the right order to reach our goal. This process can become arbitrarily complicated  and it may require some experience and intuition to figure out, which transformations will bring us closer to our goal in a given situation. It can also turn out to be impossible for a given equation. Two of the most basic transformations are: (1) add or subtract a constant from both sides, (2) multiply or divide both sides by any constant except zero. There are many more operations that may be ok in a given situation, but some of them come with pitfalls and require special care. For example, you may multiply or divide both sides by the unknown $x$, too - as long as $x$ is not zero, but since $x$ is unknown, you may not know whether or not that's the case. You may also take the square of both sides, but then you need to be aware that this may create extra solutions wich solve only the squared equation but not the original one, so you may need to sieve out these extra solutions later, etc. We won't go deeper into this here because we need to address a more basic problem first: Even for these most basic transformations given above, we may already run into problems. We could solve $3 + x = 5$, but what about $5 + x = 3$? Let's try to isolate $x$ by subtracting $5$ from both sides: $5 + x - 5 = 3 - 5$ which simplifies to $x = 3 - 5$. But what is $3-5$? There is no such natural number! But maybe there's something else...

%In general, adding or subtracting any given constant number from both sides is an allowed operation.
% multiply both sides by the same number (but not with zero)
% divide both sides by the saem number (we must assume that both sides are divisible by that number)
% taking reciprocals of both sides 
% sqauring both sides...with some care
% But: what if applying a transformation results in a
%\medskip
% "subtractability"

\subsection{Integer Numbers}
Within the natural numbers, the equation $3 + x = 5$ has no solution. We could be satisfied with this answer and just say: ok, that particular equation has no solution. In general, this may be reasonable - not every equation that we can write down in mathematics needs to have a solution. But maybe we can do better in this case. If we want to have any chance to solve an equation that doesn't admit a solution within the natural numbers, then our sought solution must be something else. What could that something else be? Well, we need to be able to work with it just like we work with numbers - we want to be able to add, subtract, multiply etc. and we want to be able to mix these new (so far, hypothetical) objects in our operations with the natural numbers that we already have. Thus, we will apparently need some not-so-natural numbers or number-like objects. The first set of "unnatural" numbers that we will define are the negative numbers. Together with the set of natural numbers, they form the set of integer numbers, denoted as $\mathbb{Z}$ (from the german word "Zahl" which means "number"). In reality, you can not take away 5 apples from 3 apples but many bank accounts allow you to go into debt. If you have 30 coins deposited and withdraw 50 coins, your account will go into debt by 20 coins which you will have to pay back later - you've taken a loan. Mathematically, such loan or debt can be modeled by negative numbers, i.e. numbers that are less than zero. That means, in order to arrive at zero (i.e. nothing), you need to add something (i.e. put something in). That's indeed a rather unnatural behavior.  A negative number is written with a minus sign in front of it: a debt of 20 is written as $-20$ and we say: "minus twenty" or "negative twenty". Every natural number except zero has a negative counterpart. In this context, the old natural numbers from 1 upwards are also called positive numbers. Zero is neither positive nor negative - it's just neutrally zero. 

\medskip
In the context of abstract algebra, the number zero is indeed called the "neutral element" with respect to addition because adding zero to any number gives as result just that very same number. Zero is neutral in the sense that it doesn't change anything. In general, a neutral element (with respect to some binary operation) has the property that if it occurs as one of the operands, the result of the operation will just be the other operand. In this context, a negative number like $-5$ is then called the "inverse element" with respect to the number $5$ and vice versa. Two elements are said to be inverses of one another with respect to some binary operation, if putting them as arguments into the operation, the result will be the neutral element of that operation. So, the fact that $5 + (-5) = 0$ shows that 5 and -5 are inverses of each other with respect to addition because 0 is indeed the neutral element of addition. For both, 5 and -5, we call 5 the "absolute value" of the number and we capture the idea that 5 is positive and -5 is negative by its "sign" which can be either positive (plus) or negative (minus). Instead of 5, we may also write $+5$, but the plus sign is usually omitted.

\medskip
We have \emph{defined} the negative numbers in terms of how we want them to behave with respect to addition. But how should they behave with respect to multiplication? We should we want here? It would be convenient if our old laws for multiplication (associativity, commutativity, distributivity) would continue to hold for our new numbers and also hold when we mix old and new numbers in such an operation. If we can make this work, we wouldn't need to unlearn any algebraic laws and could work with our new numbers in exactly in the same way that already works so well for our old numbers. And indeed, it is possible to make this work in one and only one way. We have to define multiplication for the integers in such a way, that the absolute value of the result is equal to the product of the absolute values of the operands and the sign of the result is negative, iff one of the operands is negative and the other is positive. In particular, the product of two negative numbers has to be positive.


\subsection{Rational Numbers}
Motivated by the unsolvability of a particular equation within the realm of natural numbers, we have successfully extended the set of numbers that we have to work with. Now, we are able to solve a far greater set of equations! But there are still some other equations that we cannot yet solve. Consider the equation $3 \cdot x = 5$. Trying to isolate $x$ on the LHS calls for dividing both sides by 3. That will give $x = 5/3$, but just like the expression $3-5$ didn't represent any natural number, the expression $5/3$ does not represent any integer number. By trying to solve a particular equation, we have discovered yet another kind of number. It is called a rational number. I have once heard a phrase like $5/3$ is the number that you would get, if $5$ would be divisible by $3$.

...TBC...

% rules for arithmetic, countability (Cantor's 1st diagonal argument), decimal expansion (fixed and floating point), explain the notion of a field - the rational numbers are the first example of a field


%\paragraph{Arithmetic Operations}
%... In higher mathematics, specifically in abstract algebra, it turns out that a more convenient way to think about arithmetic operations is to only consider addition and multiplication as basic and cast subtraction and division in terms of these two basic operations and so called inverse elements. The inverse element of a rational number $q$ with respect to addition is just its negative version, i.e. $-q$. The inverse element of $q$ with respect to multiplication is the reciprocal of $q$, written as $1/q$ or $q^{-1}$. The set of rational numbers is the first set in which both of these inverses exist for any given $q$, except for $q=0$ which has no multiplicative inverse. 



%\subsection{Intervals}
%So far, we've built up our number system as $\mathbb{N \subset Z \subset Q}$. In order to approach our next set of numbers, we'll need a little preliminary, so we'll take a small break from extending our set of numbers and have a little interlude. We will need to define what an interval is
% -define what open and closed intervals are
% -A subset I of the real number line R is called an interval if for all r1,r2 \in I and
%  all x \in R with r_1 < x < r_2, we have: x \in I


%https://www.youtube.com/watch?v=d9z-DV9bLfw
%https://en.wikipedia.org/wiki/Interval_(mathematics)



% Hmm - it appears it's the other way around: we first need real numbers and then define intervals based on that. I think, I wanted to defin e intervals first and the define real numbers in terms of dedkind cuts which themselves are defined as dividng the rationla number line Q into two intervals?

% Maybe define minimu, maximum, infimum and supremum and then define real numbers as adding all the possible supremums and infimums - like we add sqrt(2) by saying it is the supremum of the set x \in Q: x^2 <=2



\subsection{Real Numbers}
OK, nice - so the set of rational numbers is a field. We can add, subtract, multiply and divide to our hearts content without ever encountering any object that does not belong to the set (with the sole exception of division by zero). Is that finally good enough or could we still want more? The pattern of coming up with unsolvable equations continues: Consider the equation $x^2 = 2$. We can take the square root of both sides to get the solution $x = \sqrt{2}$. If the square root of two is a rational number, then we can indeed solve that equation within the rational numbers - but is it? The answer turns out to be a resounding nope! I could now present a proof for this fact (and a mathematically minded reader would rightfully demand one) but I think that would distract too much from the main line, so I just ask you to take my word for it or look it up elsewhere - there are actually many very different proofs for that fact. Besides the square root of two, there are many more numbers that are still "missing". Our number line still has "gaps" and in a sense that can be made explicit, actually "most" numbers are still missing. If we "fill in" these missing numbers, we arrive at what we call the \emph{real numbers}. Defining the process of filling in the missing numbers is rather technical and can be skipped on a first reading.



\subsubsection{$\star$ Completeness}

\paragraph{Dedekind Completeness}




% https://en.wikipedia.org/wiki/Least-upper-bound_property
% https://en.wikipedia.org/wiki/Completeness_of_the_real_numbers
% https://en.wikipedia.org/wiki/Cauchy_sequence

% https://en.wikipedia.org/wiki/Least-upper-bound_property#Logical_status
% The least-upper-bound property is equivalent to other forms of the completeness axiom, such as the convergence of Cauchy sequences or the nested intervals theorem.


\paragraph{Cauchy Completeness}

\subsubsection{$\star$ Order}
The real numbers are an ordered field and they are Dedekind complete. Any other field that has these two properties is isomorphic to the real numbers. TBC...

% https://en.wikipedia.org/wiki/Ordered_field


\subsubsection{$\star$ Density ans Sparsity}
OK - so we have now filled in the gaps between the rational numbers and thereby obtained a continuous set of numbers. How many new numbers did we get? Apparently, infinitely many - but can we specify this some more? The rational numbers are already \emph{dense} in the sense that in between \emph{any} pair of rational numbers $q_1, q_2$, there are infinitely many more  rational numbers. You can see this by explicitly constructing one rational number that is between $q_1, q_2$, for example, their average: $q_3 = (q_1 + q_2) / 2$. It is clearly again a rational number and it is also clearly in between the two original rational numbers $q_1$ and $q_2$. Now you can use your new $q_3$ together with either $q_1$ or $q_2$ to construct another rational number between $q_3$ and $q_1$ (or $q_2$). You can repeat this process ad infinitum and thereby generate infinitely many new rational numbers that are all in between $q_1$ and $q_2$. Yet, despite being so dense, they are also sparse within the reals in the sense that the vast majority of all  real numbers on the number line is irrational. Making size comparisons between infinite sets is a somewhat advanced topic for which we will need (some bits of) the theory of cardinalities and/or measure theory. We will get into some of the the details of that in later chapters. For now, it shall suffice to say that that the rationals are a set of measure zero within the reals and that the cardinality of the reals is strictly greater than that of the rationals....TBC...

% -explain completeness, Cauchy sequences
% -maybe briefly explain Dedekind cuts (that's why we introduced the intervals before the
%  real numbers)
% -intervals, uncountability (Cantor's 2nd diagonal argument), 
% -algebraic numbers (as % subset...but wait - they can be complex, too)
% -explain how operations like add/mul/pow are extended to the reals
% -maybe that should explain it for rationals first, then approximate reals by rationals
%  and take a limit
% -maybe the explanantion for the rationals should be in the section of the rationals.

% https://www.youtube.com/watch?v=iGpUS3rejso
% at 26:35: we want a set where every nonempty subset that is bounded from above has a supremum. That's what we call "complete". ..but there's also that defintion in terms of cauchy sequences which is probably equivalent.



\subsection{Complex Numbers}
We have established the set of real numbers forms an ordered, Dedekind-complete field. Does "complete" mean, we are now done with building our number system? Unfortunately not yet. There is still one kind of equation left that we can write down with our basic arithmetic operators but which we still cannot yet solve. But this one will really be the last one, so we see the light at the end of the tunnel! The equation, I'm talking about, is $x^2 = -1$. Solving for $x$ formally yields $x = \sqrt{-1}$, but we know that any real number muliplied by itself will produce a nonnegative result, so this weird square root of minus one cannot be a real number. We will now do a similar step as we did when postulating the existence of negative numbers. Those can be thought of as resulting from taking a natural number and multiplying it by some, then hypothetical and unnatural, new object - namely, by the number $-1$, which we could call the "negative unit" (a nonstandard term, I just made up) and which up to then, wasn't a thing. In analogy, we now define the square root of minus one as the "imaginary unit", which is indeed standard terminology, denoted by $\i$. Just like multiplying a natural number $n$ by the "negative unit" $-1$ gave us the negative counterpart of $n$, namely $-n$, multiplying any real number $r$ by the imaginary unit $\i$ gives us the imaginary counterpart of $r$, namely $\i r$. The general law $\sqrt{a b} = \sqrt{a} \sqrt{b}$ of square roots now allows us to take square roots of any negative number: we just factor out the square root of minus one, for example: $\sqrt{-9} = \sqrt{9} \sqrt{-1} = 3 \i$. The square root of minus nine is three times the imaginary unit.

..TBC...

% Actually, real numbers already can be seen as kind of 2D but the 2nd dimension is only a binary sign, i.e. allows only two possible values. We could see the negative numbers as the positive number ray reflected around the origin. Then we could generalize them by observing that a reflection can also be seen as a rotation by 180° and allow arbitrary angles (Verify if that is true - reflections reverse orientations, rotations don'T - I think). We would get number rays at all angles. We would naturally arrive at a polar form of complex numbers. The absolute value is the radius and we have an angle, too. Geometrically, a reflection is a discrete yes/no operation whereas rotation is a continuous one. Negation (i.e. multiplication by -1) is reflection. Reflection is rotation by 180°, so "half-reflection" (i.e. multiplying by sqrt(-1)) should be rotation by 90°. Thinking about numbers in this way, we realize that the bidirectional horizontal real number line, as we usually draw it on a sheet of paper, is actually built from two number rays, positive and negative, with a zero added right in the middle. Looking at the number rays geometrically on the 2D sheet of paper and realizing that reflections are just very special kinds of rotations on our 2D sheet, we suddenly see all the other possible directions of the number ray. They have been there all along, right there on our 2D sheet of paper. We just didn't see them due to a mental blockade. A whole new dimension has just opened up in our mental image of what a number is. Mind = blown!

% View complex numbers as encoding geometric transformations. Adding is just a translation. Multiplication can do scaling, reflection and rotation. So, with complex numbers, we can do all the relevant geometric transformations of Euclidean geometry. Quadruples of complex numbers can encode even more transformations: the Moebius transforms. These include translations, scalings and rotations as subset - these are those where the denominator is 1. I think, the group can be created by also allowing inversion and therefore division. 

% what about equations that do not only involve algebraic operations? what if an equation involves transcendental functions? will there still be always a complex solution? or could it make sense to define yet more numbers to solve more equations....like, say tanh(x) = 2...will this have complex solutions? maybe that's the reason why analytic functions are always unbounded over the complex plane, such that they can be inverted for any input?

% https://en.wikipedia.org/wiki/Real_number


% remind of the "unnaturalness" of negative numbers - make the point that imaginary numbers are no more unnatural than negative numbers

% Polar form: mention different names for the two components: radius / magnitude / modulus / absolute value / amplitude and angle / argument / phase ...and how they belong together mag/phs, amp/phs, rad/ang, mod/arg - maybe prefer to use rad/ang for their clear geometric meaning

% Explain geometric interpretation of complex addition and multiplication. Is there also a geometric interpretation of exponentiation? The real part of an exponent changes the length, the imaginary part changes the angle? So, it's beset interpreted when the basis in in polar form and the exponent is in cartesian form?

% Complex numbers cannot be ordered in a way that cooperates nicely with multiplication and addition - maybe one could come up with orderings that cooperate with one of these, though: ordering by real/image should cooperate with addition and ordering accordine to mag/phase may cooperate with multiplication (VERIFY!)


\subsubsection{Solving Equations, revisited}
....ToDo: explain fundamental theorem of algebra....
% I don't know anymore what I intended to talk about here. Maybe just recapitulate or state that we are now able to solve all algebraic equations? Maybe state the fundamental theorem of algebra? It would fit here well.

\subsubsection{Are we done?}
I promised that $x^2 = -1$ will be the last equation that we can write down using only the numbers we have and our basic arithmetic operations and that the complex numbers will enable us to solve all such algebraic equations. But you may ask what about $0 \cdot x = 1$? Solving for $x$ formally gives $x = 1/0$ and we know that we can't (yet?) divide by zero. Is it conceivable that we again broaden our notion of what a number is and invent another kind of "imaginary unit" for the strange object $1/0$? Perhaps a sort of "infinite unit"? In certain contexts, the set of complex numbers is indeed augmented by an additional element, called the "point at infinity", denoted as $\infty$, and the resulting set is called the extended complex plane. But we cannot really compute with that $\infty$ thing in the same way as we do with actual numbers. Trying to come up with a number system in which division by zero is defined would force us to break certain laws of arithmetic. For example, we would have to give up on the tried and true law $0 + 0 = 0$. ...TBC...

% Having $\infty$ available, we can indeed write down things like $1/0 = \infty$ and also $1 / \infty = 0$. But there is a caveat: We also have $2/0 = \infty, -3/0 = \infty, (4 + 5 \i) / 0 = \infty $ etc. Any finite complex number divided by zero would have to give infinity as well. There is only one point at infinity in the extended complex plane. But why? Can't we, for example, make $2 \cdot \infty$ be something different than $\infty$? In the context of real numbers, we indeed at least distinguish between positive and negative infinity. I don't really know the answer to that but I think the current state of the art is that from  $1/0 = \infty$ we can't conclude that $1 = 0 \cdot \infty$, so this notion of infinity doesn't really help us to solve equations uniquely so the usefulness of that idea is a bit limited. In most situations, the equation $0 \cdot x = 1$ is usually indeed just accepted to be unsolvable. Maybe someday someone will come up with a concept of a number that makes that equation solvable, too. I tried myself - and failed, so far....

% It doesn't work if we want to maintain the law: 0+0 = 0, see:
%https://www.youtube.com/watch?v=v05rI0lGk6A&t=348s

\subsubsection{Digest}
The complex numbers are by many people seen as very mysterious, inscrutable and unintuitive objects. By analogy, I tried to make the point that they are actually no more mysterious or more imaginary than the negative numbers, but I'm not sure, whether I buy that argument myself. So, if you still feel uncomfortable about them and consider them mysterious - don't worry: you are not alone. Welcome to the club! The complex numbers may be indeed a bit hard to digest. Perhaps for this reason, in my high school math education, they were never mentioned - the reals were, where they stopped. I find that sad because it feels a bit like stopping one step before the finishing line. Maybe that has changed and/or is different in other countries, which I certainly hope because from a mathematical point of view, the complex numbers - and not the reals - mark the logical and satisfying endpoint of a progression of discovering ever more numbers.

% also talk about the extended complex plane, i.e. with the additional point at infinity

\subsection{Other Numbers}
What? There are still other numbers? Didn't I just say, the complex numbers mark the endpoint? It is indeed true that we do not need to invent other numbers if the goal is to be able to solve algebraic equations (at least, if we exclude the nasty division-by-zero case). But there may be other goals and other applications in which other notions of a "number" may be useful. That raises the question, what a number actually is. We initially used (natural) numbers to count things. Then, negative numbers were included to model the state of having "less than zero" of a quantity, i.e. a debt. The rational numbers were introduced to model splitting a quantity into equally sized chunks. Later we broadened our scope to use real numbers for a continuously measurable quantity such as a length in geometry. This was necessary because we discovered (or rather, I told you) that not all such lengths can be expressed by rational numbers. Finally, the complex numbers were introduced for a more abstract reason: the desire to be able to solve algebraic equations. In the solution process, roots of negative numbers occured and complex numbers allowed us to deal with them. ...tbc...

\subsubsection{Cardinal Numbers}
Cardinal numbers are used to characterize the "cardinality", i.e. the size, of a set. If the set is finite, its size is just the number of elements - which is a natural number.  tbc..
% https://en.wikipedia.org/wiki/Cardinal_number
%https://en.wikipedia.org/wiki/Continuum_hypothesis

%https://en.wikipedia.org/wiki/Cardinal_number#Cardinal_multiplication
% for example $|\mathbb{Q}| = |\mathbb{N} \times \mathbb{N}| = |\mathbb{N}|$, $|\mathbb{C}| = |\mathbb{R} \times \mathbb{R}| = |\mathbb{R}|$, which may be counterintuitive at first. In general $\kappa \cdot \mu = max(\kappa, \mu)$ if $\kappa$ or $\mu$ is infinite and both are nonzero.

\subsubsection{Ordinal Numbers}
We may use numbers not only to count "how many" things there are in total but also to put things into a particular order. One object is the \emph{first}, another is the \emph{second}, then comes the \emph{third} and so on. ...tbc...
% cardinal numbers give the size of a set, ordinal numbers give the index of an object in a tuple?

\medskip
todo: make small paragraphs about: ordinal, cardinal, algebraic, transcendental, Gaussian, hyperreal, dual, p-adic, surreal, clifford numbers (aka multivectors), quaternions, octonions, sedenions, modular integers, Galois fields - what about matrices as representations? ...what about wheel theory?


%https://en.wikipedia.org/wiki/Ordinal_number

% perhaps a generalized set of numbers is any set of objects that contains at least some natural numbers as subset and in which we have a notion of addition and multiplication

%

\subsection{Machine Numbers}
There are infinitely many numbers and even a single real number may need infinitely many digits to be represented exactly. Computers can only deal with finite things, so in order to work with numbers in a computer, we will need to cut some corners. In a modern computer, you will find two basic forms of number representations that are realized on the hardware level: integers and floating point numbers where the latter are an attempt at modeling the real numbers. From these basic types supported directly by the machine's hardware, more complicated number types can be constructed in software. It's important to be aware of the implications of the finite number representations in a computer. Integers may "overflow", i.e. wrap around - what's actually being done is modular arithmetic where the modulus is a large'ish power of 2, often large enough to be safe from wrap-around (like $2^{31}$ or $2^{63}$), but sometimes, when dealing with really large integers, this may not be enough. Some higher level languages, like Python or Mathematica, don't have this issue because they will use software implementations of arbitrary size integers, if needed. But in lower level languages like C++ or Java, you need to be aware of this possibility. With floating point numbers, overflow is usually less of an issue (although it can happen, too) but they will need some rounding in most cases. This introduces small errors which are usually tolerable on a per-operation level but these errors may accumulate in a long sequence of operations in such a way to render the final result a completely meaningless "random" number. Whether or not that happens depends on the exact sequence of operations, i.e. the algorithm. Two algorithms which are supposed to give the same result if all operations are assumed to work with exact values may actually deliver wildly different accuracy of the result in real life on a computer. This is a question of so called "numerical stability" and addressed in the context of numerical analysis which is also a field of mathematics which we'll encounter later.

% two's complement, IEEE754 float
% give algorithms (in sage or pytthon) to compute with arbitrary size integers and multiprecision floating point numbers

\begin{comment}

-introduce sum and product notation, factorial, binomial coeffs

-say something about countably and uncountably infinite sets (defined via power sets) - real numbers are uncountable but it's yet unknown if their cardinality is aleph1 or aleph2, i think
-numbers in the computer: 2-complement integers, IEEE754 float (half/single/double/quad), big int, big float, overflow, precision, precision loss in subtraction of similar values

Resources:
-Climbing past the complex numbers. https://www.youtube.com/watch?v=q3Tbf-d9sE4
 Explains the Cayley-Dickson construction of C,H,O,S and beyond. See also:
 https://en.wikipedia.org/wiki/Cayley%E2%80%93Dickson_construction
 
https://www.youtube.com/watch?v=PQAhC1M93C8  Can any Number be a Base?
...explains nicely different bases including base 1, negative, rational, algebraic transcendental
and complex bases. Explains how e arises as minimum of the radix economy function b/ln(b)

\end{comment}


