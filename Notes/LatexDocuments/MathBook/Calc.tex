\chapter{Calculus}
Calculus is the study of continuous functions and provides the tools for computing slopes of tangents, areas under curves and identifying important features such as extrema and inflection points. These tools are mainly derivatives and integrals. Functions can also be unknowns which in the context of calculus can be determined by differential- or integral equations. This is especially important in the physical sciences. Calculus also provides tools to approximate a given function by simpler functions that are well understood, most notably by Taylor- and Fourier series. This is important in numerical computations. Series themselves are also an important object of study in calculus. A series is an infinite sum of all the elements of an infinite sequence of numbers. Besides their role in the usual ("infinitesimal") calculus, sequences and series can also be studied by a discrete version of calculus which closely parallels the regular calculus. The discrete counterparts of derivatives, integrals and differential equations are differences, sums and difference equations. The discrete calculus may involve a step size $h$ (when absent, it's usually assumed to be unity) and the continuous calculus may be seen as the limiting case, when the stepsize approaches zero. To figure out what happens in this limiting process, calculus also studies the idea of a limit itself. This idea is actually the foundation of the usual calculus on which the ideas of derivatives and integrals are based.