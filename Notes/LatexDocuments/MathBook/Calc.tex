\chapter{Calculus}
Calculus is the study of continuous functions and provides the tools for computing slopes of tangents, areas under curves and identifying important features of a function such as extrema and inflection points. The new tools are mainly derivatives and integrals which, together with our old algebraic tools of solving equations, will enable us to solve such problems. Functions can also be unknowns which in the context of calculus can be determined by differential- or integral equations. This is especially important in the physical sciences. Calculus also provides tools to approximate a given function by simpler functions that are well understood, most notably by Taylor- and Fourier series. This is important in numerical computations. Series themselves are also an important object of study in calculus. A series is an infinite sum of all the elements of an infinite sequence of numbers. Besides their role in the usual ("infinitesimal") calculus, sequences and series can also be studied by a discrete version of calculus which closely parallels the regular calculus. The discrete counterparts of derivatives, integrals and differential equations are differences, sums and difference equations. The discrete calculus may involve a step size $h$ (when absent, it's usually assumed to be unity) and the continuous calculus may be seen as the limiting case, when the stepsize approaches zero. To figure out what happens in this limiting process, calculus also studies the idea of a limit itself. This idea is actually the foundation of the usual calculus on which the ideas of derivatives and integrals are based. Derivatives and integrals are founded on the general idea of making an approximation that contains some parameter. We can tweak that parameter and by making it smaller, we can get a more accurate approximation. Then, we take our approximation to the limit where our "accuracy" parameter becomes zero and that will yield the exact result.


%Making that parameter, we can choose, how accurate the approximation should be. 

% general theme: make an approximation that contains a precision parameter and then take that approximation to the limit


% Historically, perhaps the first mathematical concept that we nowadays subsume under calculus is the computation of areas of various shapes. This is actually an idea from geometry and the ancient greeks, being mostly geometers\footnote{and number theorists}, already developed some methods for that which work in certain special cases like the circle. ...

% Later, the idea of an "instantaneous rate of change" became a topic of interest. This could be geometrically interpreted as the slope of a function at a given point which we envision as the slope of a startight line that is tangent to the respective function at the point of interest. ...

% The beginnings of calculus as we know it today started informally with non-rigorous concepts like "infinitesimal" quantities which are quantities that are supposed to be greater that zero (such that we are allowed to divide by them) but smaller than any real number. At that time calculus was called "infinitesimal calculus" [VERIFY] and the whole field lacked the rigor that we normally would expect from mathematics. It was filled with appeals to intuition which were mostly correct but couldn't be rigorously justified at the time ...

% Later, all these ideas of infinitesimal quantities were abandoned and the whole topic of calculus was rephrased in terms of limits and convergence. The modern view puts the idea of a limit at the foundation of all things calculus and considers derivatives and integrals as special cases of this idea. Other special cases are infinite sums and products and also limiting values (aka fixed points) of arbitrary algorithms. Even later, the ideas of infinitesimals were resurrected and sort of rehabilitaed in a field called "non-standard analysis". ...

\begin{comment}

https://en.wikipedia.org/wiki/Calculus
https://en.wikipedia.org/wiki/Mathematical_analysis
https://en.wikipedia.org/wiki/List_of_calculus_topics

adage: "Algebra is about euqations, analysis is about inequalities"

https://www.youtube.com/watch?v=G-29AZf-NkU  Die Monster der Analysis



Every Type of Math in 7 Minutes - Part 1: Calculus and Analysis
https://www.youtube.com/watch?v=OHac-N9KhYw




\end{comment}