\section{Differentiation}

\subsection{The Derivative}
Consider an arbitrary function $f(x)$ and imagine that we want to compute the slope of the graph at a given $x_0$. We could approximate it by considering the point $(x_0, f(x_0))$ and another point on the graph a small distance $h$ further to the right $(x_0+h, f(x_0+h))$ and compute the "rise over run" quotient $(f(x_0+h) - f(x_0)) / h$ which would give us the slope of a secant. For smaller and smaller $h$, the approximation would get better and better because the secant would approach the tangent. Now we consider the limit as $h$ approaches zero. If that limit exists, it actually \emph{is} our desired slope of the tangent. We denote that tangent slope at $x_0$ as $f'(x_0)$. This leads us to the following definition:
\begin{equation}
  f'(x_0) = \lim_{h \rightarrow 0} \frac{f(x_0 + h) - f(x_0)}{h}
\end{equation}
We note that we could compute this limit at any value of $x_0$ where that limit exists, so if the limit exists everywhere, we can actually replace $x_0$ by $x$ because $x_0$ just an arbitrary variable name anyway which we may choose any way we want, as long as the name doesn't collide with any other names in the equation. Then, this definition actually defines a whole new function $f'(x)$. This new function $f'$ ("f prime") is called the derivative of $f$.

\paragraph{Example} Let $f(x) = x^2$. Plugging it into the definition above, using $x$ in place of $x_0$ as explained, we get:
\begin{equation}
  f'(x) = \lim_{h \rightarrow 0} \frac{(x + h)^2 - x^2}{h}
        = \lim_{h \rightarrow 0} \frac{x^2 + 2 h x + h^2 - x^2}{h} 
        = \lim_{h \rightarrow 0} \frac{2 h x + h^2}{h} 
        = \lim_{h \rightarrow 0} 2 x + h
        = 2 x
\end{equation}
so the derivative of $f(x) = x^2$ is $f'(x) = 2 x$.

\paragraph{Notations} Writing $f'$ with the prime for the derivative of $f$ is only one of several different notations. It is called the Lagrange notation. Especially in physics and when the independent variable is time $t$, you will also often see a notation with a dot above the function name, here $f$, such that the derivative of $f(t)$ with respect to $t$ would be written as: $\dot{f}(t)$. This is called Newton notation. Yet another notation is due to Leibniz and that notation writes a $\frac{d}{dx}$ in front of the $f$ such that $f'(x) = \frac{d}{dx} f(x) = \frac{d f(x)}{dx}$. The Leibniz notation can be convenient when manipulating equations because in certain algebraic transformations, the $dx$ can indeed be treated like a denominator in a fraction. The $\frac{d}{dx}$ in front of the $f$ can actually also be seen as an \emph{operator} that "acts on" the function $f$. There is another such operator notation that uses just a $D$ instead, i.e. $f' = D f$. I have suppressed the argument here, i.e. wrote just $f$ instead of $f(x)$. This is also common practice. In another operator notation, the argument of the operator $D$, which is the function $f$, is put into brackets: $f' = D[f]$. You may also encounter notations like this:
\begin{equation}
	\frac{d}{d x} f(x) \bigg\rvert_{x=a}
\end{equation}
This means: take the derivative of $f$ with respect to $x$, then evaluate the resulting expression at $x = a$.

%ToDo: Introduce the "evaluated at" notation with the vertical bar on the right.

\paragraph{An Alternative Definition}
An alternative definition that you won't usually find in textbooks is this:
\begin{equation}
 f'(x) = \lim_{t \rightarrow 1} \frac{f(t x) - f(x)}{ t x - x }
\end{equation}
If we apply this definition to the function $f(x) = x^2$, we get:
\begin{equation}
f'(x) = \lim_{t \rightarrow 1} \frac{(t x)^2 - x^2}{ t x - x }
      = \lim_{t \rightarrow 1} \frac{t^2 x^2 - x^2}{ t x - x }
      = \lim_{t \rightarrow 1} \frac{x^2 (t^2 - 1)}{ x (t - 1)}
      = x \lim_{t \rightarrow 1} \frac{(t^2 - 1)}{(t - 1)}         
\end{equation}
where in the last step, we noted that $x$ does not depend on $t$ so it could be dragged out of the limit. We now see that the derivative is proportional to $x$ and what is left is to figure out the constant of proportionality from the limit, that involves only $t$ but not $x$ anymore. We factor the numerator as $t^2 - 1 = (t-1)(t+1)$ to get:
\begin{equation}
f'(x) = x \lim_{t \rightarrow 1} \frac{(t-1)(t+1)}{(t - 1)}     
      = x \lim_{t \rightarrow 1} t+1
      = 2 x
\end{equation}
which gives us the same result. This alternative definition is also related to the so called quantum derivative which is basically the same definition just without taking the limit. Sometimes, this alternative definition is much easier to evaluate than the standard definition. That's why I wanted to mention it. This is especially true when $f(x) = x^n$. With the standard definition, you'd have to expand a term of the form $(x+h)^n$ using the binomial theorem whereas with the alternative definition, things cancel nicely. However, usually, we do not evaluate derivatives via the definition over a limit (standard or alternative) anyway. Instead, we use...
%Fortunately, we don't have to go through this tedious procedure via the limit every time we need to compute a derivative.

\subsection{Differentiation Rules}
Instead of evaluating derivatives from first principles every time, we see one, we have a lookup list of derivatives of all important elementary functions that has been assembled once and for all by the technique above or otherwise and we have a couple of rules how to compute derivatives of more complicated functions that are created in various ways from the more basic functions (btw: Did I already say that a big theme in math is finding computational shortcuts? This is a nice example). These rules are for any two functions $f$ and $g$ and any constant $a$:

\medskip
\begin{tabular}{c c c l}
  $(a f)'$        &$=$& $ a f'$              & Homogeneity \\
  $(f + g)'$      &$=$& $f' + g'$            & Sum rule (aka Additivity) \\
  $(f \cdot g)'$  &$=$& $f' g + g' f$        & Product rule \\
  $(f / g)'$      &$=$& $(f' g - g' f)/g^2$  & Quotient rule \\
  $(f( g))'$      &$=$& $g' \cdot f'(g)$     & Chain rule \\
\end{tabular}
\medskip
% are there more rules? what about logarithmic differentiation and derivative of inverse?

The two first properties, homogeneity and additivity, taken together constitute the important property of linearity. We may state the linearity property also in a single formula: $(a f + b g)' = a f' + b g'$ where $b$ is another constant. It can easily be verified that this single formula is equivalent to the first two rules in the list above. One lesser known formula is [VERIFY!]:
\begin{equation}
\label{Eq:DerivativeViaH}
 \frac{d}{d x} f(x) 
 = \lim_{h \rightarrow 0} \frac{f(x + h) - f(x)}{h}
 = \frac{d}{d h} f(x + h) \bigg\rvert_{h=0}
\end{equation}
whose usefulness lies mostly in its generalizations to the directional derivative and Gateaux derivative which we will encounter later. There, we will see a similar definition via a limit and the RHS will tell us, how to actually evaluate the limit....but that's for later... 
% Where did I get this formula from? I think, I derived it myself? Maybe via a Taylor expansion of f? Or via specializing to 1D the directional derivative which itself is more easily to "see" intuitively? Or maybe we can define a bivariate function f(x,h) = f(x+h) and derive it from there using partial derivatives? Or maybe evaluate the limit directly from the bivariate function?

%todo: list derivatives of elementary functions

\subsection{Elementary Derivatives}
We'll now give a list of some of the most important derivatives of elementary functions. This list (or a more comprehensive version of it that can be found elsewhere), together with the differentiation rules above, will allow us to mechanically evaluate derivatives of all functions that can be constructed from these elementary functions by algebraic operations or composition (i.e. nesting).
...

\subsection{Higher Order Derivatives}
With the derivative, we have an operation to produce a new "derived" function from a given function. There's nothing that stops us to apply the same operation to the derived function as well. And then iterate that process as often as we like. That's the way, higher order derivatives are created.

...TBC...

% give all different notations $f'', f''', f^{(k)}, \frac{d^k}{d x^k}, D^k, \ddot{f}, \dddot{f}$




\begin{comment}
	
This video has an interesting alternative definition of the derivative:

https://www.youtube.com/watch?v=XfWgfZ5V2qI  New Definition of the Derivative

f'(x) = \lim_{t \rightarrow 1} \frac{f(tx) - f(x)}{t x  - x}

which is often algebraically simpler to evaluate - especially for powers of x. No binomial theorem is needed. 

It has also this useful formula:
(t^n - 1) = (t-1) (1 + t + t^2 + t^3 + ... + t^{n-1})	
which might be integrated somewhere in the elementary algebra section
	
\end{comment}

