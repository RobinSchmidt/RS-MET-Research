\section{More Exotic Calculus Topics}

%===================================================================================================
\subsection{Integral Equations}

% transforming DEs into IEs and vice versa
% why would one want to express a DE as IE? Maybe to apply certain numerical solution techniques?
% see "The Art of VA Filter Design"

%===================================================================================================
\subsection{Discrete Calculus}
% Integrals are replaced by sums, derivatives by finite differences.

% Why don't they teach Newton's calculus of 'What comes next?'
% https://www.youtube.com/watch?v=4AuV93LOPcE


%===================================================================================================
\subsection{Operational Calculus}

% https://www.youtube.com/watch?v=N8Rxlc-fGr0

% https://www.youtube.com/watch?v=PAyx3Dxfhe8

% Could be picked up more seriously in functional analysis

% The craziest definition of the derivative you have ever seen!
% https://www.youtube.com/watch?v=PoSB9HK4kxg


%===================================================================================================
\subsection{Umbral Calculus}
% https://www.youtube.com/watch?v=D0EUFP7-P1M

% Could it be framed as a bridge back from continuous to discrete where the bridge from discrete to continuous is given by limits?



%===================================================================================================
\subsection{Fractional Calculus}
We know how to take a derivative which we also call the first derivative. We also know how to take a 2nd, 3rd, 4th, etc. derivative. A 0th derivative is also no problem: it's just the function itself. We can even meaningfully take the $-1$st derivative (i.e. the "minus first" or "negative first" derivative). What we want is that taking the $-1$st derivative and then the 1st derivative cancel each other out, i.e. give an identity operation. To achieve that, we may just integrate once\footnote{And commit the sin of forgetting about the integration constant. ToDo: Explain how to deal with it more thoroughly. Maybe we need to talk about left- and right inverse operations. Integrate-then-differentiate will indeed always give back the original function $f$ but doing it the other way around may lose constants.}. So, it looks like, we have a meaningful way to define the $n$th derivative for all integer $n$. But what about rational or even real $n$? Or maybe even complex $n$? Is there a way to meaningfully define a "half-derivative", say? If such an operation exists, we would expect it to behave in such a way that when we apply it twice, we will get the same result as when just taking the ordinary $1$st derivative. It can indeed be done and how to do it is the topic of fractional calculus. ...TBC...

% Imaginary derivative of x
% https://www.youtube.com/watch?v=tMalym_n8zM

% https://en.wikipedia.org/wiki/Fractional_calculus


% Half-Derivative: Between a Function and its Derivative
% https://www.youtube.com/watch?v=SzNO8q0rlPc

%===================================================================================================
\subsection{Non-Newtonian Calculus}
The standard calculus that we use in everyday math is based on the definition of the derivative as:
\begin{equation}
  f'(x) = \lim_{h \rightarrow 0} \frac{f(x + h) - f(x)}{h}
\end{equation}
In the context here, we will call this standard calculus "Newtonian calculus" because it was invented by Isaac Newton\footnote{Gottfried Wilhelm Leibniz would strongly disagree with that. The consensus among historians of mathematics today is that they both invented it independently. And they had a bitter rivalry about who should be credited.}. With other definitions of the "derivative", we may obtain other calculi. For example, consider the following definition:
\begin{equation}
  f^*(x) = \lim_{h \rightarrow 0} \left(  \frac{f(x + h)}{h}  \right)^{\frac{1}{h}} 
\end{equation}

% f*(x) = e^(f'(x) / f(x))


...TBC...


% 
% Product integral - Geometric calculus, Non-Newtonian calculus
% https://www.youtube.com/watch?v=zxJ61HTRPBs

% The magical geometric derivative.
% https://www.youtube.com/watch?v=lQ_AdAFVsaM

%===================================================================================================
\subsection{Stochastic Calculus}

% Stochastic Process, Filtration | Part 1 Stochastic Calculus for Quantitative Finance
% https://www.youtube.com/watch?v=ocnhBnL-DBc
% https://www.youtube.com/watch?v=ocnhBnL-DBc&list=PLvtFb3DXIpiBkpUpgpjmGmNExpJhwcHyy

%===================================================================================================
\subsection{Non-Standard Analysis}

% derivatives "defined" without limits?!?
% https://www.youtube.com/watch?v=7DZIvUptfec

%===================================================================================================
%\subsection{Stochastic Calculus}

% Introducing Weird Differential Equations: Delay, Fractional, Integro, Stochastic!
% https://www.youtube.com/watch?v=eJaHCZ3ITIc
% I think, delay-differential equations are a hybrid between differential and difference equations


% https://en.wikipedia.org/wiki/Integral_equation
% https://en.wikipedia.org/wiki/Delay_differential_equation

\begin{comment}

Every Unsolved Problem in Calculus
https://www.youtube.com/watch?v=An71KvzRd2c
-Casas-Alvera conjecture, Riemann conjecture, Navier-Stokes existence and smoothness, Jacobian 
 conjecture


Maybe make a section Real Analysis
-Should explain: intermediate value theorem, mean value theorem,
 https://en.wikipedia.org/wiki/Real_analysis#Important_results
-See:  https://www.youtube.com/watch?v=vV7ZuouUSfs  Do we even need the real numbers??
 This video show how certain things that we take for granted do not hold when trying to develop 
 calculus on the rational numbers Q. Step functions would be considered continuous according to the
 definition when the step occurs at an irrational number. It's possible to introduce steps at 
 irrationals by not defining the function value *at* the step. For example, one can say:
 f(x) = 1 for x > sqrt(2) and 0 for x < sqrt(2). We don't have to say what f is a sqrt(2). This f
 would be considered to be continuous over Q. ...which is not what we want. For the function 
 f(x) = x^2 - 2, the intermediate value theorem fails. For f(x) = 1/(x^2 - 2), the mean value 
 theorem fails. The extreme value theorem fails for f(x) = (x^2 - 2)^2
 


\end{comment}