\section{Integration} 


\subsubsection{Integration by Parts} 
Write the function $f(x)$ to be integrated as a product $u(x)v(x)$ such that the remaining integral over $u' v$ in the formula is easier to evaluate than the original one. This usually means to choose a $u$ that simplifies under differentiation and/or to choose a $v'$ that simplifies under integration. The formula for a definite integral is:
\begin{equation}
  \int_a^b u(x) v'(x) \, dx = \Big[u(x) v(x)\Big]_a^b - \int_a^b u'(x) v(x) \, dx
\end{equation}
With $u = u(x), du = u'(x) dx, v = v(x), dv = v'(x) dx$, an indefinite integral can be written using the differentials $du, dv$ as:
\begin{equation}
  \int u \, dv \ =\ uv - \int v \, du
\end{equation}
The formulas can be derived from the product rule of differentiation.

\paragraph{Example} Let's find the indefinite integral of $f(x) = 2 x \ln x \, dx$ using $u' = 2 x, v = \ln x$ such that $u = x^2, v' = 1/x$ and $v' u = x^2/x = x$. Note that the roles of $u,v$ are reversed here. They are just dummy names and you'll find formulas using both conventions, so a bit of flexibility is good:
\begin{align}
  \int 2 x \ln x \, dx = x^2 \ln x - \int x \, dx = x^2 \ln x - \frac{x^2}{2} + C
\end{align}
...yeah, the $+ C$ has been neglected (i.e. taken to be zero) in the formulas above. It's more convenient to just add the integration constant at the very end of the calculation.


%\newline % we want some vertical space - newline doe not really cut it
See also:
\href{https://en.wikipedia.org/wiki/Integration_by_parts}{wikipedia.org/wiki/Integration\_by\_parts}

% Applications: finding adjoint operators (move integral from one function to the other)
% https://www.youtube.com/watch?v=aG5tFA8GJ78 Linear Operators and their Adjoints (Nathan Kutz)


% ToDo: Substitution, Logarithmic integration, integration of inverse?


\begin{comment}

Differentiation is a purely mechanic process: given the list of elemenary derivatives and the differentiation rules, we can easily compute an expression for the derivative of any function. In may be tedious and the resulting expressions may become unwieldy, but in principle we know exactly what to do. This makes differentiation a perfect task for a computer and every computer algebra system will readily do this. Integration is much more difficult. Given an arbitrary function, it may not be immediately clear which rule should be applied in order to find an antiderivative. For many functions, an antiderivative that can be expressed as a closed form formula involving only elementary functions may not even exist. However, there is an algorithm that can compute any elementary antiderivative, if it exists or prove the nonexistence if it doesn't. That algorithm is the Risch algorithm and it is so complicated that, at the time of this writing, most computer algebra systems do not implement it in its full glory but rather in some simplified and less general form.

https://mathoverflow.net/questions/374089/does-there-exist-a-complete-implementation-of-the-risch-algorithm
https://mathematica.stackexchange.com/questions/140088/does-mathematica-implement-risch-algorithm-if-it-does-in-which-cases

-as example for computing a Riemman integral from the definiton, use f(x) = x^2 and integrate it from zero to some variable upper limit b
-define a sum S_N = \sum_{k=1}^N (k dx)^2 dx with dx defined as (b-a)/N with a=0, so it's just dx=b/N
-this leads to S_N = dx^3 \sum_k k^3
-using \sum_k k^3 = (2N^3+3N^2+N)/6 via Faulhaber's formula we can get rid of the sum
-this leads to an expression containing powers of N in the numerator and N^3 in the denominator
-when N -> inf only the leading term remains and we get \lim_{N -> inf} S_N = b^3/3
-this should just serve as a demonstartion that in princpile, it's possible to evaluate integrals from the definitions, using rules of limits and closed form sum formulas as an analogy how it is in princpile possible to evaluate derivatives using limits
-state that nobody actually evaluates integrals like that, just like with derivatives

\end{comment}