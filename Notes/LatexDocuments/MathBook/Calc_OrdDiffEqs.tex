\section{Ordinary Differential Equations} 
Ordinary differential equations (abbreviated as ODEs) are equations in which the unknown quantity is not just a single variable like $x$, i.e. a number, but rather a full function $f:\mathbb{R} \rightarrow \mathbb{R}$. The equation may contain the function itself as well as derivatives of it. The presence of derivatives is what makes it a "differential" equation. The goal is to find a function $f = f(x)$ that obeys a certain given relationship between itself and its own derivatives. For example, the function $f(x) = \e^x$ obeys the relation $f = f'$ and the function $f(x) = \sin (x)$ obeys the relation $f = -f''$. These two are perhaps the simplemost nontrivial examples of ODEs and are practically very relevant. In ODEs, what we are \emph{given} is the relation like $f = -f''$ and what is to be found, i.e. what is \emph{unknown}, is $f$ itself. Consider the slightly more complicated example $f + a_1 f' + a_2 f'' = 0$ for some given constant coefficients $a_1, a_2$. We see that for $a_1 = 0, a_2 = 1$, this ODE reduces to the ODE for the sine: $f = -f''$ and for $a_1 = -1, a_2 = 0$, we get the ODE for the exponential function $f = f'$. Solving the differential equation means to find a function $f = f(x)$, which, when it is being plugged into the ODE and all the derivatives are evaluated, yields a true statement. It can straightforwardly (albeit tediously) be verified by evaluating the derivatives that $f = c_1 \e^{-\sigma x} + c_2 \e^{\sigma x}$ is indeed a solution to $f + a_1 f' + a_2 f'' = 0$ when we choose $\sigma = (a_1 + \sqrt{a_1^2 - 4 a_2})/(2 a_2)$. We will actually go through this example in more detail later. For now, I'll just say that \emph{verifying} that some given function is a solution to an ODE is an easy task but \emph{finding} such a solution in the first place is hard. This is a bit like with finding vs verifying antiderivatives but even more so - meaning that finding the solution is even harder in general. Some people even refer to the process of finding a solution to a differential equation as "integrating the differential equation" and integration is indeed often needed as a subtask in the process of solving an ODE. I'm not so fond of that terminology (it sounds unnecessarily obtuse) and will just call it "solving the ODE" rather than "integrating the ODE".

...TBC...

% give the solution to the damped oscillator equation and encourage the reader to verify that it is a solution indeed. Give sage and mathematic code to find the solution

% Wolfram alpha gives the solution only in complex form:
% DSolve[f[x] + a_1 f'[x] + a_2 f''[x] == 0, f[x], x]

% maybe use concrete coeffs: 9 y + 2 y' + 5 y'' = 0, y(0)=0, y'(0)=1
% https://www.wolframalpha.com/input?i=9+y+%2B+2+y%27+%2B+5+y%27%27+%3D+0%2C+y%280%29%3D0%2C+y%27%280%29%3D1
% this gives a nice damped oscillation




% https://en.wikipedia.org/wiki/Harmonic_oscillator#Damped_harmonic_oscillator
% https://en.wikipedia.org/wiki/Ordinary_differential_equation

\subsection{Solution Methods}

\subsubsection{Analytical Methods}

\paragraph{Separation of Variables}

\paragraph{Making an Ansatz}

% https://www.wolframalpha.com/input?i=derivative+of+e%5E%28a+x%29+sin%28w+x%29
% https://www.wolframalpha.com/input?i=second+derivative+of+e%5E%28a+x%29+sin%28w+x%29

% Maybe make a subsection "Determining the Parameters" with subsubsections "Initial Value Probelms" and "Boundary Value Probelms"

\subsubsection{Numerical Methods}

\paragraph{Euler's Method}

\paragraph{Runge-Kutta Methods}

\paragraph{Implicit Methods}



\subsection{Theory of Linear ODEs}
For the important subset of linear ODEs, we actually do have a full blown analytic solution theory. Much of this theory will closely parallel and build upon the solution theory of linear systems of equations that we know from linear algebra. ...TBC...


% https://en.wikipedia.org/wiki/Linear_differential_equation

\subsubsection{Linear ODEs with Constant Coefficients}


%\subsubsection{Linear ODEs with Polynomial Coefficients}


% https://en.wikipedia.org/wiki/Sturm%E2%80%93Liouville_theory
% https://de.wikipedia.org/wiki/Sturm-Liouville-Problem

\begin{comment}
-solving an ODE can be seen as a certain generalization of solving an integral - explain how
-give ODE of damped oscillator: f + a_1 f' + a_2 f'' = 0. For $a_1 = 0, a_2 = 1$ we get the ODE for exp, for $a_1 = -1, a_2 = 0$ we get the ODE for sin.

\end{comment}