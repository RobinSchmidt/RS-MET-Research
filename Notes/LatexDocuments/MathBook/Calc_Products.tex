\section{Infinite Products}
Recall the product sign from page \pageref{Sec:SumsAndProducts} and let's combine that idea with the idea of taking infinite an sum embodied in equation (\ref{Eq:InfiniteSum}). But now we want to take an infinite product instead of a sum. That is, we want to make sense of expressions like this:
\begin{equation}
\label{Eq:InfiniteProduct}
P = \lim_{n \rightarrow \infty} \prod_{k=k_0}^n a_k
  = \prod_{k=k_0}^{\infty} a_k
\end{equation}
where, again, $(a_k)$ is some given sequence. Here, "given" means that we'll typically have a formula to calculate $a_k$ for a given $k$, i.e. we have a formula in which $k$ appears that computes $a_k$. We also have given ourselves the flexibility to let the index $k$ start at some prescribed value $k_0$. It's often $0$ or $1$ but it may be some other number as well. 

% https://en.wikipedia.org/wiki/Infinite_product
% https://mathworld.wolfram.com/InfiniteProduct.html
% https://de.wikipedia.org/wiki/Produkt_(Mathematik)#Unendliche_Produkte

% https://www.math.ualberta.ca/~isaac/math324/s12/zeta.pdf
% definition on page 2:
% -divergence to zero: 
%  -infinitely many factors are zero or
%  -the factors converge to zero
% -It doesn't define convergence to zero - it requires that the final result must be nonzero for
%  a product to count as convergent

%===================================================================================================
\subsection{Convergence}
Just as we did with infinite sums (aka series), we are interested in the question whether or not a given infinite product converges to a finite number. One could perhaps guess, that the definition of convergence for infinite products is entirely analogous to the definition of convergence for infinite sums, namely, that a product converges if and only if the sequence of partial products converges. However, the actual definition is a bit more subtle and takes into account the special role of the number zero for multiplication. The actual definition of convergence ensures that a convergent product is zero if and only if at least one of its factors is zero. 
%A necessary (but not sufficient) condition for an infinite product to converge is that the individual factors must approach one. That is analogous to the requirement that summands must approach zero in infinite sums.
For infinite sums, a necessary (but not sufficient) condition for convergence was that the individual summands approach zero as $n \rightarrow \infty$. For infinite products, the analogous condition is that the individual factors must approach one. This is one reason why it is convenient to express the individual factors $a_k$ in the form $1 + h_k$ with $h_k = a_k - 1$. In products, it makes more sense to think about, how far the individual factors deviate from 1 rather than from 0 and the $h_k$ give us that information more directly than the $a_k$. They are the offset from $1$. In German, these $h_k$ values have a name: they are called "Kerne" (meaning "cores" or "kernels") but I have not yet found an English term for those (TODO: figure out if there is one!). So, we may write our products in either of the two forms:
\begin{equation}
 P = \prod_{k=k_0}^{\infty} a_k = \prod_{k=k_0}^{\infty} (1 + h_k) 
\end{equation}

% -The factors must approach 1 and do so fast enough - that's a necessary (but not sufficient)
%  condition

% 08a - Infinite Products
% https://www.youtube.com/watch?v=QPpjjyHDTMA
% -at 12:33 - example p_n = \prod_{2}^{n} (1 + 1/k)


% Relation Between Infinite Sums and Products: Convergence Conditions
% https://www.youtube.com/watch?v=F7jhQVsa_Go 


%---------------------------------------------------------------------------------------------------
\subsubsection{Some Examples}
To get a feeling for how infinite products may converge or diverge, we'll now look at a couple of simple examples.

\paragraph{Divergence to Infinity}
Consider the following finite product:
\begin{equation}
 p_n = \prod_{k=2}^n  \left(1 + \frac{1}{k} \right)
     = \prod_{k=2}^n  \left(\frac{k + 1}{k} \right)
     = \frac{3}{2}   \cdot \frac{4}{3} \cdot \frac{5}{4} \cdot \frac{6}{5} 
       \cdot \ldots \cdot \frac{n}{n-1} \cdot \frac{n+1}{n}
     = \frac{n+1}{2}
\end{equation}
It simplified so nicely because we could cancel the 3 in the numerator of the first factor with the denominator 3 in the 2nd factor and so on. We are dealing with a telescopic product and the only non-canceling numerator is the last one given by $n+1$ and the only non-canceling denominator is the first one given by $2$. So far, there is nothing infinite going on - it's all just a finite product. Now we look at what happens when we let $n$ approach infinity. Clearly, $p_n = \frac{n+1}{2}$ also diverges to infinity in this case. So we can say that:
\begin{equation}
 P = \prod_{k=2}^{\infty} \left(1 + \frac{1}{k} \right)
   = \lim_{n \rightarrow \infty} p_n 
   = \lim_{n \rightarrow \infty} \prod_{k=2}^n  \left(1 + \frac{1}{k} \right)
   = \lim_{n \rightarrow \infty} \frac{n+1}{2} 
   = \infty
\end{equation}
The infinite variant of the product diverges to infinity. The individual factors in the product approach 1 from above - but not quickly enough to lead to a convergent infinite product. This is somewhat analogous to the divergence of the harmonic series where we sum terms of the form $\frac{1}{n}$. These terms approach zero but not fast enough to make the infinite sum convergent.


\paragraph{Divergence to Zero}
Now let's consider the finite product:
\begin{equation}
 p_n = \prod_{k=2}^n  \left(1 - \frac{1}{k} \right)
     = \prod_{k=2}^n  \left(\frac{k - 1}{k} \right)
     = \frac{1}{2} \cdot \frac{2}{3} \cdot \frac{3}{4} \cdot \frac{4}{5} 
       \cdot \ldots \cdot \frac{n-2}{n-1} \cdot \frac{n-1}{n}
     = \frac{1}{n}
\end{equation}
The only difference is that now we subtract $\frac{1}{k}$ from $1$ in the factors instead of adding it like in the product before. The factors now approach one from below rather than from above. Again we are dealing with a telescopic product that simplifies nicely. In this case, it simplifies to $p_n = \frac{1}{n}$. Most of the numbers from $1$ to $n$ appear in the numerator and the denominator and cancel each other out. The only ones that remain are $1$ in the numerator and $n$ in the denominator. If we now again let $n$ approach infinity to turn the finite into an infinite product, then $p_n = \frac{1}{n}$ clearly approaches zero:
\begin{equation}
 P = \prod_{k=2}^{\infty} \left(1 - \frac{1}{k} \right)
   = \lim_{n \rightarrow \infty} p_n 
   = \lim_{n \rightarrow \infty} \prod_{k=2}^n  \left(1 - \frac{1}{k} \right)
   = \lim_{n \rightarrow \infty} \frac{1}{n} 
   = 0
\end{equation}
That is so even though none of the individual factors is zero - in fact, the smallest factor that occurs is $\frac{1}{2}$. If we would be dealing with an infinite sum rather than a product, we would consider such a sum convergent - namely, it would converge to zero. With infinite products, however, the terminology is a bit different. We say that the product \emph{diverges to zero}. 

\medskip
Q: What about a product of the form $P = \prod_{k=2}^{\infty} \left(1 + \frac{(-1)^k}{k} \right)$? We'll alternatingly get factors that are greater and less than 1. Will that converge like the alternating harmonic series does?

% What about a product with alternating signs in the offset:
%   P = \prod_{k=2}^{\infty} \left(1 + \frac{-1^k}{k} \right)
% we'll alternatingly get factors that are greater and less than 1.

%Such a product is not considered to be convergent and instead is said to \emph{diverge to zero}.

%\begin{equation}
% P = \prod_{k=2}^{\infty} \left(1 - \frac{1}{k} \right)
%   = \lim_{n \rightarrow \infty} \prod_{k=2}^n  \left(1 - \frac{1}{k} \right)
%\end{equation}

% 08a - Infinite Products
% https://www.youtube.com/watch?v=QPpjjyHDTMA
% consider the product 
%  p_n = \prod_{k=2}^n  \left(1 + \frac{1}{k} \right)
% the factors are now (1 + 1/k) rather than a (1 - 1/k). It's also a telescopic product - but it
% diverges to infinity

\paragraph{Convergence to a Non-Zero Value}
Recall that the harmonic series, i.e. the infinite sum over terms of the form $\frac{1}{k}$ was divergent but the sum of the reciprocals of the squares $\frac{1}{k^2}$ was convergent. This was so because the terms approached zero faster than in the $\frac{1}{k}$ case. Let's try to let the factors in our infinite product approach 1 faster by using factors of the form $1 - \frac{1}{k^2}$ rather than $1 - \frac{1}{k}$. So, we'll consider the product:
\begin{equation}
 p_n = \prod_{k=2}^n  \left(1 - \frac{1}{k^2} \right)
     = \prod_{k=2}^n  \left(\frac{k^2-1}{k^2} \right)
     = \frac{n+1}{2n}
\end{equation}
The last step is not supposed to be obvious but it can be guessed by looking at the first couple of products and then be proved by induction.  If we again let $n$ approach infinity to turn the finite into an infinite product, then this time $p_n$ approaches a nonzero value, namely $\frac{1}{2}$.
\begin{equation}
\label{Eq:ProdOneMinusOneOverKSquared}
P = \prod_{k=2}^{\infty} \left(1 - \frac{1}{k^2} \right)
  = \lim_{n \rightarrow \infty} \prod_{k=2}^{n} \left(1 - \frac{1}{k^2} \right)
  = \lim_{n \rightarrow \infty} p_n
  = \lim_{n \rightarrow \infty} \frac{n+1}{2n}
  = \frac{1}{2}
\end{equation}
So, we can say that the infinite product converges to $\frac{1}{2}$. Now we are getting closer to the behavior that we want to eventually call convergent. If the limit of the sequence of partial products approaches some \emph{finite value} and that value is \emph{not zero}, then we are certainly justified to call the product convergent. In some sources, that is indeed the definition of a convergent product: the limit of partial products must converge to a finite nonzero value. We are aiming for a somewhat more liberal definition here, though - one that allows us to also call a product convergent when the value is zero due to one or more of the factors being zero.

%---------------------------------------------------------------------------------------------------
\subsubsection{The Special Role of Zero}
In the context of multiplication and products, the number zero has a special role - it annihilates everything. If just a single factor in a (finite or infinite) product is zero, then the whole product is zero as well. It is also the only real (and complex) number that has no multiplicative inverse.

\paragraph{Convergence to Zero}
We have seen an example of an infinite product that had zero as a result even though none of its individual factors was zero. We called that product \emph{divergent to zero}. Not all products that have zero as result are said to "diverge" to zero, though. Some are also just zero because one or more of the factors is itself zero. Such products may also \emph{converge to zero}. To see what the difference between the two notions is, consider the infinite product:
\begin{equation}
P = \prod_{k=1}^{\infty} \left(1 - \frac{1}{k^2} \right)
\end{equation}
It looks almost like the one defined in (\ref{Eq:ProdOneMinusOneOverKSquared}) which was convergent to the number $\frac{1}{2}$. But our product here has a small but important difference. The index $k$ now starts at $1$ rather than $2$. That has the effect that the very first factor in the product evaluates to zero. This, of course, turns the value of the whole product to zero as well. We annihilated an otherwise convergent product by introducing a single factor of zero. In the definition of convergence that we are aiming for, we want to still call such a product convergent. We will say that it \emph{converges to zero}. In general, we want to call a product that evaluates to zero as convergent to zero, when its infinite tail converges to a nonzero number but in an initial section, there may be finitely many factors of zero. The zeroness of the final product is not caused by an infinite tail that decays to zero but rather by one or more of the factors $a_k$ being zero.

%\begin{equation}
% p_n = \prod_{k=2}^n  \left(1 - \frac{1}{k^2} \right)
%     = \prod_{k=2}^n  \left(\frac{k^2-1}{k^2} \right)
%     = \frac{n+1}{2n}
%\end{equation}
%The last step is not supposed to be obvious but it can be guessed by looking at the first couple of products and then be proved by induction.  If we again let $n$ approach infinity to turn the finite into an infinite product, then this time $p_n$ approaches a nonzero value, namely $\frac{1}{2}$.
%\begin{equation}
%P = \lim_{n \rightarrow \infty} \prod_{k=2}^{n} \left(1 - \frac{1}{k^2} \right)
%  = \lim_{n \rightarrow \infty} p_n
%  = \lim_{n \rightarrow \infty} \frac{n+1}{2n}
%  = \frac{1}{2}
%\end{equation}
%So, we can say that the infinite product converges to $\frac{1}{2}$. If we now consider the product in which the index does start at $1$ rather than at $2$, we get an initial first factor of zero which collapses the whole product to zero as well. This product has a value of zero because one of the factors happened to be zero - but its tail end converged to a nonzero finite number. Only in such a case, we want to say that the product \emph{converges to zero}. [VERIFY!]

%---------------------------------------------------------------------------------------------------
\subsubsection{Definition of Convergence}
We now have all our desiderata in place and are now ready to craft the general definition of convergence for products. We call an infinite product $P = \prod_{k = k_0}^{\infty} a_k$ convergent, if: (1) there exists an $N$ such that all $a_k \neq 0$ for all $k \geq N$. (2) The sequence of partial products $p_n = \prod_{k = N}^{n} a_k$ approaches a finite nonzero value $P_N$ as $n \rightarrow \infty$. If $p_n \rightarrow 0$ as $n \rightarrow \infty$, we say that the product diverges to zero.

% Some authors also just flatly disallow a convergent product to have a value of zero and call a product convergent if and only if the sequence of partial products converges to a nonzero value. With such a definition "convergence to zero" is not a thing. There's only divergence to zero, divergence to infinity and convergence to a finite nonzero value. But maybe we need to allow convergence to zero if we wnat to formulate certain theorems in a convenient way - those that say: the product converges if and only if the sum of ... converges

\paragraph{Consequences of our Definition}
Let's now see what have we achieved with this somewhat subtle definition. Firstly, we note that a convergent product is zero if and only if one of its factors is zero. This is a behavior that we wanted to have. The definition of convergence only looks at the infinite tail of the product starting at some index $N$ after which no zeros for the factors $a_k$ are allowed to occur anymore. A finite initial section $\prod_{k=k_0}^{N-1} a_k$ that may contain zero values for the factors $a_k$ is ignored - it's not taken into account for the question whether or not we call the product convergent. That is also in line with our notion of convergence for infinite sums. The value of a finite initial section has no effect on the convergence. If that finite initial section has at least one zero value but the infinite tail converges to a nonzero finite value, the product evaluates to zero and we say that it converges to zero. If the infinite tail approaches zero, then the product also evaluates to zero, but this time, we say that it diverges to zero - regardless of whether or not the finite initial section contains zero factors. ...VERIFY...TBC....

% The behavior of a finite initial section does not affect the convergence

\paragraph{Absolute Convergence}
Like with series, there is also a notion of absolute convergence for products. Unlike with series, we do not just take the absolute values of the $a_k$. Instead, we take the absolute values of the $h_k$. So we define: An infinite product $\prod_{k=k_0}^{\infty} (1 + h_k)$ is said to be \emph{absolutely convergent} if the product $\prod_{k=k_0}^{\infty} (1 + |h_k|)$ is convergent. Every product that is absolutely convergent is also convergent but not vice versa. As with series, absolute convergence is a stronger requirement.


% Would 

% Consequences of that defintion....

% -We want a definition that respects the property of finite products: a convergent product is 
%  zero iff and only if one of its factors is zero. 
% -Explain convergence to zero vs divergence to zero





%---------------------------------------------------------------------------------------------------
\subsubsection{Convergence Criteria}

%With this definition of the $h_k$, we can state the following:

\paragraph{Cauchy Criterion} The infinite product $\prod_{k=k_0}^{\infty} a_k$ converges if and only if for every $\varepsilon > 0$ there exists an index $N_{\varepsilon}$ such that for all $n,m$ with $m > n \geq N_{\varepsilon}$ the inequality 
\begin{equation}
 \Bigg| 1 -\prod_{k=n+1}^{m} a_k \Bigg| < \varepsilon
\end{equation}
holds. That means that beyond some index $N_{\varepsilon}$ that depends on the chosen $\varepsilon$, we can take arbitrarily long partial products from the tail and they are all very close to 1, namely differ from 1 at most by $\varepsilon$.

% https://www.sciencedirect.com/topics/mathematics/cauchy-criterion
% https://proofwiki.org/wiki/Definition:Cauchy%27s_Criterion_for_Products#google_vignette
% https://site.uvm.edu/jmwilson/files/2021/03/24221s_12.pdf

\paragraph{Theorem} If there exists an index $m$ such that all $h_k$ for $k \geq m$ have the same sign, then the infinite product $\prod_{k=k_0}^{\infty} (1 + h_k)$ has the same convergence as the infinite sum $\sum_{k=k_0}^{\infty} h_k$. That means: either both converge or both diverge. Or stated yet another way: the product converges if and only if the sum converges. 

\paragraph{Theorem} The infinite product $\prod_{k=k_0}^{\infty} (1 + h_k)$ converges absolutely if and only if  the infinite sum $\sum_{k=k_0}^{\infty} h_k$ converges absolutely.

\paragraph{Theorem} If for a given product $\prod_{k=k_0}^{\infty} (1 + h_k)$ the associated series  $\sum_{k=k_0}^{\infty} h_k$ converges then: (1) If the squared series $\sum_{k=k_0}^{\infty} h_k^2$ converges then the product also converges.  (2) If the squared series $\sum_{k=k_0}^{\infty} h_k^2$ diverges then the product also diverges. If the (non-squared) series $\sum_{k=k_0}^{\infty} h_k$ diverges, this criterion says nothing about the convergence of the product. It may or may not converge and other criteria need to be used to figure that out.




% has the same convergence as the squared series $\sum_{k=k_0}^{\infty} h_k^2$. 

% see Arens, Zusatzmaterial, pg 29

%(1) If the squared series $\sum_{k=k_0}^{\infty} h_k^2$


%...TBC...

% ak = 1 + hk  ->  hk = ak - 1





%===================================================================================================
\subsection{Functions Expressed via Products}
Some of our favorite functions have expressions based on infinite products. One way to find such an expression for a given function $f(x)$ is to consider the roots of the function, i.e. the values of $x$ where $f(x) = 0$. Let's call these roots $r_k$ where the index $k$ ranges over all the roots, however many there may be. If we know all the roots $r_k$ and construct a function in one of the two ways:
\begin{equation}
 p(x) = \prod_{k} (x - r_k)  \qquad \text{or} \qquad
 q(x) = \prod_{k} (1 - \frac{x}{r_k})
\end{equation}
then $p(x)$ and $q(x)$ will have the same roots as $f(x)$. Of course, if the product is infinite, we may have to worry about convergence and even if it is finite, it is not clear whether or not the product will match the function anywhere else. If the function $f(x)$ happens to be a polynomial, though, we can actually confidently say that when we include all of the roots of $f(x)$ into our product function $p(x)$, then $p(x)$ must actually indeed be equal to $f(x)$ up to a constant scale factor. The same is true for $q(x)$ but with a different scale factor [VERIFY!]. If we now remember that a power series expansion of a function can be regarded as a sort of "infinite polynomial", we may plausibly hope that we can express our function $f(x)$ with such an infinite product. In practice, the form of $q(x)$ tends to be easier to deal with. First, it has the desirable form of products where the factors are of the form $(1 + \ldots)$. I think, it also tends to be less prone to divergence away from the roots (VERIFY!). To turn one form into the other, one just needs to multiply a linear factor of the form $(x - r_k)$ with an appropriate constant, namely $-1 / r_k$. That could be problematic if one of the $r_k$ is zero, though. But that is a problem that can easily be solved by pulling these factors out of the product. ...TBC...

% To tame the polynomial, we may want to multiply each factor of the form (x - r_k) by some number c
%  that turns the factor into the form (1 - x/r_k)

% ToDo: 
% -Explain how one can come up with a product representation of a given function. 
% -If we know the zero-crossings/roos r_k  of the function, we may try to express f(x) as
%    f(x) = c  \prod_{k=0}^{$\infty$} (x - r_k)
%  where c is an overall scaling factor that somehow needs to figured out. Maybe it could make sense
%  to give each linear factor (x - r_k) its own scaling factor and use factors of the form
%  c_k (x - r_k)  and maybe try to express the factor in the form of (1 - x_k / r_k) because we 
%  generally like factors of the form (1 + something).
% -Maybe if we have a MacLaurin or Taylor series of the function given:
%  f(x) = a_0 + a_1 x + a_2 x^2 + a_3 x^3 + ...
%  we can factor out one linear factor (x - r_k) at a time?
%
% 05b - Euler's Solution of the Basel Problem
% https://www.youtube.com/watch?v=wbirLVLNYNk

%---------------------------------------------------------------------------------------------------
\subsubsection{Factoring the Sine Function}
A nice infinite product representation of the sine function was found be Euler and is given by:
\begin{equation}
\sin(x) = x \cdot \prod_{k=1}^{\infty} \left( 1 - \frac{x^2}{k^2 \pi^2}  \right)
        = \ldots \left(1 + \frac{x}{3 \pi}  \right)
                 \left(1 + \frac{x}{2 \pi}  \right) 
                 \left(1 + \frac{x}{  \pi}  \right)
                 x 
                 \left(1 - \frac{x}{  \pi}  \right)
                 \left(1 - \frac{x}{2 \pi}  \right) 
                 \left(1 - \frac{x}{3 \pi}  \right)
          \ldots
\end{equation}
He "factored out" all the infinitely many roots of the sine function that occur at all the integer multiples of $\pi$. Each positive integer $k$ gives one factor of $(1 + \frac{x}{k \pi})$ and one of $(1 - \frac{x}{k \pi})$. These pairs of factors can be combined into a single $(1 - \frac{x^2}{k^2 \pi^2})$ factor. The $k=0$ case is treated separately to avoid division by zero. That's the single $x$ in the middle. He discovered this formula in the course of solving the so called Basel problem. This is the problem of finding the value of the infinite sum $\sum_{k=1}^{\infty} \frac{1}{k^2}$. Euler found the value of this sum to be $\frac{\pi^2}{6}$ and his investigations led also to the definition of the famous Riemann zeta function. By evaluating the product formula at $x = \frac{\pi}{2}$, one obtains:
\begin{equation}
 1 = \frac{\pi}{2} \cdot \prod_{k=1}^{\infty} \left( 1 - \frac{1}{4 k^2}  \right)
   = \frac{\pi}{2} \cdot \prod_{k=1}^{\infty} \left( \frac{4 k^2 - 1}{4 k^2}  \right)
   = \frac{\pi}{2} \cdot \prod_{k=1}^{\infty} \left( \frac{(2k-1)(2k+1)}{(2k)^2}  \right)   
\end{equation}
and solving that for $\pi$ gives:
\begin{equation}
\pi = 2 \cdot \prod_{k=1}^{\infty} \left( \frac{(2k)^2 }{(2k-1)(2k+1)} \right)   
\end{equation}
which is known as the \emph{Wallis product} representation of $\pi$. Due to its slow convergence, the formula is useless for practical computation of $\pi$, though. But it's an interesting factoid nonetheless. 

% ToDo: explain how the product form of the sine can be used to find formulas for infinite sums of
% the form sum_k 1/k^2, sum_k 1/k^4, sum_k 1/k^6, ... It somehow works by comparing coefficients of
% the Taylor expansion and the expanded product form...I think. zetamath has a video about that.

% The Basel Problem Part 2: Euler's Proof and the Riemann Hypothesis
% https://www.youtube.com/watch?v=FCpRl0NzVu4&list=PLbaA3qJlbE93DiTYMzl0XKnLn5df_QWqY&index=3

% https://en.wikipedia.org/wiki/List_of_trigonometric_identities#Infinite_product_formulae

% Can Sine be Factored?
% https://www.youtube.com/watch?v=s2oO4g-13sc
% -Explains why the $\prod_{k} (x - r_k)$ form for sine doesn't coverge but the
%  $\prod_{k} (1 - \frac{x}{r_k})$ does. It's because in the 2nd form, the factors approach 1 
%  whereas in the 1st case, they oscillate and grow bigger.
% -At around 15 min, it has a power series expansion for 
%  sin(x)/x = 1 - x^2/3! + x^4/5! - x^6/7! + ... = \sum_{k=0}^{\infty} x^{2k} / (2*k+1)!
% -This could be useful for approximating the sinc function in practical applications.
% -At around 19 min, mentions the Weierstrass factorization theorem which states that entire
%  functions can be factored in this way (like Euler factored the sine function). See:
%  https://en.wikipedia.org/wiki/Weierstrass_factorization_theorem
%  This theorem is relevant here and should be mentioned.

% At the outset, it is not obvious that the roots alone are enough to determine the sine function uniquely. Not only is it conceivable that the actual sine function needs a constant factor in front of it - we could even multiply the product by an arbitrary function (perhaps one that doesn't itself have real zeros such as exp(x)) and the resulting would still have exactly the same zeros as the sine function. At least along the real axis. It may be a different story once we widen our perspective to the complex plane (exp does have complex roots - or does it?)

% ToDo: 
%
% - Construct a function in a way similar to sin but with roots at  \i k \pi  rather than  k \pi.
%   Maybe we'll get sinh?
%
% - Maybe explain that the factorization of sin worked only because the roots at k \pi are the only
%   roots of sin. Would there have been additional, "invisible" roots somewhere else in the 
%   complex plane, it would not have worked out, I think.
%
% - Construct a function that has roots (of order 1) at all the Gaussian integers in the complex 
%   plane. Star with f(z) = \prod_{n=-inf}^{inf} \prod_{m=-inf}^{inf} (1 - z/(n + i*m)) and 
%   transform it into  f(z) = z \prod_{n=1}^{inf} \prod_{m=1}^{inf} a_{nm}  with
%   a_{nm} = (1-z/(n+im)) (1-z/(n-im)) (1-z/(-n+im)) (1-z/(-n-im)) = ...simplify
%   I think, it should be bi-periodic with periods 1 and i such that the values on unit square
%   already define it fully. What is the shape there? Maybe compare it to the Weierstrass 
%   P-function

%---------------------------------------------------------------------------------------------------
\subsubsection{The Weierstrass Factorization Theorem}
...TBC...ToDo: Explain hat this theorem ensures that every \emph{entire function} can be factored in a way that generalizes the procedure that Euler used to factor the sine function. Explain what "entire functions" are. Maybe that should be a topic for complex analysis

% https://en.wikipedia.org/wiki/Weierstrass_factorization_theorem
% https://de.wikipedia.org/wiki/Weierstra%C3%9Fscher_Produktsatz


%---------------------------------------------------------------------------------------------------

\begin{comment}

See Arens, Zusatzmaterial, page 26 ff.
-The definition of convergence for products is not the obvious one and quite subtle. There is
 a notion of convergence to zeor as well as divergence to zero.

-For a product to converge, the individual factors must approach 1 (necessary but not sufficient
 condition, I guess)

Relation Between Infinite Sums and Products: Convergence Conditions
https://www.youtube.com/watch?v=F7jhQVsa_Go
-If all a_n > 0, we can just take the logarithm of the product to transform it into a sum. The 
 product converges  iff the sum converges
-If all a_n >= 0, the poduct over (1 + a_n) converges iff the sum over a_n converges


ToDo:
-Give Euler's product form of cosine. Look up the zetamath channel. IIRC, it has a good video 
 about this
-Explain my own approximation to sine/cosine somewhere in the codebase - dig it out

Playlists:
https://www.youtube.com/watch?v=1y_1P2zTInc&list=PLGqN-SK0TWHCVN66i8KVZxDjzIoq8IouB

https://www.youtube.com/watch?v=awHcEvjMTSI&list=PLEhxwgh_ZIHjA-XDPbXkXysyf5jCAcerg

https://www.youtube.com/watch?v=X_QCaoNNQko  Infinite Products: Example 1
https://www.youtube.com/watch?v=gpnKVKPfPyQ  Infinite Products- Example 2


https://www.youtube.com/watch?v=xMdfnPNGlWM  How to write 1/(1-x) as an infinite PRODUCT!


https://www.youtube.com/watch?v=9cvmDT3_TKc  a great first infinite product!

Relation Between Infinite Sums and Products: Convergence Conditions
https://www.youtube.com/watch?v=F7jhQVsa_Go  

Is it true that for convergent products with
prod_{k=1}^{\infty} a_k = P   we also have   prod_{k=1}^{\infty} \frac{1}{a_k} = \frac{1}{P}
-I guess, that may be the case in case of convergence to a nonzero value. But if P converges
 (or diverges) to zero, then maybe 1/P diverges to infinity which would make sense.

A typical trick to deal with products is to take the logarithm to turn it into a sum


The Mysterious Spiral of The Infinity Tetration
https://www.youtube.com/watch?v=LG7m-6svKlA


\end{comment}