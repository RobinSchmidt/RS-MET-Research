\section{Infinite Products}
Recall the product sign from page \pageref{Sec:SumsAndProducts} and let's combine that idea with the idea of taking infinite sum embodied in equation (\ref{Eq:InfiniteSum}). But now we want to take an infinite product instead of a sum. That is, we want to make sense of expressions like this:
\begin{equation}
\label{Eq:InfiniteProduct}
P = \lim_{n \rightarrow \infty} \prod_{k=0}^n a_k
  = \prod_{k=0}^{\infty} a_k
\end{equation}
where, again, $(a_k)$ is some given sequence. Here, "given" means that we'll typically have a formula to calculate $a_k$ for a given $k$, i.e. we have a formula in which $k$ appears that computes $a_k$. ...TBC...

% https://en.wikipedia.org/wiki/Infinite_product


\subsection{Convergence}
Just as we did with infinite sums (aka series), we are interested in the question whether or not a given infinite product converges to a finite number. One could perhaps guess, that the definition of convergence for infinite products is entirely analogous to the definition of convergence for infinite sums, namely, that a product converges if and only if the sequence of partial products converges. However, the actual definition is a bit more subtle and takes into account the special role of the number zero for multiplication. The actual definition of convergence ensures that a convergent product is zero if and only if at least one of its factors is zero. A necessary condition for an infinite product to converge is that the individual factors must approach one. That is analogous to the requirement that summands must approach zero in infinite sums.  ...TBC...

% -The factors must approach 1 and do so fast enough - that's a necessary (but not sufficient)
%  condition

\subsubsection{The Role of Zero}

\paragraph{Divergence to Zero}
Consider the finite product:
\begin{equation}
 p_n = \prod_{k=2}^n  \left(1 - \frac{1}{k} \right)
     = \frac{1}{2} \cdot \frac{2}{3} \cdot \frac{3}{4} \cdot \frac{4}{5} 
       \cdot \ldots \cdot \frac{n-2}{n-1} \cdot \frac{n-1}{n}
     = \frac{1}{n}
\end{equation}
Most of the numbers from $1$ to $n$ appear in the numerator and the denominator and cancel each other out. The only ones that remain are $1$ in the numerator and $n$ in the denominator. We are dealing with a telescopic product here. If we now let $n$ approach infinity to turn the finite into an infinite product, $p_n = \frac{1}{n}$ clearly approaches zero. That is so even though none of the individual factors is zero - in fact, the smallest factor that occurs is $\frac{1}{2}$. Such a product is not considered to be convergent and instead is said to \emph{diverge to zero}.

%\begin{equation}
% P = \prod_{k=2}^{\infty} \left(1 - \frac{1}{k} \right)
%   = \lim_{n \rightarrow \infty} \prod_{k=2}^n  \left(1 - \frac{1}{k} \right)
%\end{equation}

\paragraph{Convergence to Zero}
Not all products that have zero as result diverge to zero, though. Some do also \emph{converge to zero}. To see what the difference between the two notions is, consider the finite product:
\begin{equation}
 p_n = \prod_{k=2}^n  \left(1 - \frac{1}{k^2} \right)
     = \prod_{k=2}^n  \left(\frac{k^2-1}{k^2} \right)
     = \frac{n+1}{2n}
\end{equation}
The last step is not supposed to be obvious but it can be shown by induction.  If we again let $n$ approach infinity to turn the finite into an infinite product, then this time $p_n$ approaches a nonzero value, namely $\frac{1}{2}$.
\begin{equation}
P = \lim_{n \rightarrow \infty} \prod_{k=2}^{n} \left(1 - \frac{1}{k^2} \right)
  = \lim_{n \rightarrow \infty} p_n
  = \lim_{n \rightarrow \infty} \frac{n+1}{2n}
  = \frac{1}{2}
\end{equation}
So, we can say that the infinite product converges to $\frac{1}{2}$. If we now consider the product in which the index does start at $1$ rather than at $2$, we get an initial first factor of zero which collapses the whole product to zero as well. This product has a value of zero because one of the factors happened to be zero - but its tail end converged to a nonzero finite number. Only in such a case, we want to say that the product \emph{converges to zero}. [VERIFY!]

\subsubsection{Definition of Convergence}
Let's now cook that up into a general definition. We call an infinite product convergent, if ...TBC...


\paragraph{Consequences of our Definition}

% Consequences of that defintion....



% -We want a definition that respects the property of finite products: a convergent product is 
%  zero iff and only if one of its factors is zero. 
% -Explain convergence to zero vs divergence to zero

%===================================================================================================
\subsection{Functions Expressed via Products}
Some of our favorite functions have expressions based on infinite products. 

\paragraph{Sine}
A nice infinite product representation of the sine function was found be Euler and is given by:
\begin{equation}
 \sin(x) = x \cdot \prod_{k=1}^{\infty} \left( 1 - \frac{x^2}{k^2 \pi^2}  \right)
\end{equation}
By evaluating it at $x = \frac{\pi}{2}$, one obtains:
\begin{equation}
 1 = \frac{\pi}{2} \cdot \prod_{k=1}^{\infty} \left( 1 - \frac{1}{4 k^2}  \right)
   = \frac{\pi}{2} \cdot \prod_{k=1}^{\infty} \left( \frac{4 k^2 - 1}{4 k^2}  \right)
   = \frac{\pi}{2} \cdot \prod_{k=1}^{\infty} \left( \frac{(2k-1)(2k+1)}{(2k)^2}  \right)   
\end{equation}
and solving that for $\pi$ gives:
\begin{equation}
\pi = 2 \cdot \prod_{k=1}^{\infty} \left( \frac{(2k)^2 }{(2k-1)(2k+1)} \right)   
\end{equation}
which is known as the \emph{Wallis product} representation of $\pi$. Due to its slow convergence, the formula useless for practical computation of $\pi$, though. But it's an interesting factoid nonetheless.


%---------------------------------------------------------------------------------------------------

\begin{comment}

See Arens, Zusatzmaterial, page 26 ff.
-The definition of convergence for products is not the obvious one and quite subtle. There is
 a notion of convergence to zeor as well as divergence to zero.

-For a product to converge, the individual factors must approach 1 (necessary but not sufficient
 condition, I guess)

Relation Between Infinite Sums and Products: Convergence Conditions
https://www.youtube.com/watch?v=F7jhQVsa_Go
-If all a_n > 0, we can just take the logarithm of the product to transform it into a sum. The 
 product converges  iff the sum converges
-If all a_n >= 0, the poduct over (1 + a_n) converges iff the sum over a_n converges


ToDo:
-Give Euler's product form of cosine. Look up the zetamath channel. IIRC, it has a good video 
 about this
-Explain my own approximation to sine/cosine somewhere in the codebase - dig it out


\end{comment}