\section{Infinite Products}
Recall the product sign from page \pageref{Sec:SumsAndProducts} and let's combine that idea with the idea of taking infinite sum embodied in equation (\ref{Eq:InfiniteSum}). But now we want to take an infinite product instead of a sum. That is, we want to make sense of expressions like this:
\begin{equation}
\label{Eq:InfiniteProduct}
P = \lim_{n \rightarrow \infty} \prod_{k=0}^n a_k
  = \prod_{k=0}^{\infty} a_k
\end{equation}
where, again, $(a_k)$ is some given sequence. Here, "given" means that we'll typically have a formula to calculate $a_k$ for a given $k$, i.e. we have a formula in which $k$ appears that computes $a_k$. ...TBC...

% https://en.wikipedia.org/wiki/Infinite_product
% https://mathworld.wolfram.com/InfiniteProduct.html

% https://www.math.ualberta.ca/~isaac/math324/s12/zeta.pdf
% definition on page 2:
% -divergence to zero: 
%  -infinitely many factors are zero or
%  -the factors converge to zero
% -It doesn't define convergence to zero - it requires that the final result must be nonzero for
%  a product to count as convergent

\subsection{Convergence}
Just as we did with infinite sums (aka series), we are interested in the question whether or not a given infinite product converges to a finite number. One could perhaps guess, that the definition of convergence for infinite products is entirely analogous to the definition of convergence for infinite sums, namely, that a product converges if and only if the sequence of partial products converges. However, the actual definition is a bit more subtle and takes into account the special role of the number zero for multiplication. The actual definition of convergence ensures that a convergent product is zero if and only if at least one of its factors is zero. A necessary condition for an infinite product to converge is that the individual factors must approach one. That is analogous to the requirement that summands must approach zero in infinite sums.  ...TBC...

% -The factors must approach 1 and do so fast enough - that's a necessary (but not sufficient)
%  condition

% 08a - Infinite Products
% https://www.youtube.com/watch?v=QPpjjyHDTMA
% -at 12:33 - example p_n = \prod_{2}^{n} (1 + 1/k)

\subsubsection{Some Examples}
To get a feeling for how infinite products may converge or diverge, we'll now look at a couple of simple examples.

\paragraph{Divergence to Infinity}
Consider the following finite product:
\begin{equation}
 p_n = \prod_{k=2}^n  \left(1 + \frac{1}{k} \right)
     = \prod_{k=2}^n  \left(\frac{k + 1}{k} \right)
     = \frac{3}{2}   \cdot \frac{4}{3} \cdot \frac{5}{4} \cdot \ldots \cdot 
       \frac{n}{n-1} \cdot \frac{n+1}{n}
     = \frac{n+1}{2}
\end{equation}
It simplified so nicely because we could cancel the 3 in the numerator of the first factor with the denominator 3 in the 2nd factor and so on. We are dealing with a telescopic product and the only non-canceling numerator is the last one given by $n+1$ and the only non-canceling denominator is the first one given by $2$. So far, there is nothing infinite going on - it's all just a finite product. Now we look at what happens when we let $n$ approach infinity. Clearly, $p_n = \frac{n+1}{2}$ also diverges to infinity in this case. So we can say that:
\begin{equation}
 P = \prod_{k=2}^{\infty} \left(1 + \frac{1}{k} \right)
   = \lim_{n \rightarrow \infty} p_n 
   = \lim_{n \rightarrow \infty} \prod_{k=2}^n  \left(1 + \frac{1}{k} \right)
   = \lim_{n \rightarrow \infty} \frac{n+1}{2} 
   = \infty
\end{equation}
The infinite variant of the product diverges to infinity. The individual factors in the product approach 1 from above - but not quickly enough to lead to a convergent infinite product. This is somewhat analogous to the divergence of the harmonic series where we sum terms of the form $\frac{1}{n}$. These terms approach zero but not fast enough to make the infinite sum convergent.


\paragraph{Divergence to Zero}
Now let's consider the finite product:
\begin{equation}
 p_n = \prod_{k=2}^n  \left(1 - \frac{1}{k} \right)
     = \frac{1}{2} \cdot \frac{2}{3} \cdot \frac{3}{4} \cdot \frac{4}{5} 
       \cdot \ldots \cdot \frac{n-2}{n-1} \cdot \frac{n-1}{n}
     = \frac{1}{n}
\end{equation}
The only difference is that now we subtract $\frac{1}{k}$ form $1$ instead of adding it like in the product before. The factors approach one from below. Again we are dealing with a telescopic product that simplifies nicely. In this case, it simplifies to $p_n = \frac{1}{n}$. Most of the numbers from $1$ to $n$ appear in the numerator and the denominator and cancel each other out. The only ones that remain are $1$ in the numerator and $n$ in the denominator. If we now again let $n$ approach infinity to turn the finite into an infinite product, $p_n = \frac{1}{n}$ clearly approaches zero. That is so even though none of the individual factors is zero - in fact, the smallest factor that occurs is $\frac{1}{2}$. If we would be dealing with an infinite sum rather than a product, we would consider such a sum convergent - namely, it would converge to zero. With infinite products, however, the terminology is a bit different. We say that the product \emph{diverges to zero}. 

%Such a product is not considered to be convergent and instead is said to \emph{diverge to zero}.

%\begin{equation}
% P = \prod_{k=2}^{\infty} \left(1 - \frac{1}{k} \right)
%   = \lim_{n \rightarrow \infty} \prod_{k=2}^n  \left(1 - \frac{1}{k} \right)
%\end{equation}

% 08a - Infinite Products
% https://www.youtube.com/watch?v=QPpjjyHDTMA
% consider the product 
%  p_n = \prod_{k=2}^n  \left(1 + \frac{1}{k} \right)
% the factors are now (1 + 1/k) rather than a (1 - 1/k). It's also a telescopic product - but it
% diverges to infinity

% Is it true that for convergent products with
% prod_{k=1}^{\infty} a_k = P   we also have   prod_{k=1}^{\infty} \frac{1}{a_k} = \frac{1}{P}
% -I guess, that may be the case in case of convergence to a nonzero value. But if P converges
%  (or diverges) to zero, then maybe 1/P diverges to infinity which would make sense.


\paragraph{Convergence to a Non-Zero Value}
...tbc...

%\subsubsection{The Role of Zero}


\paragraph{Convergence to Zero}
Not all products that have zero as result diverge to zero, though. Some do also \emph{converge to zero}. To see what the difference between the two notions is, consider the finite product:
\begin{equation}
 p_n = \prod_{k=2}^n  \left(1 - \frac{1}{k^2} \right)
     = \prod_{k=2}^n  \left(\frac{k^2-1}{k^2} \right)
     = \frac{n+1}{2n}
\end{equation}
The last step is not supposed to be obvious but it can be guessed by looking at the first couple of products and then be proved by induction.  If we again let $n$ approach infinity to turn the finite into an infinite product, then this time $p_n$ approaches a nonzero value, namely $\frac{1}{2}$.
\begin{equation}
P = \lim_{n \rightarrow \infty} \prod_{k=2}^{n} \left(1 - \frac{1}{k^2} \right)
  = \lim_{n \rightarrow \infty} p_n
  = \lim_{n \rightarrow \infty} \frac{n+1}{2n}
  = \frac{1}{2}
\end{equation}
So, we can say that the infinite product converges to $\frac{1}{2}$. If we now consider the product in which the index does start at $1$ rather than at $2$, we get an initial first factor of zero which collapses the whole product to zero as well. This product has a value of zero because one of the factors happened to be zero - but its tail end converged to a nonzero finite number. Only in such a case, we want to say that the product \emph{converges to zero}. [VERIFY!]

\subsubsection{Definition of Convergence}
Let's now cook that up into a general definition. We call an infinite product convergent, if ...TBC...

% Some authors also just flatly disallow a convergent product to have a value of zero and call a product convergent if and only if the sequence of partial products converges to a nonzero value. With such a definition "convergence to zero" is not a thing. There's only divergence to zero, divergence to infinity and convergence to a finite nonzero value. But maybe we need to allow convergence to zero if we wnat to formulate certain theorems in a convenient way - those that say: the product converges if and only if the sum of ... converges

\paragraph{Consequences of our Definition}

% Consequences of that defintion....



% -We want a definition that respects the property of finite products: a convergent product is 
%  zero iff and only if one of its factors is zero. 
% -Explain convergence to zero vs divergence to zero

%===================================================================================================
\subsection{Functions Expressed via Products}
Some of our favorite functions have expressions based on infinite products. 

% ToDo: 
% -Explain how one can come up with a product representation of a given function. 
% -If we know the zero-crossings/roos r_k  of the function, we may try to express f(x) as
%    f(x) = c  \prod_{k=0}^{$\infty$} (x - r_k)
%  where c is an overall scaling factor that somehow needs to figured out. Maybe it could make sense
%  to give each linear factor (x - r_k) its own scaling factor and use factors of the form
%  c_k (x - r_k)  and maybe try to express the factor in the form of (1 - x_k / r_k) because we 
%  generally like factors of the form (1 + something).
% -Maybe if we have a MacLaurin or Taylor series of the function given:
%  f(x) = a_0 + a_1 x + a_2 x^2 + a_3 x^3 + ...
%  we can factor out one linear factor (x - r_k) at a time?

\paragraph{Sine}
A nice infinite product representation of the sine function was found be Euler and is given by:
\begin{equation}
 \sin(x) = x \cdot \prod_{k=1}^{\infty} \left( 1 - \frac{x^2}{k^2 \pi^2}  \right)
\end{equation}
By evaluating it at $x = \frac{\pi}{2}$, one obtains:
\begin{equation}
 1 = \frac{\pi}{2} \cdot \prod_{k=1}^{\infty} \left( 1 - \frac{1}{4 k^2}  \right)
   = \frac{\pi}{2} \cdot \prod_{k=1}^{\infty} \left( \frac{4 k^2 - 1}{4 k^2}  \right)
   = \frac{\pi}{2} \cdot \prod_{k=1}^{\infty} \left( \frac{(2k-1)(2k+1)}{(2k)^2}  \right)   
\end{equation}
and solving that for $\pi$ gives:
\begin{equation}
\pi = 2 \cdot \prod_{k=1}^{\infty} \left( \frac{(2k)^2 }{(2k-1)(2k+1)} \right)   
\end{equation}
which is known as the \emph{Wallis product} representation of $\pi$. Due to its slow convergence, the formula useless for practical computation of $\pi$, though. But it's an interesting factoid nonetheless.

% The Basel Problem Part 2: Euler's Proof and the Riemann Hypothesis
% https://www.youtube.com/watch?v=FCpRl0NzVu4&list=PLbaA3qJlbE93DiTYMzl0XKnLn5df_QWqY&index=3

% https://en.wikipedia.org/wiki/List_of_trigonometric_identities#Infinite_product_formulae

%---------------------------------------------------------------------------------------------------

\begin{comment}

See Arens, Zusatzmaterial, page 26 ff.
-The definition of convergence for products is not the obvious one and quite subtle. There is
 a notion of convergence to zeor as well as divergence to zero.

-For a product to converge, the individual factors must approach 1 (necessary but not sufficient
 condition, I guess)

Relation Between Infinite Sums and Products: Convergence Conditions
https://www.youtube.com/watch?v=F7jhQVsa_Go
-If all a_n > 0, we can just take the logarithm of the product to transform it into a sum. The 
 product converges  iff the sum converges
-If all a_n >= 0, the poduct over (1 + a_n) converges iff the sum over a_n converges


ToDo:
-Give Euler's product form of cosine. Look up the zetamath channel. IIRC, it has a good video 
 about this
-Explain my own approximation to sine/cosine somewhere in the codebase - dig it out

Playlists:
https://www.youtube.com/watch?v=1y_1P2zTInc&list=PLGqN-SK0TWHCVN66i8KVZxDjzIoq8IouB

https://www.youtube.com/watch?v=awHcEvjMTSI&list=PLEhxwgh_ZIHjA-XDPbXkXysyf5jCAcerg

https://www.youtube.com/watch?v=X_QCaoNNQko  Infinite Products: Example 1
https://www.youtube.com/watch?v=gpnKVKPfPyQ  Infinite Products- Example 2


https://www.youtube.com/watch?v=xMdfnPNGlWM  How to write 1/(1-x) as an infinite PRODUCT!


https://www.youtube.com/watch?v=9cvmDT3_TKc  a great first infinite product!

Relation Between Infinite Sums and Products: Convergence Conditions
https://www.youtube.com/watch?v=F7jhQVsa_Go  


\end{comment}