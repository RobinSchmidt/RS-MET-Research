\section{Infinite Products}
Recall the product sign from page \pageref{Sec:SumsAndProducts} and let's combine that idea with the idea of taking infinite sum embodied in equation (\ref{Eq:InfiniteSum}). But now we want to take an infinite product instead of a sum. That is, we want to make sense of expressions like this:
\begin{equation}
\label{Eq:InfiniteProduct}
P = \lim_{n \rightarrow \infty} \prod_{k=0}^n a_k
  = \prod_{k=0}^{\infty} a_k
\end{equation}
where, again, $(a_k)$ is some given sequence. Here, "given" means that we'll typically have a formula to calculate $a_k$ for a given $k$, i.e. we a formula in which $k$ appears that computes $a_k$. ...TBC...

\begin{comment}

See Arens, Zusatzmaterial, page 26 ff.

-For a product to converge, the individual factors must approach 1 (necessary but not sufficient
 condition, I guess)

Relation Between Infinite Sums and Products: Convergence Conditions
https://www.youtube.com/watch?v=F7jhQVsa_Go
-If all a_n > 0, we can just take the logarithm of the product to transform it into a sum. The 
 product converges  iff the sum converges
-If all a_n >= 0, the poduct over (1 + a_n) converges iff the sum over a_n converges


ToDo:
-Give Euler's product form of cosine. Look up the zetamath channel. IIRC, it has a good video 
 about this
-Explain my own approximation to sine/cosine somewhere in the codebase - dig it out


\end{comment}