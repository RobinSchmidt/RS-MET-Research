\section{Real Analysis} 



\begin{comment}


Make a section about "Real Analysis". There, give all the weird functions that are used as
(counter)examples.

https://www.youtube.com/watch?v=G-29AZf-NkU  Die Monster der Analysis
https://www.youtube.com/watch?v=uaIBoBaSzJs  a classic real analysis example
https://www.youtube.com/watch?v=Z_DZZJbTNCw  the ruler (function) is nowhere differentiable
https://www.youtube.com/watch?v=Jz8VCv1MIYE  Ableitungen à la carte (Borels Lemma)  5:23
 -> f(x) = \sum_{k=0}^{\infty} \e^{-\sqrt{2^k}} \cos( 2^k x)
 is smooth everywhere but analytic nowhere

Important theorems in real analysis:
https://en.wikipedia.org/wiki/Intermediate_value_theorem
https://en.wikipedia.org/wiki/Mean_value_theorem
https://en.wikipedia.org/wiki/Extreme_value_theorem
https://en.wikipedia.org/wiki/Bolzano%E2%80%93Weierstrass_theorem
https://en.wikipedia.org/wiki/Completeness_of_the_real_numbers
https://en.wikipedia.org/wiki/List_of_real_analysis_topics#Fundamental_theorems


Maybe try to define important calculus ideas purely in terms of rational numbers. There will be problems. For example, certain sequences of numbers may converge to an irrational limit. Point them out by examples (maybe e, pi, sqrt(2), ...) to justify the necessity of introducing the real numbers.


the gateway proof to formal calculus
https://www.youtube.com/watch?v=owkryUc66VA
-proves the irrationality of sqrt(2)


the real numbers are stranger than you think
https://www.youtube.com/watch?v=b23cXLdp1o8
-The Cantor set is uncountable. That's very counterintuitive because it appears to be much sparser 
 than the rationals which are countable. Both have Lebesgue measure zero.
-Measure is about how a set is distributed


Yes, you can do induction on the real numbers! #SoME4
https://www.youtube.com/watch?v=mNJhQ36h70s


A Nonintegrable Derivative - #SoME4
https://www.youtube.com/watch?v=_yiW6XC6rN4

\end{comment}