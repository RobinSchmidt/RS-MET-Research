\section{Series}

\subsection{Sequences}
A \emph{sequence} can be defined as a function from the natural into the real numbers: $f: \mathbb{N} \rightarrow \mathbb{R}$. Sometimes the codomain can also be the complex numbers. The input can be seen as an index for an element of the sequence. Sometimes it is more convenient to restrict the domain to the positive natural numbers, i.e. consider functions $f: \mathbb{N}^+ \rightarrow \mathbb{R}$. You will find both definitions for sequences in the literature. Here, we will mostly use the former and I will give a special disclaimer when the latter is used. Some example sequences are:
\begin{equation}
 a_n = (-1)^n,           \quad 
 a_n = \frac{1}{n},      \quad
 a_n = \frac{(-1)^n}{n}, \quad 
 a_n = \frac{1}{n^2},    \quad
 a_n = \frac{1}{2^n},    \quad  
 a_n = \sqrt{n}
\end{equation}
The examples with an $n$ or $n^2$ in the denominator can only be meaningfully defined as functions from the positive naturals into the reals because otherwise we would have a division by zero for the first (actually zero-th) element, i.e. when $n=0$. For the last two, plugging in $n=0$ poses no problem because $2^0=1$ and $\sqrt{0} = 0$. So, for those which would have a division by zero for $n=0$, we would chose a domain of $\mathbb{N}^+$ while for the others, we could choose $\mathbb{N}$. We could also say that they are all defined on $\mathbb{N}$ and make a special case definition $a_0 = 0$ for the problematic ones. The function definition would then look a bit ugly but is entirely legitimate. Sequences themselves are a sort of preliminary to what we actually want to build up to: namely series. These are infinite sums of sequence elements.

% https://en.wikipedia.org/wiki/Sequence

\subsection{Operations}

\subsubsection{Pointwise Operations}
Just like we can do with any functions $f: \mathbb{R} \rightarrow \mathbb{R}$, we can of course also add, subtract, multiply and divide two functions $f: \mathbb{N} \rightarrow \mathbb{R}$ pointwise. The fact that the domain is restricted to the naturals does not change anything about that. That means: we can just perform our usual arithmetic operations on sequences element-wise.

\subsubsection{Convolution aka Cauchy Product}
Another binary operation that we can perform on two sequences $(a_n),(b_n)$ is \emph{convolution}, also known as the \emph{Cauchy product} between the sequences. We will denote the operation symbol for convolution by an asterisk $\ast$ and define the operation as:
\begin{equation}
 (c_n) = (a_n) \ast (b_n) = (b_n) \ast (a_n) \quad \text{with} \quad
 c_n = \sum_{k=0}^n a_k b_{n-k} = \sum_{k=0}^n b_k a_{n-k}
\end{equation}
where, from the given formula, you will already correctly infer that the operation is commutative.
[VERIFY!]. It is also associative and distributive over addition and subtraction. This distributivity, together with the fact that scaling one of the inputs by a factor yields an output scaled by the same factor constitutes the important property of \emph{bilinearity}. That means, the operation is linear in both of its input arguments.
% https://mathworld.wolfram.com/CauchyProduct.html

\paragraph{Deconvolution}
With some caveats, it is actually possible to undo a convolution. We will call this operation \emph{deconvolution}. ...TBC...

\paragraph{Continuous Convolution}
As a side note, an analoguos operation of convolution can also be defined between two functions $f,g: \mathbb{R} \rightarrow \mathbb{R}$. In this case, the convolution is defined via an integral instead of a sum:
\begin{equation}
 h(t) = f(t) \ast g(t) = g(t) \ast f(t) \quad \text{with} \quad
 h(t) = \int_{-\infty}^{\infty} f(\tau) g(t-\tau) \, d \tau 
      = \int_{-\infty}^{\infty} g(\tau) f(t-\tau) \, d \tau
\end{equation}
This is not relevant in the context of series but it's sometimes good to point out the connections to other topics. I have used $t$ for the input and $\tau$ for the dummy integration variable here because this is the way, you often find it stated. This operation occurs a lot in signal processing where $t$ stands for time and $\tau$ for a time-lag or delay.




\subsection{Convergence}
When faced with a sequence of numbers, an important question is its behavior when the index $n$ approaches infinity. Specifically, we are interested, if the output of the sequence approaches some finite number or not. There are a couple of things that could happen when $n$ approaches infinity: (1) the numbers converge to some finite number, (2) the numbers diverge to infinity, (3) the numbers stay finite but jump around and approach nothing. In the third case, we will mostly find situations where the sequence alternates between positive and negative values. The example sequence $(-1)^n$ is of that kind because $(-1)^n$ will be $+1$ for even $n$ and $-1$ for odd $n$ and thereby induce this alternating behavior. The sequence $1/n$ approaches zero, so that would count as convergent. The sequence $(-1)^n / n$ is an alternating version of the former. It alternates between positive and negative values but the absolute values become smaller and smaller. It also converges to zero but in an alternating way. The sequence $1/n^2$ also converges to zero and does so even faster than $1/n$ and $1/2^n$ converges to zero yet faster.

%This sequence has a name: it's called the \emph{alternating harmonic series}.

%Theorems: pointwise sum of sequences converes to sum of limits, etc.




\subsection{Infinite Sums aka Series}
When we have a given sequence $(a_n)$, we can use it to define a new sequence, namely the sequence of its partial sums. Let's call that sequence $(s_n)$. Its $n$th element is given by the sum over all elements up to $n$ of our input sequence $(a_n)$:
\begin{equation}
 s_n = \sum_{k=0}^n a_k
\end{equation}
Just like with any sequence, we can ask whether this sequence of partial sums converges or diverges when we let $n$ approach infinity. What we have then created is an infinite sum. Such infinite sums have a special name: they are called \emph{series}.
...TBC...


\subsubsection{Some Sums}
ToDo: give formulas for finite and infinite geometric series, infinite alternating harmonic series


\subsection{Power Series}
A power series is a series that involves a variable $x$. We imagine that we have a given sequence $(a_n)$ and we will intepret this as a sequence of coefficients of a series of powers of $x$, i.e. a sort of infinite polynomial. We will consider the series:
\begin{equation}
 f(x) = \sum_{n=0}^\infty a_n (x-c)^n
\end{equation}
where $c$ is some fixed constant. It's often zero but for generality, I've already included it. On the left hand side, I already suggestively have written $f(x)$ to indicate that this "infinite polynomial" may be used to define a function - at least for those values of $x$, for which the infinite sum converges to a finite value. If the series converges for any $x$ at all, it will do so inside a disc centered at $c$ when we assume that $c$ and $x$ can be complex numbers. The radius of this disc is called the radius of convergence. In general, we could also allow complex coefficients $a_n$. For the time being, we'll focus our attention to real $x,c,a_n$ all being numbers, though.
% https://en.wikipedia.org/wiki/Radius_of_convergence


\paragraph{Convolution Revisited}
When we defined the convolution operation between sequences, it may have seemed somewhat arbitrary and unmotivated. In light of power series, we may give it some interpretation that also justifies why it is also called the Cauchy "product". Recall that to mutliply two (finite) polynomials, we have to convolve their lists of coefficients. The Cauchy product is basically the infinite version of that. It's a generalization in the sense that it includes the finite case as special case when we assume that the sequences just become identically zero after some $n$. The product sequence is the sequence of coefficients of an infinite product polynomial, so to speak...TBC...that can perhaps be explained better

\medskip
...TBC..ToDo: explain how to compute radius of convergence - distance to nearest singularity

%explain 

\subsubsection{Taylor Expansion}
One truly remarkable result of calculus is that a lot of functions can be expressed as such a power series. For many important functions (including our favorites $\exp, \sin, \cos$), we will even find an infinite radius of convergence. Taylor's theorem tells us how to find the coefficients and the formula is actually pretty simple. First, we pick an expansion point, i.e. the center of our expansion. That's the $c$ in the formula given above but here, we'll call it $x_0$. The $k$th coefficient is given the $k$th derivative of $f$ at our expansion point $x_0$ divided by $k!$. So the formula for the Taylor series is:
\begin{equation}
\label{Eq:TaylorSeries}
f(x) = \sum_{k=0}^\infty \frac{f^{(k)}(x_0)}{k!} (x-x_0)^k
%     = \sum_{k=0}^n \frac{f^{(k)}(x_0)}{k!} (x-x_0)^k + \mathcal{O}(x-x_0)^n
\end{equation}
For this formula to work, we require a few things. First, $f$ must be smooth at $x_0$ such that we actually can take all the derivatives. Second, we require $x$ to be within the radius of convergence of the series, i.e. close enough to $x_0$. There are some rather special functions that are indeed smooth but have a radius of convergence of zero. But we'll forget about that complication right now. For most functions, it works just fine. 
% maybe give a formula that splits it into a finite sum and a remainder

\medskip
The practical usefulness of this Taylor expansion lies in the fact that when we just take a finite sum instead of an infinite one, we obtain an approximation to our function. The more terms we use, the better our approximation becomes (provided that the argument is within the radius of convergence - of course). Due to that feature, Taylor expansions are used a lot in numerical computations to approximate all kinds of functions. They are also used a lot in physics to simplify complicated problems by making approximations. There, often just first or second order approximations are used, i.e. approximations up to the linear or quadratic term. Let's see what happens, when we only use a single term, i.e. only the $k=0$ term. The formula tells us to evaluate the $0$th derivative of $f$, which is $f$ itself, at $x_0$, divide by $0!$, i.e. by one and multiply that by $(x-x_0)^0 = 1$. So what we get is just $f(x_0)$. The $0$th order Taylor polynomial around $x_0$ approximates $f$ as the constant function $f(x_0)$. Not a very impressive approximation but at least it has the correct function value at $x_0$. If we take one term more, we obtain a linear approximation of $f$ around $x_0$. It will have a matching function value and a matching first derivative. The approximant will be a tangent to our actual function at $(x_0, f(x_0))$. The next approximation will additionaly match the second derivative and create a paraboloic approximation. And so on.

ToDo: maybe move that into an "Applications" section

\paragraph{Example} Let's consider the function $f(x) = 1 / (1 + x^2)$ and let's pick $x_0 = 0$ as expansion point. ToDo: draw plot of the function and a couple of approximants - show how convergence breaks down at +-1 because of the poles at +-i

\medskip
...TBC... explain aproximation properties and error term

% https://en.wikipedia.org/wiki/Taylor%27s_theorem



\subsection{Trigonometric Series}
We now want to look at a completely different kind of series. It will again involve a variable $x$ but this time, we will not use powers of $x$ but instead trigonometric functions, specifically sines and cosines. We will look at series of the form:
\begin{equation}
\label{Eq:FourierSeries}
 f(x) = \frac{a_0}{2} + \sum_{k=1}^\infty \left(  a_k \cos(k x) + b_k \sin(k x) \right)
\end{equation}
As with power series, the $f(x)$ on the left hand side indicates that we intend to use such a series to define a function of $x$. By construction, our so defined function will be periodic with a period of $2\pi$. This can easily be seen as follows. The $a_0/2$ term will just give us a constant offset and that's periodic with any period. The $k=1$ term gives a sine and cosine which are the prototypical functions with period $2\pi$. For higher $k$, the sin/cos pair will oscillate $k$ times faster, so it will have a period of $2\pi/k$. But of course, something that is periodic with period $2\pi/k$ is also periodic with period $2\pi$. So, all the higher terms will also be $2\pi$-periodic. They will just undergo $k$ cycles when our slowest sinusoid (for $k=1$) undergoes one cycle.

\medskip
There are different conventions for dealing with the $a_0$ coefficient which represents the constant part of the function, in electrical engineering also known as the DC (directed current) term. ...TBC..

% https://en.wikipedia.org/wiki/Fourier_series
% Leupold, Vol2, pg 50 ff
% maybe use \tau instead of 2 \pi 
% maybe assume a period of 2 pi - having to carry aorund the p and the omega makes things unnecessarily complicated. Yes, it's more general but the application to functions with arbitrary period can better be done on a higher level by stretching the functions appropriately. we don't want to deal with that inside the nitty gritty of Fourier analysis and it's more consistent with most books
% maybe use the convention that pulls the a0/2 out of the sum and starts the sum at 1. this relieves us from special casing the analysis equation for a_0 and also simplifies the explanation of the periodicity

%...TBC...

%explain different conventions - some use $2 \pi x$ as argument, some pull out $a_0/2$, etc.

\subsubsection{Fourier Expansion}
The Taylor expansion expressed arbitrary functions as (potentially) infinite weighted sums of powers of the variable $x$. The coefficients were computed using derivatives of the function at one single point. Fourier expansions, on the other hand, express arbitrary periodic functions as (potentially) infinite sums of sines and cosines of $x$. The coefficients are calculated by a definite integral over one period. The formulas are:
\begin{equation}
\label{Eq:FourierCoeffs}	
 a_k = \frac{1}{\pi} \int_{-\pi}^{\pi} f(x) \cos(k x) \, dx, \quad
 b_k = \frac{1}{\pi} \int_{-\pi}^{\pi} f(x) \sin(k x) \, dx 
\end{equation}
We can intepret equation \ref{Eq:FourierSeries} as a synthesis formula and equation \ref{Eq:FourierCoeffs} as an analysis formula in the sense that the former is used to synthesize (i.e. generate, create) a periodic function $f(x)$ whereas the latter takes a periodic function $f(x)$ as input, analyzes it, and spits out the synthesis coefficients. In signal processing terms,  the analysis formula actually computes a cross-correlation between the input function $f(x)$ and the respective sine or cosine function. The integration can also be performed over some other interval of length $2\pi$ like $[a,a+2\pi]$ for some $a$. 

\medskip
The requirements on $f$ are quite different from those for Taylor series. First of all and most obviously, we require $f$ to be periodic. Here, we restricted our attention to $2 \pi$-periodic functions, but the theory can easily be applied to functions with other periods by stretching the function appropriately along the $x$-axis. For a Taylor series, we required $f$ to be smooth because we had to take (infinitely many) derivatives to compute the coeffs. For a Fourier series, $f$ does not even need to be continuous. We only require that the integrals in the analysis equation must exist and be finite. This is a much less strict requirement. If the function has jump discontinuities, it will converge to average of the values on the left and right at the jump. If we use only finitely many terms, there will be some overshoot and wiggle around the jump. This is called Gibbs's phenomenon. Taking more terms will make these wiggles narrower in width but not smaller in height. Their height will settle to something like $9\%$ of the jump height when taking ever more but still finitely many terms. They will disappear completely only due to vanishing width in the limit of infinitely many terms.

\subsubsection{Magnitude and Phase}
It is well known that a weighted sum of a sine and a cosine of the same frequency can also be expressed as just a sine or just a cosine. For this, we need give the argument an appropriate phase shift $\varphi_k$ and scale the result by an appropriate amplitude factor $A_k$. The phase shift and the scale factor depend on the relative and total amounts of the sine and cosine components in the following way.:
\begin{equation}
 A_0 = \frac{a_0}{2},           \;\;
 A_k = \sqrt{a_k^2 + b_k^2},    \;\;
 \varphi_k = \atan2(b_k, a_k),  \qquad
 a_k = A_k \sin(\varphi_k),     \;\;
 b_k = A_k \cos(\varphi_k)   
\end{equation}
where the $A_k = \ldots$ formula applies only to $k \neq 0$. I have also given the conversion formula for the other direction. With these new coeffs, we can express the series as:
\begin{equation}
f(x) = A_0 + \sum_{k=1}^\infty A_k \sin(k x +\varphi_k) 
     = A_0 + \sum_{k=1}^\infty A_k \cos(k x +\varphi_k - \frac{\pi}{2})
\end{equation}
[VERIFY!] 
% Seems like we can't get rid of $A_0$ when we start the sum at $k=0$. phi_0 = 0 because atan2(0,x) is 0° for any x and sin(0) = 0 and cos(-pi/2) = 0. So the DC would be missing in the resythesis. But wouldn't it be nicer if the formula would cover the edge case also? can that be made to work? We would just need to ensure that phi_0 = pi/2, such that sin(phi_0) = cos(0) = 1, I think. 
% https://en.wikipedia.org/wiki/Sine_and_cosine#Identities
\newline



\medskip
...TBC... interpretation as amplitude and phase, symmetries (odd - no cosines, even - no sines)
%Leup2, pg 65

\paragraph{Complex Formulation}
The theory of Fourier series if often being presented in its complex form. Doing so hides some of its complexity in the complex numbers. Depending on your fluency with complex numbers, this may feel like a simplification or like a mystification. The equations are shorter, yes. But they are complex - pun intended. Instead of using real valued sine and cosine functions for analysis and synthesis, one uses a single complex exponential function. This is equivalent due to Euler's formula. By doing so, one reduces the apparent complexity of having to deal with two sets of coefficients, namely the $a_n$ and $b_n$ coefficients, to a single set of $c_n$ coefficients - but these coeffs are then complex valued. Going to the complex domain actually also generalizes the applicability to functions $f: \mathbb{R} \rightarrow \mathbb{C}$ as well, so we get a little bit more than just a simpli/mystification. In the complex formulation, the synthesis and analysis equations look like:
\begin{equation}
\label{Eq:FourierSeriesComplex}
 f(x) = \sum_{k=-\infty}^\infty c_k  e^{\i k x}, \qquad
 c_k = \frac{1}{2\pi} \int_{-\pi}^{\pi} f(x) e^{-\i k x}  \, dx
\end{equation}
We could put synthesis and analysis equation together in one line and it's still shorter than the real valued analysis equation alone. Notice also how there's no special case for $c_0$ anymore. It may seem a little bit strange that the summation in the analysis equation now starts at minus infinity. Shouldn't the real and imaginary parts of the non-negatively indexed coeffs $c_0, c_1, \ldots$ already contain the same amount of information as the pairs $(a_0,b_0),(a_1,b_1), \ldots$ from the real formulation? Why do we need the additional coeffs with negative indices? Indeed, when the input function is purely real valued, the negatively indexed coeffs will be redundant, but remember that we now allow, in general, complex valued functions $f(x)$ functions, too. For real valued functions $f$, the $c_k$ will be conjugate symmetric, i.e. $c_{-k} = \overline{c_k}$. [VERIFY]. A $c_k$ with a negative index $k$ formally represents the coefficient for a negative frequency. That just means the complex phasor rotates clockwise rather than counterclockwise in the corresponding $c_k  e^{\i k x}$ term in the synthesis equation. This flips the sign of the sine part, the cosine part stays the same [VERIFY]

...TBC...explain negative frequencies and how conjugate symmetry of coeffs yields real signals


\paragraph{Aperiodic Functions}
So far, we assumed $f$ to be periodic but the theory can actually be generalized to deal with (some) aperiodic functions, too. This leads to the idea of the Fourier transform which will be treated in more detail in the section about integral transforms. At this point, I'll just say that the sum in the synthesis equation will be replaced by an integral, our discrete Fourier coefficients $a_n, b_n$ will be replaced by a continuous function of frequency and in the analysis equations, we will have to perform an integration from minus to plus infinity instead of over one period. 




\subsubsection{Connection to Power Series}
Taylor and Fourier series seem to be totally disconnected things. However, when we widen our perspective to the complex domain, a beautiful connection between these two emerges...

\medskip
ToDo: bring stuff from Weitz video "Taylor trifft Fourier"
% https://www.youtube.com/watch?v=cXxsSdbfEEg
%\hyperlink{https://www.youtube.com/watch?v=cXxsSdbfEEg}{youtube.com/watch?v=cXxsSdbfEEg}

