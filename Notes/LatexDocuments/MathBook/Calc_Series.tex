\section{Series}

\subsection{Sequences}
A \emph{sequence} can be defined as a function from the natural into the real numbers: $f: \mathbb{N} \rightarrow \mathbb{R}$. Sometimes the codomain can also be the complex numbers. The input can be seen as an index for an element of the sequence. Sometimes it is more convenient to restrict the domain to the positive natural numbers, i.e. consider functions $f: \mathbb{N}^+ \rightarrow \mathbb{R}$. You will find both definitions for sequences in the literature. Here, we will mostly use the former and I will give a special disclaimer when the latter is used. Some example sequences are:
\begin{equation}
 a_n = (-1)^n,           \quad 
 a_n = \frac{1}{n},      \quad
 a_n = \frac{(-1)^n}{n}, \quad 
 a_n = \frac{1}{n^2},    \quad
 a_n = \frac{1}{2^n},    \quad  
 a_n = \sqrt{n}
\end{equation}
The examples with an $n$ or $n^2$ in the denominator can only be meaningfully defined as functions from the positive naturals into the reals because otherwise we would have a division by zero for the first (actually zero-th) element, i.e. when $n=0$. For the last two, plugging in $n=0$ poses no problem because $2^0=1$ and $\sqrt{0} = 0$. So, for those which would have a division by zero for $n=0$, we would chose a domain of $\mathbb{N}^+$ while for the others, we could choose $\mathbb{N}$. We could also say that they are all defined on $\mathbb{N}$ and make a special case definition $a_0 = 0$ for the problematic ones. The function definition would then look a bit ugly but is entirely legitimate. Sequences themselves are a sort of preliminary to what we actually want to build up to: namely series. These are infinite sums of sequence elements.

% https://en.wikipedia.org/wiki/Sequence

\subsection{Operations}

\subsubsection{Pointwise Operations}
Just like we can do with any functions $f: \mathbb{R} \rightarrow \mathbb{R}$, we can of course also add, subtract, multiply and divide two functions $f: \mathbb{N} \rightarrow \mathbb{R}$ pointwise. The fact that the domain is restricted to the naturals does not change anything about that. That means: we can just perform our usual arithmetic operations on sequences element-wise.

\subsubsection{Convolution aka Cauchy Product}
Another binary operation that we can perform on two sequences $(a_n),(b_n)$ is \emph{convolution}, also known as the \emph{Cauchy product} between the sequences. We will denote the operation symbol for convolution by an asterisk $\ast$ and define the operation as:
\begin{equation}
 (c_n) = (a_n) \ast (b_n) = (b_n) \ast (a_n) \quad \text{with} \quad
 c_n = \sum_{k=0}^n a_k b_{n-k} = \sum_{k=0}^n b_k a_{n-k}
\end{equation}
where, from the given formula, you will already correctly infer that the operation is commutative.
[VERIFY!]. It is also associative and distributive over addition and subtraction. This distributivity, together with the fact that scaling one of the inputs by a factor yields an output scaled by the same factor constitutes the important property of \emph{bilinearity}. That means, the operation is linear in both of its input arguments.
% https://mathworld.wolfram.com/CauchyProduct.html

\paragraph{Deconvolution}
With some caveats, it is actually possible to undo a convolution. We will call this operation \emph{deconvolution}. ...TBC...

\paragraph{Continuous Convolution}
As a side note, an analoguos operation of convolution can also be defined between two functions $f,g: \mathbb{R} \rightarrow \mathbb{R}$. In this case, the convolution is defined via an integral instead of a sum:
\begin{equation}
 h(t) = f(t) \ast g(t) = g(t) \ast f(t) \quad \text{with} \quad
 h(t) = \int_{-\infty}^{\infty} f(\tau) g(t-\tau) \, d \tau 
      = \int_{-\infty}^{\infty} g(\tau) f(t-\tau) \, d \tau
\end{equation}
This is not relevant in the context of series but it's sometimes good to point out the connections to other topics. I have used $t$ for the input and $\tau$ for the dummy integration variable here because this is the way, you often find it stated. This operation occurs a lot in signal processing where $t$ stands for time and $\tau$ for a time-lag or delay.




\subsection{Convergence}
When faced with a sequence of numbers, an important question is its behavior when the index $n$ approaches infinity. Specifically, we are interested, if the output of the sequence approaches some finite number or not. There are a couple of things that could happen when $n$ approaches infinity: (1) the numbers converge to some finite number, (2) the numbers diverge to infinity, (3) the numbers stay finite but jump around and approach nothing. In the third case, we will mostly find situations where the sequence alternates between positive and negative values. The example sequence $(-1)^n$ is of that kind because $(-1)^n$ will be $+1$ for even $n$ and $-1$ for odd $n$ and thereby induce this alternating behavior. The sequence $1/n$ approaches zero, so that would count as convergent. The sequence $(-1)^n / n$ is an alternating version of the former. It alternates between positive and negative values but the absolute values become smaller and smaller. It also converges to zero but in an alternating way. The sequence $1/n^2$ also converges to zero and does so even faster than $1/n$ and $1/2^n$ converges to zero yet faster.

%This sequence has a name: it's called the \emph{alternating harmonic series}.

%Theorems: pointwise sum of sequences converes to sum of limits, etc.




\subsection{Infinite Sums aka Series}
When we have a given sequence $(a_n)$, we can use it to define a new sequence, namely the sequence of its partial sums. Let's call that sequence $(s_n)$. Its $n$th element is given by the sum over all elements up to $n$ of our input sequence $(a_n)$:
\begin{equation}
 s_n = \sum_{k=0}^n a_k
\end{equation}
Just like with any sequence, we can ask whether this sequence of partial sums converges or diverges when we let $n$ approach infinity. What we have then created is an infinite sum. Such infinite sums have a special name: they are called \emph{series}.
...TBC...


\subsubsection{Some Sums}
ToDo: give formulas for finite and infinite geometric series, infinite alternating harmonic series


\subsection{Power Series}
A power series is a series that involves a variable $x$. We imagine that we have a given sequence $(a_n)$ and we will intepret this as a sequence of coefficients of a series of powers of $x$, i.e. a sort of infinite polynomial. We will consider the series:
\begin{equation}
 f(x) = \sum_{n=0}^\infty a_n (x-c)^n
\end{equation}
where $c$ is some fixed constant. It's often zero but for generality, I've already included it. On the left hand side, I already suggestively have written $f(x)$ to indicate that this "infinite polynomial" may be used to define a function - at least for those values of $x$, for which the infinite sum converges to a finite value. If the series converges for any $x$ at all, it will do so inside a disc centered at $c$ when we assume that $c$ and $x$ can be complex numbers. The radius of this disc is called the radius of convergence. In general, we could also allow complex coefficients $a_n$. For the time being, we'll focus our attention to real $x,c,a_n$ all being numbers, though.
% https://en.wikipedia.org/wiki/Radius_of_convergence


\paragraph{Convolution Revisited}
When we defined the convolution operation between sequences, it may have seemed somewhat arbitrary and unmotivated. In light of power series, we may give it some interpretation that also justifies why it is also called the Cauchy "product". Recall that to mutliply two (finite) polynomials, we have to convolve their lists of coefficients. The Cauchy product is basically the infinite version of that. It's a generalization in the sense that it includes the finite case as special case when we assume that the sequences just become identically zero after some $n$. The product sequence is the sequence of coefficients of an infinite product polynomial, so to speak...TBC...that can perhaps be explained better

...TBC..ToDo: explain how to compute radius of convergence

%explain 

\subsubsection{Taylor Expansion}
One truly remarkable result of calculus is that a lot of functions can be expressed as such a power series. For many important functions (including our favorites $\exp, \sin, \cos$), we will even find an infinite radius of convergence. Taylor's theorem tells us how to find the coefficients and the formula is actually pretty simple. 

...TBC...give formula for coeffs, explain aproximation properties and error term

% https://en.wikipedia.org/wiki/Taylor%27s_theorem



\subsection{Trigonometric Series}
We now want to look at a completely different kind of series. It will again involve a variable $x$ but this time, we will not use powers of $x$ but instead trigonometric functions, specifically sines and cosines. We will look at series of the form:
\begin{equation}
 f(x) = \sum_{n=0}^\infty \left(  a_n \cos(n \omega x) + b_n \sin(n \omega x) \right)
\end{equation}
where $\omega$ is some fixed constant. As with power series, the $f(x)$ on the left hand side indicates that we intend to use such a series to define a function of $x$. By construction, our so defined function will be periodic with a period of $p = \frac{2\pi}{\omega}$. This can easily be seen by considering a couple of values for $n$. For $n = 0$, the argument of sine and cosine becomes $0$ as well. The sine of zero is zero and the cosine of zero is one, so the $n=0$ term will just give us the constant function with value $a_0$. The $n=1$ term gives a sine and cosine of period $p$, the $n=2$ term gives a sin/cos pair with period $p/2$. Of course, when something is periodic with period $p/2$, it is also periodic with period $p$. And the same is true for any $n$: a function that is periodic with period $p/n$ is also periodic with period $p$. So, all the higher terms will also be $p$-periodic. They will just undergo $n$ cycles when our slowest sinusoid (for $n=1$) undergoes one cycle.

% https://en.wikipedia.org/wiki/Fourier_series
% Leupold, Vol2, pg 50 ff
% maybe use \tau instead of 2 \pi 

%...TBC...

%explain different conventions - some use $2 \pi x$ as argument, some pull out $a_0/2$, etc.

\subsubsection{Fourier Expansion}
The Taylor expansion expressed arbitrary functions as (potentially) infinite weighted sums of powers of the variable $x$. The coefficients were computed using derivatives of the function at one single point. Fourier expansions, on the other hand, express arbitrary periodic functions as (potentially) infinite sums of sines and cosines of $x$. The coefficients are calculated by a definite integral over one period. The formulas are:
\begin{equation}
 a_0 = \frac{1}{p} \int_0^p f(x) \, dx, \quad
 b_0 = 0, \quad
 a_n = \frac{2}{p} \int_0^p f(x) \cos(n \omega x) \, dx, \quad
 b_n = \frac{2}{p} \int_0^p f(x) \sin(n \omega x) \, dx 
\end{equation}

...TBC...explain the convergence properties, requirements on $f$ ((square?)-integrable?), explain hwo the formulas compute a correlation, why this special case for $a_0$



\paragraph{Aperiodic Functions}
So far, we assumed $f$ to be periodic but the theory can actually be generalized to deal with (some) aperiodic functions, too. ...TBC...




\subsubsection{Connection to Power Series}
ToDo: bring stuff from Weitz video "Taylor trifft Fourier"