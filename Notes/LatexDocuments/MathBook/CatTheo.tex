\chapter{Category Theory}
% Maybe make that a chapter in the structural math part

Category theory is an area of math that attempts to structure and organize mathematics itself. This section here is meant to only give a very superficial birds eye overview. Category theory has been called the "mathematics of mathematics" and - less charitably - as "abstract nonsense". Its relation to set theory has been compared to that of higher level programming languages to machine code. Category theory describes a given field of mathematics in terms of a so called \emph{category}. Such a category consists of a class of \emph{objects} and relationships between those objects, called \emph{morphisms}. You can picture the objects as nodes (as in a multigraph, see [REF needed to Graph Theory section]) and the morphisms as directed edges, depicted as arrows. Morphisms are also sometimes called \emph{arrows}. The category must also have a notion of composing morphisms. When there's an arrow from node $A$ to node $B$ and another arrow from node $B$ to node $C$ then this induces an arrow from node $A$ to node $C$. This sort of \emph{composition} of morphisms into other morphisms can be pictured as traversing a path in the multigraph. The composition of morphisms must be associative. As final ingredient for a category, we need for each node a special kind of morphism, called an \emph{identity}. An identity is a morphism/arrow that goes from a node $A$ to itself and is meant to convey an abstraction of the idea of an identity function - a sort of neutral element in the realm of morphisms.
%ToDo: composition (-> paths), identites (-> loopy edges)

% https://en.wikipedia.org/wiki/Multigraph

\paragraph{The category "Set"} is perhaps the most immediate and prototypical example of a category. In Set, the objects are sets and the morphisms are functions. Composition of morphisms is the usual compositon of functions and the identites are of course the respective identity functions for each set. Don't make the mistake of thinking about set elements as objects. The sets themselves are the objects and their internal structure is considered opaque. From a categorical point of view, we can't look into the sets. We can't talk about individual elements at all in categorical terms - the reason being that, in general, the objects can be of a completely different nature and don't need to have a concept of "elements". One object in Set would be $\mathbb{N}$, another one would be $\mathbb{R}$, another one would be $\{0,1\}$, etc. The category Set has a node for every possible set and for every possible ordered pair $(A,B)$ of nodes (i.e. sets), it has bunch of directed edges (arrows) where each such arrow stands for a possible function from set $A$ to set $B$. For example, there would be an arrow called "$\sqrt{\; \;}$" from $\mathbb{N}$ to $\mathbb{R}$. There would also be a "$\sqrt{\; \;}$" from $\mathbb{R}$ to $\mathbb{R}$ and one from $\mathbb{N}$ to $\mathbb{N}$. As morphisms, they would all be considered different entities even though they are supposed to stand for the same mathematical operation of taking the square root. Each morphism carries along with it its source (or domain) and target (or codomain). Functions can have certain properties and the most important ones are injectivity, surjectivity and bijectivity. Within the framework of set theory, these properties are defined in a way that references set elements. For example, a function from $A$ to $B$ is defined to be bijective, iff every element $a \in A$ maps to a unique element $b \in B$ and vice versa.

% Other examples:
% category of proofs: objects: propositions, arrows: deductions

\paragraph{Iso-, Mono- and Epimorphisms}
Abstracting the idea of bijective functions to a more general setting in which the objects are not necessarily sets and the morphisms are not necessarily functions leads to the categorical idea of an \emph{isomorphism} which is a particular kind of morphism with certain additional properties. The key here is that in the definition of what properties such an isomorphism must satisfy, we are not allowed to talk about "elements". The only things that we are allowed to talk about are categorical terms like objects, morphisms, compositions and identities. We must capture the idea of bijectivity in terms of these and only these ideas. It goes like this: A morphism $f: A \rightarrow B$ is an isomorphism, if there exists a morphism $g: B \rightarrow A$ such that $g \circ f = id_A$ and  $f \circ g = id_B$. In this context, $g$ is the inverse morphism of $f$ and itself an isomorphism as well. In terms of elements when we are in Set: we take an element $a \in A$, apply $f$ to map it to an element $b \in B$, then apply $g$ to map $b$ back to an element $a' \in A$, then $a' = a$, i.e. we're back to where we started so the whole roundtrip from set $A$ to set $B$ and back to $A$ reduces to an identity operation in $A$. Stated without mentioning elements and therefore generalizable: Applying $f$ first, then $g$ yields a composed morphism that equals the identity in $A$ and applying $g$ first, then $f$ yields the identity in $B$. Note how at no point are we talking about the internal structure of the objects like we did when referencing set elements in the definition of bijective functions. The categorical definition of an isomorphism is not even focusing on the objects at all but it's all about the morphisms. The objects are just the backdrop and our main attention goes to the morphisms. We are indeed looking at things from a higher level than in set theoretical descriptions where we looked \emph{inside} the details of the objects. This is typical of category theoretical definitions and theorems. Likewise, a monomorphism $f: A \rightarrow B$ is characterized by the property that for all morphisms $g,h: C \rightarrow A$, we have that $f \circ g = f \circ h$ implies $g = h$. Monomorphisms generalize the idea of an injective (i.e. left-unique) function. An epimorphism $f: A \rightarrow B$ is defined by the property that for all morphisms $g,h: B \rightarrow C$, we have that $g \circ f = h \circ f$ implies $g = h$. Epimorphisms generalize the idea of a surjective (i.e. right-total) function. [VERIFY] If you don't immediately see why these definitions indeed capture and generalize the ideas of injectivity and surjectivity (I certainly didn't), try reversing the implications. For injectivity/monomorphisms, rewrite $(f \circ g = f \circ h) \Rightarrow (g = h)$ as the equivalent statement $(g \neq h) \Rightarrow (f \circ g \neq f \circ h)$ and draw some picture where $g \neq h$. Then do something similar for surjectivity/epimorphisms. Then pat yourself on the back for having grasped a rather abstract and obscure categorical definition. [TODO: maybe add a figure that shows this].
%pg 482 in Ehrig et al



\begin{comment}

https://www.youtube.com/watch?v=H0Ek86IH-3Y
Initial objects have arrow to all objects, terminal object arrows from all objects

endomorphism, automorphism, homomorphism - the set of morphisms is also sometimes called "Hom" - why?

ToDo: 
-explain the categorical analogs of injective and surjective functions (mono- and epimorphisms)
 -explain why the definitions of mono- and epimorphisms work, i.e. capture the desired idea
 -reverse the implications: g != h  ->  g°f != h°f etc. and draw pictures where g != h and show how that implies the RHS
-give other examples of categories: FinSet, Grp, Vect, deductive systems, Graph
-explain subcategories in this context - many other examples actually are subcategories of Set
-examplin categorical products and coproducts
-explain functors, natural transformations

https://en.wikipedia.org/wiki/Category_theory
https://plato.stanford.edu/entries/category-theory/#Exam

https://www.youtube.com/watch?v=SmXB2K_5lcA  Category Theory for Programmers: Chapter 1 - Category

https://www.youtube.com/watch?v=1TvNeFLGMrE  Abstrakter Unsinn? Was ist Kategorientheorie?

https://math.jhu.edu/~eriehl/context.pdf

https://texample.net/tikz/examples/tag/diagrams/
https://texample.net/tikz/examples/labeled-chain/

The Mathematician's Weapon | An Introduction to Category Theory, Abstraction and Algebra
https://www.youtube.com/watch?v=FQYOpD7tv30
 "Category theory is the abstraction of composition"
 has some nice examples of categories
 https://www.youtube.com/watch?v=5Ykrfqrxc8o&list=PLoCKNPo3VR0I2wqT2wemCNIlpjdy_Ry_q&index=2
 https://www.youtube.com/watch?v=DrldYpmwN5s

Maybe draw a diagram for Set with some example objects (sets) and arrows (functions)
Q\{0} x Q  ->  Q:  (a,b) -> a/b
R  ->  {0,1}:  a -> 1, if a rational, 0 otherwise 

\medskip 
Of course, there's much more to say about category theory but this very brief overview shall suffice for a book that attempts to focus on applied math.


The First Real Application of Category Theory #SoME3
https://www.youtube.com/watch?v=Njx2ed8RGis

Universal Construction | Category Theory and Why We Care
https://www.youtube.com/watch?v=HJ9-yvZ-Toc
- Terminal objects correspond to singleton sets. There's exactly one possible arrow from
  any object in the category to the terminal object.
- All terminal objects are isomorphic to one another which is why they are somtimes lumped together
  into "the" terminal object.
- In the category of propositional logic, the proposition "True"  is the terminal object



https://www.youtube.com/watch?v=yP2RjVD-cZ0  A gentle introduction to category theory

\end{comment}


