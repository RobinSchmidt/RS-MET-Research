\chapter{Combinatorics}
Combinatorics is concerned with counting all the possible ways to do a certain thing. This "thing" could be something like "select $k$ elements from a set of size $n$" or "create all possible acylic graphs of size $n$" or "find all ways to express a natural number $n$ as sum of smaller numbers" or "find all the possible ways that 5 rolls of a die could result in a sum less than 17". Things like that. The common theme is to count the number of different possibilities to do that thing. The result of such a counting tally will always be a discrete quantity or, more specifically, a natural number. Therefore, combinatorics is firmly situated in the realm of \emph{discrete mathematics} which is the mathematics that deals with things that can be enumerated, i.e. put into a (potentially infinitely long) list. The protypical example of a set of enumerable things is the set of natural numbers - other examples are integer numbers and the rational numbers\footnote{Yes, rational numbers are indeed enumerable - we'll see that later in axiomatic set theory.} but also things that are more common in computer science like trees, graphs, strings, etc. Combinatorics lays a lot of groundwork for other mathematical fields such as set theory, graph theory, number theory and probability theory. It is therefore appropriate to put the topic first in the "discrete math" part of the book. By the way, the examples of possible questions "select $k$ elements..." were taken from these fields - in that order. ...TBC...

% Discrete mathematics is the kind of mathematics that can represented exactly on a computer. To deal with continuous mathematics, we have to make do with approximations



\begin{comment}

https://en.wikipedia.org/wiki/Combinatorics
https://de.wikipedia.org/wiki/Abz%C3%A4hlende_Kombinatorik
https://en.wikipedia.org/wiki/Generating_function

Catalan, Stirling, Bernoulli ... numbers

Superpermutations: the maths problem solved by 4chan
https://www.youtube.com/watch?v=OZzIvl1tbPo
-shortest sequence of elements a,b,c,... that contains all possible permutation. For example,
 "aba" is a superpermtation of the elements a,b because it contains both possible permutations, namely
 "ab" and "ba" as substrings

\end{comment}