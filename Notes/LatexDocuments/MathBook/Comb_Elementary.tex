\section{Elementary Combinatorics} 
% Maybe call it "enumerative combinatorics"

%\subsection{Selecting Elements from a Set} 
\subsection{Collections of Set Elements} 
% Maybe rename to "Element Selections", "Choices", "Collections"

% Leupold, Vol. 2 pg 238 has a nice overview
% the table on pg 332 is also nice - reproduce it - but with formulas for the numbers

% Weitz pg 177 has also a nice introdcution

% Notation: the set has n elements and we choose k from it

% https://de.wikipedia.org/wiki/Abz%C3%A4hlende_Kombinatorik
% https://en.wikipedia.org/wiki/Enumerative_combinatorics
% "The problem of finding a closed formula is known as algebraic enumeration, and frequently involves deriving a recurrence relation or generating function and using this to arrive at the desired closed form"
% "a simple asymptotic approximation may be preferable" ...Stirlings approximations

% https://de.wikipedia.org/wiki/Abz%C3%A4hlende_Kombinatorik#Begriffsabgrenzungen
% has a translation table between the German and English terms and also other nice tables

% https://en.wikipedia.org/wiki/Permutation#Other_uses_of_the_term_permutation

% The terminology seems a mess

\paragraph{Permutations}
Let's say that we have $n$ different (i.e. distinguishable) objects and we want to know, how many ways there are, to put these objects into a different order. For example, the possible orders for the $3$ objects $a,b,c$ are $abc$, $acb$, $bac$, $bca$, $cab$, $cba$. These are $6$ different orders. We call these different orderings of a number of $n$ different objects \emph{permutations}. The term "permute" means something like "reorder" where identity (i.e. do-nothing) reordering also counts as trivial edge case (as is typical in math). The number of \emph{permutations} of $n$ distinguishable objects is given by factorial of $n$:
\begin{equation}
P_n = 1 \cdot 2 \cdot 3 \cdot \ldots \cdot n = n!
\end{equation}
This is easy to understand when we recognize that there are exactly $n$ different ways to choose an element for the first position because we have $n$ elements to choose from. Then, after the first element has been chosen, we have left $n-1$ elements to choose for the second position, then $n-2$ for the third and so on. We get $n\cdot(n-1)\cdot(n-2)\cdot\ldots\cdot1$ ways all in all. I stressed the point that the objects need to be distinguishable. This is the usual case when the objects are elements of a mathematical \emph{set} because set elements are by definition distinguishable. There is a more general notion of a \emph{multiset} in which there may be indistinguishable objects. If the $n$ objects are not all distinguishable such that they fall into $k$ equivalence classes of sizes $m_1, m_2, \ldots, m_k$ with $m_1 + m_2 + \ldots + m_k = n$, then there are:
\begin{equation}
P_n^{(m_1,m_2,\ldots,m_k)} 
= \frac{n!}{m_1! m_2! \ldots m_k!} 
= \binom{n}{m_1,m_2,\ldots, m_k}
\end{equation}
distinguishable permutations. These ares sometimes called \emph{multiset permutations} or \emph{permutations with repetition} [VERIFY!]. The numbers $m_i$ are called the \emph{multiplicities}. The quantity is also called the multinomial coefficient and it is a generalization of the binomial coefficient "$n$-choose-$k$" and the notation on the right hand side is a corresponding generalization of the usual binomial coefficient notation $\binom{n}{k}$. ...TBC... mention asymptotic approximations such as Stirling's formula and continuous variants as the product function(or its shifted version, the Gamma-function, point out relation to expanding an expression like $(\sum_{i=1}^{n} x_i)^p$ - here $n$ is the number of variables and $p$ is the power/exponent)

% Leupold, pg 329
% https://en.wikipedia.org/wiki/Permutation#Permutations_of_multisets
% https://en.wikipedia.org/wiki/Multinomial_theorem#Multinomial_coefficients

% Nice example:
% https://www.tutorchase.com/notes/cie-a-level/maths/4-2-2-arrangements-with-repetition

% %\pageref{Sec:FactorialsAndBinomCoeffs}

\paragraph{Combinations}
Imagine we have a set of size $n$ and we are asked how many different subsets of size $k$ we can form from this set. That is: how many different ways are there to choose $k$ elements from a set of $n$ elements. It doesn't matter in which order we chose the elements - as long as the total set of chosen elements is the same, our created subset is considered to be the same (remember that in sets, order doesn't matter - it isn't even a thing\footnote{Unless you \emph{define} an order \emph{on top} of the set - but a set all by itself does not have a notion of order built into it.}). We call these different subsets of size $k$ of a given set of size $n$ \emph{combinations} or, more specifically, $k$-combinations. The number of possible $k$-combinations from a set of size $n$ is given by the binomial coefficient:
\begin{equation}
C_n^k = \binom{n}{k} 
      = \frac{n!}{k! (n-k)!} 
      = \frac{n \cdot (n-1) \cdot (n-2) \cdot \ldots \cdot (n-k+1)}{k!}
      = \frac{\prod_{m=n-k+1}^{n} m}{k!}
\end{equation}
which is, for this reason, typically pronounced as "$n$-choose-$k$". In some situations, we may be allowed to choose an element more than once such that we do not form a (sub)set anymore but instead a multiset with elements picked from a given set. For the number different possible such combinations with repetition we get:
\begin{equation}
\prescript{r}{}{C}_n^k
= \binom{n+k-1}{k}
= \frac{n+k-1}{(n-1)! k!}
= \left(\!\!{n\choose k}\!\!\right)
\end{equation}
where the top-left superscript $r$ stands for "with repetition"and the right notation that looks like binomial coefficient notation with double parentheses is sometimes used for this. This number is sometimes pronounced as "$n$-multichoose-$k$". VERIFY...TBC...Explain Pascal's triangle

% https://en.wikipedia.org/wiki/Combination#Number_of_combinations_with_repetition
% notation with double parentheses, "n-multichoose-k"

% https://tex.stackexchange.com/questions/5816/multiset-notation-in-latex
% https://en.wikipedia.org/wiki/Combination


\paragraph{Variations}
%The number of \emph{variations} of $n$ distinguishable objects is given by:
Now we have given a set of size $n$ and the task is to choose $k$ objects from it and put them into a given order. How many ways are there to do this? That is, how many different $k$-tuples can be formed from elements of a set of size $n$? If we allow an object to be chosen more than once, these tuples are is called \emph{variations with repetition} or \emph{arrangements with repetition} and there are:
\begin{equation}
\prescript{r}{}{V}_n^k = n^k
\end{equation}
of them. Again, the $r$ stands for "with repetition". In this setup, the number $k$ can even be greater than $n$. In such a case, obviously, there \emph{must} be repetitions. We could, for example, ask how many different strings of length $7$ could be formed from the $3$ letters $a,b,c$. The answer is $3^7 = 2187$. If repetitions are not allowed, we call it \emph{variations/arrangements with repetition} and there are:
\begin{equation}
V_n^k = \frac{n!}{(n-k)!} 
      = n \cdot (n-1) \cdot (n-2) \cdot \ldots \cdot (n-k+1) 
      = \prod_{m=n-k+1}^{n} m
\end{equation}
of them. ...VERIFY! ...TBC...

% https://en.wikipedia.org/wiki/Permutation#Other_uses_of_the_term_permutation
% k-permutation of n, partial permutation, sequence without repetition, variation, or arrangement

% https://en.wikipedia.org/wiki/Falling_and_rising_factorials

% https://de.wikipedia.org/wiki/Variation_(Kombinatorik)

% https://www.tutorchase.com/notes/cie-a-level/maths/4-2-2-arrangements-with-repetition
% 
% https://stackoverflow.com/questions/2366074/code-for-variations-with-repetition-combinatorics

\paragraph{Partitions}
A \emph{partition} of a set $S$ is a way of splitting it into disjoint, non-empty subsets whose union gives back the original set. For example, take the set $\{a,b,c\}$. We can partition it in the following 5 ways: $\{\{a\},\{b\},\{c\}\}, \{\{a\},\{b,c\}\}, \{\{a,b\},\{c\}\}, \{\{a,c\},\{b\}\}, \{\{a,b,c\}\}$. For a set with $n$ elements, the number of possible partitions is given by the $n$th \emph{Bell number} $B_n$. Partitions of a set can also be interpreted in terms of equivalence classes. Therefore, the number $B_n$ gives the number of possible different equivalence relations on a set of size $n$. The first $7$ Bell numbers are: $B_0=1, B_1=1, B_2=2, B_3=5, B_4=15, B_5=52, B_6=203$. The Bell numbers can be computed algorithmically in a similar way as binomial coefficients. For those, we could use Pascal's triangle. For the Bell numbers, there's Bell's triangle. It looks as follows:
\begin{center}  % doesn't work!
\begin{verbatim}
                    1
                 1     2
              2     3     5
           5     7    10    15
       15    20    27    37    52
    52    67    87   114   151   203
203   255   322   409   523   674   877
\end{verbatim}
\end{center}
The rules to compute the numbers are as follows: Start with the single $1$ in the first row. To start the next row, copy the last number from the previous row into the first position of the next row. All other numbers in the row are computed by adding the left and top-left neighbors until the row is completely filled. Then copy the last number of the row into the first position of the next row again. And so on. The sequence of Bell numbers appears in the leftmost column, i.e. in the row start numbers. They also appear in the row end numbers but index-shifted by one. 

...TBC... Q: Do the inner numbers of the Bell triangle have some meaning, too? ToDo: mention integer partitions (don't confuse these concepts).

% https://en.wikipedia.org/wiki/Partition_of_a_set
% https://en.wikipedia.org/wiki/Bell_number
% https://en.wikipedia.org/wiki/Bell_triangle  - similar to Pascal's triangle
% https://en.wikipedia.org/wiki/Partition_of_a_set#Counting_partitions
% ...non-crossing partitions / Catalan numbers
% Stirling numbers

% Not to be confused with:
% https://en.wikipedia.org/wiki/Partition_problem
% https://en.wikipedia.org/wiki/Integer_partition


\begin{comment}


https://en.wikipedia.org/wiki/Central_binomial_coefficient
https://en.wikipedia.org/wiki/Catalan_number#Applications_in_combinatorics
https://en.wikipedia.org/wiki/Stirling_number
https://en.wikipedia.org/wiki/Stirling_numbers_of_the_first_kind
https://en.wikipedia.org/wiki/Gaussian_binomial_coefficient#Combinatorial_descriptions


\end{comment}

