\section{Elementary Combinatorics} 
% Maybe call it "enumerative combinatorics"

%\subsection{Selecting Elements from a Set} 
\subsection{Collections of Set Elements} 
% Maybe rename to "Element Selections", "Choices", "Collections"

% Leupold, Vol. 2 pg 238 has a nice overview
% the table on pg 332 is also nice - reproduce it - but with formulas for the numbers

% Weitz pg 177 has also a nice introdcution

% Notation: the set has n elements and we choose k from it

% https://de.wikipedia.org/wiki/Abz%C3%A4hlende_Kombinatorik
% https://en.wikipedia.org/wiki/Enumerative_combinatorics
% "The problem of finding a closed formula is known as algebraic enumeration, and frequently involves deriving a recurrence relation or generating function and using this to arrive at the desired closed form"
% "a simple asymptotic approximation may be preferable" ...Stirlings approximations

% https://de.wikipedia.org/wiki/Abz%C3%A4hlende_Kombinatorik#Begriffsabgrenzungen
% has a translation table between the German and English terms and also other nice tables

% https://en.wikipedia.org/wiki/Permutation#Other_uses_of_the_term_permutation

% The terminology seems a mess

\paragraph{Permutations}
Let's say that we have $n$ different (i.e. distinguishable) objects and we want to know, how many ways there are, to put these objects into a different order. For example, the possible orders for the $3$ objects $a,b,c$ are $abc$, $acb$, $bac$, $bca$, $cab$, $cba$. These are $6$ different orders. We call these different orderings of a number of $n$ different objects \emph{permutations}. The term "permute" means something like "reorder" where identity (i.e. do-nothing) reordering also counts as trivial edge case (as is typical in math). The number of \emph{permutations} of $n$ distinguishable objects is given by:
\begin{equation}
P_n = 1 \cdot 2 \cdot 3 \cdot \ldots \cdot n = n!
\end{equation}
This is easy to understand when we recognize that there are exactly $n$ different ways to put an element into the first position because we haven $n$ elements to choose from. Then, there remain $n-1$ different ways to put some other element into the second position and so on because after the first element has been chosen, we have $n-1$ elements left to choose from. And so on. We get $n\cdot(n-1)\cdot(n-2)\cdot\ldots\cdot1$ ways all in all. I stressed the point that the objects need to be distinguishable. This is the usual case when the objects are elements of a mathematical \emph{set} because set elements are by definition distinguishable. There is a more general notion of a \emph{multiset} in which there may be indistinguishable objects. If the $n$ objects are not all distinguishable such that they fall into $k$ equivalence classes of sizes $m_1, m_2, \ldots, m_k$ with $m_1 + m_2 + \ldots + m_k = n$, then there are:
\begin{equation}
P_n^{(m_1,m_2,\ldots,m_k)} 
= \frac{n!}{m_1! m_2! \ldots m_k!} 
= \binom{n}{m_1,m_2,\ldots, m_k}
\end{equation}
distinguishable permutations. These ares sometimes called \emph{multiset permutations} or \emph{permutations with repetition} [VERIFY!]. The numbers $m_i$ are called the \emph{multiplicities}. The quantity is also called the multinomial coefficient and it is a generalization of the binomial coefficient "$n$-choose-$k$" and the notation on the right hand side is a corresponding generalization of the usual binomial coefficient notation $\binom{n}{k}$. ...TBC... mention asymptotic approximations such as Stirling's formula and continuous variants as the product function(or its shifted version, the Gamma-function)

% Leupold, pg 329
% https://en.wikipedia.org/wiki/Permutation#Permutations_of_multisets
% https://en.wikipedia.org/wiki/Multinomial_theorem#Multinomial_coefficients

\paragraph{Combinations}
Imagine we have a set of size $n$ and we are asked how many different subsets of size $k$ we can form from this set. That is: how many different ways are there to choose $k$ elements from a set of $n$ elements. It doesn't matter in which order we chose the elements - as long as the total set of chosen elements is the same, our created subset is considered to be the same (remember that in sets, order doesn't matter - it isn't even a thing\footnote{Unless you \emph{define} one order \emph{on top} of the set - but a set all by itself does not have a notion of order built into it.}). We call these different subsets of size $k$ of a given set of size $n$ \emph{combinations} or, more specifically, $k$-combinations. The number of possible $k$-combinations from a set of size $n$ is given by the binomial coefficient:
\begin{equation}
C_n^k = \binom{n}{k} 
      = \frac{n!}{k! (n-k)!} 
      = \frac{n \cdot (n-1) \cdot (n-2) \cdot \ldots \cdot (n-k+1)}{k!}
      = \frac{\prod_{m=n-k+1}^{n} m}{k!}
\end{equation}
which is, for this reason, typically pronounced as "$n$-choose-$k$". In some situations, we may be allowed to choose an element more than once such that we do not form a (sub)set anymore but instead a multiset with elements picked from a given set. For the number different possible such combinations with repetition we get:
\begin{equation}
\prescript{r}{}{C}_n^k
= \binom{n+k-1}{k}
= \frac{n+k-1}{(n-1)! k!}
= \left(\!\!{n\choose k}\!\!\right)
\end{equation}
where the top-left superscript $r$ stands for "with repetition"and the right notation that looks like binomial coefficient notation with double parentheses is sometimes used for this. This number is sometimes pronounced as "$n$-multichoose-$k$". VERIFY...TBC...

% https://en.wikipedia.org/wiki/Combination#Number_of_combinations_with_repetition
% notation with double parentheses, "n-multichoose-k"

% https://tex.stackexchange.com/questions/5816/multiset-notation-in-latex
% https://en.wikipedia.org/wiki/Combination


\paragraph{Variations}
%The number of \emph{variations} of $n$ distinguishable objects is given by:
\begin{equation}
V_n^k = \frac{n!}{(n-k)!} 
      = n \cdot (n-1) \cdot (n-2) \cdot \ldots \cdot (n-k+1) 
      = \prod_{m=n-k+1}^{n} m
\end{equation}

% https://en.wikipedia.org/wiki/Permutation#Other_uses_of_the_term_permutation
% k-permutation of n, partial permutation, sequence without repetition, variation, or arrangement

% https://de.wikipedia.org/wiki/Variation_(Kombinatorik)

% https://en.wikipedia.org/wiki/Partition_of_a_set



\paragraph{Partitions}



\begin{comment}



\end{comment}

