\chapter{Discrete Calculus}
Calculus, in the most commonly used sense of the word, refers to the field of mathematics that deals with derivatives, integrals, differential equations, sequences and series expansions. In a broader sense \emph{a} calculus is a system of rules for doing calculations. The so called \emph{discrete calculus} is a calculus that is analogous to the usual differential and integral calculus, just that we do not go the limiting step where some quantity (like a $\Delta x$ or $h$, say) approaches zero. Instead, our $h$ remains a fixed finite (i.e. non-infinitesimal) value that is chosen upfront - usually taken to be $1$. It turns out that you can always reformulate the problem into a form where $h = 1$ such it doesn't really matter which value you choose, and choosing $1$ turns out to be most convenient. ...TBC...

% Calculus of sequences, i.e. of functions N -> R

% Examples: lambda calculus, residue


\section{Difference Calculus}
Difference calculus is the part of discrete calculus that is analogous to the usual differential calculus. ...TBC...

%ToDo: define forward and backward difference, mention connection to numerical approximation of derivatives (maybe mention the central difference although it plays no big role in difference calculus)

\section{Summation Calculus}
Summation calculus is the part of discrete calculus that is analogous to the usual integral calculus. ...TBC...


%\section{Function Approximation}
% -develope the Newton-Gregory formula as analogon for Taylor series



\section{Difference Equations}
The calculus of difference equations is the part of discrete calculus that is analogous to the calculus of differential equations. ...TBC...


\subsection{The Z-Transform}

% Generating Functions , the Z-Transform

% https://en.wikipedia.org/wiki/Generating_function
% -Looks like the ordinary generating function is the z-trafo for sequences that are zero
%  for n < 0. ...adn the z-trafo usually assumes a complex argument
% -Exponential generating functions can transform recurrence relations into ODEs.

\section{Umbral Calculus}

% The Shadowy World of Umbral Calculus
% https://www.youtube.com/watch?v=D0EUFP7-P1M


% The Strange Case of the Umbral Calculus
% https://www.youtube.com/watch?v=ytZkmV-7EbM

\begin{comment}


Why don't they teach Newton's calculus of 'What comes next?'
https://www.youtube.com/watch?v=4AuV93LOPcE

Differenzenrechnung: Vom Kalkül der diskreten Analysis
https://www.youtube.com/watch?v=1-Q7Z-F4KVs

Another World Similar to Calculus | Discrete Calculus
https://www.youtube.com/watch?v=mVjBSWHlTnY


\end{comment}