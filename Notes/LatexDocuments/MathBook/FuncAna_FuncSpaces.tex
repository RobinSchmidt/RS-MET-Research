\section{Function Spaces}
Recall from linear algebra the general definition of a vector space as a set of elements, called vectors, that can be added together to give another vector or multiplied by a scalar (i.e. a number) to give another vector. Now notice that we can do these two things with functions just the same: adding two functions pointwise gives a perfectly valid other function and multiplying a function by a number also gives another bona fide function. It can be easily verified that all the required vector-space rules such as associativity, commutativity, distributivity, neutral elements, etc. are satisfied for these operations, when applied to functions. That means, we can actually view the set of all real-valued functions $f: \mathbb{R \rightarrow R}$ as a vector space! This vector space, however, has a fundamentally different nature than those we have seen before. In order to uniquely determine a function $f$, we will have to say what its function value will be for any given input value $x$. If the domain of the function is the whole real number line (or even just a finite interval of it), that means, we must prescribe (uncountably) infinitely many function values, so the dimensionality of the space of all functions that map real numbers to real numbers is apparently also (uncountably) infinite. We will also encounter subspaces of this huge space whose dimensionality will be only countably infinite or even just finite.

\subsection{Scalar Products}
In the $N$-dimensional vector space $\mathbb{R}^N$, the standard scalar product of two vectors $\mathbf{u,v}$ was defined as the sum over the element-wise products:
\begin{equation}
 \langle \mathbf{u,v} \rangle = \sum_{i=1}^{N} u_i v_i
\end{equation}
That motivates the definition of the scalar product of two real-valued functions $f,g$, both defined on some interval $(a,b)$, by analogy as:
\begin{equation}
 \langle f,g \rangle = \int_a^b f(x) g(x) \; dx
\end{equation}
This is one possible definition of a scalar product on a space of functions and it is in fact the one that is most often used, but there are some other variations that should be mentioned, too. ...
% introduce the general from with a weight and maybe also the complex form, maybe also with a positive definite kernel, what about the integration limits, there are also scalar products involving derivatives

\subsection{Norms and Distances}
In $\mathbb{R}^N$, the Euclidean norm of a vector was defined as the square root of the scalar product of the vector with itself. We can define a norm for functions in a completely analogous way. However, we don't necessarily need a scalar product to define a norm. Another way to define a norm would be to just take the maximum value of the function within a given interval. ...
% L_p-norm, maximum-norm, norms induced by a scalar product, distances (norms of differences)

\subsection{Special Function Spaces}
Often we are not interested in the vast and unstructured set of all possible real functions but only in a subset of it, i.e. in a set of functions that meet some additional criteria. We may, for example, demand that the functions have to be continuous, differentiable, (square-)integrable, bounded, periodic, monotonic, symmetric, etc. It turns out that many of these subsets are actually also subspaces in the sense that addition of two such functions or multiplication of one of them by scalar will always give a new function that also belongs to the set. Some of these specific function spaces are important enough that they have been given special names, so we shall now familiarize ourselves with this terminology.

\subsubsection{Continuity and Differentiability}
One way to categorize functions is by considering, how smooth they are. Does a function $f$ have spikes, jumps, corners, etc? That idea is captured by the number of continuous derivatives that a function has. If a function is $k$ times continuously differentiable, then it is said to be in the class $C^k$. If a function is just itself continuous but has a discontinuous first derivative, it is in the class $C^0$ but not in the class $C^1$. An example of such a $C^0$ but not $C^1$ function would be the absolute value $f(x) = |x|$. In general, the classes with lower $k$ are less restrictive, i.e. allow more functions. But the classification is meant in an inclusive sense, i.e. a function that is in $C^k$ is automatically also in $C^j$ if $j < k$. If a function can be differentiated $k+1$ times, it means that the $k$-th derivative is not only continuous but even differentiable which is a stronger requirement. That means, a function that can be differentiated $k+1$ times is automatically at least in $C^k$. If a function can be differentiated infinitely often, it is said to be in the class $C^\infty$. Such functions are also called \emph{smooth}. If, in addition to be infinitely differentiable, it also has a convergent Taylor series at every point of its domain, it said to be in $C^\omega$ and also called \emph{analytic}\footnote{The letter $\omega$ is also used in other areas of math (ordinal numbers, nonstandard analysis) to denote "infinity". I'm not sure, if that usage there is related but to be honest, I can't see how. It seems like the $\omega$ from ordinal numbers would mean exactly the same thing as the $\infty$ symbol as it is used here. So maybe drawing a connection would be a bit hasty.}. Analytic functions are in some sense even nicer than just smooth ones.
ToDo: maybe use notation $C^k(\mathbb{R})$

% Can we generalize this idea to negative k? C^{-1} could be the space of "functions" that
% contain Dirac-spikes, C^{-2} could contain derivatives of Dirac spikes (i.e. bipolar
% spikes), etc.

% Integrating a function increases its k-value by 1, I think. Yeah - has to be because 
% differentiation reduces k by 1 by definition.

%...TODO: define $C^k$ spaces including $C^\infty$ and $C^\omega$

% Bump functions are in  $C^\infty$ but not in $C^\omega$. Explain why!

% https://en.wikipedia.org/wiki/Smoothness
% https://mathworld.wolfram.com/C-kFunction.html

\subsubsection{Integrability} Functions can also be categorized according to their integrability. By analogy with the classification based on differentiability, you might guess that we consider repeated integration of a function. But that's not how it works. Instead, we consider the integral of the $p$-th power of the absolute value of the function. For example, for functions $f: \mathbb{R} \rightarrow \mathbb{R}$, we may consider the integral of the absolute value of $f$ over the whole domain: $\int_{-\infty}^{+\infty} |f(x)| \; dx$ and classify those functions for which this integral is finite as belonging to the class of \emph{absolutely integrable} functions over the real numbers. Next, we may introduce an exponent $p$ to whose power we raise the absolute value and look at $\int_{-\infty}^{+\infty} |f(x)|^p \; dx$ and requiring that integral to be finite. For the special case $p=2$, we call the class of functions \emph{square integrable} and for general $p$, we call the functions \emph{$p$-th power integrable}. Generalizing further, we may consider functions $f: \mathbb{R}^n \rightarrow \mathbb{R}$ and look at $\int_{\mathbb{R}^n} |f(x_1, x_2,\ldots,x_n)|^p \; dx_1 dx_2 \ldots dx_n$ and requiring that integral to be finite. Generalizing even further, we may consider functions from any set $S$ for which a measure $\mu$ is defined into the real numbers\footnote{I guess, it could be generalized even further to allow the codomain be any set $T$ on which a suitable absolute value function is defined. $T =\mathbb{R}$ or $T = \mathbb{C}$ are two obvious choices.} $\mathbb{R}$ such that $f: S \rightarrow \mathbb{R}$ and requiring $\int_S |f|^p \; d \mu $ to be finite. This integral is apparently meant to be understood as a Lebesgue integral - so we require that the $p$-th power of the absolute value of $f$ must be Lebesgue integrable and the integral must be finite. We denote this set of functions by $\mathcal{L}^p(S,\mu)$ and call it the set of \emph{measurable functions} [VERIFY!]

%When talking about measures, it is clear


%any set of objects $T$ on which an absolute value is defined (such as $T =\mathbb{R}$ or $T = \mathbb{C}$)

% The normed vector space ( L p ( S , μ ) , ‖ ⋅ ‖ p ) {\displaystyle \left(L^{p}(S,\mu ),\|\cdot \|_{p}\right)} is called L p L^{p} space or the Lebesgue space of p p-th power integrable functions

%This idea could be generalized to functions $f: \mathbb{R}^n \rightarrow \mathbb{R}$ by considering $\int_{\mathbb{R}^n} |f(x_1, x_2,\ldots,x_n)|^p \; dx_1 dx_2 \ldots dx_n$ and requiring that integral to be finite.

%That is, for some given $f$, we consider the integral $\int_S |f|^p \; d \mu$ where $S$ is some set on which $f$ is defined (i.e. a subset of the domain of $f$) and $\mu$ is some measure that is defined on $S$ [VERIFY].

% https://en.wikipedia.org/wiki/Lp_space#Lp_spaces_and_Lebesgue_integrals

%\begin{equation}
%\int_S |f|^p \; d \mu
%\end{equation}

%Rather than considering repeated integration (as we did in the case), we consider

%TODO: define $L^p$ spaces named after Lebesgue, explain $\mathcal{L}^p, \ell^p$

% https://en.wikipedia.org/wiki/Lp_space#Lp_spaces_and_Lebesgue_integrals

% https://en.wikipedia.org/wiki/Lp_space
% https://en.wikipedia.org/wiki/Absolutely_integrable_function
% https://en.wikipedia.org/wiki/Square-integrable_function
% https://mathworld.wolfram.com/SquareIntegrable.html

% https://de.wikipedia.org/wiki/Lp-Raum

% https://en.wikipedia.org/wiki/Measurable_function
% https://mathworld.wolfram.com/MeasurableFunction.html


\subsubsection{Normed Function Spaces}
Let's go back the expression $\int_{-\infty}^{+\infty} |f(x)|^p \; dx$ that defines the class of $p$-th power integrable functions $f: \mathbb{R} \rightarrow \mathbb{R}$ by the requirement that this integral must be finite. So far, we really only looked at the cases $p=1$ and $p=2$. We initially intended to let $p$ be a positive natural number but we now want to widen our perspective to allow real $p \geq 1$ and even allow for $p = \infty$. In order to get nice behavior when we let $p$ go to infinity, we will now look at the expression $(\int_{-\infty}^{+\infty} |f(x)|^p \; dx)^{1/p}$. As long as we were just concerned with the finiteness of the integral and $p$ was assumed to be a finite positive number, it didn't really matter whether or not we do the $(\ldots)^{1/p}$ step at the end. But for what we want to do now, namely defining a norm that is still meaningful when $p \rightarrow \infty$, we need this additional reciprocal exponent\footnote{At least, I think this is the reason why we formerly left out the $p$-th root. Consider what happens to the integral expression when $p \rightarrow \infty$: for $|f| < 1$ the integrand will go to zero and for $|f| > 1$ the integrand will go to infinity. It appears like taking the $p$-th root after integration tames this behavior and we get a meaningful limit when $p \rightarrow \infty$.}. We define the $p$-norm of a function $f: S \rightarrow \mathbb{R}$ as
\begin{equation}
 \norm{f}_p = \left( \int_S |f|^p \; d\mu \right)^{1/p}
\end{equation}


 ...TBC...TODO: introduce $p$-norm and $L^p$ spaces as space of equivalence classes of functions from an $\mathcal{L}^p$ space

% https://en.wikipedia.org/wiki/Norm_(mathematics)#p-norm

% https://web.archive.org/web/20201024063542/http://faculty.bard.edu/belk/math461/LpFunctions.pdf

\paragraph{Banach Space}

% https://de.wikipedia.org/wiki/Banachraum
% -vollständiger normierter Vectorraum

\paragraph{Hilbert Space}

% https://de.wikipedia.org/wiki/Hilbertraum
% -Banachraum, dessen Norm durch ein Skalarprodukt induziert ist.

% https://en.wikipedia.org/wiki/Lp_space#Special_cases
% L^2 is the only Hilber space among the L^p spaces

\paragraph{Sobolev Space}

% https://de.wikipedia.org/wiki/Sobolev-Raum
% -Funktionenraum von schwach differenzierbaren Funktionen, der zugleich ein Banachraum ist. 


%\subsubsection{Analytic Functions}
% Smooth Functions, Differentiable Functions
%\subsubsection{Square Integrable Functions}
% finite support

%\subsubsection{Polynomials}
% powers, power series, orthogonal sets of polynomials (Legendre, Chebychev, etc.)

%\subsubsection{Periodic Functions}

% Hilbert-/Banach-/Sobolev-spaces, maybe drag the "Test Functions" subsection to here - it may fit here even better, maybe after analytic functions - maybe use the order: analytic, test, square-integrable as Weitz in his video about distributions

% https://en.wikipedia.org/wiki/Banach_space
% https://en.wikipedia.org/wiki/Hilbert_space
% https://en.wikipedia.org/wiki/Sobolev_space

\begin{comment}

-what are the function spaces that have a countably infinite dimension? Maybe those functions whose Taylor
 series converges everywhere belong to it
-scalar-product, norm, 
-say something about functions \mathbb{R^M \rightarrow R^N}$ - how does these fit into these scheme
-what about functions defined only on the integers or rationals?
-what about complex functions - in particular, how does the scalar product generalize

-completeness: after a norm and therefore a distance was defined, we may define Cauchy sequences of functions which are sequences of functions whose elements get arbitrarily close together with respect to the defined distance. if the limit of such a sequence also belongs to the function space, the space is said to be complete. give an example of an incomplete space and its completion...maybe the space of rational functions? could be a nice analogy to the rational numbers. consider the sequence of "Butterworth" functions f_n(x) = 1 / (1 + x^n). their limit is a step function which is not a rational function and therefore not in the set (note that the discontinuity itself is not the problem here - rational functions can be discontinuous, too due to the presence of poles - but poles are actually a different kind of discontinuity)

-norm equivalence: two norms are equivalent, iff each Cauchy sequence that converges with respect to one norm also converges with respect to the other norm and vice versa

important named function spaces:
Banach: complete, normed
Hilbert: Banach, where norm is induced by an inner product 
Sobolev: Hilbert, where differentiation is always allowed (all alements are differentiable)

\end{comment}