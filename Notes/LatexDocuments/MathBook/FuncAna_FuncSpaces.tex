\section{Function Spaces}
Recall from linear algebra the general definition of a vector space as a set of elements, called vectors, that can be added together to give another vector or multiplied by a scalar (i.e. a number) to give another vector. Now notice that we can do these two things with functions just the same: adding two functions pointwise gives a perfectly valid other function and multiplying a function by a number also gives another bona fide function. It can be easily verified that all the required vector-space rules such as associativity, commutativity, distributivity, neutral elements, etc. are satisfied for these operations, when applied to functions. That means, we can actually view the set of all real-valued functions $f: \mathbb{R \rightarrow R}$ as a vector space! This vector space, however, has a fundamentally different nature than those we have seen before. In order to uniquely determine a function $f$, we will have to say what its function value will be for any given input value $x$. If the domain of the function is the whole real number line (or even just a finite interval of it), that means, we must prescribe (uncountably) infinitely many function values, so the dimensionality of the space of all functions is apparently also (uncountably) infinite. We will also encounter subspaces of this huge space whose dimensionality will be only countably infinite or even just finite.

\subsection{Scalar Products}

\subsection{Norms and Distances}
% L_p-norm, maximum-norm, norms induced by a scalar product, distances (norms of differences)

\subsection{Special Function Spaces}
% Hilbert-/Banach-/Sobolev-spaces, space of all polynomials, space of all analytic functions, periodic functions with some period P, square-integrable functions

\begin{comment}

-what are the function spaces that have a countably infinite dimension? Maybe those functions whose Taylor
 series converges everywhere belong to it
-scalar-product, norm, 
-say something about functions \mathbb{R^M \rightarrow R^N}$ - how does these fit into these scheme
-what about functions defined only on the integers or rationals?
-what about complex functions - in particular, how does the scalar product generalize

\end{comment}