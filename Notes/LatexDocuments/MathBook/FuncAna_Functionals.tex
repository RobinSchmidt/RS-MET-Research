\section{Functionals} 

\subsection{Lebesgue Integration}
In functional analysis, we will encounter a lot of integrals. Sometimes, however, we will have to deal with functions or function-like objects that are not Riemann integrable. A new notion of integrals is desired that will give the same results in the case of Riemann integrable functions but is applicable to a wider range of objects. That's what the Lesbesgue integral does for us. To define the Lebesgue integral, we will first need to define the notion of a measure.

\subsubsection{Measure Theory}

\subsubsection{The Lebesgue Integral}


\subsection{Test Functions}
For the following material, we will need to work with a set of functions with some special properties. We want them to be smooth, which means they should be differentiable infinitely often everywhere. We also want them to have a finite support, which means they should be identically zero everywhere except within some finite interval. The set of all functions that satisfy these two requirements will be called the set of "test functions". [TODO: explain where the name comes from]. At first glance, it may be a bit surprising that functions that satisfy both of these criteria even exist (it certainly suprised me). To make a function identically zero outside a given interval typically requires a piecewise definition and such piecewise definitions tend to make the functions non-smooth at the junctions of the pieces. However, such functions do indeed exist, and a standard example is the so called bump function:
\begin{equation}
 f(x) = ...
\end{equation}
It is indeed infinitely often differentiable everywhere, even at the potentially problematic junctions [TODO: show it]. It is, however, not analytic there: a power series expansion centered at these points will not converge to the function $f$ [TODO: figure out to what it converges, if at all (maybe to the zero-function?)].


\subsection{Distributions}



\begin{comment}

-dual space: set of all linear maps from agivne space into the (real) numbers

https://en.wikipedia.org/wiki/Lebesgue_integration
https://de.wikipedia.org/wiki/Lebesgue-Integral

https://en.wikipedia.org/wiki/Distribution_(mathematics)
https://en.wikipedia.org/wiki/Bump_function
https://en.wikipedia.org/wiki/Mollifier

\end{comment}