\section{Integral Transforms}


\begin{comment}

-Fourier and Laplace Transform
 -can be seen as linear integral operators that take a function to another function
  ...i think, it operates on the space of square-integrable functions and produces square integrable
  functions as output, too?

https://en.wikipedia.org/wiki/List_of_Fourier-related_transforms
https://en.wikipedia.org/wiki/List_of_transforms


https://en.wikipedia.org/wiki/Hartley_transform
https://en.wikipedia.org/wiki/Hilbert_transform
https://en.wikipedia.org/wiki/Mellin_transform
https://en.wikipedia.org/wiki/Gabor_transform

what about wavelet transform? Morlet

eigenfuncs of fourier-trafo:
http://www.systems.caltech.edu/dsp/ppv/papers/journal08post/PPVIETEeigenFT.pdf
https://www.tandfonline.com/doi/abs/10.1080/09747338.2008.11673800?journalCode=tije20




The Fourier Transform on L2 - What they don't tell you
https://www.youtube.com/watch?v=etZy8a32kcc
-Hermite functions are eigenfunctions of Fourier transform (they are an eigenbasis)

See also:
https://en.wikipedia.org/wiki/Hermitian_wavelet
https://en.wikipedia.org/wiki/Hermite_transform
https://en.wikipedia.org/wiki/Hermite_polynomials

I think, the Hermite polynomials can be obtained from successively differentiating 
f(x) = e^(-x^2). See the following SageMath code:

f0 = exp(-x^2)
f1 = diff(f0, x)
f2 = diff(f1, x)
f3 = diff(f2, x)
f4 = diff(f3, x)
f5 = diff(f4, x)
f0, f1, f2, f3, f4, f5

The result is always some polynomial times e^(-x^2). Although, the actualy Hermite polynomials seem
to have different signs in the coefficients.

\end{comment} 