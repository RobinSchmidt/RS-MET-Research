\section{Integral Transforms}

\subsection{The Fourier Transform}

% ToDo: 
% -Definition of the FT 
%  -introduce it as limiting case for the Fourier series when the period grows to infinity
%  -mention the different conventions for the normalization. Maybe use the 
%   symmetric convention here (with 1/sqrt(pi) in front of both integrals)
% -convolution theorem
% -introduce 2D version and maybe variations like sine and cosine trafos

\subsection{The Laplace Transform}

% What 99% of Stanford students don't understand about Laplace theorem
% https://www.youtube.com/watch?v=_ngYWe58toc

% https://www.math.purdue.edu/~caiz/MA527-cai/lectures/Table%20of%20Laplace%20Transforms.pdf

\begin{comment}

-Fourier and Laplace Transform
 -can be seen as linear integral operators that take a function to another function
  ...i think, it operates on the space of square-integrable functions and produces square integrable
  functions as output, too?

https://en.wikipedia.org/wiki/List_of_Fourier-related_transforms
https://en.wikipedia.org/wiki/List_of_transforms


https://en.wikipedia.org/wiki/Hartley_transform
https://en.wikipedia.org/wiki/Hilbert_transform
https://en.wikipedia.org/wiki/Mellin_transform
https://en.wikipedia.org/wiki/Gabor_transform

what about wavelet transform? Morlet

eigenfuncs of fourier-trafo:
http://www.systems.caltech.edu/dsp/ppv/papers/journal08post/PPVIETEeigenFT.pdf
https://www.tandfonline.com/doi/abs/10.1080/09747338.2008.11673800?journalCode=tije20

-Radon transform


The Fourier Transform on L2 - What they don't tell you
https://www.youtube.com/watch?v=etZy8a32kcc
-Hermite functions are eigenfunctions of Fourier transform (they are an eigenbasis)

See also:
https://en.wikipedia.org/wiki/Hermitian_wavelet
https://en.wikipedia.org/wiki/Hermite_transform
https://en.wikipedia.org/wiki/Hermite_polynomials
https://mathphys.uva.es/files/2020/07/CelGadOlmo_0720.pdf  Hermite functions and Fourier series

I think, the Hermite polynomials can be obtained from successively differentiating 
f(x) = e^(-x^2). See the following SageMath code:

f0 = exp(-x^2)
f1 = diff(f0, x)
f2 = diff(f1, x)
f3 = diff(f2, x)
f4 = diff(f3, x)
f5 = diff(f4, x)
f0, f1, f2, f3, f4, f5

The result is always some polynomial times e^(-x^2). Although, the actual Hermite polynomials seem
to have different signs in the coefficients.


Uncovering the Structure of the Fourier Transform: From Theory to MATLAB
https://www.youtube.com/watch?v=j734acXy-VM


Operators take functions as input and produce functions as output. In this sense, they can be considered as 2nd order functions. Does it make sense to consider 3rd order functions? Some object that takes an operator as input and returns another operator? Maybe composing operators with a given fixed operator could be such a 3rd order function? Or if we take the analogy with matrices: matrices can be applied to vectors to produce new vectors - in this sense, they can be regarded as operators on the space of vectors. But matrices can also "act on" other matrices by means of matrix multiplication. So maybe linear higher order operators (which take operators to operators) can always also be considered as normal operators (that take functions to functions)?





\end{comment} 