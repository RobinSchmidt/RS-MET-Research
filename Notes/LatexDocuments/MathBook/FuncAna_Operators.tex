\section{Operators}


\begin{comment}

-linear operators: integral, differential, projection (best approximation from a subspace), multiply by fixed function, shift, rotate?, solutions to ODE as operator? driving-function goes in, solution comes out?
-eigenfunctions and eigenvalues
-adjoint operators - how to compute (see videos/notes/books by Nathan Kutz), something with integration by parts
-solving operator equations, Fredholm alternative
-seems like in inifinite dimensional spaces, linear operators can be unbounded (how so? - find example) and the bounded linear operators are precisely those that give rise to continuous mappings -> figure out, this seems to be an important theme

-Legendre Trafo: inverts a function y(x) = grad(f(x)) into x(y) = grad(g(y)) where f,g are scalar fields and x,y are vectors: g(y) = x(y) \cdot y - f(x(y)), see Ahrens add-on, pg 201, it maps a scalar-field to another scalar field
https://en.wikipedia.org/wiki/Legendre_transformation
https://math.stackexchange.com/questions/4208589/integral-kernel-of-the-legendre-transform

https://en.wikipedia.org/wiki/Operator_(mathematics)

https://en.wikipedia.org/wiki/Operator_algebra


interesting ones from here:
https://en.wikipedia.org/wiki/List_of_mathematic_operators
-geom/arith mean (sort of running mean, i guess - mean, so far)
-logarithmic derivative
-Schwarzian derivative
-Legendre transform
-dual curve?
-evolute, involute
-arc length
-inverse
-reflection: F[y] = y(-x)

-auto correlation? cross correlation with given function

https://en.wikipedia.org/wiki/Weierstrass_transform  (convolution with Gaussian -> smoothing)
...what about mollifiers?
https://en.wikipedia.org/wiki/Mollifier

don't onfuse these:
https://en.wikipedia.org/wiki/Legendre_transform_(integral_transform)
https://en.wikipedia.org/wiki/Legendre_transformation



Example operators:
https://en.wikipedia.org/wiki/Volterra_operator      just the integral
https://en.wikipedia.org/wiki/Composition_operator   
https://en.wikipedia.org/wiki/Sturm%E2%80%93Liouville_theory
https://en.wikipedia.org/wiki/Schwarzian_derivative


\end{comment} 