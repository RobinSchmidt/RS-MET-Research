\chapter{Geometry}
Geometry is the study of shapes that live in a certain (vector) space. That "space" can be the well known 3D space that we happen to live in or just a 2D subspace of it, such as a plane or the surface of a sphere or torus or whatever other 3D object. It can also be a higher dimensional space that we can't easily visualize anymore such as the 4D spacetime that we actually happen to live in. Of special interest are features of these shapes that remain invariant under a certain set of transformations like, for example, the distance between two points or the angle between two intersecting lines which both remain invariant under translations and rotations. These sets of transformations under consideration actually form groups: they can be chained one after another and the combined transformation is another element from the set. 

\medskip
I used terms like "point", "line", "distance", etc. assuming that everyone has an intuitive understanding of what these terms mean. In a more modern, abstract and axiomatic approach, one starts with a bunch of primitive notions and then proceeds to make postulates (i.e. state axioms) about the relationships of these primitives. From these axioms, one then derives theorems via the rules of formal logic as usual. Such a set of primitives, postulates about them and the theorems that follow from them defines "a geometry". In this lingo, you can define different geometries by using different sets of primitives and/or postulates. At this point, the terms for the primitives are not yet supposed to carry any meaning - the words are to be understood as empty shells. Filling in the missing interpretations, i.e. explaining what a "point" or "line" is supposed to mean (typically in a mix of set-theoretic and natural language), creates "a model" for "a geometry". There can be different models for the same geometry and, of course, there can be different geometries.

\medskip
The best known and most basic geometry is the Euclidean geometry of the plane. The postulates of this geometry could be stated als follows: (1) we can always draw a straight line from a point to another point, (2) we can extend any straight line indefinitely, (3) we can draw a circle with given radius and center, (4) all right angles are equal to each other, (5)\footnote{Euclid himself used a different, more complicated 5th axiom. I've given Playfair's parallel axiom here which is an equivalent replacement.} for a given line and a given point which is not on the line, we can find a unique line that goes through the point and is parallel to the given line. The last one is called the parallel axiom and plays a special role. It's the most complicated one, not as immediately obvious as the others, Euclid himself tried to avoid to use it in proofs as much as possible and it is the one that historically led to the first non-Euclidean geometries by replacing it by another postulate. For example, by postulating that there are at least two such lines, you get hyperbolic geometry. By postulating that no such line exists, one may arrive at elliptic or spherical geometry. In the former, the lines cross in one point, in the the latter, in two points. We still haven't said what "point" or "line" means. Euclid himself gave some definitions such as "a point is something that has no parts", "a line is something that has only length but no width", etc. But these Euclidean definitions are not very helpful because they can actually only be correctly understood by someone who already intuitively knows what a point or line is.

% Q: Is Playfair's axiom uncoditionally equivalent to Euclid's 5th axiom or only under the condition of Euclid's other 4 axioms? I guess, it depends on what other axioms are used in the proof of the equivalence. If we can proof Playfair from Euclid's 5th and vice versa without using any of the other 4 axioms, then they would be unconditionally equivalent? It seems like an analogy to linear independence of vectors - we can use 4 fixed ones and then have some freedom to choose the 5th...the axioms are like "linearly independent" basis vectors - but here the "linear" indepencence must be replaced by something like a "proofly" or "theoremly", "theorectically" independence.

% Here we have something similar:
%  https://en.wikipedia.org/wiki/Axiom_of_constructibility
% "The axiom of constructibility implies the axiom of choice (AC), given Zermelo–Fraenkel set theory without the axiom of choice (ZF). "


\medskip
In the late 19th century, David Hilbert proposed a new foundation for geometry. He starts by stating what the primitive notions are while emphasizing that we should not yet assign any meaning to them. These primitives are: points, lines, incidence, betweenness and congruence. "Incidence" is also called containment. Going against Hilbert's intention, I'll now tell you what incidence is supposed to mean: it's meant to model the circumstance that a point lies on a line or that a line passes through a point or that two lines cross. In all these cases, the objects of interest (lines, points) coincide in some way. "Congruence" is meant to model "sameness" in some way. "Points", "lines" and "betweenness" are meant to model exactly what you think they do. Of Hilbert's axioms, there are 20 (or 21, if the parallel axiom is included) so I won't list them all here. If you are interested, you can look them up on wikipedia for example. If the parallel axiom is left out, the resulting geometry is called "absolute" geometry. Absolute geometry contains only those theorems which are true in all geometries that sastisfy all axioms except the parallel axiom or any of its (equivalent or non-equivalent) replacements.
%...tbc...maybe state Hilbert's axioms...
% (1) given two points, there exists exactly one line that contains them, (2) a given line contains at least two points, (3) there exist three points that are not contained in a single line, (4) if point B is between points A and C, then A,B,C are distinct points contained in the same line and B is also between C and A, (5) given two points A,B, there exists a point B that is between them, (6) if three points are contained in a line, then exactly one of them is between the other two ...oh - actually there are a lot more axioms - too many to list them all here - that would be too much of a distraction

\medskip
A bit earlier, a contemporary of Hilbert, Felix Klein, published what became to be known as the Erlangen program. It is a proposal of classifying geometries by means of group theory. One considers a particular group of geometric transformations such as translations, rotations, scaling, shearing, projection, Moebius transforms, etc. One then classifies different geometries according to the concepts that remain invariant under these transformations. For example, if the group of transformations consists only of rotations, translations and reflections then any combination of such transformations will preserve all distances between pairs of points and therefore also all notions that depend on these distances such as lengths, areas, radii, angles, shapes, etc.. Therefore, all these notions are meaningful concepts within the resulting class. This set of transformations is also called the set of rigid transformations or the Euclidean group, denoted $E(n)$ for $n$-dimensional Euclidean spaces. Allowing only rotations and translations but no reflections, also orientation will be preserved and the group is called the special Euclidean group, denoted as $SE(n)$. Adding back reflections and allowing additionally uniform scaling, one gets the group of similarity transformations $S(n)$. There are many more such groups and the Erlangen program investigates the structures of them and their relationships - like whether one is a subgroup of another etc..

\medskip
For the ancient greeks, geometry was all about "constructing" new points from a given set of old points by means of using only a straightedge and a compass. Measurement of lengths, angles, etc. was not allowed. This approach is now sometimes called constructive geometry or synthetic geometry and was largely disconnected from the world of numbers. This is to be constrasted with so called analytic geometry, in which each point is represented by a tuple of coordinates (i.e. numbers!) on which we can perform calculations. The advent of analytic geometry was a great revolution because it allowed for the first time to reach into all the tools of algebra to perform "constructions" of new points. "Measurements", in the form of computations, are now also allowed and become an essential new tool. This revolution was made possible by the invention of the cartesian coordinate system, named after Rene Descartes (although he wasn't the sole inventor). Within the framework of analytic geometry, vectors become the tool of choice to represent points and matrices are used to represent a lot of important transformations that need to be applied to the points. Construction by physical tools has been replaced by pure computation! Yay - that means, we can do it on a computer, too! Analytic geometry is the mathematical idea underlying and enabling all of computer graphics!

\medskip
Besides the distinction between "synthetic" and "analytic" geometry, the various types of geometry ("Euclidean", "hyperbolic", "spherical", etc.), there are also some other adjectives that can be placed in front of the noun "geometry" such as "elementary", "algebraic", "differential", "fractal", "discrete", etc. We'll now take a look at what some of these terms mean in more detail...

\begin{comment}

https://en.wikipedia.org/wiki/Synthetic_geometry
https://en.wikipedia.org/wiki/Straightedge_and_compass_construction

https://en.wikipedia.org/wiki/Rigid_transformation

https://en.wikipedia.org/wiki/Similarity_(geometry)
https://en.wikipedia.org/wiki/Congruence_(geometry)
https://en.wikipedia.org/wiki/Erlangen_program

https://en.wikipedia.org/wiki/Playfair%27s_axiom

What defines "a" geometry? A set of axioms? Or a set/group of transformations? Or a metric? What is a model of a geometry?
I think, "a geometry" is indeed define by a set of axioms and a model for such a geometry is defined by a set of transformations and a metric? To state the axioms, we need a couple of basic concepts, such as "points", "lines", "angles", etc.

-axiomatic approach by Euclid, later by Hilbert
-classification of geometries (Felix Klein's Erlangen program)

https://en.wikipedia.org/wiki/Geometrization_conjecture#The_eight_Thurston_geometries

https://en.wikipedia.org/wiki/Erlangen_program

https://en.wikipedia.org/wiki/Poincar%C3%A9_half-plane_model

https://en.wikipedia.org/wiki/Hyperbolic_geometry#Models_of_the_hyperbolic_plane

https://en.wikipedia.org/wiki/Euclidean_geometry
https://en.wikipedia.org/wiki/Spherical_geometry
https://en.wikipedia.org/wiki/Elliptic_geometry
https://en.wikipedia.org/wiki/Hyperbolic_geometry

https://en.wikipedia.org/wiki/Hilbert%27s_axioms
https://en.wikipedia.org/wiki/Foundations_of_geometry

-synthetic vs. analytic geometry

https://www.youtube.com/watch?v=EmLzMYr6uHU


Maybe it should go into the Applications part (under Art)

Misc Geometry (Friezes, Tilings, Tesselations):

A New Pattern in Nature
https://www.youtube.com/watch?v=2HoUK9kYu4Q  

Der Satz von Bolyai-Gerwien und Hilberts drittes Problem (Weihnachtsvorlesung 2014)
https://www.youtube.com/watch?v=40Mt9WdSNEk 

Friese und ihre Symmetrien (Mai-Vorlesung 2014, Teil 1 von 2)
https://www.youtube.com/watch?v=gMe7yNLWR24

Parkettierungen (Mai-Vorlesung 2014, Teil 2 von 2)
https://www.youtube.com/watch?v=MtK8rBLwwOU

Discovery of the Aperiodic Monotile - Numberphile
https://www.youtube.com/watch?v=_ZS3Oqg1AX0

Finally, a true Aperiodic Monotile!
https://www.youtube.com/watch?v=IfVwelta1fE

The Infinite Pattern That Never Repeats
https://www.youtube.com/watch?v=48sCx-wBs34

How a Hobbyist Solved a 50-Year-Old Math Problem (Einstein Tile)
https://www.youtube.com/watch?v=A1BhOVW8qZU

The Shiny New Shape That Aperiodically Tessellates!
https://www.youtube.com/watch?v=sLQrHz7CQf4

How a Hobbyist Created An Infinite Pattern That Never Repeats
https://www.youtube.com/watch?v=tbZG7EPzLFg

Hexagons are the Bestagons
https://www.youtube.com/watch?v=thOifuHs6eY

Heesch Numbers and Tiling - Numberphile
https://www.youtube.com/watch?v=6aFcgATW9Mw

Mathematicians Just Discovered These Shapes!
https://www.youtube.com/watch?v=mB8h0-pxEgE

5 and Penrose Tiling - Numberphile
https://www.youtube.com/watch?v=QTrM-UVcgBY

The ALMOST Platonic Solids
https://www.youtube.com/watch?v=_QxrkEqOrWM

there are 48 regular polyhedra
https://www.youtube.com/watch?v=_hjRvZYkAgA

FINALLY! A Good Visualization of Higher Dimensions
https://www.youtube.com/watch?v=sZqGWy0hxe8&t=235s


Fünf auf einen Schlag
https://www.youtube.com/watch?v=lhxDyASxbLE
-geometrische Flächenberechnungen



Other artistic applications:
-Interference patterns can be beautiful - especially when animated

\end{comment}