\section{Algebraic Geometry}
Algebraic geometry investigates the sets of zeros of multivariate polynomials and the geometrical shapes that are defined by these zero sets. Consider as example the bivariate polynomial $f(x,y) = x^2 + y^2 - r^2$ for some given constant $r$. Those points $(x,y)$ in the $xy$-plane for which this function produces zero as output define a geometric shape - in this case, a circle of radius $r$ centered at the origin. You can have multiple simultaneous equations that must be equal to zero and of course, you can also have more than two input variables. It's also possible to consider complex solutions, in fact, that's the more typical setting. In general, these sets of solutions to polynomial equations are called "algebraic varieties" and these are the main subject of interest in algebraic geometry.

\medskip
It is also possible to work in the field of rational numbers $\mathbb{Q}$ rather than in the field of complex numbers $\mathbb{C}$ or even in finite fields such as $\mathbb{Z}_p$ for a given prime number $p$. ...TBC..

\medskip
Fun fact: algebraic geometry and geometric algebra are two fields of mathematics that have nothing to do with each other. Isn't that weird? I really wonder, how connections between these fields could look like. Maybe we could start by interpreting the variables in the equations as multivectors instead of real or complex numbers? I have no idea but I think there just \emph{should} be some connection!




%===================================================================================================
\subsection{Algebraic Varieties}
An \emph{algebraic variety} is the set of solutions to a system of polynomial equations $f_1(\mathbf{x}) = 0, \ldots, f_m(\mathbf{x}) = 0$ where $\mathbf{x} = (x_1, \ldots, x_n)$ and the $f_i$ are multivariate polynomials, i.e. polynomials with $n$ input variables that we may collect in an input vector $\mathbf{x}$ whenever this is notationally convenient. In general, this set of solutions is a set of points in an $n$-dimensional space. The dimensionality of the variety will typically be given by $n-m$ where $n$ is the number of variables and $m$ the number of equations [VERIFY!]. ...TBC...

%---------------------------------------------------------------------------------------------------
\subsubsection{Parameterizations}
Defining a shape as the solution set of a system of equations is also called an \emph{implicit definition}. Sometimes it is more convenient to have a more explicit representation of the set of points. A fully explicit representation would be, if we could isolate one variable and express it explicitly as function of all the others. But requiring such a functional representation is often too restrictive - there are some shapes that simply cannot represented that way. For example, we could try to represent the circle explicitly as $y = \sqrt{r^2 - x^2}$ but that would give us only the upper half the circle unless we interpret the square-root as multi-valued function, which is already inconvenient in this very simple toy example and in more complicated cases, things may get out of hand even more. If a fully explicit representation is not possible or practical, the second best thing is to let all the coordinates be functions of a set of some auxiliary variables which we call parameters. In the case of a circle, that could look like $x(t) = \cos(t), y(t) = \sin(t)$ for $t \in [0, 2 \pi]$. In such a representation we call $t$ the parameter.
...TBC...

\paragraph{Finding Rational Parameterizations}
The parameterization of the circle via the cosine and sine functions given above is not the only possible way to parameterize the circle. It will probably be the most common one in most practical contexts (such as computer graphics) but from an algebraic point of view, it has a flaw: it requires usage of transcendental functions. Wouldn't it be much nicer in an algebraic context if we could use purely algebraic functions to parameterize it? We can and in fact, we need only rational functions - not even roots are needed. To find a rational parameterization of an implicitly given curve, we sometimes can proceed as follows:

...TBC...

\paragraph{From Parameterization to Equation}
Now we look at the inverse problem: We assume that we have a rational parameterization given and we ask what the corresponding implicit equation is. ...TBC...

% circle: ACRS, pg 2, general: 68 - has also algorithm for the inverse problem

% important questions: find a (rational) parameterization, is the zero set a manifold? what topological invariants does it have?

% Bezout's Theorem


% https://en.wikipedia.org/wiki/Algebraic_variety
% https://mathworld.wolfram.com/AlgebraicVariety.html



%---------------------------------------------------------------------------------------------------
\subsubsection{Examples}

\paragraph{Circles, Spheres and Hyperspheres}
As a simple example, consider 2D space and the single equation $x^2 + y^2 - 25 = 0$. This equation defines a circle of radius $5$ centered at the origin of the $xy$-plane. If we look at 3D space, we could define a sphere of radius $r$ via the polynomial equation in 3 variables $x^2 + y^2 + z^2 - r^2 = 0$. Note that the set of points that satisfies this equation is only the surface of the sphere, not the full ball. So, the dimensionality of the set of points that satisfies this equation is 2, not 3. 

\paragraph{Conic Sections}
A circle is defined by a bivariate polynomial of degree 2, i.e. a quadratic polynomial. The circle is therefore also called a curve of degree 2 or quadratic curve. Besides the circle, all conic sections, i.e. sets of solutions of equations of the form $0 = A x^2 + B y^2 + C x y + D x + E y + F$, fall into this category. They are called conic sections because these shapes also arise from intersecting a cone with a plane.

\paragraph{Elliptic Curves}
The set of solutions of $y^2 = x^3 + a x + b$ for some given values $a,b$ is called an "elliptic curve". These are not to be confused with actual ellipses, which are special cases of conic sections. Rather, elliptic curves represent a canonical form of the generalization of conic sections up to degree 3 (verify!).



\paragraph{Torus}

%simple examples: cassini curves, lemniskate, torus

% AVRS, pg 72: 
% -Nodal cubic curve: y^2 = x^2 + x^3
% -Folium of Descartes: x^3 + y^3 = 3 x y

% https://en.wikipedia.org/wiki/Lemniscate
% https://de.wikipedia.org/wiki/Lemniskate

% -Give implicit equation and parametrization(s) whereever possible and if possible also an explicit
%  formula




%---------------------------------------------------------------------------------------------------
\subsubsection{Problematic Intersection Points}
Consider the situation where we have two polynomial equations $f(x,y) = 0$ and $g(x,y) = 0$ in 2D and we want to find all the intersection points of the two curves that are defined by the two polynomials respectively. Such intersection points can be found by solving the equation $f(x,y) = g(x,y)$ for pairs $(x,y)$ that satisfy the equation. For simplicity, let's assume that $g(x,y) = ax + by + c$, i.e. $g$ is only of degree one such that defines a straight line. In an ideal world, we would like to see exactly $n$ intersection points between the curve $f$ and the line $g$ if the polynomial $f$ has degree $n$. We can indeed make that (and more) work by tweaking the problem setting appropriately whenever we encounter a problematic situation in our current setting.

\paragraph{Intersections at Complex Coordinates}
At first, we assume that our "current setting" takes place in the usual Euclidean $xy$-plane which we also call $\mathbb{R}^2$. If we take $f(x,y) = x^2 + y^2 + 1$ and $g(x,y) = 0$, we can easily convince ourselves that no pair $(x,y)$ will ever satisfy the equation $x^2 + y^2 + 1 = 0$. Except when we allow $x$ and $y$ to be complex valued. In that case $x = 0, y = \i$ would work, for example. So, our first tweak to the problem setting setting is that we move it from $\mathbb{R}^2$ to $\mathbb{C}^2$. This will not solve all of our problems but it gets us some way into the right direction.

% -solution: use the field of complex numbers

\paragraph{Intersections at Infinity}
A second problem that we may run into is exemplified by the following situation: let $f(x,y) = 2 x - y + 1$ and $g(x,y) = 2 x - y + 2$. These define two parallel lines that obviously intersect nowhere - not even when $x$ and $y$ are allowed to be complex. The solution to this problem is, informally, that we "add points at infinity" and say that two parallel lines intersect at such a point at infinity ...TBC...

% -solution: consider also intersections at infinity

\paragraph{Multiplicities of Intersections}
Consider $f(x,y) = x^2 - y$ and $g(x,y) = y$. If we draw the curves defined by setting each equation individually to zero, we will draw the parabola $y = x^2$ for $f$ and the horizontal line $y = 0$, i.e. the $x$-axis, for $g$. These two curves do not intersect but rather touch at $(x,y) = (0,0)$. The straight line $g$ is tangential to the parabola $f$ at this point. Such points of tangency highlight yet another problem that we might run into.
 ...TBC...

% perturb g a bit - shift it a bit up - the point of tangency turns into two points of intersection. In general, in teh complex domain, we will see higher order saddles so we will get more intersection points by such a perturbation when there is a higher "order of tangency" (verify!)

% -tangency, tangetial intersections or rather touching points
% -solution: count intersections with multiplicity

%===================================================================================================
\subsection{Projective Space}

\subsubsection{Homogeneous Coordinates}
What we informally and vaguely called "adding points at infinity" will now be fleshed out for real. One way to "add points at infinity" to a given space is to switch to so called homogeneous coordinates. To get a feeling for how that works, let's start with our usual Euclidean plane $\mathbb{R}^2$. We'll actually eventually apply the idea to $\mathbb{C}^n$, but imagining to start with $\mathbb{R}^2$ is okay for intuition building and this intuition happens to be also practically relevant for computer graphics. The points in our original space $\mathbb{R}^2$ are given as pairs $(x,y)$. What we now do is to add an extra coordinate which we call $w$ such that our points are now given as triples $(x, y, w)$. If $w \neq 0$, then these triples will encode the point $(x/w, y/w)$ in our original space. If $w=0$, they kind of still do represent $(x/w, y/w)$ but since $w = 0$, these points will be "at infinity" - and it's not just any infinity but an infinity that lies ahead in a specific direction, namely into the direction of the vector $(x,y)$. In some sense, we have added directional infinities to our space. We can envision these points as lying on a perimeter of a circle with infinite radius such that this perimeter is infinitely far away. The fact that we encode a 2D vector $(x,y)$ by a triple $(x,y,w)$ by saying that we get our 2D vector back via $(x/w, y/w)$ implies that the "encoding" of a given finite $(x,y)$ vector is not unique. For example, the triple $(15,12,3)$ encodes the same 2D vector as the triple $(5,4,1)$, namely the 2D vector $(5,4)$. It turns out that for $w \neq 0$, we can always normalize our triple in such a way that $w=1$: We simply divide all 3 components by whatever $w$ currently is.
...TBC...

% https://en.wikipedia.org/wiki/Homogeneous_coordinates

% Explain why the space is called "projective". When w != 0, we may normalize the triple wo w=1. We can interpret the w=1 plane as a slice of xyw-space and if we cut out this slice, we get our original xy-plane back

\subsubsection{Homogeneous Polynomials}
When we are working with homogeneous coordinates, we give the space one extra coordinate. We use $(n+1)$ dimensional space but we still want to use polynomials to describe shapes in $n$-dimensional space by way of zero-sets of polynomial equations. That puts some constraints on the kinds of polynomials that we can use for this purpose. In 2D, if $(x : y : 1)$ is a point such that $p(x,y,1) = 0$, we'd better make sure that this occurs if and only if $p(x/w, y/w, w) = 0$ for any $w$ we may choose. Otherwise the shape would not be well defined. The way we ensure this is called homgenization. We take our original polynomial in $x,y$ and multiply each monomial in it by an appropriate power of $w$ such that each monomial has the same total degree. Being homogeneous is indeed the property that we need to satisfy our constraint, because if a multivariate polynomial $f(\mathbf{x})$ is homogeneous of degree $n$, then we have $f(k \mathbf{x}) = k^n f(\mathbf{x})$ for any input multiplier $k$, so in particular, $f(k \mathbf{x}) = 0$ whenever $f(\mathbf{x}) = 0$ [VERIFY!].

...TBC...

%If a polyno

%===================================================================================================
\subsection{Elliptic Curves}

\subsubsection{Elliptic Curves} 
% asterisk means: can be skipped on first reading
Here, we will consider a certain subset of bivariate polynomial equations that turned out to be in the sweet spot between being enough to actually tackle them and complex enough to allow for some interesting features. I'm talking about the so called elliptic curves. But didn't we already talk about ellipses as one particular type of conic section? Yeah - an \emph{elliptic curve} is actually \emph{not an ellipse} but rather the set of solutions of a bivariate polynomial equation of degree 3. ToDo: explain etymology if the term

\paragraph{$\star$ Group Structure} % star means: can be skipped on first reading
The set of rational points on such an elliptic has an interesting property: together with a suitably defined binary operation between these points, they form an algebraic structure known as commutative group. That basically means that we can plug two rational points into our operation and the operation will produce another rational point on the curve. In the context of elliptic curve theory, this operation is usually denoted  with $+$ and called addition - but it is \emph{not} just the regular vector addition. ...TBC...
% see Elliptic Tales, pg 116
% mention that i works also when we don't work over Q but over a finite field
% The points on the elliptic curve




\subsection{$\star$ The Abstract Perspective}
This section can be skipped on a first reading of the book. It references rather advanced ideas that are properly introduced only (much) later - specifically \emph{ring theory} from abstract algebra.




\subsubsection{Polynomial Rings}
Algebraic geometry, when viewed as the business of trying to find the set of solutions of multivariate polynomial equations, is actually about quite intuitive ideas. However, when you open a textbook on algebraic geometry written for real mathematicians, you will discover that it will talk about very abstract ideas that sound like they have nothing to do with geometry. In the abstract algebra perspective of algebraic geometry, we consider quotient rings of polynomial rings. For example, in 2D, when we have a bivariate polynomial like $p(x,y) = x^2 + y^2 - 1$, we will consider the ring $R = \mathbb{R}[x,y] \setminus (x^2 + y^2 - 1)$ and analyze its properties with the arcane machinery of ring theory. ...TBC...

% Is this the ring of all polynomials that are zero on p? Or is it the ring of all polynomials
% but we are allowed only to plug in values for which p(x,y) = 0 i.e. only points on the unit
% circle?
% Our ring $R$ is the ring of all polynomials $q(x,y)$ that

% What is algebraic geometry?
% https://www.youtube.com/watch?v=MflpyJwhMhQ
%
% -Maybe this chapter should have a last section: Ring Theoretic Perspective or similar which 
%  can be skipped on a first reading
%
% -Maybe bring a 1D example: take a polynomial like p(x) = (x+2)*(x-3) = x^2 - x - 6 with 
%  roots at {-2,3}
% -Consider the ring R[x] \ p(x)
% -It is the ring of all polynomils that has zeros at -2,3. The set of all these polynomials
%  is considered to form an eqivalence class.
% -Any polynomial that is zero at -2 and 3 must have p(x) as factor, so any polynomial with p
%  as factor is considered to be equivalent...right...or wrong?

%
% -Or is it like we we consider the ring of all polynomials but we allow only to plug in 
%  points (x,y) for which the polynomial is zero?



\begin{comment}



https://de.wikipedia.org/wiki/Algebraische_Geometrie
https://de.wikipedia.org/wiki/Algebraische_Variet%C3%A4t
https://de.wikipedia.org/wiki/Hilbertscher_Basissatz
https://de.wikipedia.org/wiki/Gr%C3%B6bnerbasis

https://en.wikipedia.org/wiki/Algebraic_geometry
https://en.wikipedia.org/wiki/Algebraic_variety

https://en.wikipedia.org/wiki/Zero_of_a_function#Zero_set
https://en.wikipedia.org/wiki/Projective_plane


https://en.wikipedia.org/wiki/Elliptic_curve
https://en.wikipedia.org/wiki/Algebraic_curve
https://en.wikipedia.org/wiki/Weierstrass_elliptic_function


https://en.wikipedia.org/wiki/K3_surface
https://en.wikipedia.org/wiki/Calabi%E2%80%93Yau_manifold

-conic sections in 2D and 3D, classification
-elliptic curves
-lemniskate, contour plots
-polynomials with integer/rational coeffs -> integer and rational solutions -> elliptic curve cryptography
-what about allowing more general classes of functions?
-what about multiple simultaneous equations - what does that imply for the dimensionality of the solution set?
-what are important features of the solution sets? connectedness?

-seems like differential geometry deals with parameteric representations and algebraic geometry with implicit representations of geometric shapes (manifolds?)
-is there a way to convert between these representations? ..yeah...this seems to be actually a difficult research area

Tikz:
https://texample.net/tikz/examples/feature/angles/
https://texample.net/tikz/examples/feature/foreach/

https://pbelmans.ncag.info/blog/2010/11/11/howto-draw-algebraic-curves-using-pgftikz/

-write a c++ function that takes in an implicit function f(x,y) = 0 and spits out a vector of vectors of 2D points (the outer vector is because we may have multiple curves), this can then be converted into svg or tikz code (via suitable code-generators that take as input the coordinate vectors)
-with this code, generate tikz figures for elliptic curves, maybe x^2 + y^3 + k = 0 for different values of k

Fun fact

Wie muss man fragen, um "schöne" Antworten zu bekommen? (Der Satz von Bezout):
https://www.youtube.com/watch?v=dyehHqozzqE




\end{comment}