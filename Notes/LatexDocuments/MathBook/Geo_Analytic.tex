\section{Analytic Geometry}
Analytic geometry is about representing the positions of points within space by numbers and doing calculations with these numbers. These numbers are called "coordinates" and are to be understood with respect to a given coordinate system which has to be chosen beforehand. From points, more complex objects like lines, planes, shapes, etc. can be made up and we are interested in computing things like the distance of a point from a line or plane etc.. The term "analytic" is, in my humble opinion, a bit misleading here because "analysis" is usually taken to be a more rigorous version of calculus but in analytic geometry, we typically do more basic vector- and matrix-algebra stuff and the more calculusy things are usually tackled later in "differential" geometry. However, "analytic" geometry, just like "elementary" geometry, is also mostly a bunch of formulas that are useful for creating more complex geometric algorithms. The main distinctive feature is that these formulas will typically involve representations of our shapes in terms of vectors of coordinates. This is actually the more typical representation we use in a computer, at least in the context of graphics but also for physical simulations. 

%===================================================================================================
\subsection{The $2D$ Euclidean Plane $\mathbb{R}^2$}
The simplest setting to do analytic geometry in is the $2$-dimensional plane also known as the Euclidean plane which we denote as $\mathbb{R}^2$ due to the fact that we use two real numbers to specify a location in this space. This is the place where Euclid himself developed all the geometric notions that we learn today in school. The objects we will deal with are points, lines, curves, triangles and polygons. The quantities we are interested in are distances, angles and areas. ...TBC...

%---------------------------------------------------------------------------------------------------
\subsubsection{Points, Vectors and Vertices}
Our eventual goal is to describe geometric shapes and calculate certain features of them. The most primitive object that we need to build our representations of shapes from is a \emph{point}. A point is just a location in a space - in this case, a location in the $2D$ plane. When we build polygons from points, we will also call these points \emph{vertices} (singular: \emph{vertex}) and we may use \emph{vectors} to represent these points. Conceptually, a point is actually not the same thing as a vector. A point is just a location in space and it doesn't seem to be a meaningful idea to add two locations in space together, for example. A vector, on the other hand, is meant to encode a length and direction (of movement, for example) and it does indeed make geometric sense to add a vector to a point to get another point. Vectors are things that we can freely add together anyway and the result will always be another vector. Such an addition does make geometric sense as well: we first do one translation, then the other - the sum of the two vectors encodes to total translation. Even though it doesn't make sense to \emph{add} two points, it does make sense to \emph{subtract} two points - and the result will be the vector that tells us, how to move from one point to the other. This seems all very messy. In practice, however, the distinction between points and vectors is rarely made and points are just identified with their position vectors, i.e. vectors that go from the origin of the coordinate system to the point. So, in this setup, the terms point, vector and vertex kinda all mean the same basic mathematical object, namely a vector, but the usage of different terms can be viewed as a hint to what specific role a given vector currently plays. If we say vertex, the role is typically to be a corner of a polygon. If we say point, the role is typically a general location. If we say vector, the role may be a translation or displacement. But whatever the role is, the representation will always be a pair of real numbers, i.e. a $2D$ vector.

...TBC...TODO: introduce different notations for points and vectors.

%In this context, the points are also often called "vertices" and they can be represented by vectors. 

%---------------------------------------------------------------------------------------------------
\subsubsection{Lines}
After points (or vectors or vertices), the next object we need is the \emph{line}. A line can be defined in various ways. Perhaps the most obvious way is to use two points because given two distinct points, there is always exactly one line that passes through both of them. It is noteworthy that by line, we mean the infinitely extended line that goes through the two points. If we mean just the segment of the lines in between the two points, we call it a \emph{line segment}. In the great scheme of things, lines can be viewed as $1D$ spaces in their own right - and points would be the edge case of $0D$ spaces. Let's start with two points represented via their position vectors $\mathbf{p}_1, \mathbf{p}_2$ and their difference vector $\mathbf{v}$ which encodes how to move from $\mathbf{p}_1$ to $\mathbf{p}_2$:
\begin{equation}
\mathbf{p}_1 = \begin{pmatrix} x_1 \\ y_1 \end{pmatrix}, \quad
\mathbf{p}_2 = \begin{pmatrix} x_2 \\ y_2 \end{pmatrix}, \qquad
\mathbf{v}   = \mathbf{p}_2 - \mathbf{p}_1 
             = \begin{pmatrix} x_2 - x_1 \\ y_2 - y_1 \end{pmatrix}
             = \begin{pmatrix} v_x \\ v_y \end{pmatrix}
\end{equation}
In this case, $\mathbf{p}_1$ and $\mathbf{p}_2$ are really playing the role of points in the sense that they encode locations whereas $\mathbf{v}$ is a proper vector in the narrower sense of encoding a displacement. The situation is depicted below.

\begin{tikzpicture}
[thick, >=stealth', 
  dot/.style = { draw, fill = black, circle, inner sep = 0pt, minimum size = 4pt }
]

% Coordinate system:
\draw[->] (-5, 0) -- (8,0) coordinate[label = {below:$x$}] (xmax);
\draw[->] ( 0,-1) -- (0,5) coordinate[label = {left:$y$}]  (ymax);

% Line:  
\draw     (-4,-1) -- (6,4)  node[pos=0.85, below right] {};

% Points 1p, p2 on the line:  
\draw (2,2) node[dot, label = {above:$\mathbf{p}_1$}]{};
\draw (4,3) node[dot, label = {above:$\mathbf{p}_2$}]{};

% Vector v and its components:
\draw[red, <->] (2,2) -- (4,2);
\draw[red] (3,2) node[below] {$v_x$};
\draw[red, <->] (4,2) -- (4,3);
\draw[red] (4,2.5) node[right] {$v_y$};
\draw[red, ->] (2,2) -- (4,3);
\draw[red] (3,3) node[below] {$\mathbf{v}$};

% Marks on the x- and y-axes:
\draw (2,0) node[dot, label = {below:$x_1$}]{};  
\draw (4,0) node[dot, label = {below:$x_2$}]{};
\draw (0,2) node[dot, label = {left:$y_1$}]{};
\draw (0,3) node[dot, label = {left:$y_2$}]{};
% ToDo: try to use tick-marks instead - or just drop horizontal and vertical lines
% Maybe draw a grid, put centered (maybe wrap into a figure)

% x,y intercepts:
\draw (-2,0) node[dot, label = {below:$a$}]{};  
\draw (0,1) node[dot, label = {left:$b$}]{}; 

% angle alpha:
\draw (-1,0) arc (0:27.5:1);
\draw (-1.2,0.2) node[label = {center:$\alpha$}]{};

% Can be used as example for drawing the tick marks on the axes:
%\draw [shift={(1,0)}, color=black] (0pt,2pt) -- (0pt,-2pt) node [below] {$x_1$};
% the shift determines the position

\node[align=left] at (10,3.0) 
{
\begin{tabular}{l l}
  $m = \frac{y_2-y_2}{x_2-x_1} = \frac{v_y}{v_x} = -\frac{b}{a}$ & Slope \\ 
  $\tan(\alpha) = m $                          & Angle\\  
  $\mathbf{p}(t) = \mathbf{p}_1 + t (\mathbf{p}_2 -\mathbf{p}_1)$ & Parametric form \\
  $y = m x + b$ & Explicit form \\
  $x/a + y/b = 1$ & Intercept form \\

\end{tabular}
};
\end{tikzpicture}
% draw in v (maybe in blue), maybe also its x- and y- components
% maybe draw in the normal vector n

% adapted from:
% https://texample.net/tikz/examples/linear-regression/

% see also:
% https://texample.net/tikz/examples/upper-riemann-sum/

..TBC..give different forms of a line and formulas to convert between them, formulas for angles, distance point-line, projection of points onto lines, ...

% -explicit form: not applicable to vertical lines (m would have to be infinite)
% -intercept form: not applicable to horizontal or vertical lines (because a or b would
%  have to be infinite)
% -talk about the degrees of freedom - there are 3. So, the 2-point form contains more 
%  information than we need (which makes sense, because it actually encodes a line *segment*)

% points can be seen as 0D objects, lines as 1D

% Leupold, pg 174-176:
% 3.44:  m = tan(alpha) = (y2-y1)/(x2-x1)     Anstiegswinkel
% 3.45:  r = r1 + t*(r2 - r1)                 Parameterdarstellung
% 3.46:  r = r1 + t * u                       u = r2-r1
% 3.47:  (y-y1)/(x-x1) = (y2-y1)/(x2-x1)      Zweipunkteform, (2-point form?)
% 3.48:  y-y1 = m*(x-x1)                      Punktrichtungsform
% 3.49:  y = m*x + b                          Normalform, (explicit?)
% 3.50:  x/a + y/b = 1                        Abschnittsform
% 3.51:  Ax + By + C = 0                      allgemeine Form, (implicit? kinda like Hesse
%                                             normal form of planes?)
%
% m: slope, a: x-intercept, b: y-intercept, r1 = (x1,y1), r2 = (x2,y2)

% To compute the intercepts a,b, use the parametric form in coordinate form:
%   x(t) = x_1 + t (x_2 - x_1)
%   y(t) = y_1 + t (y_2 - y_1)
% To find b: find t such that x(t) = 0, plug it into y(t). That gives: b = y_1 - x_1 * m
% To find a: find t such that y(t) = 0, plug it into x(t). That gives: a = x_1 - y_1 / m

% Leupold

%---------------------------------------------------------------------------------------------------
\subsubsection{Curves}

%---------------------------------------------------------------------------------------------------
\subsubsection{Triangles}


%---------------------------------------------------------------------------------------------------
\subsubsection{Polygons}


%---------------------------------------------------------------------------------------------------
\subsubsection{Conic Sections}

\paragraph{Circles and Ellipses}

\paragraph{Hyperbolas}

\paragraph{Parabolas}




%ToDo: representation as 2D vectors, points, lines, formulas for: distance between points, between point and line, angle between lines, ...


%===================================================================================================
\subsection{The $3D$ Euclidean Space $\mathbb{R}^3$}
When we up our game by one dimension from the Euclidean plane, we arrive at the $3$-dimensional Euclidean space. Points, lines, triangles and curves are still a thing - but they will now exist in $3$ rather than $2$ dimensions. But we also get a new type object to deal with: arbitrary planes. In the $2D$ case, there was \emph{the} plane which was the backdrop for all the objects. Now planes themselves are objects within the larger $3D$ space and they can have any orientation within that larger space. We even get curves surfaces which was also not a thing in $\mathbb{R}^2$. Polygons can still make sense in $\mathbb{R}^3$, but we need to be a bit more careful because it's not guaranteed anymore that all the vertices of a polygon lie in the same plane. This was not an issue in $2D$ space. ...TBC...

%ToDo: representation as 3D vectors, points, lines, planes, formulas for: distance between point and line, point and plane, intersection point of line with plane, intersection line of 2 planes, intersection point of 3 planes, angle between plane and line, plane and plane, ...


%===================================================================================================
\subsection{The $nD$ Euclidean Space $\mathbb{R}^n$}
Having seen the Euclidean spaces of dimension $2$ and $3$, it's now time to generalize to Euclidean spaces of arbitrary dimension. ...TBC...

%\subsection{The Euclidean }

%---------------------------------------------------------------------------------------------------
\subsubsection{Lines}
Given a point $\mathbf{p}$ and a nonzero vector $\mathbf{v}$, both elements of $\mathbb{R}^n$, we can consider the following set of points:
\begin{equation}
 L 
 =
 \{\mathbf{p} + \mathbb{R} \mathbf{v} \} 
 = 
 \{ \mathbf{p} + t \mathbf{v}, \; t \in \mathbb{R} \}
\end{equation}
This set of points represents a line in $\mathbb{R}^n$. The expression in the middle is just a shorthand notation for the expression on the right. Multiplying the vector $\mathbf{v}$ by the set of real numbers $\mathbb{R}$ is meant in the sense that we take $\mathbf{v}$ and multiply it by every possible real number $t \in \mathbb{R}$. This line $L$ goes through the point $\mathbf{p}$ and has a direction of $\mathbf{v}$. The line is given in the so called \emph{parametric form}\footnote{Well, I think, for a parametric form, one would more commonly write out a separate equation for each coordinate if $n$ is known and small enough like $2$ or $3$. For general unknown $n$, this vector form can be seen as a more abstract way of writing down a parametric form.} with the parameter $t$. It's also called the \emph{vector form}. We can imagine that, as $t$ sweeps through the real numbers, a point sweeps along the line. For each value of $t$, we obtain some point on the line by just plugging the value $t$ into the formula. With $t=0$, we are at $\mathbf{p}$. With $t=1$, we are at $\mathbf{p + v}$. With $t=\frac{1}{2}$, we are at $\mathbf{p} + \frac{1}{2} \mathbf{v}$ which is exactly halfway between $\mathbf{p}$ and $\mathbf{p + v}$. And so on. The vector form is not the only way to define a line. One can also use two (distinct) points $\mathbf{p}_1,\mathbf{p}_2$ on the line and two real numbers $t_1, t_2$ that satisfy $t_1 + t_2 = 1$ and express the line as the set of points:
\begin{equation}
 L = \{ t_1 \mathbf{p}_1 + t_2 \mathbf{p}_2,  \quad t_1, t_2 \in \mathbb{R}, t_1 + t_2 = 1 \}
\end{equation}
This is called a \emph{two point form}. It may also be called a representation in terms of \emph{barycentric coordinates}, although that terminology is less commonly used for lines - it's more commonly used for planes that are defined by 3 points or $n$-dimensional hyperplanes that are defined by $n+1$ points, but it is the proper $1D$ analog. You may wonder why we introduce the variable $t_2$ and don't just use $1 - t_1$ instead (and then drop the index altogether). The reason is that this would work only for the 1D case, i.e. the line, but writing it like above will readily generalize to barycentric representations of higher dimensional objects. Yet another way to define a line is by means of an implicit equation of the form...TBC....TODO: give other forms of a line (implicit, explicit, etc.), mention that it can be generalized from $\mathbb{R}^n$ to vector spaces over any field - the complex numbers are interesting in the context of algebraic geometry, give the ways to convert between the different forms.

% Two lines are parallel, when they have parallel direction vectors. Two vectors are parallel when one is a scalar multiple of the other

% https://en.wikipedia.org/wiki/Line_(geometry)#Definition
% https://en.wikipedia.org/wiki/Barycentric_coordinate_system

% https://math.stackexchange.com/questions/2931233/the-vector-form-and-parametric-forms-of-a-line-given-a-point-and-direction-or

%---------------------------------------------------------------------------------------------------
\subsubsection{Planes}
Given a point $\mathbf{p}$ and two linearly independent\footnote{Linear independence implies that both must be nonzero.} vectors $\mathbf{v, w}$, consider the following set of points:
\begin{equation}
 \mathbf{p} + \mathbb{R} \mathbf{v}  + \mathbb{R} \mathbf{w}
 \quad = \quad
 \mathbf{p} + s \mathbf{v} + t \mathbf{w}, \; s,t \in \mathbb{R}
\end{equation}
This set of points represents a plane in $\mathbb{R}^n$. 

%---------------------------------------------------------------------------------------------------
\subsubsection{Hyperplanes}
After having seen lines and planes, the pattern continues. We can take some fixed point $\mathbf{p}$ together with a set of $k$ linearly independent vectors $\mathbf{v}_1, \mathbf{v}_2, \ldots, \mathbf{v}_k$ and consider the set:
\begin{equation}
 \mathbf{p} + \sum_{i=0}^{k} t_i \mathbf{v}_i, \; t_i \in \mathbb{R}
\end{equation}
This set defines a $k$-dimensional hyperplane in $\mathbb{R}^n$ where $k < n$. Well, we could perhaps allow $k=n$ as edge case as well, but then we would just get the whole space $\mathbb{R}^n$ and that's not interesting. We are interested in substructures of $\mathbb{R}^n$. But $k > n$ does really not make any sense at all, not even as edge case, because then we can't even find a set of $k$ linearly independent vectors anymore.

% https://web.evanchen.cc/handouts/bary/bary-short.pdf

%---------------------------------------------------------------------------------------------------
%\subsection{Quadrics}

% https://en.wikipedia.org/wiki/Quadric_(algebraic_geometry)
% Maybe do it in algebraic geometry


% Give equation of a cone and explain why conic sections are called like that. It's because the
% curves arise as intersection between a cone and a plane. I think, the equation is z^2 = x^2 + y^2
% but I'm not sure about that.







\begin{comment}


Geometrische Interpretation linearer Abbildungen
https://www.youtube.com/watch?v=EQ5Xct2YyLk


Weitz: Analytische Geometrie
https://www.youtube.com/watch?v=7rkkHIjHtNI
https://www.youtube.com/watch?v=7rkkHIjHtNI&list=PLb0zKSynM2PBYzz6l37rWH3B_n_7P40QP

\end{comment}