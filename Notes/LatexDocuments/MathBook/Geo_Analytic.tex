\section{Analytic Geometry}
Analytic geometry is about representing the positions of points within the 2D plane or 3D space $n$D by numbers and doing calculations with these numbers. These numbers are called \emph{coordinates} and are to be understood with respect to a given coordinate system which has to be chosen beforehand. From points, more complex objects like lines, planes, shapes, etc. can be made up and we are interested in computing things like the distance of a point from a line or plane etc.. The term "analytic" is, in my humble opinion, a bit misleading here because "analysis" is usually taken to be a more rigorous version of calculus but in analytic geometry, we typically do more basic vector- and matrix-algebra stuff and the more calculusy things are usually tackled later in "differential" geometry. However, "analytic" geometry, just like "elementary" geometry, is also mostly a bunch of formulas that are useful for creating more complex geometric algorithms. The main distinctive feature is that these formulas will typically involve representations of our shapes in terms of vectors of coordinates. This is actually the more typical representation we use in a computer, at least in the context of graphics but also for physical simulations. There are 3 common types of formulaic representations of geometric objects: \emph{explicit}, \emph{implicit} and \emph{parametric}. An explicit representation means that one coordinate is expressed as function of the others - for example in 3D as $z = f(x,y)$. This would describe a surface above the $xy$-plane where the function $f$ gives the height of the surface above the given $xy$-point in the plane. Envision the function $f$ as describing a terrain above the $xy$-plane. Such an explicit form is only of limited applicability, though because it is possible only for certain shapes. Implicit and parametric representations are universally applicable. An implicit representation is just an equation in all coordinates of the form $F(x,y,z) = 0$ or a system of such equations that must be satisfied simultaneously. In general, the dimensionality of the represented object will be that of the space minus the number of equations [VERIFY!]. For example, with one equation in 3D, we would get a 2D surface. In a parametric representation, we use auxiliary variables, called parameters, to describe the object. The number of these parameters gives the dimension of the object and we will have $n$ functions of these parameters when we are in $n$D space. For example, a 1D curve in 3D space would be described by 3 functions of a single variable as $x = x(t), y = y(t), z = z(t)$ where $t$ is the parameter. A surface in 3D would use $x = x(u,v), y = y(u,v), z = z(u,v)$ where $u,v$ are the parameters. In analytic geometry, we mostly encounter simple shapes described by linear or sometimes quadratic implicit equations. That corresponds to parametric representations that can be expressed in terms of elementary functions like $\sin, \cos, \sinh, \cosh$. Generalizing to more complicated equations - like arbitrary polynomials - is usually reserved for the more advanced topic of \emph{algebraic geometry}, which is mostly concerned with implicit representations. Another way to advance the topic is to move on to \emph{differential geometry}, which mostly deals with parametric representations and has special interest in the curvature of the represented objects. \emph{Analytic geometry} lays the groundwork for both of these fields. 


% ToDo: mention polar coordinates - it's another type of explicit representation. what about % shperical coordinates
%
% mention synthetic geometry

% https://en.wikipedia.org/wiki/Analytic_geometry
% https://en.wikipedia.org/wiki/Synthetic_geometry

%===================================================================================================
\subsection{The $2D$ Euclidean Plane $\mathbb{R}^2$}
The simplest setting to do analytic geometry in is the $2$-dimensional plane also known as the Euclidean plane which we denote as $\mathbb{R}^2$ due to the fact that we use two real numbers to specify a location in this space. This is the place where Euclid himself developed all the geometric notions that we learn today in school. The objects we will deal with are points, lines, curves, triangles and polygons. The quantities we are interested in are distances, angles and areas. ...TBC...

%---------------------------------------------------------------------------------------------------
\subsubsection{Points, Vectors and Vertices}
Our eventual goal is to describe geometric shapes and calculate certain features of them. The most primitive object that we need to build our representations of shapes from is a \emph{point}. A point is just a location in a space - in this case, a location in the $2D$ plane. When we build polygons from points, we will also call these points \emph{vertices} (singular: \emph{vertex}) and we may use \emph{vectors} to represent these points. Conceptually, a point is actually not the same thing as a vector. A point is just a location in space and it doesn't seem to be a meaningful idea to add two locations in space together, for example. A vector, on the other hand, is meant to encode a length and direction (of movement, for example) and it does indeed make geometric sense to add a vector to a point to get another point. Vectors are things that we can freely add together anyway and the result will always be another vector. Such an addition does make geometric sense as well: we first do one translation, then the other - the sum of the two vectors encodes the total translation. Even though it doesn't make sense to \emph{add} two points, it does make sense to \emph{subtract} two points - the result will be the vector that tells us, how to move from one point to the other. This seems all very messy. In practice, however, the distinction between points and vectors is rarely made and points are just identified with their position vectors, i.e. vectors that go from the origin of the coordinate system to the point. So, in this setup, the terms point, vector and vertex kinda all mean the same basic mathematical object, namely a vector, but the usage of different terms can be viewed as a hint to what specific role a given vector currently plays. If we say vertex, the role is typically to be a corner of a polygon. If we say point, the role is typically a general location. If we say vector, the role may be a translation or displacement. But whatever the role is, the representation will always be a pair of real numbers, i.e. a $2D$ vector.

\paragraph{Distance Between Points}
In geometry, we are interested in the geometric relationships between geometric objects. perhaps the simplemost example of such a relationship is the distance between two points. as we are doing Euclidean geometry here, we will use the Euclidean distance. Given two points $\mathbf{p} = (p_x, p_y)$ and $\mathbf{q} = (q_x, q_y)$, the Euclidean distance between them is given by:
\begin{equation}
\dist(\mathbf{p,q}) = |\mathbf{p} - \mathbf{q}| = \sqrt{(p_x - q_x)^2 + (p_y - q_y)^2}
\end{equation}
Different distance measures are possible, but then we will leave the realm of Euclidean geometry. Non-Euclidean geometry is a topic for later.

\paragraph{Center of Mass}
If we have a collection of $N$ points $\mathbf{p}_1, \ldots, \mathbf{p}_N$, their \emph{center of mass}, also known as \emph{barycenter} or \emph{balance point}, is given by their arithmetic average. If we denote it by $\mathbf{P}$, we have:
\begin{equation}
\mathbf{P} = \frac{1}{N} \sum_{i = 1}^N \mathbf{p}_i
\end{equation}
This quantity is usually not so relevant in geometry, but for the sake of completeness.


%https://en.wikipedia.org/wiki/Center_of_mass

%...TBC...TODO: introduce different notations for points and vectors. Define distance between two points

% What about the center of mass of a set of points. Barycenter

%In this context, the points are also often called "vertices" and they can be represented by vectors. 

%---------------------------------------------------------------------------------------------------
\subsubsection{Lines}
After points (or vectors or vertices), the next object we need is the \emph{line}. A line can be defined in various ways. Perhaps the most obvious way is to use two points because given two distinct points, there is always exactly one line that passes through both of them. It is noteworthy that by line, we mean the infinitely extended line that goes through the two points. If we mean just the segment of the lines in between the two points, we call it a \emph{line segment}. In the great scheme of things, lines can be viewed as $1D$ spaces in their own right - and points would be the edge case of $0D$ spaces. Let's start with two distinct points represented via their position vectors $\mathbf{p}_1, \mathbf{p}_2$ and their difference vector $\mathbf{v}$ which encodes how to move from $\mathbf{p}_1$ to $\mathbf{p}_2$:
\begin{equation}
\mathbf{p}_1 = \begin{pmatrix} x_1 \\ y_1 \end{pmatrix}, \quad
\mathbf{p}_2 = \begin{pmatrix} x_2 \\ y_2 \end{pmatrix}, \qquad
\mathbf{v}   = \mathbf{p}_2 - \mathbf{p}_1 
             = \begin{pmatrix} x_2 - x_1 \\ y_2 - y_1 \end{pmatrix}
             = \begin{pmatrix} v_x \\ v_y \end{pmatrix}
\end{equation}
In this case, $\mathbf{p}_1$ and $\mathbf{p}_2$ are really playing the role of points in the sense that they encode locations whereas $\mathbf{v}$ is a proper vector in the narrower sense of encoding a displacement. The situation is depicted below using as example $\mathbf{p}_1 = (2,2), \mathbf{p}_2 = (4,3)$.


\begin{tikzpicture}
[thick, >=stealth', 
  dot/.style = { draw, fill = black, circle, inner sep = 0pt, minimum size = 6pt }
]

% Grid:
\draw[very thin,color=lightgray] (-3.2,0.0) grid (4.2,4.2);

% Coordinate system:
\draw[->] (-5, 0) -- (8,0) coordinate[label = {below:$x$}] (xmax);
\draw[->] ( 0,-1) -- (0,5) coordinate[label = {left:$y$}]  (ymax);

% Line:  
\draw     (-4,-1) -- (6,4)  node[pos=0.85, below right] {};

% Points 1p, p2 on the line:  
\draw (2,2) node[dot, label = {above:$\mathbf{p}_1$}]{};
\draw (4,3) node[dot, label = {above:$\mathbf{p}_2$}]{};

% Vector v and its components:
\draw[red, <->] (2,2) -- (4,2);
\draw[red] (3,2) node[below] {$v_x$};
\draw[red, <->] (4,2) -- (4,3);
\draw[red] (4,2.5) node[right] {$v_y$};
\draw[red, ->] (2,2) -- (4,3);
\draw[red] (3,3) node[below] {$\mathbf{v}$};

% Marks on the x- and y-axes:
% ToDo: try to use tick-marks instead - or just drop horizontal and vertical lines
% Maybe draw a grid
\draw [shift={(2,0)}, color=black] ( 0pt,+3pt) -- ( 0pt,-3pt) node [below] {$x_1$};
\draw [shift={(4,0)}, color=black] ( 0pt,+3pt) -- ( 0pt,-3pt) node [below] {$x_2$};
\draw [shift={(0,2)}, color=black] (+3pt, 0pt) -- (-3pt, 0pt) node [left]  {$y_1$};
\draw [shift={(0,3)}, color=black] (+3pt, 0pt) -- (-3pt, 0pt) node [left]  {$y_2$};

% x,y intercepts:
\draw [shift={(-2,0)}, color=black] ( 0pt,+3pt) -- ( 0pt,-3pt) node [below] {$a$};
\draw [shift={( 0,1)}, color=black] (+3pt, 0pt) -- (-3pt, 0pt) node [left]  {$b$};

% angle alpha:
\draw (-1,0) arc (0:27.5:1);
\draw (-1.25,0.16) node[label = {center:$\alpha$}]{};

% Equations:
\node[align=left] at (9.5,2.0) 
{
\begin{tabular}{l l}
  $\mathbf{p}(t) = \mathbf{p}_1 + t (\mathbf{p}_2 -\mathbf{p}_1)$ & 2-point form \\
  $\mathbf{p}(t) = \mathbf{p}_1 + t \mathbf{v}$ & Parametric vector form \\  
  $Ax + By + C = 0$                             & Implicit coordinate form \\   
  $y = m x + b$                                 & Explicit form \\
  $\frac{x}{a} + \frac{y}{b} = 1$               & Intercept form \\
\end{tabular}
};
\end{tikzpicture}
% https://en.wikipedia.org/wiki/Linear_equation
% https://en.wikipedia.org/wiki/Linear_equation#Equation_of_a_line
% ToDo: maybe wrap into a figure, center it and move the formulas out.


where also 4 different forms of the line equation are given. There are yet more forms to write down a line equation but these are some of the more common ones. We'll see some others later. The given names of the forms are not all super official, though. I call a form "coordinate form" when we explicitly use the coordinates $x,y$ and I call it vector form, when vector notation such as $\mathbf{p}$ is used. The slope $m$, angle $\alpha$ and the $x$- and $y$-intercepts $a,b$ can be computed as: 
\begin{equation}
 \tan(\alpha) = m = \frac{y_2-y_1}{x_2-x_1} = \frac{v_y}{v_x} = -\frac{b}{a}, \qquad
 a = x_1 - \frac{y_1}{m}, \; b = y_1 - m x_1
\end{equation}
The denomination "$y$-intercept" can be understood as: "where the line intercepts the $y$-axis" - likewise for "$x$-intercept". The parametric form is also commonly seen in coordinate notation as:
\begin{equation}
x(t) = x_1 + t (x_2 - x_1) = x_1 + t v_x, \qquad
y(t) = y_1 + t (y_2 - y_1) = y_1 + t v_y
\end{equation}
With my naming scheme, this form could then be called "parametric coordinate form" - again, this is not an official term. The coefficients $A,B,C$ for the "implicit coordinate form" can be computed by:
\begin{equation}
 A = y_1 - y_2 = -v_y, \quad B = x_2 - x_1 = v_x, \quad C = x_1 y_2 - x_2 y_1
\end{equation}
These coefficients are not uniquely determined, though. Because the right hand side of $Ax + By + C = 0$ is zero, we are free to multiply the $A,B,C$-coefficients all through by the same nonzero constant without changing the line that is encoded by this implicit line equation. 

\paragraph{Hesse Normal Form}
It may make sense to normalize the implicit line equation by requiring that $A^2 + B^2 = 1$. To achieve that, we need to scale all coefficients $A,B,C$ by $s = 1 / |\mathbf{v}| = 1 / \sqrt{A^2 + B^2}$. For that to work, we need to require that $A$ and $B$ are not simultaneously zero. That's usually a requirement for those coefficients anyway. From the formulas above, we see that they could only be simultaneously zero, if our two points $\mathbf{p}_1, \mathbf{p}_2$ would actually be one and the same point. That's of course not allowed. We said that the two points must be distinct. Let's define:
\begin{equation}
 s = \frac{1}{\sqrt{A^2 + B^2}}, \qquad
 n_x = s A, \;
 n_y = s B, \;
 d   = s C, \qquad
 \mathbf{n} = \begin{pmatrix} n_x \\ n_y \end{pmatrix}
\end{equation}
With these definitions in place, we can write down our implicit line equation also in the following form:
\begin{equation}
 \mathbf{n} \cdot \mathbf{p} + d = 0 \quad \text{or} \quad
 n_x p_x + n_y p_y + d = 0
\end{equation}
when we take $\mathbf{p} = (p_x, p_y)$. The right variant is still an "implicit coordinate form" but now with normalized coefficients. The left variant might be called "implicit vector form" but it's also known as the \emph{Hesse normal form}. The vector $\mathbf{n}$ is a vector of unit length that is perpendicular to our line and it's called a \emph{normal vector}. In this context, "normal" usually means "perpendicular". Normal vectors more commonly used for planes and more general surfaces - normal vectors to surfaces are very common in the context of 3D rendering engines - but the idea can be applied to a line, too. The value $d$ has a geometric interpretation, too: It's a (signed) \emph{distance} of the line from the origin - so using the letter $d$ actually makes sense.  [VERIFY all of this!].

% https://en.wikipedia.org/wiki/Hesse_normal_form

\paragraph{Lines as Sets}
I have given a lot of equations that \emph{define} a line in the sense that these formulas can be used to produce points on the line (in the case of the explicit and parametric forms) or can be used to check whether or not some given point is on the line (in the case of the implicit forms). I have not yet really said, what a line actually \emph{is} in a mathematical sense. So, I will now do just that: a line $L$ is a \emph{set of points}. We can write down this set using set builder notation as follows: 
\begin{equation}
L = \{ \mathbf{p} \in \mathbb{R}^2 : \mathbf{n} \cdot \mathbf{p} + d = 0 \}
  %= \{ \mathbf{p} + t \mathbf{v}, \;\; t \in \mathbb{R}, \; \mathbf{p,v} \in \mathbb{R}^2   \}
  = \{ \mathbf{p} + t \mathbf{v}, \; t \in \mathbb{R}  \}
  = \{\mathbf{p} + \mathbb{R} \mathbf{v} \} 
\end{equation}
where $\mathbf{n}, \mathbf{v}, d$ are given constants, $t$ is a dummy variable for a parameter and the rightmost notation is an abbreviation for the parametric form in the middle. The left form is again the implicit form using the normal vector.
...TBC...

\paragraph{Lines as Data}
We have seen various ways to define a line. As programmers, we may ask ourselves what data structure we should use to store a line in the computer. Some forms of the line equation were redundant in terms of data representation. The parametric forms use four real variables - either two points or a point and a vector. But these forms actually define not just a line itself but a line segment which includes already some more information than just the line itself. How many real (i.e. floating point) numbers do we actually \emph{need} to store the data of a line? The most parsimonious ways to define a line got away with just two real numbers - namely the explicit form $y = mx + b$ and the intercept form $x/a + y/b = 1$, so it surely appears that a line has two degrees of freedom. But these two forms are unfortunately not universally applicable. Maybe we could do some hacks with using the infinite values of IEEE floating point numbers for edge cases. Or we could look at the implicit forms. The Hesse normal form is generally applicable but needs one vector and one scalar to define a line. These are 3 real numbers all in all. But the vector is supposed to be normalized to unit length which takes away one degree of freedom. So - again - we can conclude that a general line has two degrees of freedom. Maybe to encode a line with the smallest amount of data, one could use the angle of the normal vector together with $d$ from the Hesse normal form. But that's a rather inconvenient encoding for most of our desired computations. But maybe for data storage purposes, that could be a useful way to do it. For practical geometric computations, one usually opts for the 2-point form. Mathematical lines do not have start- and endpoints but the lines that we want to draw on a screen usually do. So, the additional information stored in the 2-point form is usually quite useful in practice. So, although it's not the mathematically purest and not the most economical form to store a line, using two points is often a quite practical and convenient data structure for representing lines in geometric algorithms.

\paragraph{Distance Between Point and Line}
Back to geometry. As said - we are interested in relationships between geometric objects. The next relationship of interest is the distance between a point and a line. What we mean by that is, of course, the shortest possible distance. That means that we want to figure out the length of a straight path from the point that hits the line from a perpendicular direction. We don't want to approach the line in a slanted way because such a path would clearly be longer. Observe that the Hesse normal form of the line is an equation that we can treat as a formula for a function into which we can plug in arbitrary points $\mathbf{p}$. If the function returns zero, it means our input point $\mathbf{p}$ was, in fact, on the line. It turns out that, in general, the returned value can be interpreted as a signed distance of that point from the line. The distance is then just the absolute value of that. So, given a point $\mathbf{p}$ and a line $L$ in terms of $\mathbf{n}, d$, i.e. in Hesse normal form, we can compute the distance between $\mathbf{p}$ and $L$ as:
\begin{equation}
\dist(\mathbf{p}, L) = | \mathbf{n} \cdot \mathbf{p} + d |
\end{equation}

\paragraph{Intersection of Two Lines}
To compute an intersection between two lines $L_1, L_2$, let's assume the lines are given in parametric form like so:
\begin{equation}
L_1: \mathbf{p} + s \mathbf{u}, \quad
L_2: \mathbf{q} + t \mathbf{v}
\end{equation}
To find a common point, we must require that $\mathbf{p} + s \mathbf{u} = \mathbf{q} + t \mathbf{v}$. This vector equation represents a system of two linear equations for $s$ and $t$ which we may readily solve. Remember that $\mathbf{p,u,q,v}$ are all just known vectors. To find the intersection point, we can then either plug $s$ into the equation for $L_1$ or $t$ into the equation for $L_2$. ...TBC...


\paragraph{Angle Between Lines}
Let's now assume that we have two lines $L_1, L_2$ given and we want to compute the angle between them. If both lines are given in the form ...TBC...





% Verify all formulas numerically

% The coeffs for the implicit form can be multiplied through by any nonzero constant.
% Give coordinate form of implcit equation: x(t) = x_1 + t (x_2 - x_1), ...
% Explain that the vector (A,B) is a normal to the line. Maybe draw it in. Explain which forms are
% convenient in which situations.
% Give different forms of a line and formulas to convert between them, formulas for angles,
% distance point-line, projection of points onto lines, ...
% Explain that the set of lines forma a 2D vector space over R (I think)

% Maybe use n_x, n_y, d inszead of A,B,C. (n_x, n_y) is a normal vector, d the distance to the 
% origin
% draw in v (maybe in blue), maybe also its x- and y- components
% maybe draw in the normal vector n

% adapted from:
% https://texample.net/tikz/examples/linear-regression/
% https://texample.net/tikz/examples/upper-riemann-sum/


% -explicit form: not applicable to vertical lines (m would have to be infinite)
% -intercept form: not applicable to horizontal or vertical lines (because a or b would
%  have to be infinite)
% -talk about the degrees of freedom - there are 3. So, the 2-point form contains more 
%  information than we need (which makes sense, because it actually encodes a line *segment*)

% Leupold, pg 174-176:
% 3.44:  m = tan(alpha) = (y2-y1)/(x2-x1)     Anstiegswinkel
% 3.45:  r = r1 + t*(r2 - r1)                 Parameterdarstellung
% 3.46:  r = r1 + t * u                       u = r2-r1
% 3.47:  (y-y1)/(x-x1) = (y2-y1)/(x2-x1)      Zweipunkteform, (2-point form?)
% 3.48:  y-y1 = m*(x-x1)                      Punktrichtungsform
% 3.49:  y = m*x + b                          Normalform, (explicit?)
% 3.50:  x/a + y/b = 1                        Abschnittsform
% 3.51:  Ax + By + C = 0                      allgemeine Form, (implicit? kinda like Hesse
%                                             normal form of planes?)
%
% m: slope, a: x-intercept, b: y-intercept, r1 = (x1,y1), r2 = (x2,y2)

% To compute the intercepts a,b, use the parametric form in coordinate form:
%   x(t) = x_1 + t (x_2 - x_1)
%   y(t) = y_1 + t (y_2 - y_1)
% To find b: find t such that x(t) = 0, plug it into y(t). That gives: b = y_1 - x_1 * m
% To find a: find t such that y(t) = 0, plug it into x(t). That gives: a = x_1 - y_1 / m
%
% Maybe given an implicit equation in vector form: \mathbf{n} \cdot \mathbf{p} + d = 0
% where n is the unit normal vector n_x = A / |v|, n_y = B / |v|, d = C / |v|

%---------------------------------------------------------------------------------------------------
\subsubsection{Curves}
Curves are a generalization of lines. Equations of curves can also be given in various forms. The most common ones are again: parametric, implicit and explicit. All these 3 forms can be written down in a form using vector notation or by writing out one equation for each coordinate. For example, the unit circle can be written down in implicit form as $x^2 + y^2 = 1$, in explicit form as $y = \pm \sqrt{1 - x^2}$ and in parametric form as $(x,y) = (\cos t, \sin t), t \in [0, 2 \pi]$. For the purpose of drawing a curve on a screen using some vector graphics library, the parametric form is usually the most convenient. The explicit form is convenient when we are dealing with a curve that represents a function - but not all curves do represent functions so the applicability of that form is rather limited. The explicit form of the circle that I have written down above is somewhat of a hack using this $\pm$ sign to indicate that I want both values of the square root - but such a thing isn't really allowed for a proper function definition. You understand what I mean because you are a human (I guess), but a computer wouldn't accept that as an explicit definition - "explicit" is usually synonymous with "functional" as in "defines a function $y = f(x)$". We might get away with broadening our view to allowing multifunctions in explicit curve definitions - but I digress.

...TBC...give some interesting examples - maybe the lemniscate, an elliptic curve, etc.

%---------------------------------------------------------------------------------------------------
\subsubsection{Triangles}


%---------------------------------------------------------------------------------------------------
\subsubsection{Polygons}


%---------------------------------------------------------------------------------------------------
\subsubsection{Conic Sections}

\paragraph{Circles and Ellipses}

\paragraph{Hyperbolas}

\paragraph{Parabolas}




%ToDo: representation as 2D vectors, points, lines, formulas for: distance between points, between point and line, angle between lines, ...


%===================================================================================================
\subsection{The $3D$ Euclidean Space $\mathbb{R}^3$}
When we up our game by one dimension from the Euclidean plane, we arrive at the $3$-dimensional Euclidean space. Points, lines, triangles and curves are still a thing - but they will now exist in $3$ rather than $2$ dimensions. But we also get a new type object to deal with: arbitrary planes. In the $2D$ case, there was \emph{the} plane which was the backdrop for all the objects. Now planes themselves are objects within the larger $3D$ space and they can have any orientation within that larger space. We even get curves surfaces which was also not a thing in $\mathbb{R}^2$. Polygons can still make sense in $\mathbb{R}^3$, but we need to be a bit more careful because it's not guaranteed anymore that all the vertices of a polygon lie in the same plane. This was not an issue in $2D$ space. ...TBC...

%ToDo: representation as 3D vectors, points, lines, planes, formulas for: distance between point and line, point and plane, intersection point of line with plane, intersection line of 2 planes, intersection point of 3 planes, angle between plane and line, plane and plane, ...


%===================================================================================================
\subsection{The $nD$ Euclidean Space $\mathbb{R}^n$}
Having seen the Euclidean spaces of dimension $2$ and $3$, it's now time to generalize to Euclidean spaces of arbitrary dimension. ...TBC...

%\subsection{The Euclidean }

%---------------------------------------------------------------------------------------------------
\subsubsection{Lines}
Given a point $\mathbf{p}$ and a nonzero vector $\mathbf{v}$, both elements of $\mathbb{R}^n$, we can consider the following set of points:
\begin{equation}
 L 
 =
 \{\mathbf{p} + \mathbb{R} \mathbf{v} \} 
 = 
 \{ \mathbf{p} + t \mathbf{v}, \; t \in \mathbb{R} \}
\end{equation}
This set of points represents a line in $\mathbb{R}^n$. The expression in the middle is just a shorthand notation for the expression on the right. Multiplying the vector $\mathbf{v}$ by the set of real numbers $\mathbb{R}$ is meant in the sense that we take $\mathbf{v}$ and multiply it by every possible real number $t \in \mathbb{R}$. This line $L$ goes through the point $\mathbf{p}$ and has a direction of $\mathbf{v}$. The line is given in the so called \emph{parametric form}\footnote{Well, I think, for a parametric form, one would more commonly write out a separate equation for each coordinate if $n$ is known and small enough like $2$ or $3$. For general unknown $n$, this vector form can be seen as a more abstract way of writing down a parametric form.} with the parameter $t$. It's also called the \emph{vector form}. We can imagine that, as $t$ sweeps through the real numbers, a point sweeps along the line. For each value of $t$, we obtain some point on the line by just plugging the value $t$ into the formula. With $t=0$, we are at $\mathbf{p}$. With $t=1$, we are at $\mathbf{p + v}$. With $t=\frac{1}{2}$, we are at $\mathbf{p} + \frac{1}{2} \mathbf{v}$ which is exactly halfway between $\mathbf{p}$ and $\mathbf{p + v}$. And so on. The vector form is not the only way to define a line. One can also use two (distinct) points $\mathbf{p}_1,\mathbf{p}_2$ on the line and two real numbers $t_1, t_2$ that satisfy $t_1 + t_2 = 1$ and express the line as the set of points:
\begin{equation}
 L = \{ t_1 \mathbf{p}_1 + t_2 \mathbf{p}_2,  \quad t_1, t_2 \in \mathbb{R}, t_1 + t_2 = 1 \}
\end{equation}
This is called a \emph{two point form}. It may also be called a representation in terms of \emph{barycentric coordinates}, although that terminology is less commonly used for lines - it's more commonly used for planes that are defined by 3 points or $n$-dimensional hyperplanes that are defined by $n+1$ points, but it is the proper $1D$ analog. You may wonder why we introduce the variable $t_2$ and don't just use $1 - t_1$ instead (and then drop the index altogether). The reason is that this would work only for the 1D case, i.e. the line, but writing it like above will readily generalize to barycentric representations of higher dimensional objects. Yet another way to define a line is by means of an implicit equation of the form...TBC....TODO: give other forms of a line (implicit, explicit, etc.), mention that it can be generalized from $\mathbb{R}^n$ to vector spaces over any field - the complex numbers are interesting in the context of algebraic geometry, give the ways to convert between the different forms.

% Two lines are parallel, when they have parallel direction vectors. Two vectors are parallel when one is a scalar multiple of the other

% https://en.wikipedia.org/wiki/Line_(geometry)#Definition
% https://en.wikipedia.org/wiki/Barycentric_coordinate_system

% https://math.stackexchange.com/questions/2931233/the-vector-form-and-parametric-forms-of-a-line-given-a-point-and-direction-or

%---------------------------------------------------------------------------------------------------
\subsubsection{Planes}
Given a point $\mathbf{p}$ and two linearly independent\footnote{Linear independence implies that both must be nonzero.} vectors $\mathbf{v, w}$, consider the following set of points:
\begin{equation}
 \mathbf{p} + \mathbb{R} \mathbf{v}  + \mathbb{R} \mathbf{w}
 \quad = \quad
 \mathbf{p} + s \mathbf{v} + t \mathbf{w}, \; s,t \in \mathbb{R}
\end{equation}
This set of points represents a plane in $\mathbb{R}^n$. 

%---------------------------------------------------------------------------------------------------
\subsubsection{Hyperplanes}
After having seen lines and planes, the pattern continues. We can take some fixed point $\mathbf{p}$ together with a set of $k$ linearly independent vectors $\mathbf{v}_1, \mathbf{v}_2, \ldots, \mathbf{v}_k$ and consider the set:
\begin{equation}
 \mathbf{p} + \sum_{i=0}^{k} t_i \mathbf{v}_i, \; t_i \in \mathbb{R}
\end{equation}
This set defines a $k$-dimensional hyperplane in $\mathbb{R}^n$ where $k < n$. Well, we could perhaps allow $k=n$ as edge case as well, but then we would just get the whole space $\mathbb{R}^n$ and that's not interesting. We are interested in substructures of $\mathbb{R}^n$. But $k > n$ does really not make any sense at all, not even as edge case, because then we can't even find a set of $k$ linearly independent vectors anymore.

% https://web.evanchen.cc/handouts/bary/bary-short.pdf

%---------------------------------------------------------------------------------------------------
%\subsection{Quadrics}

% https://en.wikipedia.org/wiki/Quadric_(algebraic_geometry)
% Maybe do it in algebraic geometry


% Give equation of a cone and explain why conic sections are called like that. It's because the
% curves arise as intersection between a cone and a plane. I think, the equation is z^2 = x^2 + y^2
% but I'm not sure about that.







\begin{comment}


Geometrische Interpretation linearer Abbildungen
https://www.youtube.com/watch?v=EQ5Xct2YyLk


Weitz: Analytische Geometrie
https://www.youtube.com/watch?v=7rkkHIjHtNI
https://www.youtube.com/watch?v=7rkkHIjHtNI&list=PLb0zKSynM2PBYzz6l37rWH3B_n_7P40QP

\end{comment}