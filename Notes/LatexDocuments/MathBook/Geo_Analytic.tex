\section{Analytic Geometry}
Analytic geometry is about representing the positions of points within space by numbers and doing calculations with these numbers. These numbers are called "coordinates" and are to be understood with respect to a given coordinate system which has to be chosen beforehand. From points, more complex objects like lines, planes, shapes, etc. can be made up and we are interested in computing things like the distance of a point from a line or plane etc.. The term "analytic" is, in my humble opinion, a bit misleading here because "analysis" is usually taken to be a more rigorous version of calculus but in analytic geometry, we typically do more basic vector- and matrix-algebra stuff and the more calculusy things are usually tackled later in "differential" geometry. However, "analytic" geometry, just like "elementary" geometry, is also mostly a bunch of formulas that are useful for creating more complex geometric algorithms. The main distinctive feature is that these formulas will typically involve representations of our shapes in terms of vectors of coordinates. This is actually the more typical representation we use in a computer, at least in the context of graphics but also for physical simulations. In this context, the points are also often called "vertices" and they can be represented by vectors. Conceptually, a point is actually not the same thing as a vector. A point is just a location space and it doesn't seem to be a meaningful idea to add two locations in space together, for example. A vector, on the other hand, is meant to encode a length and direction and it does indeed make geometric sense to add a vector to a point to get another point. In practice, however, this distinction is rarely made and points are just identified with their position vectors, i.e. vectors that go from the origin of the coordinate system to the point. Vectors are things that we can freely add together anyway and the result will always be another vector.

\subsection{The 2D Plane}

ToDo: representation as 2D vectors, points, lines, formulas for: distance between points, between point and line, angle between lines, ...



\subsection{The 3D Space}

ToDo: representation as 3D vectors, points, lines, planes, formulas for: distance between point and line, point and plane, intersection point of line with plane, intersection line of 2 planes, intersection point of 3 planes, angle between plane and line, plane and plane, ...