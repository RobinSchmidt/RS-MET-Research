\section{Differential Geometry}
In algebraic geometry, we considered zero sets of systems of polynomial equations. Under certain circumstances, these zero sets defined a manifold and sometimes, we could find a parametric representation of such a manifold, i.e. a formula, that takes as input one or more independent parameters and spits out a point on the manifold as output. Differential geometry (DG) typically takes a parametric representation, i.e. a vector valued function of possibly multiple arguments, of such a manifold as a starting point. Since we will need to take work with (partial) derivatives of such a parametric representation, we will assume that all the functions are differentiable sufficiently often.

\medskip
DG can be roughly split into "classical" or "elementary" DG and "modern" DG. The former deals with curves in the 2D plane and with curves or surfaces in 3D space and is typically expressed in (a subset of) the language of vector calculus. The latter deals with the completely general case of $kD$ manifolds in $nD$ space where $k \leq n$ and is typically expressed in (a subset of) the language of differential forms (aka exterior calculus) and/or tensor calculus. Important applications of classical DG are in computer graphics, modern DG features prominently in general relativity.

\medskip
A very important, maybe the most important, concept in DG is the idea of curvature.

% curves vs surfaces vs general manifolds, intrisic vs extrinsic

...tbc...