\section{Differential Geometry}
In algebraic geometry, we considered zero sets of systems of polynomial equations. Under certain circumstances, these zero sets define a manifold and sometimes, we could find a parametric representation of such a manifold, i.e. a formula, that takes as input one or more independent parameters and spits out a point on the manifold as output. Differential geometry (DG) typically takes a parametric representation, i.e. a vector valued function of possibly multiple arguments, of such a manifold as a starting point. Since we will need to take work with (partial) derivatives of such a parametric representation, we will assume that all the functions are differentiable sufficiently often.

\medskip
DG can be roughly split into "classical" or "elementary" DG and "modern" DG. The former deals with curves in the 2D plane and with curves or surfaces in 3D space and is typically expressed in (a subset of) the language of vector calculus. The latter deals with the completely general case of $kD$ manifolds in $nD$ space where $k \leq n$ and is typically expressed in (a subset of) the language of differential forms (aka exterior calculus) and/or tensor calculus. Important applications of classical DG are in computer graphics, modern DG features prominently in general relativity.

\medskip
A very important, maybe the most important, concept in DG is the idea of curvature.

% curves vs surfaces vs general manifolds, intrisic vs extrinsic

...tbc...



\begin{comment}
	
In algebraic geometry, we deal with a manifold that is defined by an implicit equation of the form f(x,y) = 0 in 2D or f(x,y,z) = 0 in 3D or a system of such equations like f(x,y,z) = 0, g(x,y,z) = 0 in 3D. The equations are assumed to be polynomials but (I guess) the general ideas can be generalized to arbitrary equations. The manifold is typically of a dimensionality given by the dimensionality of the embedding space minus the number of equations (verify!). When I say "typically", I mean that there may be other cases but these can be considered as somehow being degenerate edge cases (verify!). For example, if we have have 1 equation in 3 variables, we normally get a 2D manifold, i.e. a surface in 3D space. 

In differential geometry, we also deal with manifolds but here, the manifolds are not defined via an implicit equation but rather by a set of parametric equations. We can imagine the parametrization to define a map of the manifold. We may have different parametrizations (i.e. maps or charts) for different regions. The totality of all these maps (i.e. the set of all the maps) is called an atlas. Ideally, the maps should cover the whole manifold. Different maps may overlap with respect to the region that they cover. If they do overlap, they should agree in the regions of overlap. ...tbc...

In some sense, algebraic geometry is simpler than differential geometry because it just involves solving algebraic equations. Differential geometry, on the other hand, uses a lot of calculus which is usually considered to be a more advanced math topic than algebra. But in some other ways, differential geometry might also be considered easier that algebraic geometry because in DG we have the luxury of getting a parametrization of our manifold handed to us on a silver plate. Such a parametrization is for many purposes a much more convenient description of the manifold than the unwieldy implicit equation that AG gives us. One could actually say that the availability of the parametrization \emph{enables} us to make use of a lot of calculus that would unavailable otherwise.


https://mathworld.wolfram.com/Atlas.html


ToDo:
-There are at least 2 ways of defining a surface in 3D: 
 -parametrically as x(u,v), y(u,v), z(u,v) - that's the DG approach
 -implicitly as F(x,y,z) = 0 - that's the AG approach
-An explicit form z = f(x,y) can be seen a special case of both:
 parametric: x(u,v) = u, y(u,v) = v, z(u,v) = f(u,v)
 implicit:   F(x, y, f(x,y)) = 0
-Explain, how we can get a local parametrization from an implicit equation using
 partial derivatives...I'm not sure, if that's possible but I think, it should be.
 See:
 https://tutorial.math.lamar.edu/classes/calciii/parametricsurfaces.aspx
-Maybe we could also try to convert locally to an explicit form. It would become locally
 a 1D root-finding problem of F(x,y,z) = 0. Here, x,y are plugged in and can be seen as 
 parameters and we want to find z.
-Explain how we may get an implicit form from a given parametrization


%An Amazing Theorem for Tangents (Fabricius-Bjerre Theorem)
%https://www.youtube.com/watch?v=7_UclB2dB0o

\end{comment}