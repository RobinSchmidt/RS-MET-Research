\section{Fractal Geometry}



\begin{comment}

-Fractals can be defined via 
 (1) line-drawings (Hilbert curve, dragon curve) vs defined as 
 (2) of some region (Mandelbrot set, Julia set)
 (3) algorithmically (Sierpinsi triangle)
-(1) is connected to curves in differential geometry - we define a curve via a function f: R -> R^2
 (2) is more like algebraic geometry - we define a set by a property: S = {(x,y) \in R^2 : ...}
 (3) like elementary geometry - start with triangle, duplicate, scale, move, etc.
-I thinlk, (2) can also be formulated in terms of (discrete) dynamical systems. There are also 
 fractals arising from continuous dynamical systems, I think.
-Self similarity
-Fractal Dimension
-Space filling curves


\end{comment}