\section{Fractal Geometry}



\begin{comment}

-Fractals can be defined via 
 (1) line-drawings (Hilbert curve, dragon curve) vs defined as 
 (2) of some region (Mandelbrot set, Julia set)
 (3) algorithmically (Sierpinsi triangle)
-(1) is connected to curves in differential geometry - we define a curve via a function f: R -> R^2
 (2) is more like algebraic geometry - we define a set by a property: S = {(x,y) \in R^2 : ...}
 (3) like elementary geometry - start with triangle, duplicate, scale, move, etc.
-I thinlk, (2) can also be formulated in terms of (discrete) dynamical systems. There are also 
 fractals arising from continuous dynamical systems, I think.
-Self similarity
-Fractal Dimension
-Space filling curves


Das mathematische Geheimnis hinter Chaos
https://www.youtube.com/watch?v=vXkcS0QciTU

Is the Logistic Map hiding in the Mandelbrot Set? | #SoME3
https://www.youtube.com/watch?v=1uW-x2HxMOI

This equation will change how you see the world (the logistic map)
https://www.youtube.com/watch?v=ovJcsL7vyrk
-The video makes it seem like the bifurcation diagram and Mandelbrot set are two sides of the same
 coin - but they actually apply to different maps. The Mandelbrot map actually has its own
 bifurcation diagram. It looks qualitatively similar to the bifurcation diagram of the logistic map
 but the details are different. When considering both maps:
   Logistic:    x_{n+1} = r x_n (1 - x_n) = r x_n - r x_n^2
   Mandelbrot:  z_{n+1} = z_n^2 + c
 we see that the parameter r in the logictic map plays a different role than the parameter c in the
 Mandelbrot map. In the logistic map, r scales the linear and quadratic coeff in a polynomial. In
 the Mandelbrot map, c is the constant term.

\end{comment}