\chapter{Googology}
Googology is the branch of math that investigates extremely big numbers and the extremely fast growing functions that are used to produce them. The name comes from the number $10^{100}$ which is called a \emph{googol}. On ordinary scales, that number would perhaps be considered rather large - it's a 1 with a 100 zeros after it - however, in the context of googology, that specimen is actually tiny. Somewhat bigger is a \emph{googolplex} which is defined to be 10 to the power of a googol, i.e. $10^{10^{100}}$. We see that to produce these numbers, we have used a mathematical operation that is quite famous for escalating quickly - namely exponentiation.  ...TBC...


%The rather is called a \emph{googol}

% -explain why it's called a googol
% -explain googolplex
% -explain etymology of google


\section{Hyperoperations}

\subsection{Knuth's Arrow Notation}



\section{Fast Growing Functions}




\begin{comment}

Is googology part of discrete math, too? It's about fast growing functions that typically take
natural arguments.

Examples of fats growing functions:
-Ackermann, Busy Beavers, Tree, Rayos number, tetration, pentation, hyperoperations, 
 Graham's number, ...
https://en.wikipedia.org/wiki/Ackermann_function
https://en.wikipedia.org/wiki/Large_numbers
https://en.wikipedia.org/wiki/Fast-growing_hierarchy

How Big Is Graham’s Number? (S1EP04)
https://www.youtube.com/watch?v=Tw5AaS0qNgs

The Maths Behind Big Numbers (S1EP03)
https://www.youtube.com/watch?v=tWh6wRM87sc

The Boundary of Computation
https://www.youtube.com/watch?v=kmAc1nDizu0
-But I think, the non-computability of BB(n) has not so much to do with it's size but with the
 halting problem. In priciple, it may be possible to come up with small non-computable
 numbers, I think. How about BB(n)/BB(n+1) - it should be close to zero even for moderately
 large n - but it's a rational number, so I'm not sure, if that counts. Maybe define a number
 BB*(n) as follows: Let a BB-machine print not only 0s and 1s on a tape but rather numbers
 from a given range - say 0..9 - acroding to some rule (like step-number mod 10). The output is
 not how many numbers it has printed but the last printed number. It's just as non-computable as
 the regular BB-number - but it's clearly in the range 0..9


Hexation and Graham's Number
https://www.youtube.com/watch?v=-zv454ePhsw

You CANNOT imagine this number! (Graham's Number)
https://www.youtube.com/watch?v=qjkhCO-1JMM
-It ends in 7. ...How do we know that?

Tetration pentation hexation heptation octation (How fast by adding 1 arrow)
https://www.youtube.com/watch?v=VXiRv7csFa0

https://googology.fandom.com/wiki/Googology_Wiki
https://googology.fandom.com/wiki/Googology
https://en.wikipedia.org/wiki/Large_numbers

\end{comment}