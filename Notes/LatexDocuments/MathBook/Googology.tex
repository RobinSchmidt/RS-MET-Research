\chapter{Googology}
Googology is the branch of math that investigates extremely big numbers and the extremely fast growing functions that are used to produce them. The name comes from the number $10^{100}$ which is called a \emph{googol}\footnote{If that reminds you of the name of a company famous for its internet search engine, then that's no coincidence - the company name is a reference to the googol.}. On ordinary scales, that number would perhaps be considered rather large - it's a 1 with a 100 zeros after all - however, in the context of googology, a googol is actually a rather tiny specimen. Somewhat bigger is a \emph{googolplex} which is defined to be 10 to the power of a googol, i.e. $10^{10^{100}}$. Children may like to play the game "who can name the biggest number". Obviously it's always the child that answers last. It just needs to take the biggest number that has been named so far and add 1 to it to create an even bigger number. Or multiply it by 2 or take the factorial of it or raise it to its own power or whatever. So, that game seems to be rather pointless. ...TBC...

% -however, there have been serious competitions like that - Rayo's number resulted from it
% -there obviously need to be some rules to avoid such a kind "cheating". It was a contest at MIT
%  in 2007

%The rather is called a \emph{googol}

% -explain why it's called a googol
% -explain googolplex
% -explain etymology of google

% -what about infinity plus one?
% -Compare it to combinatorial explosions ..well - I guess, the tree-function counts as
%  combinatorial?


\section{Hyperoperations}
We see that to produce a googol, we have used a mathematical operation that is quite famous for escalating quickly - namely exponentiation. But for our purposes, it doesn't escalate quickly enough, so let's see how we can come up with operations that lead to even faster growth. Taking a step back, we note that exponentiation can be viewed as repeated multiplication. Multiplication can itself be viewed as repeated addition which in turn can be viewed as repeated incrementation. From that point of view, we may reasonably ask ourselves what the next operation in this hierarchy should be. Obviously, it would have to be repeated exponentiation. ...TBC...


% Tetration pentation hexation heptation octation

\subsection{Knuth's Arrow Up Notation}

\subsection{Graham's Number}


\subsection{Rayo's Number}

% The Daddy of Big Numbers (Rayo's Number) - Numberphile
% https://www.youtube.com/watch?v=X3l0fPHZja8


% https://googology.fandom.com/wiki/Large_Number_Garden_Number
% https://googology.fandom.com/wiki/Little_Bigeddon
% https://googology.fandom.com/wiki/Sasquatch
% https://googology.fandom.com/wiki/Maximum_shifts_function
% https://googology.fandom.com/wiki/Fish_numbers
% https://googology.fandom.com/wiki/BIG_FOOT


\section{Fast Growing Functions}
% -Start with incrementation f_0(n) = n+1
% -The next level is obtained by iterating the previous level n times, so
%  f_1(n) = f_0(f_0( ...f_0(n) )) = 2n with n nested evaluations of f_0
%  ...or is that even right?

% Fastest growing functions Ranking (Using Fast growing hierarchy)
% https://www.youtube.com/watch?v=F7JHcglMh4s

% The Hierarchy of Big Functions || n^n greater than n! greater than e^n greater than n^100
% https://www.youtube.com/watch?v=DZvqwuWuai0
% -focuses on "normal" functions - not so much about Googology

% Large numbers ranking (Update)
% https://www.youtube.com/watch?v=Dh4UMKpygNo

\subsection{The Fast Growing Hierarchy}



\subsection{The Tree Function}

\subsection{The Ackermann Function}

\subsection{The Busy Beavers Function}







\begin{comment}

-Is googology part of discrete math, too? It's about fast growing functions that typically take
 natural arguments.
 
-What can we actually compute in googology? Maybe initial, final or general finite sections of the
 string of digits? In some video, it was mentioned that it is known that Graham's number is known to
 end in 7. How can such a thing be proven?
 
-If you want to play the game "who can name the biggest number", it's obviously always the person
 who answers last - one can just use the biggest number so far and add 1 - or multiply by 2 - or 
 take the factorial of it - or whatever.

-What's the point of it all? Is it just an intellectual game? Or can it tell us something about the
 limits of mathematics?


https://en.wikipedia.org/wiki/Ackermann_function
https://en.wikipedia.org/wiki/Large_numbers
https://en.wikipedia.org/wiki/Fast-growing_hierarchy

How Big Is Graham’s Number? (S1EP04)
https://www.youtube.com/watch?v=Tw5AaS0qNgs

The Maths Behind Big Numbers (S1EP03)
https://www.youtube.com/watch?v=tWh6wRM87sc

The Boundary of Computation
https://www.youtube.com/watch?v=kmAc1nDizu0
-But I think, the non-computability of BB(n) has not so much to do with it's size but with the
 halting problem. In priciple, it may be possible to come up with small non-computable
 numbers, I think. How about BB(n)/BB(n+1) - it should be close to zero even for moderately
 large n - but it's a rational number, so I'm not sure, if that counts. Maybe define a number
 BB*(n) as follows: Let a BB-machine print not only 0s and 1s on a tape but rather numbers
 from a given range - say 0..9 - acroding to some rule (like step-number mod 10). The output is
 not how many numbers it has printed but the last printed number. It's just as non-computable as
 the regular BB-number - but it's clearly in the range 0..9


Hexation and Graham's Number
https://www.youtube.com/watch?v=-zv454ePhsw

You CANNOT imagine this number! (Graham's Number)
https://www.youtube.com/watch?v=qjkhCO-1JMM
-It ends in 7. ...How do we know that?

Tetration pentation hexation heptation octation (How fast by adding 1 arrow)
https://www.youtube.com/watch?v=VXiRv7csFa0



How do we know the last digit of Graham's number is 7?
https://www.youtube.com/watch?v=-F7dpyVm7c4







https://googology.fandom.com/wiki/Googology_Wiki
https://googology.fandom.com/wiki/Googology
https://en.wikipedia.org/wiki/Large_numbers

https://googology.fandom.com/wiki/List_of_googolisms

\end{comment}