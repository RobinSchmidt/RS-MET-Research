\section{Linear Systems of Equations}
One important area of application of matrices and vectors is the solution of systems of linear equations.

% solvability, rank (may also be filed under matrix features - maybe introduce the concept here and mnetion it there again)

% solution structure: particular solution plus general solution of homogeneous system
% explain, why that structure arises
% A solutions of the homogenous system gives zero by definition, so adding any multiple it to a
% particular aolution does not destroy the solution property...or something
% This solution structure is really only relevant for (consistent) singular systems that have a 
% whole space of solutions. If the solution is unique, I think we get the special case where the
%% space spanned by the solution of the homogeneus system is 0-dimensional...or soemthing?


\begin{comment}

-"Algebra" is generally about solving equations. Questions liek: 
 -How many solutions are there?
 -How can we find them? Is there a systematic algorithm to produce the solutions?
 -Is there some structure to the set of solutions.
  -In case of linear algebra: the structure of the solution set of a linear system of equations is:
   x_g = x_p + x_h where: x_g is the general solution, y_p is a particular solution and y_h is the
   homogeneous solution. The latter is a subspace of the space we are seeking solutions in that is 
   given by the solution of the corresponding
   
\end{comment}