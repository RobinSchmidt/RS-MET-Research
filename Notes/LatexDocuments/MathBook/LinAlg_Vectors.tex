\section{Vectors and Matrices}

\paragraph{Vectors}
Consider a point in the 2D plane or in 3D space. To represent it mathematically, we would need a pair or a triple of numbers respectively. An $n$-dimensional vector can generally be thought of as an $n$-tuple of numbers. For intuition building, it's best to think of real numbers although in general, other kinds of numbers may also be allowed in the more general case. 
%A vector 


\paragraph{Matrices}


\paragraph{Well, actually...}
Strictly speaking, an $n$-tuple of numbers is not what a vector really $is$ by its nature. By its nature, a vector is a geometric entity such as a point in space or an arrow with a direction and length (or magnitude). The $n$-tuple or numbers is a specific representation of that geometric entity that depends on the coordinate system that we have chosen. But take that statement as a foreshadowing to a more advanced viewpoint. For the purposes of this section, it's totally okay to picture a vector as a tuple of numbers.

\medskip
Moreover, when we look at things in a more strict way, we need to distinguish between points and vectors. ...TBC...


%===================================================================================================
%\subsection{Vectors}

%===================================================================================================
%\subsection{Matrices}


%===================================================================================================
\subsection{Operations}

%---------------------------------------------------------------------------------------------------
\subsubsection{Vector Operations}
\paragraph{Scalar Product}
\paragraph{Vector Products}
% cross-product, triple-product, wedge-product

%---------------------------------------------------------------------------------------------------
\subsubsection{Matrix Operations}
\paragraph{Addition and Multiplication}
\paragraph{Inversion}


%===================================================================================================
\subsection{Linear Systems of Equations}

% solvability, rank (may also be filed under matrix features - maybe introduce the concept here and mnetion it there again)



%===================================================================================================
\subsection{Special Features}

%---------------------------------------------------------------------------------------------------
\subsubsection{Vector Features}

\paragraph{Vector Norms}
% are a measure of length

\paragraph{Orthogonality and Angles}
% Angles between a are defined in terms of the cosine of the norm
% 

%---------------------------------------------------------------------------------------------------
\subsubsection{Matrix Features}

\paragraph{Determinant}

\paragraph{Norms}

\paragraph{Orthogonality}

\paragraph{Symmetry}

\paragraph{Misc Special Matrices}
% Toeplitz, circulant, unitary (maybe file under orthogonal - it's the complex version), tringular, ...










%===================================================================================================
\subsection{Matrix Decompositions}


%---------------------------------------------------------------------------------------------------
\subsubsection{Eigendecomposition aka Diagonalization}


%---------------------------------------------------------------------------------------------------
\subsubsection{Jordan Normal Form}
A diagonalization of a matrix is unfortunately not always possible [TODO: state conditions]. A Jordan normal form is one of the two second best things that we can do ...TBC...

% interpretation of the Jordan cells as representing a pair of complex conjugate eigenvalues?

%---------------------------------------------------------------------------------------------------
\subsubsection{Singular Value Decomposition}



%---------------------------------------------------------------------------------------------------
\subsubsection{Additive Decompositions}



%===================================================================================================
\subsection{Important Facts and Formulas}