\chapter{Logic}


%===================================================================================================
\section{Propositional Logic}

% -also known as zeroth order logic
% -Connectives (and, or, not, nand, xor, ...)

\subsection{Connectives}

$\neg, \wedge, \vee, \xor, \nand, \then$


% https://en.wikipedia.org/wiki/Exclusive_or
% https://en.wikipedia.org/wiki/Logical_connective

%===================================================================================================
\section{Predicate Logic}


%\subsection{Notation}

\subsection{Quantifiers}

$\exists, \nexists, \exists!, \exists_1, \forall$

% -also known as first order logic
% -Variables, Formulas
% -Quantors: \exists, \forall, \exists ! (it exists exactly one)

% https://en.wikipedia.org/wiki/Quantifier_(logic)
% https://de.wikipedia.org/wiki/Quantor

%===================================================================================================
\section{Proof Techniques}


% https://jdhsmith.math.iastate.edu/class/BookOfProof.pdf
% -Really good free ebook

%---------------------------------------------------------------------------------------------------
\subsection{Deduction}
Logical \emph{deduction} is the process of deriving new true statements from given true statements. In this context, the given statements are called \emph{premises} and the derived statements are called \emph{conclusions}. The process of deriving new statements from known ones is called \emph{inference} and there are certain rules that we need to obey to draw valid conclusions from a given set of premises.

% https://de.wikipedia.org/wiki/Deduktion

% https://home.uni-leipzig.de/methodenportal/deduktion_induktion/
% -Deduktion schließt vom allgemeinen auf das spezielle
% -Induktion schließt vom speziellen auf das allgemeine

% https://en.wikipedia.org/wiki/Logical_consequence
% https://en.wikipedia.org/wiki/Propositional_calculus

\subsubsection{Rules of Inference}
% https://en.wikipedia.org/wiki/Rule_of_inference

\paragraph{Modus Ponens}
% https://en.wikipedia.org/wiki/Modus_ponens

\paragraph{Modus Tollens}
% https://en.wikipedia.org/wiki/Modus_tollens

\paragraph{Contraposition}
% https://en.wikipedia.org/wiki/Contraposition

\paragraph{Resolution}
% -Resolution
%  https://de.wikipedia.org/wiki/Resolution_(Logik)
% https://en.wikipedia.org/wiki/Resolution_(logic)

% https://de.wikipedia.org/wiki/Schlussregel
% https://de.wikipedia.org/wiki/Abduktion

%---------------------------------------------------------------------------------------------------
\subsection{Induction}
Proof by induction is a proof technique that can be used to prove a statement $\varphi(n)$ that should hold for any natural number $n$ greater or equal to some given $n_0$ (which is often $0$ or $1$). The technique works by first showing directly that the statement holds true for $n_0$. It then goes on by showing that whenever the statement holds for some $n$, then it automatically follows that it must also hold for $n+1$. It then follows that $\varphi(n)$ holds for any $n \geq n_0$.

\paragraph{Example: Gauss's summation formula} A well known formula for the sum over all natural numbers up to some given $n$ is $\sum_{k=1}^n k = \frac{n (n+1)}{2}$. We will use induction to prove that the formula works for every $n \in \mathbb{N}$. We need to first show it directly for $n=1$. We have $\sum_{k=1}^1 k = 1$ and we have $\frac{1 (1+1)}{2} = 1$, so the formula works for $n=1$. ...TBC...

% https://en.wikipedia.org/wiki/Mathematical_induction




%===================================================================================================
\section{Higher Order Logic}
Propositional logic and predicate logic is also called zeroth and first order logic respectively indicating a hierarchy of logical systems. The position in this hierarchy determined by what the variables that appear in the formulas stand for. That is formalized in the idea of the \emph{domain of discourse} which is the universe of things that we can talk about.

\medskip
In 0th order logic, i.e. propositional logic, the variables just stand for truth values. That means, any variable $x$ that appears in a formula can be assigned either true or false and that's it. Variables have no inner structure. Our domain of discourse is just the relation between logical formulas themselves - like is one formula equivalent to another, does one imply the other, etc. We cannot yet talk about objects from the world, so to speak. ...TBC...

%For example, the formula $(x \rightarrow y) \wedge x$ implies the much simpler formula $y$.

\medskip
In 1st order logic, the variables that were formerly just atomic truth values now become predicates over real world objects like numbers, sets, set elements, etc. We can now express things like "There exists a number such that...". We use quantifiers like $\exists$ and $\forall$ to introduce a variable. ...TBC...

\medskip
In 2nd order logic, we can additionally quantify over predicates. For example, we could write down the law of the excluded middle as: $\forall \varphi \forall x: (\varphi(x) \vee \neg \varphi(x))$. The formula says that for any predicate $\varphi$ and for any variable $x$, either $\varphi(x)$ or $\neg \varphi(x)$ is true\footnote{Shouldn't the law use an exclusive or, i.e. say $\varphi(x) \xor \neg \varphi(x)$? I mean, the use of inclusive or does not render the statement wrong - but it doesn't seem to express that $\varphi$ and $\neg \varphi$ are mutually exclusive. Figure out!}.

% -bound variables

%we get variables that stand for sets or set elements and we can quantify over such variables by means of quantifiers like $\exists$ and $\forall$.

% Propositional logic deal with propositions and the relations between them. 




% https://en.wikipedia.org/wiki/Atomic_formula
% https://en.wikipedia.org/wiki/Propositional_variable

% https://en.wikipedia.org/wiki/Domain_of_discourse

% https://en.wikipedia.org/wiki/Propositional_calculus

% https://en.wikipedia.org/wiki/First-order_logic
%  In propositional logic, these sentences themselves are viewed as the individuals of study, and might be denoted, for example, by variables such as p and q. They are not viewed as an application of a predicate

%https://en.wikipedia.org/wiki/Second-order_logic
% First-order logic quantifies only variables that range over individuals (elements of the domain of discourse); second-order logic, in addition, quantifies over relations.

%https://en.wikipedia.org/wiki/Higher-order_logic
%https://en.wikipedia.org/wiki/Type_theory

\begin{comment}

% For the logic chapter that should come before this chapter:
%\begin{eqnarray}
%\forall x : \varphi(x) \Leftrightarrow \neg \exists x : \neg \varphi(x) \\
%\exists x : \varphi(x) \Leftrightarrow \neg \forall x : \neg \varphi(x)
%\end{eqnarray}
% Equations between quantors

% https://en.wikipedia.org/wiki/Theory_(mathematical_logic)#First-order_theories
% https://en.wikipedia.org/wiki/Method_of_analytic_tableaux
% https://en.wikipedia.org/wiki/Resolution_(logic)
% https://en.wikipedia.org/wiki/G%C3%B6del%27s_completeness_theorem
% https://en.wikipedia.org/wiki/List_of_first-order_theories

https://en.wikipedia.org/wiki/Model_theory


https://en.wikiversity.org/wiki/Logical_implication
https://simple.wikipedia.org/wiki/Implication_(logic)
https://en.wikipedia.org/wiki/Material_conditional

https://en.wikipedia.org/wiki/Logical_consequence

\end{comment}