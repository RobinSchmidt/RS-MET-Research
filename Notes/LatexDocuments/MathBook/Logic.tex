\chapter{Logic}


%===================================================================================================
\section{Propositional Logic}

% -Connectives (and, or, not, nand, xor, ...)


%===================================================================================================
\section{Predicate Logic}

% -also known as first order logic
% -Variables, Formulas
% -Quantors: \exists, \forall, \exists ! (it exists exactly one)


%===================================================================================================
\section{Proof Techniques}


% https://jdhsmith.math.iastate.edu/class/BookOfProof.pdf
% -Really good free ebook

%---------------------------------------------------------------------------------------------------
\subsection{Deduction}
Logical \emph{deduction} is the process of deriving new true statements from given true statements. In this context, the given statements are called \emph{premises} and the derived statements are called \emph{conclusions}. The process of deriving new statements from known ones is called \emph{inference} and there are certain rules that we need to obey to draw valid conclusions from a given set of premises.

% https://de.wikipedia.org/wiki/Deduktion

% https://home.uni-leipzig.de/methodenportal/deduktion_induktion/
% -Deduktion schließt vom allgemeinen auf das spezielle
% -Induktion schließt vom speziellen auf das allgemeine

\subsubsection{Rules of Inference}
% https://en.wikipedia.org/wiki/Rule_of_inference

\paragraph{Modus Ponens}
% https://en.wikipedia.org/wiki/Modus_ponens

\paragraph{Modus Tollens}
% https://en.wikipedia.org/wiki/Modus_tollens

\paragraph{Contraposition}
% https://en.wikipedia.org/wiki/Contraposition

\paragraph{Resolution}
% -Resolution
%  https://de.wikipedia.org/wiki/Resolution_(Logik)
% https://en.wikipedia.org/wiki/Resolution_(logic)

%---------------------------------------------------------------------------------------------------
\subsection{Induction}
Proof by induction is a proof technique that can be used to prove a statement $\varphi(n)$ that should hold for any natural number $n$ greater or equal to some given $n_0$ (which is often $0$ or $1$). The technique works by first showing directly that the statement holds true for $n_0$. It then goes on by showing that whenever the statement holds for some $n$, then it automatically follows that it must also hold for $n+1$. It then follows that $\varphi(n)$ holds for any $n \geq n_0$.

\paragraph{Example: Gauss's summation formula} A well known formula for the sum over all natural numbers up to some given $n$ is $\sum_{k=1}^n k = \frac{n (n+1)}{2}$. We will use induction to prove that the formula works for every $n \in \mathbb{N}$. We need to first show it directly for $n=1$. We have $\sum_{k=1}^1 k = 1$ and we have $\frac{1 (1+1)}{2} = 1$, so the formula works for $n=1$. ...TBC...

% https://en.wikipedia.org/wiki/Mathematical_induction





\begin{comment}

% For the logic chapter that should come before this chapter:
%\begin{eqnarray}
%\forall x : \varphi(x) \Leftrightarrow \neg \exists x : \neg \varphi(x) \\
%\exists x : \varphi(x) \Leftrightarrow \neg \forall x : \neg \varphi(x)
%\end{eqnarray}
% Equations between quantors

% https://en.wikipedia.org/wiki/Theory_(mathematical_logic)#First-order_theories
% https://en.wikipedia.org/wiki/Method_of_analytic_tableaux
% https://en.wikipedia.org/wiki/Resolution_(logic)
% https://en.wikipedia.org/wiki/G%C3%B6del%27s_completeness_theorem
% https://en.wikipedia.org/wiki/List_of_first-order_theories

% https://en.wikipedia.org/wiki/Second-order_logic
% https://en.wikipedia.org/wiki/Higher-order_logic

\end{comment}