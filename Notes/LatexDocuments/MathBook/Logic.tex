\chapter{Logic}


%===================================================================================================
\section{Propositional Logic}
Propositional logic deals with statements that can be either true or false. These statements are called propositions. The propositions are typically represented by variable symbols such as $p$ and $q$. For example, a proposition could be defined as $p = \text{"It is hot outside"}$ and another one could be $q = \text{"Chocolate melts in the heat"}$ and yet another one could be $r = \text{"The chocolate is outside"}$ and a last one could be $s = \text{"The chocolate is melting"}$. Propositional logic is concerned with formalizing processes like concluding $s$ from $p,q,r$. The machinery is the set of multivariate functions from the set $\{0,1\}$ to itself. So, $0$ and $1$ are the two possible values that our variables like $p,q,\ldots$ can take on and we interpret $1$ as "the proposition is true" and $0$ as "the proposition is false". In the context of logic, the constants $0$ and $1$ are also often denoted as $\bot$ and $\top$ respectively. In this case, they are sometimes read as "bottom" and "top" respectively - but you can also just read them as "false" and "true", if you prefer that (I do). The symbol $\top$ for "true" (or "top") even looks like the letter T, so it's easy to remember which is which. We want plug such logical values into logical functions which take one or more logical values as inputs and spit out another such logical value which is also either $0$ or $1$. We'll denote the set $\{0,1\}$ (or, alternatively $\{\bot, \top\}$) as $\mathbb{B}$ in honor to George Boole\footnote{George Boole: English mathematician, 1815-1864, Inventor of Boolean algebra}. We are concerned with functions $f$ of the type $f: \mathbb{B}^n \rightarrow \mathbb{B}$. We'll typically build such multivariate Boolean functions from a couple of basic bivariate Boolean functions of the type $f: \mathbb{B} \times \mathbb{B} \rightarrow \mathbb{B}$ and possibly the only interesting univariate Boolean function, namely the negation which maps $0$ to $1$ and $1$ to $0$.  Propositional logic is also known as \emph{zeroth order logic} indicating that there is a hierarchy of logical systems with various levels. Indeed, one level up, we'll find predicate logic which is also known as first order logic. 

%%...TBC...introduce symbols for true and false $\top, \bot$ - "top" and "bottom"

%  Propositional logic is the foundation of all higher order

% https://en.wikipedia.org/wiki/Boolean_algebra

%---------------------------------------------------------------------------------------------------
\subsection{Connectives}
Logical connectives are symbols that are used to create more complex logical expressions from simpler ones. They work with logical variables like the basic arithmetic operators do with numerical variables. They typically take two inputs and produce one output\footnote{The only exception is the negation which takes only a single input. It's somewhat analogous to the unary minus in arithmetic.}. Like arithmetic operators, we write them in infix notation between the variables. A non-exhaustive table of such connectives is given here:

\medskip
\begin{tabular}{c l l}
\label{Tab:LogicConnectives}
  Symbol           & Name                  & Meaning      \\
  $\neg p$         & Negation              & not $p$       \\
  $p \wedge q$     & Conjunction           & $p$ and $q$    \\
  $p \vee q$       & Disjunction           & $p$ or $q$ (maybe both)   \\
  $p \then q$      & Material Implication  & $p$ implies $q$, if $p$ then $q$   \\  
  $p \mequiv q$    & Material Equivalence  & $p$ if and only if $q$, $p$ iff $q$   \\    
  $p \xor q$       & Exclusive or, Xor     & either $p$ or $q$ but not both   \\ 
  $p \nand q$      & Non-Conjunction, Nand & Not both, Not ($p$ and $q$)   \\   
\end{tabular}
\medskip

As said, the list is not exhaustive. There are 16 possible different functions $f: \mathbb{B}^2 \rightarrow \mathbb{B}$ and we have listed only the $6$ most commonly used ones of them. We do not need a symbol for all of them because we can construct the missing ones from those in the table. In fact, we don't even need all of the connectives given in the table. A set of connectives from which all possible Boolean functions can be constructed by a suitable formula is called a \emph{functionally complete} set of connectives. For example, the set $\{\neg, \wedge, \vee\}$ is functionally complete. It is actually even overcomplete already and we could make do with  $\{\neg, \wedge\}$ but having the $\vee$ available as well makes it more convenient to write down and read(!) logical formulas. As an extreme example, the singleton set $\{\nand\}$ containing only the nand function is also functionally complete. It's not fun to read or write logical formulas when only the nand operator is permitted - but it is possible to express every possible logical formula solely in terms of nand. This fact has some technological relevance in the realm of chip design.

...TBC..TODO: give truth tables for all of them, explain relation between $\then$ and $\Rightarrow$ and $\mequiv$ and $\Leftrightarrow$. The double-stroke arrow symbols are used on the meta level whereas the single-stroke arrows are used on the object level or formula level. There are also $\vdash$ and $\vDash$ and also have meanings similar to $\then$ and $\Rightarrow$. Explain the subtle differences. Give other names for the connectives - there are a lot of akas

% https://en.wikipedia.org/wiki/Functional_completeness
% https://en.wikipedia.org/wiki/Sheffer_stroke  aka nand
% https://en.wikipedia.org/wiki/Exclusive_or
% https://en.wikipedia.org/wiki/Logical_connective
% https://en.wikipedia.org/wiki/Logical_equivalence#Relation_to_material_equivalence
% https://en.wikiversity.org/wiki/Logical_implication
% https://en.wikipedia.org/wiki/Logical_biconditional
% -explain difference between material conditional and logical implication
% -explain difference between material equivalence and logical equivalence/biconditional
%  material biconditional or equivalence or biimplication or bientailmen

% What about constant functions that always output 0 or 1

% How to Read Logic
% https://www.youtube.com/watch?v=JVjv0CCzHok

% Can we find a pair of logical connectives that behave like addition and multiplication of numbers,
% i.e. obey all the required laws (assiciativity, etc.)?

%---------------------------------------------------------------------------------------------------
\subsection{Boolean Algebra}
It turns out that some of the logical connectives obey certain rules of calculation that we are familiar with from the algebraic operations over numbers. Specifically, the "and" operation behaves similar to multiplication and the "or" operation behaves similar to addition. ...TBC... list rules: associativity, commutativity, distributivity (both operations distribute over the other - that's an important difference to "normal" algebra), de Morgan's laws, neutral and absorbing elements, etc.

% p \vee q = q \vee q,  p \wedge q = q \wedge q
% p \vee 0 = p   Neutral element of disjunction
% p \vee 1 = 1   Absorbing element of disjunction
%
% p + (q * r) = (p + q) * (p + r)
% p + p = p  Idempotence
% p * p = p  Idempotence
% (p + q) * p = p
% (p * q) + p = p
% ....see ABoAA pg 9-10




%---------------------------------------------------------------------------------------------------
\subsection{Deduction}
Logical \emph{deduction} is the process of deriving new true statements from given true statements. In this context, the given statements are called \emph{premises} and the derived statements are called \emph{conclusions}. The process of deriving new statements from known ones is called \emph{inference} and there are certain rules that we need to obey to draw valid conclusions from a given set of premises. When we write a statement like $p,q \,\vdash\, r$ then this means that if $p$ and $q$ on the left hand side are both true propositions, then the proposition $r$ on the right hand side is also true. In this notation, $p,q$ are the premises and $r$ is the conclusion. The symbol $\vdash$ is called the "turnstile" or "tee" and means something like "proves", "entails", "yields"  ...TBC...

% A logical deduction formula writes down the (comma separated) premises on the left hand side of
% the tee symbol and the conclusion on the right hand side

% https://en.wikipedia.org/wiki/Turnstile_(symbol)
% or propositional logic, it may be shown that semantic consequence and derivability are equivalent

% https://de.wikipedia.org/wiki/Deduktion

% https://home.uni-leipzig.de/methodenportal/deduktion_induktion/
% -Deduktion schließt vom allgemeinen auf das spezielle
% -Induktion schließt vom speziellen auf das allgemeine

% https://en.wikipedia.org/wiki/Logical_consequence
% https://en.wikipedia.org/wiki/Propositional_calculus

% Intro To Math Proofs (Full Course)
% https://www.youtube.com/watch?v=3czgfHULZCs
% 3:45 - kinds of statements we want to prove: equalities, (non) existence, uniqueness, 
% (mutual) implications, properties (for a given object or for all objects of a given kind)
% 11:15 - logical rules

\subsubsection{Rules of Inference}

% https://en.wikipedia.org/wiki/Rule_of_inference
% https://en.wikipedia.org/wiki/List_of_rules_of_inference
% https://en.wikipedia.org/wiki/Modus_ponens
% https://en.wikipedia.org/wiki/Modus_tollens
% https://en.wikipedia.org/wiki/Contraposition

\medskip
\begin{tabular}{c l l}
\label{Tab:RulesOfInference}
  Formula          & Name                  & Meaning      \\
  
  $p, p \then q \;\vdash\; q$                                  & 
  Modus Ponens                                                  & 
  If $p$ is true and $p$ implies $q$, then $q$ is true          \\   
  
  $\neg q, p \then q \;\vdash\; \neg p$                        & 
  Modus Tollens                                                 & 
  If $q$ is false and $p$ implies $q$ then $p$ is false         \\     
  
  $p \then q \;\vdash\; \neg q \then \neg p$                   & 
  Contraposition                                                & 
  If $p$ implies $q$ then not $q$ implies not $p$               \\   
  
  $\neg p, p \vee q \;\vdash\; q$                              & 
  Disjunctive Syllogism                                         & 
  If $p$ is false and $p$ or $q$ is true then $q$ is true       \\ 
  
  $p \then q,  q \then r \;\vdash\; p \then r$                 & 
  Hypothetical Syllogism                                        & 
  If $p$ implies $q$ and $q$ implies $r$ then $p$ implies $r$   \\   
  
  $p,q \;\vdash\; p \wedge q$                                  & 
  Conjunction Introduction                                      & 
  If $p$ is true and $q$ is true then $p$ and $q$ is true       \\ 
  
  $p \;\vdash\; p \vee q$                                      & 
  Disjunction Introduction                                      & 
  If $p$ is true then $p$ or $q$ is true                        \\    
  
  $p \wedge q \;\vdash\; p$                                    & 
  Simplification                                                & 
  If $p$ and $q$ are true then $p$ is true                      \\    
  
  $p \then q, p \then \neg q \;\vdash\; \neg p$                & 
  Reductio ad Absurdum                                          & 
  If $p$ implies $q$ and $q$s negation then $p$ is false         \\ 
  
  $p, \neg p \;\vdash\; q$                                     & 
  Deductive Explosion                                           & 
  From a contradiction follows anything                         \\   
  
  $p \then r, q \then r, p \vee q \;\vdash\; r$                & 
  Case Analysis                                                 & 
  If $p$ and $q$ both imply $r$ and $p$ or $q$ then $r$         \\ 
  
  $p \then r, q \then s, p \vee q \;\vdash\; r \vee s$         & 
  Constructive Dilemma                                          & 
  If $p$ implies $r$, $q$ implies $s$ and $p$ or $q$ then $r$ or $s$         \\ 
  
  $p \then q, q \then p \;\vdash\; p \mequiv q$                & 
  Biconditional Introduction                                    & 
  From mutual implication follows equivalence                   \\   
  
  $p \mequiv q, p \vee q \;\vdash\; p \wedge q$                & 
  Biconditional Elimination                                     & 
  If $p,q$ are equiv. and one is true then both are         \\     
  
\end{tabular}
\medskip

% Syllogism:

%Not yet complete!
% https://en.wikipedia.org/wiki/List_of_rules_of_inference#Rules_for_propositional_calculus


% https://en.wikipedia.org/wiki/Natural_deduction
% https://en.wikipedia.org/wiki/List_of_logic_symbols
% I think, we should use the \turnstile symbol instead of \Rightarrow



%---------------------------------------------------------------------------------------------------
%\subsection{Induction and Abduction}



% Sherlock Holmes NEVER 'Deduced' Anything
% https://www.youtube.com/watch?v=MJrYwh6WyF8  by Another Roof
% -in an implication p -> q, q is called antecedent and p the consequence. We may also call it 
%  premise and conclusion
% -Induction is used in scientific reasoning - one speculates about general rules from specific
%  observations. 
% -Abduction tries to find the most plausible explanation for an observation






%===================================================================================================
\section{Predicate Logic}


%\subsection{Notation}

\subsection{Quantifiers}

$\exists, \forall, \nexists, \exists!, \exists_1$
% existential, universal

% -also known as first order logic
% -Variables, Formulas
% -Quantors: \exists, \forall, \exists ! (it exists exactly one)

% https://en.wikipedia.org/wiki/Quantifier_(logic)
% https://de.wikipedia.org/wiki/Quantor
% $\neg (\forall x, p) = \exists x \neg p$

% Intro To Math Proofs (Full Course)
% https://www.youtube.com/watch?v=3czgfHULZCs
% 34:10 - some quantifier formulas

%===================================================================================================
\section{Proof Techniques}


% https://jdhsmith.math.iastate.edu/class/BookOfProof.pdf
% -Really good free ebook





\paragraph{Resolution}
% -Resolution
%  https://de.wikipedia.org/wiki/Resolution_(Logik)
% https://en.wikipedia.org/wiki/Resolution_(logic)

% https://de.wikipedia.org/wiki/Schlussregel
% https://de.wikipedia.org/wiki/Abduktion

%---------------------------------------------------------------------------------------------------
\subsection{Induction}
Proof by induction is a proof technique that can be used to prove a statement $\varphi(n)$ that should hold for any natural number $n$ greater or equal to some given $n_0$ (which is often $0$ or $1$). The technique works by first showing directly that the statement holds true for $n_0$. It then goes on by showing that whenever the statement holds for some $n$, then it automatically follows that it must also hold for $n+1$. It then follows that $\varphi(n)$ holds for any $n \geq n_0$.

\paragraph{Example: Gauss's summation formula} A well known formula for the sum over all natural numbers up to some given $n$ is $\sum_{k=1}^n k = \frac{n (n+1)}{2}$. We will use induction to prove that the formula works for every $n \in \mathbb{N}$. We need to first show it directly for $n=1$. We have $\sum_{k=1}^1 k = 1$ and we have $\frac{1 (1+1)}{2} = 1$, so the formula works for $n=1$. ...TBC...

% https://en.wikipedia.org/wiki/Mathematical_induction


% Induction on Real Numbers
% https://www.youtube.com/watch?v=bXTbGJxW_fE
% https://arxiv.org/pdf/1208.0973

% The Magic of Induction - Numberphile
% https://www.youtube.com/watch?v=DhZORrqL3xI
% -Induction: hypothesis: assume that a statement holds for a given natural number 1. step:
%  show that the statement being true for any n implies that it also holds for n+1. Example:
%  sum_{k=1}^n k = (n^2 + n) / 2 (Gauss's summation formula)
% -Strong (or complete) induction: in the iduction hypthesis, don't just assume the statement to 
%  hold for a given n but for all k <= n. Example: proof that every natural number n >= 2 has a 
%  prime divisor.
% -Well foundedness: the n|m ("n divides m") relation is "well founded" on the natural numbers. The
%  foundation are all the prime numbers in this case.
% -Computing factorials: Python has a maximum stack depth limit of 1000, so computing 
%  factorial(1000) will give a stack overflow for the standard recursive factorial implementaion.
%  But using an implementation that splits the interval into two halves, computes the products
%  of these two halves and then multiplies them, gives a tree-shaped recursive structure that is
%  much less deep. Here, the foundation are th tree leaves.

%===================================================================================================
\section{Higher Order Logic}
Propositional logic and predicate logic are also called zeroth and first order logic respectively indicating a hierarchy of logical systems. The position in this hierarchy determined by what the variables that appear in the formulas stand for. That is formalized in the idea of the \emph{domain of discourse} which is the universe of things that we can talk about.

\medskip
In 0th order logic, i.e. propositional logic, the variables just stand for truth values. That means, any variable $x$ that appears in a formula can be assigned either true or false and that's it. Variables have no inner structure. Our domain of discourse is just the relation between logical formulas themselves - like is one formula equivalent to another, does one imply the other, etc. We cannot yet talk about objects from the world, so to speak. ...TBC...

%For example, the formula $(x \rightarrow y) \wedge x$ implies the much simpler formula $y$.

\medskip
In 1st order logic, the variables that were formerly just atomic truth values now become predicates over real world objects like numbers, sets, set elements, etc. We can now express things like "There exists a number such that...". We use quantifiers like $\exists$ and $\forall$ to introduce a variable. ...TBC...

\medskip
In 2nd order logic, we can additionally quantify over predicates. For example, we could write down the law of the excluded middle as: $\forall \varphi \forall x: (\varphi(x) \vee \neg \varphi(x))$. The formula says that for any predicate $\varphi$ and for any variable $x$, either $\varphi(x)$ or $\neg \varphi(x)$ is true\footnote{Shouldn't the law use an exclusive or, i.e. say $\varphi(x) \xor \neg \varphi(x)$? I mean, the use of inclusive or does not render the statement wrong (in fact, it turns it into a tautology) - but it doesn't seem to express that $\varphi$ and $\neg \varphi$ are mutually exclusive which is clearly the intended meaning when calling it law of the \emph{excluded} middle. Figure out!}.

% The "middle" in this context could either mean: "p is neither true nor false" or it could mean: "p is true and false at the same time". The formula using the inclusive or onyl excludes the first of these possibilities, I think.

% https://www.youtube.com/watch?v=niqqm1DRTkE
% Mentions law of excluded middle at 6:20

% -bound variables

%we get variables that stand for sets or set elements and we can quantify over such variables by means of quantifiers like $\exists$ and $\forall$.

% Propositional logic deal with propositions and the relations between them. 




% https://en.wikipedia.org/wiki/Atomic_formula
% https://en.wikipedia.org/wiki/Propositional_variable

% https://en.wikipedia.org/wiki/Domain_of_discourse

% https://en.wikipedia.org/wiki/Propositional_calculus

% https://en.wikipedia.org/wiki/First-order_logic
%  In propositional logic, these sentences themselves are viewed as the individuals of study, and might be denoted, for example, by variables such as p and q. They are not viewed as an application of a predicate

%https://en.wikipedia.org/wiki/Second-order_logic
% First-order logic quantifies only variables that range over individuals (elements of the domain of discourse); second-order logic, in addition, quantifies over relations (and predicates?).

% Or (maybe) in 1st order logic, we have predicates over atoms/individuals/variables, in 2nd order logic we have predicates over 1st order predicates, in 3rd order logice predicates over 2nd order predicates and so on?
% See https://www.youtube.com/watch?v=Ws_E5jhqDVQ at 12:53

%https://en.wikipedia.org/wiki/Higher-order_logic
%https://en.wikipedia.org/wiki/Type_theory

%===================================================================================================
\section{Other Logical Systems}

\subsection{Modal Logic}
% https://en.wikipedia.org/wiki/Modal_logic

\subsection{Temporal Logic}
% https://en.wikipedia.org/wiki/Temporal_logic

\subsection{Three Valued Logic}
% https://en.wikipedia.org/wiki/Many-valued_logic

\subsection{Fuzzy Logic}
% https://en.wikipedia.org/wiki/Fuzzy_logic


\begin{comment}

% For the logic chapter that should come before this chapter:
%\begin{eqnarray}
%\forall x : \varphi(x) \Leftrightarrow \neg \exists x : \neg \varphi(x) \\
%\exists x : \varphi(x) \Leftrightarrow \neg \forall x : \neg \varphi(x)
%\end{eqnarray}
% Equations between quantors

% https://en.wikipedia.org/wiki/Theory_(mathematical_logic)#First-order_theories
% https://en.wikipedia.org/wiki/Method_of_analytic_tableaux
% https://en.wikipedia.org/wiki/Resolution_(logic)
% https://en.wikipedia.org/wiki/G%C3%B6del%27s_completeness_theorem
% https://en.wikipedia.org/wiki/List_of_first-order_theories

https://en.wikipedia.org/wiki/Model_theory


https://en.wikiversity.org/wiki/Logical_implication
https://simple.wikipedia.org/wiki/Implication_(logic)
https://en.wikipedia.org/wiki/Material_conditional

https://en.wikipedia.org/wiki/Logical_consequence

https://en.wikipedia.org/wiki/Modal_logic

A hypothesis about the psychology of  following a proof and being convinced. I observed that on myself. With every step along the proof where some theorem is being applied, I feel a sense of uncertainty: Does the theorem really apply here? Is it really applied correctly? are all the edge cases covered? So - if one does that n times along the proof and at every step, you are only 90% sure, then at the end you'll only be (0.9)^n * 100 % certain, that everything being correct - and that might be a small number, if n is big, i.e. there are many steps. Of course, in an ideal world, you are 100% sure at every step....tbc...

So, the bottom line is: a long proof (high n) with very complicated steps (steps with low confidence) often may leave you with a feeling of being unilluminated. Proofs are important for theory - but not so much from pedagogical point of view.

How to unify logic & arithmetic
https://www.youtube.com/watch?v=niqqm1DRTkE

Set Theory Part 1: Logic
https://www.youtube.com/watch?v=kOm15NAOZIM

Der Begriff des Modells in der Logik
https://www.youtube.com/watch?v=nfWzx10EPeM



https://de.wikipedia.org/wiki/Existential_Graphs

Raum und Wissen: Elemente einer Theorie epistemischen Diagrammgebrauchs (Jan Wöpking)
https://www.amazon.de/Raum-Wissen-epistemischen-Diagrammgebrauchs-Knowledge/dp/3110441667


\end{comment}