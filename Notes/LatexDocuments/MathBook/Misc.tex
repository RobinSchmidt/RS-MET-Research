\chapter{Misc Math Topics}

\section{Beautiful Formulas}

\paragraph{Euler's Formula}
The following formula:
\begin{equation}
\e^{\i \pi} + 1 = 0
\end{equation}
has been called by many mathematicians the most beautiful formula of all of mathematics because it establishes a connection between the 5 most important mathematical constants $0,1,\i,\pi,\e$. Furthermore, it makes use of each of the 3 important mathematical operations addition, multiplication, exponentiation...TBC...

% Special case of
% 0,1,i: algebra, pi: geometry, e: analysis/calculus

\paragraph{Euler Product}
For a given complex valued parameter $s$ with real part greater than one the following equation holds:
\begin{equation}
\sum_{n=1}^{\infty} \frac{1}{n^s} = 
\prod_{p \in \mathbb{P}} \frac{1}{1 - p^{-s}}
\end{equation}
The real part must be greater than one in order to have convergence of the sum on the left hand side. I think the product on the right hand side converges in a namely domain, namely for $\Re(s) > 0$ [VERIFY!]. The formula establishes a relation between a sum over \emph{all} natural numbers with a product over only the \emph{prime} numbers. The sequence of all (positive) natural numbers $(1,2,3,4,\ldots)$ is very simple and well understood but the sequence of prime numbers $(2,3,5,7,11,13,17\ldots)$ is very mysterious. Therefore, establishing a connection between a very simple sequence of numbers and a more complicated one may help to gain insights into the more complicated sequence. We can also interpret both expressions as defining a function in the domain where they converge. This function of a complex variable $s$ is an analytic function and known as the Riemann zeta function and denoted by $\zeta(s)$. In those parts of the complex domain where neither se sum nor the product converges, the function can be defined via analytic continuation. Of special interest are the zeros of the function...TBC...

\paragraph{Generalized Stokes Theorem}

