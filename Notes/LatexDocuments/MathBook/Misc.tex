\chapter{Misc Math Topics}

\section{Beautiful Formulas}
Mathematics is not just super useful in science and engineering but also very beautiful in it own right. In fact, the practical and useful aspects of it are sometimes sneered upon by (pure) mathematicians. They do it not for its practical applications but purely for their intellectual enjoyment of their ivory tower level mind blows. Math is full of formulas and some of them have inspired more awe than others. The following is a collection of certain formulas that have been called to be particularly beautiful. Such an assessment is always somewhat subjective but I think there are also some objective criteria for what makes a particular formula to be considered beautiful. It should be profound and far reaching, i.e. have huge implications, give great insights. It shouldn't be too messy to write down - ideally it would be simple to state but somewhat difficult and/or non-obvious to derive. It may reveal unexpected connections - maybe even between very different fields of mathematics. ...TBC...

% Maybe make the section a Famous Math Bits and the "Beautiful Formulas" a subsection. Maybe have also a subsection with important formulas (formulas that have a lot of practical use but are not necessarily beautiful), famous theorems, famous conjectures

\paragraph{Euler's Formula}
The following formula:
\begin{equation}
\e^{\i \pi} + 1 = 0
\end{equation}
has been called by many mathematicians the most beautiful formula of all of mathematics because it establishes a connection between the 5 most important mathematical constants $0,1,\i,\pi,\e$. Furthermore, it makes use of each of the 3 important mathematical operations addition, multiplication, exponentiation. The 5 numbers connect different fields of mathematics: $0,1$ and $\i$ come from the field of algebra, the semicircle constant $\pi$ comes from geometry and Euler's number $\e$ from calculus. ...TBC...

% Special case of
% 0,1,i: algebra, pi: geometry, e: analysis/calculus

% algebra, geometry, calculus

\paragraph{Euler Product}
For a given complex valued parameter $s$ with real part greater than one, the following equation holds:
\begin{equation}
\sum_{n=1}^{\infty} \frac{1}{n^s} = 
\prod_{p \in \mathbb{P}} \frac{1}{1 - p^{-s}}
\end{equation}
The real part must be greater than one in order to have convergence of the sum on the left hand side. I think the product on the right hand side converges in a larger domain, namely for $\Re(s) > 0$ [VERIFY!]. The formula establishes a relation between a sum over \emph{all} natural numbers with a product over only the \emph{prime} numbers. The sequence of all (positive) natural numbers $(1,2,3,4,\ldots)$ is very simple and well understood but the sequence of prime numbers $(2,3,5,7,11,13,17\ldots)$ is very mysterious. Therefore, establishing a connection between a very simple sequence of numbers and a more complicated one may help to gain insights into the more complicated sequence. We can also interpret both expressions as defining a function in the domain where they converge. This function of a complex variable $s$ is an analytic function and known as the Riemann zeta function and denoted by $\zeta(s)$. In those parts of the complex plane where neither the sum nor the product converges, the function can be defined via analytic continuation (except at the point $s=1$ where the function has a simple(?) pole). Of special interest are the zeros of the function because their placement has important implications for the distribution of prime numbers. More specifically, one is interested in the so called non-trivial zeros. They are known to lie in the so called critical strip where the real part is between zero and one. The famous Riemann hypothesis states that they all have a real part of exactly one half. The Riemann zeta function is the most important object in analytic number theory [VERIFY] ...TBC...

% Number theory

\paragraph{Generalized Stokes Theorem}
Given a manifold $M$ and a differential form $\omega$ defined on the manifold with exterior derivative $d \omega$, the following relation holds:
\begin{equation}
\int_{M} d \omega = \int_{\partial M} \omega
\end{equation}
The left hand side is an integral over the whole manifold and the integrand is the exterior derivative of the differential form. The right hand side is an integral over the boundary of the manifold and the integrand is the differential form $\omega$ itself. This equation is a vast generalization of all the integral theorems of vector calculus. It also includes the fundamental theorem of calculus as a very simple special case as well. Without much exaggeration, one could say that it encapsulates the essence of calculus. ...TBC...

% Calculus

% I think, it also encompasses soem integral formulas from complex analysis

\paragraph{Fundamental Theorem of Geometric Calculus}
The generalized Stokes theorem can be generalized even further when broadening our perspective from exterior to geometric calculus. ...TBC...

% We now must consider the geometric derivative...

\paragraph{Euler's Polyhedron Formula}
If you take any polyhedron and denote by $V$ the number of vertices, by $E$ the number of edges and by $F$ the number of faces then you will always find that:
\begin{equation}
V - E + F = 2
\end{equation}

% It applies also to planar graphs if you replace the faces with the regions/areas that are bounded by the edges and include the area outside the graph in the count.


% If you take more general geometric shapes in 3D by allowing holes in them to produce something like faceted tori, then you will find that the sum $V - E + F$ will be related to the number of holes. It's a topological invariant...

% https://en.wikipedia.org/wiki/Euler_characteristic

\paragraph{Gauss's Theorema Egregium}

% https://en.wikipedia.org/wiki/Theorema_Egregium
% https://de.wikipedia.org/wiki/Theorema_egregium

% Langland's Correspondence

\begin{comment}

The most beautiful equation in math.
https://www.youtube.com/watch?v=u7BadbVwnl4&lc=Ugxn_M9K5WiVds15A-Z4AaABAg
...has a comment on what all these equations are:
from left to right, top to bottom: Euler's identity, Navier–Stokes equation, (generalized) Stokes's theorem, Riemann–Roch theorem, Euler's factorization of the Riemann zeta function, Gauss–Bonnet theorem, ???, Hardy–Ramanujan–Rademacher partition formula

Try to find the most beautiful formulas for different ares of math:
-Abstract algebra: Lagrange's theorem (group theory), Fundamental theorem of Glaois Theory?
-Algebraic geometry: Bezout's theorem,
-Set theory: Continuum Hypothesis,
-Statistics: Central limit theorem / Gaussian distribution
-Calculus of Variations: Euler-Lagrange equation
-Tensor calculus:
-Linear algebra: SVD? rank-nullity theorem? Something about eigenvalues? the fact that a
 matrix is a zero of its own characteristic polynomial?
-Differential geometry:
-Topology: Gauss-Bonnet equation (relates topology and differential geometry)
-Functional analysis:
-functional equation for exp or log?
-Langlands correspondence?

-Euler's polyhedrrn formula:
 https://ics.uci.edu/~eppstein/junkyard/euler/
 applies also to planar graphs
-https://en.wikipedia.org/wiki/Chinese_remainder_theorem 
-More elementary: functional eq of exp, Pythagoras' themorem
-Fundamental theorem of arithmetic - maybe its generalization to factor ideals into prime ideals?

https://www.reddit.com/r/math/comments/z3yqh/what_are_the_most_important_theorems_in_all_of/

https://www.scientificamerican.com/article/these-are-the-most-beautiful-equations-in-mathematics/

https://www.livescience.com/57849-greatest-mathematical-equations.html

10 der beeindruckendsten Formeln der Mathematik
https://www.youtube.com/watch?v=uTn5av4VS5E


Chaos: The Mathematics of the Butterfly Effect
https://www.youtube.com/watch?v=TX5S6tS7jro
-A function f(x) is topologically transitive, if we can reach any region from any other region by
 repeated application of the function.


Do Numbers Exist? - Marcus du Sautoy
https://www.youtube.com/watch?v=vCIAcOJFdiU


\end{comment}