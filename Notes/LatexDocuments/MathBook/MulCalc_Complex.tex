\section{Complex Calculus}
Complex calculus is the calculus of functions that map a complex number to another complex number. Since input and output are both 2-dimensional, it makes sense to look at complex calculus from the perspective of 2D vector fields. For a function $w = f(z) = u(x,y) + \i v(x,y)$ to be differentiable in the complex sense, the partial derivatives of the component functions $u,v$ with respect to $x,y$ must exist and satisfy the Cauchy-Riemann (CR) differential equations: $u_x = v_y, u_y = -v_x$. This condition has wide ranging implications which allow a lot of simplifications to be made in the computations of path integrals. This can also help to evaluate certain definite real integrals that would otherwise be hard to do. If the function $f$ is considered as a mapping that maps points from the $z$-plane to the $w$-plane, the CR equations imply that the images of curves in the $z-$plane that intersect at a given angle, will intersect at the same angle in the $w$-plane. Such an angle-preserving mapping is also called a conformal mapping. If a function is complex differentiable at a given point $z_0$, this implies that all higher order derivatives also exist and that the function can be locally expanded in a Taylor series that converges in a circular region whose radius extends to the nearest critical point [verify!]. A critical point is...


%todo: say something about multivariable complex calculus

% e^(i*w) = i^(2w/pi)
% https://www.youtube.com/watch?v=3geVAJvJM8c  5:10
%
% a^(i*t) goes around the uni circle with speed depending on a
% https://www.youtube.com/watch?v=UOuxo6SA8Uc  25:30

% https://www.youtube.com/playlist?list=PLDcSwjT2BF_UDdkQ3KQjX5SRQ2DLLwv0R
% Essence of complex analysis  by Mathemaniac, uses a vector-field approach