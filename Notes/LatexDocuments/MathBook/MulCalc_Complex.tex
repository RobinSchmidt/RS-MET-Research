\section{Complex Calculus}
Complex calculus is the calculus of functions that map a complex number to another complex number. Since input and output are both 2-dimensional, it makes sense to look at complex calculus from the perspective of 2D vector fields. For a function $w = f(z) = u(x,y) + \i v(x,y)$ to be differentiable in the complex sense, the partial derivatives of the component functions $u,v$ with respect to $x,y$ must exist and satisfy the Cauchy-Riemann (CR) differential equations: $u_x = v_y, u_y = -v_x$. This condition has wide ranging implications which allow a lot of simplifications to be made in the computations of path integrals. If the function $f$ is considered as a mapping that maps points from the $z$-plane to the $w$-plane, the CR equations imply that the images of curves in the $z-$plane that intersect at a given angle, will intersect at the same angle in the $w$-plane. Such an angle-preserving mapping is also called a conformal mapping. If a function is complex differentiable at a given point $z_0$, this implies that all higher order derivatives also exist and that the function can be locally expanded in a Taylor series that converges in a circular region whose radius extends to the nearest singular point. If a function can be expanded in a convergent Taylor series around some point $z_0$, it is said to be \emph{analytic} at that point. A singluar point is a point is a point where the function "misbehaves" in some way. In most cases, the misbehavior is due to a division by zero. For example, the (real) function $f(x) = 1 / (1 - x^2)$ has singular points at $x = \pm 1$. Of course, we can also regard $f$ as a complex function by just allowing $x$ to be complex. But in that case, it is more common to use $z$ instead of $x$ for the name of the independent variable. The complex function $f(z) = 1 / (1 + z^2)$ has no singular points along the real axis but it does have singularities $z = \pm \i$. We will find that line integrals similar to those from $2D$ vector calculus will have a value of zero around any closed loop when that loop does not contain any singularities. And when the loop does contain singularities, the integral can be evaluated via a simple sum over some special values, called \emph{residues}, which are associated with these singularities and are comparatively easy to compute. That fact is what helps to simplify the evaluation of some definite integrals. With some ingenuity, it will even help to evaluate certain real integrals which would be hard to evaluate otherwise. Path integrals over paths that are not closed loops are actually path independent and given by simple differences between values at the endpoints of an antiderivative - just like in $1D$ calculus. But we are in $2D$, so a complex antiderivative is actually more like a potential....tbc...


\subsection{Complex Functions}
Complex functions are functions $f$ into which we can plug in a complex number $z$ as input and get out another complex number $w$ as output. We write $w = f(z)$. If we split input and output into their real and imagniray parts: $z = x + \i y, w = u  + \i v$, we can write the complex function in terms of two bivariate component functions: $u = u(x,y), v = v(x,y)$ such that $w = f(z) = f(x + \i y) = u(x,y) + \i v(x,y)$. In this form, a complex function is quite reminiscent of a vector field in $2D$ and in our interpretation of complex path integrals, we will indeed refer back to our intuitions about path integrals in $\mathbb{R}^2$.

\subsubsection{Rational Functions}
Among the simplest functions are monomials, i.e. functions of the type $f(z) = a z^n$ for some fixed complex number $a$ and some integer $n$. We already know how to multiply together two complex numbers which means we also know how to multiply a complex number $z$ by itself several times to form $z^n$. So we have everything we need to evaluate the expression $a z^n$ and the result will be another complex number. That means, this expression defines a complex function and so it is one of specimen we are going to investigate here. We also know how to add complex numbers, so we can also already take sums of terms of the form above. Such weighted sums of powers of the input variable are just our good old friends, the polynomials - just now with complex inputs and outputs and possibly also with complex coefficients. Complex division is also already defined, so we can also divide the outputs of two polynomials. That means, rational functions of a complex variable are also already defined wihtout requiring us to think much about it. In summary, for complex functions $w = f(z)$ of the form:
\begin{equation}
f(z) = \frac{\sum_{k=0}^{M} a_k z^k}{\sum_{k=0}^{N} b_k z^k}
\end{equation}
we already know what we would have to do to evaluate them for any given input $z$. For other types of functions, we'll have to put in some more thought...

% https://en.wikipedia.org/wiki/Linear_fractional_transformation
% https://en.wikipedia.org/wiki/Bilinear_transform

% maybe tackle the important function f(z) = 1/z, i.e. complex inversion, Moebius transformations
% explain Polya vector fields.

\subsubsection{Roots, Multifunctions and Algebraic Functions}
Consider the square root function again. In the definition of the real valued $f(x) = \sqrt{x}$, we already saw a foreshadowing of a potential problem. First of all, we noticed that some real numbers - namely the negative numbers - had no square roots at all. Some other real numbers - the positive ones - actually had two square roots. The two square roots of $4$ are $+2$ and $-2$, for example. We kind of dodged these issues but just saying that for negative inputs, the square root is undefined and for positive input, \emph{the} square root is \emph{defined} to be the positive solution. In the context of complex analysis, we will have to take this issue more seriously. 

\medskip
Let's first consider the problem of being unable to meaningfully define the square root for negative inputs. In the realm of complex numbers, that problem goes away completely. The purpose of defining $\i = \sqrt{-1}$ was precisely to deal with that situation. To evaluate $\sqrt{-4}$, we re-express it as $\sqrt{4} \sqrt{-1}$ using our usual laws for square roots to arrive at the solution $\sqrt{-4} = 2 \i$. And in case you wonder (you should!): yes - it is indeed allowed to use these laws. They continue to hold in the complex case. But again, $2 \i$ is not the only possible solution and $-2 \i$ is just as good a solution. Again, we actually have found two solutions. You can verify this by just squaring $2 \i$ and $-2 \i$. In both cases, you'll get $-4$.

\medskip
Secondly, let's think about what to make of the fact that we got not one but two values. When I said "solution", you may have asked yourself "solution to what equation"? It is actually a bit sloppy to say $\i = \sqrt{-1}$ and it would be more precise to say that $\i$ is some number that solves the equation $x^2 = -1$ and $-\i$ is actually a second solution to that equation. Recall that a general quadratic equation of the form $a_0 + a_1 x + a_2 x^2 = 0$ has solutions given by the formula ....


...TBC...


%The problem of not being able to define a square root (or, in fact $n$th root) for ceratin inputs will vanish - in the complex domain, negative numbers do have square roots, albeit "imaginary" ones. 

% principal values

\subsubsection{Exponential and Trigonometric functions}
Extending the domains of rational and algebraic functions from the real to the complex numbers was a straightforward process requiring only some algebra. The $n$th roots are a bit messy to deal with because we needed to introduce the concept of multivalued functions but they nevertheless did not require much higher level math tools to define. To extend the domains of the exponential and trigonometric functions, we'll need a bigger cannon from calculus: power series. The nice thing about power series is that in order to evaluate them, we only need to know how to perform additions and multiplications - and we actually indeed do know, how to add and multiply complex numbers! Using the power series representation of a function is a very common technique in math whenever we desire to define functions of arguments that are not simple numbers. The same idea can also be applied to matrices, multivectors and even operators. But here, we are interested in complex numbers. Let's recall the power series of some functions:
\begin{equation}
 \exp(x) = e^x = \sum_{k=0}^{\infty} \frac{x^k}{k!}, \quad
 \sin(x) =  \sum_{k=0}^{\infty} ...
\end{equation}
% add sin, cos, maybe sinh, cosh
The idea is now to simply allow the input $x$ to be complex valued to define these function for complex arguments. We would have to take care about the question of convergence but it turns out that these particular power series are very well behaved in that regard. They actually have an infinite radius of convergence....tbc
% Extending the domain of the real valued functions via their power series...



\subsubsection{The Complex Logarithm}
% Log - as infinite-valued multifunction. log(r e^{\i \phi}) = \log(r) + \i phi


\subsection{Complex Differentiation}
Having a little arsenal of complex functions $f: \mathbb{C} \rightarrow \mathbb{C}$ at our disposal, we can now start to do some calculus with them. Naturally, the first thing we may want to do, is to define a derivative for complex functions. Recall our definition of differentiability for vector fields in $2D$:
...tbc...

\subsubsection{The Cauchy-Riemann Equations}


\subsection{Complex Integration}

\subsection{Laurent Series}

\subsection{Analytic Continuation}
As a motivating example, consider the function $f(s)$ defined by the infinite sum:
\begin{equation}
f(s) = \sum_{n=1}^{\infty} = \frac{1}{n^s}
\end{equation}
The usage of $s$ as argument is just a common convention for this particular function which is known as the \emph{Riemann zeta function} and also often denoted as $\zeta (s)$. It is indeed a well defined and even analytic function whenever that infinite sum converges. As we know from our discussion in the series section, this sum does indeed converge for $s > 1$ when we assume $s$ to be real. More generally, if we allow $s$ to be complex, the condition for convergence is that the real part of $s$ must be greater than one. For inputs with real part less than or equal to one, the function is as of yet undefined. However, complex analysis provides us with a technique by which we can uniquely extend the domain of the function to almost the whole complex plane. That technique is called \emph{analytic continuation}. As said, for $\Re(s) > 1$, $f$ is indeed an analytic function. ...

% We have been able to extend the domains of exp and sin/cos by means of power series. that worked out because the power series of exp and sin/cos were nice enough to have an infinite radius of convergence around the expansion point zero (and, in fact, around any other expansion point as well?). 

% Hmm...but the Riemann Zeta function is not really a power series. At least not in z - but it is one in 1/z. Actually, the simplemost one - with all coeffs being 1. It's like a Laurent series with zero coeffs a_n for n >= 0 and unit coeffs fo n < 0, right? ...wait - no! That would be the case if it would be a sum over 1/s^n but it is actually a sum over 1/n^s

% explain the mapping theroem (Riemann?) see Arens add-on material (I think) - something about polygons in the complex plane


\subsection{Applications}
Functions of complex variables have numerous applications in fields like digital signal processing, fluid dynamics, quantum mechanics, (real) integration, differential equations, number theory etc. We'll now look at some of them.

\subsubsection{Rational Functions and Filter Theory}

\subsubsection{Elliptic Functions and Filter Design}

\subsubsection{Modular Forms and Number Theory}

\subsubsection{Conformal Mapping and Geometry}
% or maybe "Conformal Mapping and Fluid Dynamics"


%\subsection{Important Classes of Complex Functions}



% Some special cases of rational functions are important enough to have special names. When both numerator and denominator are only first orders polynomials, i.e. the function is of the form $f(z) = (a z  + b) / (c z + d)$, the function is also called a \emph{Moebius transformation}. In some contexts, such functions are also called \emph{linear fractional transform} and the so called \emph{bilinear transform} in digital signal processing is also a special case of such a function. Speaking of signal processing, rational functions where numerator and denominator are quadratic polynomials are alse quite common there. They occur as transfer functions of so called \emph{biquad} filters (for biquadratic) which can be sued as basic building block for all kinds of filters. Those more general filters have also rational functions as transfer functions. 


% https://en.wikipedia.org/wiki/Kramers%E2%80%93Kronig_relations

% https://www.johndcook.com/blog/applied-complex-analysis/


\begin{comment}
	
	
% But is it not really like a scalar potential (via gradient) or vector potential (via curl) - right? ToDo: figure out the exact connection between potentials and complex antiderivatives. What would the scalar potential of the vector field f(x,y) = (x^2-y^2, 2xy) be? It comes from the complex function z^2. How does it relate to the antiderivative z^3/3 of z^2?

% It has been said, that a power series around a point where $f$ is analytic converges in a disc around the expansion center that extends to the nearest singularity. What, if we try to epand the function aorund such a singular point itself? Predictably, it doesn't work - but a generalization that allows also for negative powers, called a Laurent series, may work.

% How do we find the Laurent coeffs? Apparently, we cant just evaluate the function or derivatives at the point because the function will typically be undefined at that point. It's via an integral:

% https://en.wikipedia.org/wiki/Laurent_series
% https://mathworld.wolfram.com/LaurentSeries.html

% Does this mean, this integral for a_n aggrees with the derivative for n >= 0 for regular points? Try to apply the integral formula for the coeffs to some smooth function like sin or exp. the resulting coeffs should 0 for n < 0 and agree with the Taylor coeffs for n >= 0, I guess? Yes - indeed - see formula 5 here:

% https://math.mit.edu/~jorloff/18.04/notes/topic7.pdf

% Motivate Laurent series with the example f(x) = (1+x)/(x^3-x^2) or (1+x)/(x^3+x^2) and try to expand it as power series around x0=0. We will see that we will need to allow for negative powers. Maybe for the example, the coeffs can be found by partial fraction expansion?

% https://en.wikipedia.org/wiki/Singularity_(mathematics)#Complex_analysis
% https://complex-analysis.com/content/classification_of_singularities.html
% https://www.britannica.com/topic/singularity-complex-functions
% https://en.wikipedia.org/wiki/Critical_point_(mathematics)
% https://en.wikipedia.org/wiki/Singular_point_of_a_curve
% https://en.wikipedia.org/wiki/Radius_of_convergence#Radius_of_convergence_in_complex_analysis
% https://mathworld.wolfram.com/Singularity.html


%todo: say something about multivariable complex calculus

% e^(i*w) = i^(2w/pi)
% https://www.youtube.com/watch?v=3geVAJvJM8c  5:10
%
% a^(i*t) goes around the uni circle with speed depending on a
% https://www.youtube.com/watch?v=UOuxo6SA8Uc  25:30

% https://www.youtube.com/playlist?list=PLDcSwjT2BF_UDdkQ3KQjX5SRQ2DLLwv0R
% Essence of complex analysis  by Mathemaniac, uses a vector-field approach	
	
	
https://en.wikipedia.org/wiki/Elliptic_function
https://en.wikipedia.org/wiki/Modular_form
https://en.wikipedia.org/wiki/Automorphic_form	

https://en.wikipedia.org/wiki/Hyperelliptic_curve

That looks pretty nice:
https://math.mit.edu/~jorloff/18.04/notes/

"Cauchy’s theorem is analogous to Green’s theorem for curl free vector fields"
Here, in 3.6:
https://math.mit.edu/~jorloff/18.04/notes/topic3.pdf
	
\end{comment}