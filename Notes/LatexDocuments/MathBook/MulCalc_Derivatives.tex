\section{Derivatives}

\subsection{Partial Derivatives}
Let's assume we have a bivariate function $f$, i.e. a function with two inputs $x,y$, that produces one output $z$: $z = f(x,y)$. We can visualize this as a landscape above an $(x,y)$-plane where the height is given by the function value. We can take derivatives of $f$ with respect to $x$ and with respect to $y$. In the former case, $y$ is just treated as a constant and in the latter case $x$ is treated as a constant. These two derivatives of $f$ are called partial derivatives and we denote them as $\frac{\partial f}{\partial x}$ and $\frac{\partial f}{\partial y}$, where the curly d symbol $\partial$ is a special math symbol that we read as "partial" - it's not a greek delta. The formal definition is:
\begin{equation}
 \frac{\partial f(x,y)}{\partial x} = \lim_{h \rightarrow 0} \frac{f(x+h,y) - f(x,y)}{h}, \qquad
 \frac{\partial f(x,y)}{\partial y} = \lim_{h \rightarrow 0} \frac{f(x,y+h) - f(x,y)}{h}
\end{equation}
In the text above, we wrote $\frac{\partial f}{\partial x}$ whereas in the definition, we wrote  $\frac{\partial f(x,y)}{\partial x}$. The former notation with supressed input arguments $x,y$ is just an abbreviation of the more verbose latter notation. This abbreviation makes sense when it is understood from the context that $f$ is a function of $x,y$. Another, even shorter, notation for the partial derivatives is $f_x, f_y$. For a general multivariate function with scalar output $f(\mathbf{x})$ where $\mathbf{x} = (x_0,x_1,\ldots,x_n)^T$ where $n$ is the number of inputs, we can compute partial derivatives with respect to each input dimension. We denote these as $\frac{\partial f}{\partial x_i}$ where $i = 1,\ldots,n$. We could also write $f_{x_i}$ but we will rarely use this notation because an index with an index is typographically less than ideal. The formal definition of $\frac{\partial f}{\partial x_i}$ is given by:
\begin{equation}
 \frac{\partial f(\mathbf{x}) }{\partial x_i} 
 = \lim_{h \rightarrow 0} \frac{f(\mathbf{x} + h \mathbf{e}_i ) - f(\mathbf{x})}{h}
\end{equation}
where $\mathbf{e}_i$ is the unit vector in the $i$-th coordinate direction, i.e. it has a one at position $i$ and zeros everywhere else. As said: the computation of such partial derivatives is purely mechanic and we don't need to refer to our definitions for that. Instead, we just treat the function as if it would depend only on the variable with respect to which we differentiate and treat all other variables as constants.

\subsection{Directional Derivatives}
Now let's assume we are given an arbitrary vector $\mathbf{v}$ of unit length. We define the directional derivative of $f$ into the direction of  $\mathbf{v}$ as:
\begin{equation}
\label{Eq:DirectionalDerivative}
 \frac{\partial f(\mathbf{x}) }{\partial \mathbf{v}} 
 = \lim_{h \rightarrow 0} \frac{f(\mathbf{x} + h \mathbf{v} ) - f(\mathbf{x})}{h}
 = \frac{d}{d h} f(\mathbf{x} + h \mathbf{v}) \bigg\rvert_{h=0}
\end{equation}
where the RHS is a generalization of \ref{Eq:DerivativeViaH} which tells us how we could evaluate the limit (but we'll soon learn a simpler way to evaluate it). Note that the directional derivative into any of the coordinate directions $\mathbf{e}_i$ just reduces to the partial derivative with respect to coordinate $i$, so partial derivatives can also be seen as directional derivatives into very specific directions: namely, into the directions of our coordinate axes. [TODO: verify that the RHS is actually correct - the formula is not from a book - i found it myself by analogy to the evaluation formula for the Gateaux derivative in Baerwolff, pg. 792]

\subsection{Higher Order Partial Derivatives}
Just like we could define the second derivative of a univariate function as the derivative of the derivative, we can define second order partial derivatives of a multivariate scalar function. Due to the fact that we now have $n$ input variables, we can from $n^2$ different partial derivatives. If we first find the derivative of $f$ with respect to some $x_i$ and then take the result and find the derivative with respect to $x_j$, we obtain such a second order partial derivative. We write this as $ \frac{\partial f}{\partial x_i \partial x_j}$
\begin{equation}
 \frac{\partial f}{\partial x_i \partial x_j}
\end{equation}
If the function $f$ is twice continuously differentiable, then, according to Schwartz's theorem, we will actually have the symmetry:
\begin{equation}
	\frac{\partial f}{\partial x_i \partial x_j} 
	=
	\frac{\partial f}{\partial x_j \partial x_i} 
\end{equation}
which means the order of the differentiations makes no difference. We can first take the derivative with respect to $x_i$ and then with respect to $x_j$ or the other way around. The end result will be the same.


\subsection{The Gradient Vector}
For a function $f$ with $n$ input variables (or one single $n$-dimensional input vector) and a single output, we can compute $n$ different partial derivatives. We can collect these partial derivatives into another $n$-dimensional vector. This vector is called the gradient of the function. The gradient always points into the direction in which the function increases most steeply and the length of this gradient vector tells us, how steep that steepest increase is. Note that this is a direction in the input space, not in the combined input/output space. So, if the input is 2D, such that $z = f(x,y)$ gives a height over an $(x,y)$-plane of input values, the gradient of $f$ at a particular input location $(x_0,y_0)$ in the plane gives the direction in the $(x,y)$-plane in which we need to move the input to get the steepest increase in the height value $z$. But it's \emph{not} a direction in the 3D space of $(x,y,z)$.
%[Q: what is the direction in x,y,z space? is it just x,y taken from the gradient, augmented by a z-value for which we can take the length of the gradient? that seems plausible -> make some plots! the vector should be tangential to the surface, i think]

\medskip
There are different notations for the gradient. One of them is to write $\grad f$. You may also see it in non boldface, but I want to consistently use boldface for vectors and the gradient is a vector, so boldface is appropriate. Another notation for the gradient of $f$ is $\nabla f$, where the symbol $\nabla$ is the so called "nabla" operator (a nabla is an ancient greek harp-like instrument with a triangular shape), sometimes also called "del". The nabla or del operator is defined as a symbolic vector of partial derivative symbols, for example in 2D as $\nabla = (\partial / \partial x, \partial / \partial y)^T$, so when $\nabla$ is "applied" to a function $f$ as in $\nabla f$, this expression "evaluates to" $(\partial f / \partial x, \partial f / \partial y)^T$. I use quotes here because this "application" is to be understood in a purely formal and symbolic sense.

\medskip
The gradient helps us to compute the directional derivative: It's just the scalar product between the given (unit length) direction vector $\mathbf{v}$ and the gradient vector:
\begin{equation}
 \frac{\partial f}{\partial \mathbf{v}} = \mathbf{v} \cdot \grad f
\end{equation}

\subsection{The Jacobian Matrix}
So far, we assumed that $f$ has only a single output, i.e. defines a scalar field. Now we will have a look at vector valued functions $\mathbf{f}: \mathbb{R}^n \rightarrow \mathbb{R}^m$ with $\mathbf{f(x)} = (f_1(\mathbf{x}), \ldots, f_m(\mathbf{x}))^T$ where $\mathbf{x} = (x_1,\ldots,x_n)^T$. So, our function has $n$ inputs and $m$ outputs. The $m \times n$ matrix
\begin{equation}
 \mathbf{J_f}(\mathbf{x}) = 
 \mathbf{f'}(\mathbf{x}) = 
 \begin{pmatrix}
  \frac{\partial f_1}{\partial x_1} (\mathbf{x}) & \hdots & \frac{\partial f_1}{\partial x_n} (\mathbf{x}) \\
  \vdots & \ddots & \vdots \\
  \frac{\partial f_m}{\partial x_1} (\mathbf{x}) & \hdots & \frac{\partial f_m}{\partial x_n} (\mathbf{x})
 \end{pmatrix}
\end{equation}	
is called the Jacobi matrix	of $\mathbf{f}$ at $\mathbf{x}$. It's a generalization of the derivative. The derivative of a vector-valued function is a matrix-valued function. The $k$th row of the matrix is the transpose of the gradient of the $k$th component function $f_k$. Just like 1D derivatives do, it defines a linear approximation to $\mathbf{f}$ at a given $\mathbf{x}$ in the sense that for a small vector  $\mathbf{h}$, we have:
\begin{equation}
 \mathbf{f}(\mathbf{x + h}) \approx 
 \mathbf{f}(\mathbf{x}) + \mathbf{J_f(x) \; h}
\end{equation}	
This is the reason why the individual gradients of the component functions are in the rows rather than in the columns. It ensures that we can formulate the linear approximation by left multiplication of the matrix with some small delta vector. The determinant of this matrix will play an important role in the change of variables formula for multiple integrals which we will see later. In a more terse notation, you can also suppress the index $\mathbf{f}$ and the argument $\mathbf{x}$ and just write $\mathbf{J}$ instead of $\mathbf{J_f}(\mathbf{x})$, if it's clear from the context about which $\mathbf{f}$ we are talking and that $\mathbf{f}$ is a function of $\mathbf{x}$.

% https://en.wikipedia.org/wiki/Jacobian_matrix_and_determinant
% Bärwolff, pg 392

%When we change from one coordinate system to another by a change of variables, then we will have to compensate
%If we multiply the matrix
%also mention the importance of its determinant in case of equal input and output dimensionalities - it's a generalized area/volume/hypervolume scale factor

\subsection{The Hessian Matrix}
The gradient vector and the Jacobian matrix are collections of first partial derivatives of scalar- and vector valued functions respectively. The Hessian matrix is a collection of second partial derivatives of a scalar valued function $f: \mathbb{R}^n \rightarrow \mathbb{R}$ with $f = f(\mathbf{x})$ and $\mathbf{x} = (x_1, \ldots, x_n)$. It is defined as:
\begin{equation}
 \mathbf{H}_f(\mathbf{x}) = 
	\begin{pmatrix}
		  \frac{\partial^2 f}{\partial x_1 \partial x_1} (\mathbf{x}) 
		& \hdots 
		& \frac{\partial^2 f}{\partial x_1 \partial x_n} (\mathbf{x}) 
		\\
		\vdots & \ddots & \vdots 
		\\
	      \frac{\partial^2 f}{\partial x_n \partial x_1} (\mathbf{x}) 
		& \hdots 
		& \frac{\partial^2 f}{\partial x_n \partial x_n} (\mathbf{x})
	\end{pmatrix}
\end{equation}	


...TBC...


%-https://en.wikipedia.org/wiki/Hessian_matrix#Vector-valued_functions

% mention symmetry (Schwartz' theorem)




\begin{comment}


\subsection{TODO Notes}
It seems to be generally true that for any function $f(x)$ we have
\begin{equation}
 \frac{d}{d x} f(x) = \frac{d}{d h} f(x + h) \bigg\rvert_{h=0}
\end{equation}
which seems to generalize to the directional derivative of multivariate functions as:
\begin{equation}
 \frac{\partial f(\mathbf{x}) }{\partial \mathbf{v}} 
 = \lim_{h \rightarrow 0} \frac{f(\mathbf{x} + h \mathbf{v} ) - f(\mathbf{x})}{h}
 = \frac{d}{d h} f(\mathbf{x} + h \mathbf{v}) \bigg\rvert_{h=0}
\end{equation}
which could be used to actually evaluate the directional derivative without using the formula with the scalar product with the gradient. ...verify this! This seems to be the important observation in evaluating the Gateaux derivative by analogy later...



-maybe divergence, but maybe defer that to vector calculus
https://en.wikipedia.org/wiki/Del

\end{comment} 