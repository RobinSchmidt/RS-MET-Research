


\begin{comment}

Analogous objects in exterior calculus and classic vector calculus

0-form -> multivariate function, scalar field
1-form -> vector field
2-form -> bivector field, can be indentified with a vector field in 3D
3-form -> trivector field, can be identified with a scalar field in 3D

effect of the exterior derivative $\d$ on k-forms:

0-form -> d -> 1-form -> d -> 2-form -> d -> 3-form

this is analogous to:

scalar field -> grad -> vector field -> curl -> (pseudo)vector field -> div -> (pseudo)scalar field


todo: explain, how the exterior derivative would work in 4D where the curl doesn't work anymore

The exterior derivative, i.e. the $\d$ operator, can be applied to a vector field (i.e. 1-form field) of any dimension. In this way, it generalizes the curl operator from classic 3D vector calculus to higher (and lower) dimensional spaces. The result of this application will always be a 2-form field (i.e. bivector field). In 3D - and only there - a 2-form field can be identified with a vector field because both kinds of fields have 3 independent components. You may get some strange behavior from doing such an identification, though (todo: footnote: explain what physicists mean by "pseudovectors"). In 4D, a 2-form (bivector) field would have six independent components and in 2D it would have only one, so you can't really do such an identification in these spaces. In general, a k-form in n-dimensional space will have n-choose-k independent components. You may cheat a little in 2D, though, by embedding 2D in 3D but that's arguably very unelegant. And in 4D, all is lost anyway, so classic curl is really a 3D specific thing and we need something more general. Besides working in all dimensions, the exterior derivative generalizes curl in yet another way: not only can $\d$ be applied to 1-form (vector) fields but also to 0-form (scalar) fields, in which case it acts like the classical gradient. It can also be applied to 2-form (bivector) fields, in which case it acts like the divergence. Well, at least if we are in 3D - in general, this will produce a 3-form (trivector), which in 3D can be identified with a scalar. To sum up, the exterior derivative can be seen as a generalization of the curl in 2 ways: (1) it unifies the curl with the two other first order differential operators grad and div and (2) it generalizes curl (and more) to higher dimensional spaces. The main second order differential operator from classic vector calculus, the Laplacian, would formally correspond to $\d^2$. However, this operator plays no role in exterior calculus because when applying $\d$ twice to any k-form, the result is always identically zero. This fact generalizes the vector calculus identities $curl grad f = 0$ and $div curl \mathbf{f} = 0$.

todo:
-i think, the differential 0-form is not just the scalar-field e.g. f(x), but f(x) dx - the dx belongs
 to the form


the generalized Stokes theorem generalizes the curl theorem in two ways:

(1) it can now be applied to vector-fields in arbitrary dimensions
(2) it cannot only be applied to vector fields but also scalar fields, bivector fields, etc. One tpyically does not look at vector fields per se but at 1-form fields (i.e. covector fields?) instead (or, in general, at k-form fields). But they area isomorphic, so it's ok to identify them. however, in 3D, integrating a (co)bivector fields will yield a (co)vector fields, which is the vector potential of the original bivector field...wait ...does this co-bivector thing make sense? or *is* a bivector the same thing as a covector? yes, i think, in 3D a vector is identified with a 1-form and a bivector is identified with a 2-form which is also a covector


The Generalized Stokes Theorem

in 3D:

-when $\omega$ is a 0-form (i.e. a scalar, multivariate function), the RHS is $\omega(b) - \omega(a)$ where $a,b$ are the endpoints of the curve and the LHS is a path integral over a 1-form (i.e. a vector field) along the given curve, $\d \omega$ is the gradient of $\omega$ and $\omega$ is a (scalar) potential for $\d \omega$. This is the law for integrals over potential fields. In 1D, it becomes the fundamental theorem of calculus (verify!).

-when $\omega$ is a 1-form (i.e. a vector field), the RHS is a path integral over that vector field around a closed loop, the LHS is a surface integral that integrates the curl of $\omega$ over the area of the surface that is bounded by the loop, $\omega$ is a vector potential for $\d \omega$ and $\d \omega$ is the curl of $\omega$. This is the original stokes theorem about curls. In 2D, it becomes Green's theorem. In the more general 3D setting (remember: 2D can always be thought of as being embedded in 3D), the surface can have any shape as long as it is bounded by the given loop. It is not restricted to somehow be a minimal surface (todo: add footnote) with respect to the enclosing loop. It is allowed to bulge out from the loop or even inflate like a balloon.

-when $\omega$ is a 2-form (i.e. a bivector field), the RHS is a surface integral that integrates the flux of the field through the surface over the whole (closed) surface. The LHS becomes an integral over the divergence of the field, integrated over the volume that is enclosed by the surface. This is the divergence theorem (a.k.a. Gauss' theorem).

see:
https://en.wikipedia.org/wiki/Exterior_derivative
https://www.youtube.com/watch?v=2ptFnIj71SM
https://en.wikipedia.org/wiki/Vector_calculus_identities#Second_derivative_identities

\end{comment} 