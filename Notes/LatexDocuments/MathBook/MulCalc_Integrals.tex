\section{Integrals}

\subsection{Double Integrals}

\subsection{Triple Integrals}

\subsection{Integrals in Higher Dimensions}


\subsection{The Transformation Formula}

\subsubsection{Coordinate Transformations}
% ...maybe we should have a whole section on Coordinate Systems on a higher level?
So far, we assumed that our functions are given in terms of cartesian coordinates such as $x,y,z$. There are situations where it is advantageous to use different coordinate systems. A well chosen coordinate system can simplify problems by exploiting the natural symmetries of a situation. For example, if we need to integrate a function $f(x,y)$ over a circular region centered at the origin, the expressions for the integration boundaries greatly simplify when we use polar coordinates. Maybe the function itself has circular symmetry, i.e. the value depends only on the radius but not on the angle. In such a case, the function is actually just one dimensional if we use polar coordinates. A 1D function is of course simpler than a 2D function. 
...TBC...

\subsubsection{The Formula}
%For simplicity, we will give the formula only for the 2D case

\subsubsection{Common Coordinate Systems}
ToDo: give equations (back-and-forth, if possible) for polar, cyclindrical, spherical systems and their Jacobian determinants

\begin{comment}

-Multiple integrals:
-double integrals over:
 -axis-parallel rectangles (simplest case, limits are constants)
 -triangles with two points with same y-coordinate -> y(x) is a linear function
  -can also be done using x(y) - whatever is most convenient
 -general triangles (split into two triangles of the sort above)
 -general polygons (via triangulation)
 -segments of circular annulus -> uses a coordinate transformation to polar coordinates
  -generalization to arbitrary coordinates
-triple integrals over
 -axis parallel cuboids
 -3-simplex (tetrahedron)
 -general polyeders
 -using cyclindrical, spherical and general 3D coordinates
-nD integrals
 
-line integrals (of scalar function),
-line integrals of vector fields (circulation), surface integrals (flux)
-coordinate transformations (polar, cylindricial, spherical, etc. -> jacobian determinant)
 

\end{comment} 