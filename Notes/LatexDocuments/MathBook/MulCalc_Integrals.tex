\section{Integrals}

\subsection{Double Integrals}
Double integrals are 2D integrals of functions $f: \mathbb{R}^2 \rightarrow \mathbb{R}$ with $f = f(x,y)$. We can visualize the graph of such a function as a terrain above the $xy$-plane where the $z$-coordinate is the height above the plane and is prescribed by $f$ as function of $x$ and $y$. Just like we could interpret integrals of 1D functions as areas between the $x$-axis and the function graph, we can imagine 2D integrals as the volume between the $xy$-plane and the function graph. In a definite integral in 1D, the domain of integration was just a simple interval on the $x$-axis. Here in 2D, the domain of integration can be in general an arbitrarily region in the $xy$-plane. The simplest case is when that region is an axis-parallel rectangle. In this case, ...

%-axis parallel rectangles
%-triangles
%-polygons
%-domains specified by functions


...TBC...

\subsection{Triple Integrals}

\subsection{Integrals in Higher Dimensions}


\subsection{The Transformation Formula}
% i think, it's called "change of variables" in english literature

\subsubsection{Coordinate Transformations}
% ...maybe we should have a whole section on Coordinate Systems on a higher level?
So far, we assumed that our functions are given in terms of cartesian coordinates such as $x,y,z$. There are situations where it is advantageous to use different coordinate systems. A well chosen coordinate system can simplify problems by exploiting the natural symmetries of a situation. For example, if we need to integrate a function $f(x,y)$ over a circular region centered at the origin, the expressions for the integration boundaries greatly simplify when we use polar coordinates. Maybe the function itself has circular symmetry, i.e. the value depends only on the radius but not on the angle. In such a case, the function is actually just one dimensional if we use polar coordinates. A 1D function is of course simpler than a 2D function. 
...TBC...

\subsubsection{The Formula}
%For simplicity, we will give the formula only for the 2D case

\subsubsection{Common Coordinate Systems}
ToDo: give equations (back-and-forth, if possible) for polar, cyclindrical, spherical systems and their Jacobian determinants

\begin{comment}

-Multiple integrals:
-double integrals over:
 -axis-parallel rectangles (simplest case, limits are constants)
 -triangles with two points with same y-coordinate -> y(x) is a linear function
  -can also be done using x(y) - whatever is most convenient
 -general triangles (split into two triangles of the sort above)
 -general polygons (via triangulation)
 -segments of circular annulus -> uses a coordinate transformation to polar coordinates
  -generalization to arbitrary coordinates
-triple integrals over
 -axis parallel cuboids
 -3-simplex (tetrahedron)
 -general polyeders
 -using cyclindrical, spherical and general 3D coordinates
-nD integrals
 
-line integrals (of scalar function),
-line integrals of vector fields (circulation), surface integrals (flux)
-coordinate transformations (polar, cylindricial, spherical, etc. -> jacobian determinant)
 

\end{comment} 