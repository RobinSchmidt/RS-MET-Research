\section{Partial Differential Equations}
One might think that a "partial" differential equation (abbreviated as PDE) ought to be easier to deal with than a "full" differential equation, right? Wrong! The terminology might by a bit misleading. In the context of differential equations, the term "partial" is not meant as opposite to "full" but rather refers to the fact that the equations contain partial derivatives of an unknown, to be found, multivariate function which is often denoted as $u = u(x,y,\ldots)$. I think, a more appropriate term would be "multidimensional" or "multivariate" differential equations. And they are a much more difficult subject than ordinary (one-dimensional) differential equations. Despite being difficult to tackle mathematically (i.e. solve), partial differential equations can actually be quite intuitive to understand because they are usually a plausible mathematical model for a physical situation. Modeling continuous physical phenomena is the primary application domain of PDEs. A 2-dimensional PDE could model the vibration of a string under tension, such as a guitar string, for example. The motion of the string is modeled mathematically as a function of two variables: the position along the string $x$ and the time instant $t$. We want to find a function $u(x,t)$ that tells us at every instant $t$, how much the string deviates at position $x$ from its (straight) equilibrium shape. Or, if we freeze $t$ at $t = t_0$, we may see $u(x, t_0)$ as a one-dimensional function that gives us the whole shape of the string at the instant $t_0$. In the literature of PDEs, it is customary to use the letter $u$ to denote the dependent variable, so we'll adopt that notation here, too. It is also quite common to denote the derivatives of $u$ with respect to one of the independent variables via a subscript, for example $u_t$ would be the first time derivative of $u$, i.e. the string's (local) velocity. Our space and time dependent function $u(x,t)$ arises from the physical laws that govern the motion of the string which is basically an infinitesimal version of Newton's law $F = m a$. Here $a$ is the acceleration of the string - and acceleration is the second time derivative of the excursion that we are interested in, i.e. it is $a = u_{tt}$. The force $F$ that causes the acceleration is related to the local curvature of the string given by the second spatial derivative $u_{xx}$ of the excursion. The governing equation basically says: the local acceleration is proportional to the (negative) local curvature: $u_{tt} = -k u_{xx}$ for some proportionality constant $k$. If this all makes sense to you and you are nodding along then congratulations! You have just casually understood the wave equation - one important example of a partial differential equation \footnote{Being primarily interested in sound, the wave equation is my personal favorite PDE. Everybody has a favorite partial differential equation, right? Right?!}.



%If we denote by $u_{tt}$ the second derivative of the excursion with respect to time $t$, i.e. the (local) acceleration of the string and by $u_{xx}$ the second derivative of the excursion with respect to position $x$, i.e. the local force generating mechanism, then the governing equation for our sought $u(x,t)$ is given by $u_{tt} = k u_{xx}$ for some proportionality constant $k$. Congratulations! You have just casually understood the wave equation - one important example of a partial differential equation...


...TBC...

% Make a subsection Examples and list the stuff below as subsubs and paragraphs

\subsection{Examples}
To get a feeling for how PDEs looks like and what sorts of phenomena they model, we will first have a brief look at some important examples of PDEs.
% add Black-Scholes equation and maybe some examples from other fields like chemistry and biology. I think there was a PDE for pattern formation in furs (like leopard, giraffe, zebra, etc) - look that up. What about formation of fractals like snail-shells, romanesco, etc? or sunflowers?


\subsubsection{PDEs for Static Equilibria}

\paragraph{Poisson's Equation}
% when rhs = 0, also known as Laplace's equation

% https://en.wikipedia.org/wiki/Poisson%27s_equation
% https://en.wikipedia.org/wiki/Laplace%27s_equation
% https://de.wikipedia.org/wiki/Poisson-Gleichung
% https://de.wikipedia.org/wiki/Laplace-Gleichung


\paragraph{Minimal Surfaces}

%https://en.wikipedia.org/wiki/Minimal_surface

% add minimal surfaces


\subsubsection{PDEs for Time-Varying Scalar Fields}

\paragraph{Transport Equation} ...tbc...

% https://en.wikibooks.org/wiki/Partial_Differential_Equations/The_transport_equation
% https://en.wikipedia.org/wiki/Continuity_equation
% https://www.simscale.com/docs/simwiki/numerics-background/what-is-the-transport-equation/

\paragraph{Heat Equation} 
The heat equation governs how a scalar physical quantity like temperature tends to spread out spatially over time. Imagine some clump of material that has initially a homogeneous temperature except for one single hot spot. Over time, the heat will spread out from the hot spot into the whole clump until the temperature is completely homogeneous. At the (asymptotic) end of the process, the heat from the hot spot will be dispersed over the whole object. In one, two and $n$ dimensions, the equation is:
\begin{equation}
1\text{D: } \; u_{t} = D u_{xx} , \qquad 
2\text{D: } \; u_{t} = D (u_{xx} + u_{yy}), \qquad 
n\text{D: } \; u_{t} = D \Delta u
\end{equation}
The right hand side of the equation is proportional the Laplacian of our scalar field $u$ with a proportionality factor $D$. Loosely speaking, the Laplacian measures, how much the value at the local spot deviates from the average value of the local neighborhood. The left hand side $u_t$ is the (temporal) rate of change of the quantity $u$ at any given spot. So the equation says: the more the local value deviates from the neighborhood, the faster it will change. The change will be towards the neighborhood average. The value at the spot will be dragged to the value of the local average and it will be dragged faster, when it deviates more. Differences in temperature from spot to spot will tend to even out over time. The value of $u$ at every location gets dragged towards the average until at the end, everything is at the same temperature which is the average temperature. ...TBC...


% https://en.wikipedia.org/wiki/Heat_equation


%...TBC...
% The energy has dispersed

% aka Heat Equation

\paragraph{Diffusion Equation} 
The diffusion equation is a more complicated version of the heat equation that allows the coefficient of diffusion $D$ to depend on the quantity $u$ itself such that $D = D(u)$. ...TBC...

%to be different in different spatial directions and  ...TBC...

% https://en.wikipedia.org/wiki/Diffusion_equation


\paragraph{Convection-Diffusion Equation} ...tbc...

% https://en.wikipedia.org/wiki/Convection%E2%80%93diffusion_equation


\paragraph{Wave Equation}
The wave equation is what governs the propagation of disturbances in a medium. When you throw a stone into a pond of water, a circular wavefront will be formed at the point of impact which propagates radially outward. In general, there will be some physical quantity $u$, like height of a water surface or a membrane or the pressure of a gas, which we will consider to be a function of the spatial coordinates and of time.
\begin{equation}
1\text{D: } \; u_{tt} = c^2 u_{xx} , \qquad 
2\text{D: } \; u_{tt} = c^2 (u_{xx} + u_{yy}), \qquad 
n\text{D: } \; u_{tt} = c^2 \Delta u
\end{equation}

% Loosely speaking, the wave equation basically says that the local acceleration is proportional to the local curvature. More strictly speaking the term "curvature" is not quite exact here but taking the Laplacian as something like a measure of curvature is not too far of either. The Laplacian measures, how far off the local value deviates from the average of the local neighborhood.

% https://en.wikipedia.org/wiki/Curvature#Surfaces

% maybe include the factor $c^2$ and remove the 2D case, 1D and nD should be enough and form more complicated equations, listing all 3 cases may take too much space, maybe allow for an rhs f(x), f(x,y), f(r)  where r= (x,y,z)

% Maybe also show the notation with thh d'Alembert operator

% Give the solutions

% Maybe mention that one could include "driving terms"  as a function f(x,y,t) (in the 2D case) to the right hand side. In a numerical simulation, some added driving term could actually also be applied to u_t or even u itself rather than to u_tt

%\begin{equation}
%1\text{D:  } c^2 u_{tt} + u_{xx} = f(x), \qquad 
%2\text{D:  } c^2 u_{tt} + u_{xx} + u_{yy} = f(x,y), \qquad 
%n\text{D:  } c^2 u_{tt} + \Delta u = f(\mathbf{r})
%\end{equation}

% add dispersive wave equation, nonlinear wave equation from numerical sound synthesis
 ...tbc...

% Die Wellengleichung EINFACH erklärt! | Wellen (1 von 10)
% https://www.youtube.com/watch?v=X8rfJZt8Lc4&list=PLdTL21qNWp2YtT_9KUJoz-kU6kNJ8W1Vu

% https://en.wikipedia.org/wiki/Wave_equation
% https://en.wikipedia.org/wiki/One-way_wave_equation

\paragraph{Korteweg-de-Vries Equation}
I used waves on a pond of water as an example for the sort of phenomena that the wave equation describes. Although this is a nice illustration, it was actually a bad example because surface waves on (shallow) water are better modeled by another PDE - the Korteweg-de-Vries equation. In one dimension, it reads:
\begin{equation}
 u_t + 6 u u_x + u_{xxx} = 0
\end{equation}
...TBC...[explain soliton solutions]

% The $u u_x$ term is an advection term, the $u_{xxx}$ is a dispersion term
% It has soliton solutions. Solitons are strongly stable, self-reinforcing propagating solutions. 
% https://en.wikipedia.org/wiki/Soliton

% https://en.wikipedia.org/wiki/Korteweg%E2%80%93De_Vries_equation
% https://de.wikipedia.org/wiki/Korteweg-de-Vries-Gleichung
% https://math.stackexchange.com/questions/3112564/derivation-of-2d-korteweg-de-vries-equation
% https://de.wikipedia.org/wiki/Kadomtsev-Petviashvili-Gleichung

\paragraph{Quantum Wavefunctions} ...tbc...

% aka Schroedinger Equation
% is structurally like the diffusion equation but with a twist: the coefficient is imaginary which turns the whole thing into some sort of wave equation


% Burgers Equation
% 





\subsubsection{PDEs for Time-Varying Vector Fields}

\paragraph{Fluid Dynamics} ...tbc...

% aka Navier-Stokes Equations

% https://en.wikipedia.org/wiki/Fluid_dynamics
% https://en.wikipedia.org/wiki/Navier%E2%80%93Stokes_equations

\paragraph{Electrodynamics} ...tbc...
% aka Maxwell's Equations
% maybe rename to electrodynamics

\paragraph{Magnetohydrodynamics} ...tbc...


% add equtions for magneto-hydrodynamics

% https://en.wikipedia.org/wiki/Magnetohydrodynamics
% https://de.wikipedia.org/wiki/Magnetohydrodynamik
% https://de.wikipedia.org/wiki/Gau%C3%9Fsches_Einheitensystem

\subsubsection{PDEs for Time-Varying Tensor Fields}

\paragraph{Electrodynamics Reformulated} ...tbc...

\paragraph{Gravity in General Relativity} ...tbc...
% aka Einstein's field equations




\subsection{The General Form}

% see Arens, pg 1090


\subsection{Solution Methods}

% https://en.wikipedia.org/wiki/Integrable_system

\subsubsection{Analytical Methods}

\subsubsection{Numerical Methods}




\begin{comment}


https://en.wikipedia.org/wiki/Fokker%E2%80%93Planck_equation
https://en.wikipedia.org/wiki/Convection%E2%80%93diffusion_equation
https://en.wikipedia.org/wiki/Klein%E2%80%93Kramers_equation
https://en.wikipedia.org/wiki/Master_equation




\end{comment}