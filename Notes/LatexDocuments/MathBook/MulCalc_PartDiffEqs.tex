\section{Partial Differential Equations}
One might think that a "partial" differential equation (abbreviated as PDE) ought to be easier to deal with than a "full" differential equation, right? Wrong! The terminology might by a bit misleading. In the context of differential equations, the term "partial" is not meant as opposite to "full" but rather refers to the fact that the equations contain partial derivatives of an unknown, to be found, multivariate function $u(x,y,\ldots)$. I think, a more appropriate term would be "multidimensional" or "multivariate" differential equations. And they are a much more difficult subject than ordinary (one-dimensional) differential equations. Despite being difficult to tackle mathematically (i.e. solve), partial differential equations can actually be quite intuitive to understand because they are usually a plausible mathematical model for a physical situation. Modeling continuous physical phenomena is the primary application domain of PDEs. A 2-dimensional PDE could model the vibration of a string under tension, such as a guitar string, for example. The motion of the string is modeled mathematically as a function of two variables: the position along the string $x$ and the time instant $t$. We want to find a function $u(x,t)$ that tells us at every instant $t$, how much the string deviates from its equilibrium position at position $x$. Or, if we freeze $t$ at $t = t_0$, we may see $u(x, t_0)$ as a one-dimensional function that gives us the shape of the string at the instant $t_0$. In the literature of PDEs, it is customary to use the letter $u$ to denote the dependent variable, so we'll adopt that notation here, too. It is also quite common to denote the derivatives of $u$ with respect to one of the independent variables via a subscript, for example $u_t$ would be the first time derivative of $u$, i.e. the string's (local) velocity. Our space and time dependent function $u(x,t)$ arises from the physical laws that govern the motion of the string which is basically an infinitesimal version of Newton's law $F = m a$. Here $a$ is the acceleration of the string - and acceleration is the second time derivative of the excursion that we are interested in, i.e. it is $a = u_{tt}$. The force $F$ that causes the acceleration is related to the local curvature of the string given by the second spatial derivative $u_{xx}$ of the excursion. The governing equation basically says: the local acceleration is proportional to the local curvature: $u_{tt} = k u_{xx}$ for some proportionality constant $k$. If this all makes sense to you and you are nodding along then congratulations! You have just casually understood the wave equation - one important example of a partial differential equation...



%If we denote by $u_{tt}$ the second derivative of the excursion with respect to time $t$, i.e. the (local) acceleration of the string and by $u_{xx}$ the second derivative of the excursion with respect to position $x$, i.e. the local force generating mechanism, then the governing equation for our sought $u(x,t)$ is given by $u_{tt} = k u_{xx}$ for some proportionality constant $k$. Congratulations! You have just casually understood the wave equation - one important example of a partial differential equation...


...TBC...

% Make a subsection Examples and list the stuff below as subsubs and paragraphs

\subsection{PDEs for Scalar Fields}

\subsubsection{Transport Equation}

% https://en.wikibooks.org/wiki/Partial_Differential_Equations/The_transport_equation
% https://en.wikipedia.org/wiki/Continuity_equation
% https://www.simscale.com/docs/simwiki/numerics-background/what-is-the-transport-equation/

\subsubsection{Diffusion Equation}

% aka Heat Equation

\subsubsection{Convection-Diffusion Equation}

% https://en.wikipedia.org/wiki/Convection%E2%80%93diffusion_equation

\subsubsection{Wave Equation}

%\begin{equation}
%1\text{D:  } u_{tt} + u_{xx} = 0, \qquad 
%2\text{D:  } u_{tt} + u_{xx} + u_{yy} = 0, \qquad 
%n\text{D:  } u_{tt} + \Delta u = 0
%\end{equation}
% maybe include the factor $c^2$ and remove the 2D case, 1D and nD should be enough and form more complicated equations, listing all 3 cases may take too much space, maybe allow for an rhs f(x), f(x,y), f(r)  where r= (x,y,z) 

\begin{equation}
1\text{D:  } c^2 u_{tt} + u_{xx} = f(x), \qquad 
2\text{D:  } c^2 u_{tt} + u_{xx} + u_{yy} = f(x,y), \qquad 
n\text{D:  } c^2 u_{tt} + \Delta u = f(\mathbf{r})
\end{equation}




% Burgers Equation




\subsection{PDEs for Vector Fields}

\subsubsection{Navier-Stokes Equations}

\subsubsection{Maxwell's Equations}


\subsection{PDEs for Tensor Fields}

\subsubsection{Maxwell's Equations Revisited}

\subsubsection{Einstein's Field Equations}




\begin{comment}

\end{comment}