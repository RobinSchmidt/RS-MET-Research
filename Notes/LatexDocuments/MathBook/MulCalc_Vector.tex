\section{Vector Calculus} 

\subsection{Scalar and Vector Fields}
When we are in the realm of vector calculus, the mathematical idea of a \emph{field} refers to a function $f$ that takes a vector $\mathbf{x}$, usually from $\mathbb{R}^n$, as input and produces an output - as functions do. But we must be careful not confuse this notion with the notion of a field in abstract algebra. There, a field means something else. When the output of the function is a simple scalar, i.e. a (real) number, then we call our field a \emph{scalar field}. So, scalar fields are functions $f: \mathbb{R}^n \rightarrow \mathbb{R}$. If the output of the function is itself a vector and that vector is of the same dimensionality as the input vector, we call the field a \emph{vector field}. So, vector fields are functions $\mathbf{f}: \mathbb{R}^n \rightarrow \mathbb{R}^n$. Typographically, we denote functions with vector outputs in boldface. In classic vector calculus, the dimensionality $n$ is usually 3, sometimes 2. Therefore, for convenience, we may write scalar fields as $f = f(x,y,z)$ or $f = f(x,y)$ and vector fields as $\mathbf{f}(x,y,z) = (f_1(x,y,z), f_2(x,y,z), f_3(x,y,z))^T$ or $\mathbf{f}(x,y) = (f_1(x,y), f_2(x,y))^T$. One could perhaps use $f_x, f_y, f_z$ instead of $f_1, f_2, f_3$. However, that notation is already commonly used to denote partial derivatives in the realm of partial differential equations (PDEs) which we will encounter later, so we won't use that notation here because we indeed want to use it there later. 

%One also sometimes sees the notation $u(x), v(x), w(x)$ ...really? In differential geometry, one often sees x(u,v), y(u,v), z(u,v) ...but the other way around? ...not sure

%\begin{eqnarray}
% f(x,y,z) &= 
%\end{eqnarray}

%-maybe plot an example vector field


\subsection{Differential Operators}
Operators are mathematical objects that can "act on" or be "applied to" functions to produce new functions. Differential operators are operators which are built from (partial) derivatives. In vector calculus, we will encounter operators that act on scalar- and vector fields and produce other scalar- and vector fields. In this context, one could consider taking the Jacobian and Hessian as operators that act on vector and scalar fields respectively and produce matrix fields as outputs. However, Jacobians and Hessians themselves in their original form are not commonly encountered in the realm of 3D vector calculus, as far as I know. They belong more into the realm of general multivariable calculus and/or maybe into matrix- or tensor calculus. Sometimes, the Jacobian appears in disguise as a gradient of a vector field which is the transpose of the Jacobian and we also encounter the trace of the Hessian in the form of the Laplacian operator.

%[TODO: figure out!]

%-mention how the Jacobian and Hessian fit into this scheme. I think, we may consider them as operators that produce matrix-valued functions, i.e. matrix-fields or tensor fields
%-maybe they occur in disguise as the vector gradient (Jacobian) and Laplacian (trace of Hessian)?

\subsubsection{Gradient}
We already encountered the general $n$-dimensional gradient operator that acts on a scalar field $f(\mathbf{x})$ and produces a vector field $\mathbf{grad} f(\mathbf{x})$. The gradient of scalar fields $f: \mathbb{R}^3 \rightarrow \mathbb{R}$ is indeed one of the common differential operators of classic 3D vector calculus. For such a scalar field $f = f(x,y,z)$, the gradient is defined as:
\begin{equation}
 \grad f = \nabla f = ...
\end{equation}
where we introduced two possible notations at once - one using $\grad$ and one using the nabla-operator $\nabla$. There is also a notion of a gradient of a vector field. This operator is much less common...TBC...

%-can it also be applied to vector fields? Does it then, in fact, produce the Jacobian?
%-https://en.wikipedia.org/wiki/Gradient#Gradient_of_a_vector_field


\subsubsection{Divergence}
The divergence operator acts on a vector field and produces a scalar field...TBC...

%-introduce the nabla dot f notation
%-explain hwo it generalizes to nD

\subsubsection{Curl}
The curl operator acts on a vector field and produces another vector field...TBC...

%-explain the hack for 2D (embedding 2D in 3D)
%-explain how it doesn't generalize to nD and how this traces back to the cross product

\subsubsection{Laplacian}
The Laplacian operator acts on a scalar field and produces another scalar field. It is actually the divergence of the gradient (VERIFY!) and is therefore a second order differential operator in the sense that it involves second order partial derivatives...TBC...

%-It's the trace of the Hessian, I think. When the Hessian is viewed as tensor, that also corresponds to a contraction.
%-I think, it can also be applied to vector fields?


\subsubsection{Identities}
ToDo: give a comprehensive list of identities involving the differential operators above.


% introduce nabla dot nabla and nabla-squared notation

%\subsection{Harmonic Fields}
%-Laplacian must be zero?
%-harmonic conjugates
%-https://en.wikipedia.org/wiki/Skew_gradient

\subsection{Potential Fields}
We have seen how we are able to produce new fields by applying differential operators to given fields. Fields that are so produced are special and have some convenient properties. If we are able to express a vector field as gradient of a scalar field, then the scalar field contains all the information that the original vector field contains. We can always get our original vector field back by just taking its gradient. We lost no information - yet, a scalar field is a much simpler object than a vector field. Of course, it's not always possible to find such a scalar field - but if it is, we should take advantage of it. If such a scalar field, let's call it $F$, exists for a given vector field $\mathbf{f}$, then we call $F$ a \emph{potential} for $\mathbf{f}$. A potential is a sort of antiderivative of a vector field and sometimes even called by that name. You need to be careful about different sign conventions used in the literature: some authors would define $-F$ to be the potential of $\mathbf{f}$. ...TBC...

% figure out sign conventions - i think, potentials are sometimes defined via negative gradient
% https://en.wikipedia.org/wiki/Scalar_potential
% both conventions are used - Bärwolff uses the one without the minus - mention that one finds both notations in the literature

%A vector field that can be produced by taking the gradient of a given scalar field is called a potential field

%-scalar and vector potentials ...but we need the curl be defined to define vector potentials
%maybe re-organize:  make this subsection "Scalar and vector Fields" and introduce potential fields after the "Differential Operators" or maybe even after the integrals...nah...maybe not

%-what, if the divergence is zero? Have these fields also some special name?

\subsubsection{Finding Scalar Potentials}
So, if we encounter a vector field $\mathbf{f}$ in some application, how do find its scalar potential? First things first - we should first figure out whether or not such a scalar potential even exists. Fortunately, there's a simple criterion for that: For a given vector field $\mathbf{f}$, a scalar potential $F$ exists, if and only if ...

\subsubsection{Finding Vector Potentials}



\subsection{Contour Integrals}
\subsubsection{Area of a "Curtain"}
\subsubsection{Work in a Force Field}
\subsubsection{Flux through a Curve}

\subsection{Surface Integrals}
\subsubsection{Area of a Surface}
\subsubsection{Flux through a Surface}

\subsection{Integral Theorems}

\subsubsection{Fundamental Theorem of Line Integrals}
Let $F(\mathbf{x})$ be a scalar-valued function of some vector input $\mathbf{x}$ in $\mathbb{R}^n$ and $\mathbf{f(x)} = F'(\mathbf{x}) = \grad F(\mathbf{x})$ be the gradient of $F$. We observe that $\mathbf{f} = F'$ is a vector-valued function of $\mathbf{x}$ and that according to our definition, $F$ is a potential for $\mathbf{f}$. Let $C$ be a contour with start and end points $\mathbf{a,b} \in \mathbb{R}^n$ and let $C$ be given as a parametric curve $\mathbf{c} = \mathbf{c}(t)$ with parameter $t$. Let $d\mathbf{s} = \dot{\mathbf{c}} = d\mathbf{c} / dt$ be an infinitesimal tangent vector along $C$ at the position $\mathbf{c}(t)$. Then, the following holds:
\begin{equation}
 \int_C \mathbf{f} \cdot d\mathbf{s} 
% = \int_{\mathbf{a}}^{\mathbf{b}} \mathbf{f} \cdot d\mathbf{s} 
 = F(\mathbf{b}) - F(\mathbf{a}) 
\end{equation}
Let's unpack what this formula means: In the integrand, we are forming a dot product (aka scalar product) between the gradient field $\mathbf{f}$, which is a vector field, and an infinitesimal tangent vector $d\mathbf{s}$ of our parametric contour/curve $C$. When the theorem is stated in this form, it is implicitly understood that we are evaluating $\mathbf{f}$ and $d\mathbf{s}$ at some given value for the parameter $t$. This infinitesimal dot product of two vectors is then integrated, i.e. summed, over the whole length of the curve. Via the dot product, at each point along the curve, we "measure" how much the vector field points into the local direction of our curve and that "measurement" gives a little contribution to our integral. If the vector field is locally aligned with the curve, we get the maximum contribution. If the vector field is locally perpendicular to our curve, we get no contribution at all. The right hand side is easy to understand: we just evaluate the scalar field $F$ at the two endpoints of the curve and subtract the two resulting scalar values. What this implies is that the value for the integral depends only on the endpoints $\mathbf{a,b}$ and not at all on the specific curve, i.e. on the way how we move from $\mathbf{a}$ to $\mathbf{b}$. This path-indpendence is a special convenient feature of potential fields. You cannot expect that from any general vector field. Not all vector fields can be expressed as a gradient of some scalar field. But for those that can, we can apply this theorem and thereby greatly simplify the evaluation of line integrals along paths through this field.

...TBC...

ToDo: explain, how the contour is given as parametric equations and formulate the theorem in terms of the parametric decription as an integral ove the parameter $t$, expian


%-applies to potential fields
%-could also be called "Gradient Theorem" - would fit well with "Divergence-" and "Curl Theorem"
% https://en.wikipedia.org/wiki/Gradient_theorem
% https://www.khanacademy.org/math/multivariable-calculus/integrating-multivariable-functions/line-integrals-in-vector-fields-articles/a/fundamental-theorem-of-line-integrals
% https://www.contemporarycalculus.com/dh/Calculus_all/Ch15_4.pdf
% https://math.mit.edu/~jorloff/18.04/notes/greenstheorem.pdf
% https://tutorial.math.lamar.edu/classes/calciii/fundthmlineintegrals.aspx

\subsubsection{Green's Theorem}

\subsubsection{Gauss's Divergence Theorem}

\subsubsection{Stokes's Curl Theorem}
Stokes's theorem is a 3D generalization of Green's Theorem. It relates a surface integral of a vector field over an arbitrary surface in 3D space to a line integral over the boundary of that surface. Perhaps it's more appropriate to say that Green's theorem is just a special case of Stokes' theorem in which the surface under consideration is just a region in the flat $xy$-plane. ...TBC...

%\subsubsection{Green's Formulas}





\begin{comment}

-make a section for how to compute potentials for a given evctor field
-before that, have a section for how to test, if a potential even exists 

-i'm not sure, if path integrals and surface integrals should be treated here or in the "General Concepts" section ...maybe here - there, we only do integrals of scalar fields


https://en.wikipedia.org/wiki/Matrix_calculus


\end{comment}

