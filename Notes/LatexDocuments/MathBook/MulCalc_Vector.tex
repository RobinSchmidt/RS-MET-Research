\section{Vector Calculus} 

\subsection{Fields}
The mathematical idea of a "field"

\subsubsection{Scalar Fields}
\subsubsection{Vector Fields}
\subsubsection{Potential Fields}

\subsection{Differential Operators}
\subsubsection{Gradient}
\subsubsection{Divergence}
\subsubsection{Curl}
\subsubsection{Laplacian}


\subsection{Contour Integrals}
\subsubsection{Area of a "Curtain"}
\subsubsection{Work in a Force Field}
\subsubsection{Flux through a Curve}

\subsection{Surface Integrals}
\subsubsection{Area of a Surface}
\subsubsection{Flux through a Surface}

\subsection{Integral Theorems}

\subsubsection{Fundamental Theorem of Line Integrals}
Let $F(\mathbf{x})$ be a scalar-valued function of some vector input $\mathbf{x}$ in $\mathbb{R}^n$ and $\mathbf{f(x)} = F'(\mathbf{x}) = \grad F(\mathbf{x})$ be the gradient of $F$. We observe that $\mathbf{f} = F'$ is a vector-valued function of $\mathbf{x}$ and that according to our definition, $F$ is a potential for $\mathbf{f}$. Let $C$ be a contour with start and end points $\mathbf{a,b} \in \mathbb{R}^n$ and let $C$ be given as a parametric curve $\mathbf{c} = \mathbf{c}(t)$ with parameter $t$. Let $d\mathbf{s} = \dot{\mathbf{c}} = d\mathbf{c} / dt$ be an infinitesimal tangent vector along $C$ at the position $\mathbf{c}(t)$. Then, the following holds:
\begin{equation}
 \int_C \mathbf{f} \cdot d\mathbf{s} 
% = \int_{\mathbf{a}}^{\mathbf{b}} \mathbf{f} \cdot d\mathbf{s} 
 = F(\mathbf{b}) - F(\mathbf{a}) 
\end{equation}
Let's unpack what this formula means: In the integrand, we are forming a dot product (aka scalar product) between the gradient field $\mathbf{f}$, which is a vector field, and an infinitesimal tangent vector $d\mathbf{s}$ of our parametric contour/curve $C$. When the theorem is stated in this form, it is implicitly understood that we are evaluating $\mathbf{f}$ and $d\mathbf{s}$ at some given value for the parameter $t$. This infinitesimal dot product of two vectors is then integrated, i.e. summed, over the whole length of the curve. Via the dot product, at each point along the curve, we "measure" how much the vector field points into the local direction of our curve and that "measurement" gives a little contribution to our integral. If the vector field is locally aligned with the curve, we get the maximum contribution. If the vector field is locally perpendicular to our curve, we get no contribution at all. The right hand side is easy to understand: we just evaluate the scalar field $F$ at the two endpoints of the curve and subtract the two resulting scalar values. What this implies is that the value for the integral depends only on the endpoints $\mathbf{a,b}$ and not at all on the specific curve, i.e. on the way how we move from $\mathbf{a}$ to $\mathbf{b}$. This path-indpendence is a special convenient feature of potential fields. You cannot expect that from any general vector field. Not all vector fields can be expressed as a gradient of some scalar field. But for those that can, we can apply this theorem and thereby greatly simplify the evaluation of line integrals along paths through this field.

...TBC...

ToDo: explain, how the contour is given as parametric equations and formulate the theorem in terms of the parametric decription as an integral ove the parameter $t$, expian


%-applies to potential fields
% https://www.khanacademy.org/math/multivariable-calculus/integrating-multivariable-functions/line-integrals-in-vector-fields-articles/a/fundamental-theorem-of-line-integrals
% https://www.contemporarycalculus.com/dh/Calculus_all/Ch15_4.pdf
% https://math.mit.edu/~jorloff/18.04/notes/greenstheorem.pdf
% https://tutorial.math.lamar.edu/classes/calciii/fundthmlineintegrals.aspx

\subsubsection{Green's Theorem}
\subsubsection{Gauss's Divergence Theorem}
\subsubsection{Stokes's Curl Theorem}





\begin{comment}

-make a section for how to compute potentials for a given evctor field
-before that, have a section for how to test, if a potential even exists 

-i'm not sure, if path integrals and surface integrals should be treated here or in the "General Concepts" section ...maybe here - there, we only do integrals of scalar fields

\end{comment}

