\section{Exterior Algebra}
Exterior algebra, also known as Grassmann algebra, is an extension of vector algebra. In classic vector algebra, we have (among other things) the following products defined: (1) the dot product of two vectors, which produces a scalar, (2) the product of a scalar and a vector, which produces another vector, (3) the cross product between two vectors which produces another vector. The cross product is actually a bit problematic. What it produces is not actually a proper vector anymore. Sure, it's a column of 3 numbers, but it shows some weird behaviors compared to regular vectors [TODO: transformation proerties und reflection]. It doesn't generalize to spaces of other dimensionality and it is not even associative (it's anticommutative, though). For this reason, the 3-number quantities that arise from cross products are sometimes called pseudovectors or axial vectors instead of vectors. Exterior algebra sheds some light on this odd behavior of the cross product and cleans it up by defining a new product, called the wedge product, which can for many purposes be used as a replacement for the cross product but encapsulates a much more powerful idea, is better behaved and generalizes to higher dimensions. This new kind of product can also take two vectors as input but in contrast to the cross product, its output in this case is not a vector but a new kind of object, called a bivector. In 3D, such bivectors have 3 components just like ordinary vectors, but conceptually, they represent a different idea. Building up on that, we may then also define trivectors and in general $k$-vectors, where the natural number $k$ is called the grade of the object and ranges from $0$ to $n$ where $n$ is the dimensionality of our underlying vector space. The set of all these $k$-vectors of different grades along with all the operations among them forms the exterior algebra. It includes our plain old regular vectors as 1-vectors, scalars as 0-vectors and then some more interesting stuff...

%Maybe mention that typical presentations of exterior algebra are in terms of covectors (aka 1-forms), etc rather than vectors themselves, i.e. it's done in the dual space. We phrase things here in the primal space for pedagogical reasons to build up on what is known at this point without confusing the reader with all this "dual space" noise

\subsection{The Wedge Product}
The cross product $\mathbf{c = a \times b}$ of two vectors in 3D space is usually depicted as a normal vector to the parallelogram that is formed by the two input vectors $\mathbf{a,b}$. The area of the parallelogram determines the length of this vector and the direction is perpendicular to the plane in which the parallelogram lives. If you change the order of the inputs $\mathbf{a,b}$, the output  $\mathbf{c}$ will change its sign, i.e. stick out of the parallelogram into the opposite direction. The wedge product $\mathbf{w = a \wedge b}$ can be thought of as literally and directly representing the parallelogram itself - not a normal vector to it. That's a conceptual shift of perspective that leads to a more natural and more useful way of representing planes and plane segments.

\subsubsection{Bivectors}
You may think of a vector as representing a 1-dimensional subspace of $\mathbb{R}^3$, i.e. a line with a particular direction, with some indication of size attached to it via the vector's length and some idication of orientation via its sign. By analogy, the parallelogram arising from wedging two vectors together (yes - picture it in your head exactly this way!), can be thought of as representing a 2-dimensional subspace of $\mathbb{R}^3$, i.e. a plane in which you can move in 2 independent directions. This subspace includes all points that can be reached by a linear combination of the two vectors $\mathbf{a,b}$ that make up the wedge. As with a vector, we have a size associated with this subspace via the area of the parallelogram as well as an orientation, defined by following $\mathbf{a}$ and going around the parallelogram. This oriented area element is what we call a bivector. We can think of it as a parallelogram formed by two vectors, however, it is important to note that any information about the shape of the parallelogram is lost. The bivector only represents the subspace spanned by the two directions $\mathbf{a,b}$, an area and an orientation - it has no shape associated with it. Yes, it arose from a particular shape, but the information about that shape is lost in forming the wedge product. A bivector is just an amorphous oriented plane segment and we are free to picture it in our head in any 2D shape we like as long as orientation and area is preserved. Parallelograms are a natural choice, circular discs may sometimes be appropriate, too.

\subsubsection{Trivectors}
The wedge product can do more: not only can we wedge together two vectors to form a bivector but we can also wedge together a bivector with another vector. We can write this as $\mathbf{t = (a \wedge b) \wedge c}$. First we form the wedge product of $\mathbf{a}$ and  $\mathbf{b}$ yielding a bivector and then we take the wedge product of the result with a third vector $\mathbf{c}$. This results in a so called trivector which can be pictured as the parallelepiped that is formed by extruding the parallelogram $\mathbf{a \wedge b}$ along the direction of $\mathbf{c}$. Again, the actual shapes are unimportant - if we imagine $\mathbf{a \wedge b}$ as a disc, we'd get a skewed cylinder instead of an parallelepiped. No shape information is encoded in the trivector and we are free to picture it in our heads with any 3D shape we like as long as it has the right volume.
%maybe explain relation to triple product
%explain how orientation is pictured - i think, as in and out?

\subsubsection{k-vectors}
Now we can see a pattern emerging: vectors represent oriented line segments, bivectors represent oriented plane segments and trivectors represent oriented volume segments of space. In general, a $k$-vector represents an oriented segment of a $k$-dimensional subspace of our underlying $n$-dimensional vector space. In this new view, plain old vectors are also called 1-vectors, bivectors are 2-vectors and trivectors are 3-vectors. If we work in 3D space, we can't go any higher than 3: there is no 4D subspace of 3D space (note on terminology: the 3D space itself is also considered a subspace of 3D space - it's like with subsets in the $\subseteq$ sense). But we can actually go one lower: the set of 0-vectors, i.e. the 0-dimensional subspace of 3D space, consists of our good old friends, the scalars. In 3D, we have seen that scalars can be represented by a single number, vectors by 3 numbers, bivectors also by 3 numbers and trivectors again by a single number. Does 1,3,3,1 ring a bell? It's the 3rd line of Pascal's triangle! ...and that is no coincidence! It is indeed generally true that in an $n$D vector space, the space of $k$-vectors is $n$-choose-$k$ dimensional.
%-talk about a basis for the vector, bivector and trivector spaces
%-talk about how the confusion of bivectors and vectors arises in 3D - the mathematical "coincidence" that lets us represent area-segment with vectors which has all these confusing consequences

\subsubsection{Algebraic Properties}
Let $\mathbf{a}$ be a $k$-vector and $\mathbf{b}$ be a $p$-vector. The wedge product $\mathbf{a \wedge b}$ between the two will be a $(k+p)$-vector. Like the cross product, the wedge product is anticommutative: $\mathbf{a \wedge b} = -\mathbf{b \wedge a}$ and distributive over addition: $\mathbf{a \wedge (b + c) = a \wedge b + a \wedge c }$. Unlike the cross product, the wedge product is also associative: $\mathbf{(a \wedge b) \wedge c = a \wedge (b \wedge c)}$. The anticommutativity implies that a wedge product of a $k$-vector with itself is always zero: $\mathbf{a \wedge a} = 0$. From this fact, it can also be derived that a wedge product must be zero as soon as $k+p > n$ which is a restatement of the fact $n$D space has no $k$-dimensional subspace for $k>n$. If you try to form a wedge product of a trivector with another vector in 3D, you don't magically get some sort of 4D object that lives in 3D space. Instead, you get something that would formally be 4D but is identically zero [todo: verify that!]. Note that the wedge product also encompasses the product of a scalar (0-vector) with a vector (1-vector), i.e. our old scalar-times-vector multiplication from vector algebra. Wedging together 3 1-vectors gives a trivector which in 3D has only 1 component, so it acts like a scalar. This is why trivectors in 3D are also called pseudoscalars. The wedge product between 3 1-vectors corresponds to the scalar triple product from classic vector algebra [verify!] [Q: can the vector-times-bivector be seen as the dot-product? ...nah - i don't think so - i think, for this, we need k-forms which is a different concept...i think]

%can take different types of args (vec,vec -> bivec),(vec,bivec -> trivec),(vec,scl -> vec)
%generalizes to n dimensions
%is associative and anticommutative

\subsection{The Hodge Dual}
Another important operation in exterior algebra is to take the so called Hodge-dual of a $k$-vector which will produce an $(n-k)$-vector where $n$ is the dimensionality of the space. The operation is denoted by $\star$ and pronounced "star" or maybe "Hodge-star" if it isn't clear from the context what kind of star is meant. The result of applying the Hodge star to a $k$-vector can be pictured as spanning a space given by the orthogonal complement of the $k$-vector. For example, the dual of a bivector representing a plane segment in 3D is the normal vector to that plane segment - and vice versa. The dual to a scalar is a trivector.
%dualizing a vector -> covector? dualize bivektor -> pseudovector?

%\subsection{k-forms}

%\subsection{The musical isomorphisms}

\begin{comment}

-explain covectors, co-scalars, cobivectors, etc. - i think, cobivector is a better term than 
 bicovector because the dualization of the whole space (from vectors to covectors) happens first. 
 Then there's the 2nd dualization within the space.
-
-explain k-forms: a 1-form is a device to measure the length of a vector, a 2-from measures the area of 
 a bivector, a 3-form measures the volume of a trivector. in general, a k-form takes a k-vector as input
 and produces a scalar. can we also picture it as taking k 1-vectors? i think so. when this is done, it is 
 an alternating an multilinear function. i was always confused by thinking covectors are the same thing as 1-forms...but i think that's false? or is it? maybe we can have covectors, cobivectors, etc?

resources:
https://en.wikipedia.org/wiki/Exterior_algebra
https://de.wikipedia.org/wiki/Gra%C3%9Fmann-Algebra

https://math.wikia.org/wiki/Dot_product
https://math.wikia.org/wiki/Wedge_product
https://math.wikia.org/wiki/Cross_product
https://math.wikia.org/wiki/Pseudovector
https://math.wikia.org/wiki/Scalar_triple_product
https://math.wikia.org/wiki/Vector_triple_product

https://en.wikipedia.org/wiki/Graded_vector_space

https://en.wikipedia.org/wiki/Multilinear_algebra

https://en.wikipedia.org/wiki/Hodge_star_operator
https://de.wikipedia.org/wiki/Hodge-Stern-Operator
https://www.youtube.com/watch?v=tyaWHQO-wSc
https://en-academic.com/dic.nsf/enwiki/186276
https://handwiki.org/wiki/Hodge_star_operator  good!


https://www.youtube.com/watch?v=ISKJPmuZkbY  March 9th: Fun Applications of Geometric Algebra! by Logan Lim
This video talks about both types of dualization at around 9:45. Maybe doing both to a point in R^4 
yields again a point - but in the dual space? point in R (vector) -> dualize to R* -> plane in dual 
space (vector) -> dualize within R* -> point in R* (trivector)  ...figure out!
-It also relates Geometric Algebra to Rational Trigonometry! ...cool!


\end{comment}