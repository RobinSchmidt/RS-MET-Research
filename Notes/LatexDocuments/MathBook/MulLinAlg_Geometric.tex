\section{Geometric Algebra}
Geometric algebra, also known as Clifford algebra, can be viewed as a further generalization of exterior algebra. In exterior algebra, the $k$-vectors of different grades are well separated and do not mix. That's why it's called a \emph{graded} algebra. [VERIFY!] Geometric algebra brings them all together into a shared space by means of taking a formal sum of objects of different grades. The sum is "formal" in the sense that we do not actually perform any calculations in such a sum. It's just a way to pack several different things together into a single entity but keeping them seperated by attaching some "tags" to them. This "sum" is totally analogous to the $z = a + \i b$ in complex numbers. Here, the imaginary unit $\i$ is a tag that keeps the imaginary part seperated from the real part. In fact, a plain old vector can also be written as a formal sum of the basis vectors scaled by the coordinates, so what we do here is not really something crazily new. In our step from exterior algebra to geometric algebra, we create a joint vector space from a bunch of formerly separated vector spaces. 
%The spaces of scalars, vectors , bivectors, trivectors, etc. will be merged into a single vector space of so called multivectors.  
In algebraic lingo, we are taking a \emph{direct sum} of the existing vector spaces [VERIFY]. That alone would not really give us any new structure. To achieve that, geometric algebra also introduces a new kind of product: the geometric product. If the inputs are two vectors, their geometric product is simply defined as the sum of the dot product and the wedge product (which, in this context, are also called inner and outer product respectively), where the sum is to be understood in the aforementioned formal sense. Recall that the dot product between two vectors gives a scalar and their wedge product gives a bivector. That means the geometric product of two vectors will yield a formal sum of a scalar and a bivector. In general, such combinations of scalars, vectors, bivectors and trivectors will be called multivectors. Such things may seem like very strange objects that defy any intuition. It's perhaps best to think of the differently graded $k$-vectors inside a multivector as encoding disparate aspects of the same situation. By bringing disparate quantities together in its particular ways, geometric algebra allows to write down certain systems of equations that relate different things in a single equation. As a little spoiler, a good example are Maxwell's equations of electromagnetism. In their classic formulation in the language of vector calculus, they are a system of 4 coupled partial differential equations relating the electric field, magnetic field, current density and charge density. In a formulation based on geometric algebra, these 4 equations merge into a single equation and even a very simple one. The different aspects of the situation, i.e. the different physical quantities that are involved, are encoded in the different grades of the multivector and thereby kept recoverably separated, yet packaged together in a way that is actually quite natural and useful. For some reason, geometric algebra manages to package these disparate quantities of electromagnetism together in just the right way to allow for a simple formulation of these laws. This might not be surprising, if the formalism would have been developed specifically for electromagnetism. But that's not the case. It has been developed for geometry. Many domain specific mathematical formalisms have been developed for various physical phenomena (Pauli spin matrices, Dirac matrices, Minkowski space, etc.) and it turns out that many of them can be subsumed by geometric algebra and the geometric calculus that is based on it. Some physicists take that fact as a strong indication that this is the right mathematical language for physics. Besides physicists, there's strong interest in the computer graphics community in that topic because - unsurprisingly - it's very useful for geometric computations as well. Nevertheless, it is not yet a widely known formalism and, in my humble opionion, deserves more recognition.

%As an appetizer, the Maxwell equations

%A k-vector in nD space can be specified by n-choose-k numbers.

%Geometric algebra

% https://en.wikipedia.org/wiki/Direct_sum


\begin{comment}


Notation for geometric algebras: $\mathbb{G}^{i,j,k}$ where i is the number of basis vectors that square to 1, j the number of basis vectors that square to -1 and k the number of basis vectors that square to zero (verify!). The triple $i,j,k$ is called the signature of the algebra.

Example algebras:
-G(0,0,0): real numbers
-G(1,0,0): hyperbolic numbers
-G(0,1,0): complex numbers
-G(0,0,1): dual numbers
-G(2,0,0): 2D Euclidean space
-G(3,0,0): 3D Euclidean space
-G(3,1,0): Minkowski spacetime (somtimes also G(1,3,0) is used - a matter of convention)

-the complex numbers or versions of it with a non-commuting imaginary unit appear in various sub-algebras of higher-dimensional gemoetric algebras, for example in G(2,0,0) and G(3,0,0)

ToDo: mention generalizations with arbitrary metric tensors

%...

https://www.youtube.com/watch?v=htYh-Tq7ZBI  Why can't you multiply vectors? [Dutch Game Day, 2023] 
40:29 Generalized curvature of a curve: \kappa = (f' \wedge f'') / |f'|^3
 -> For geometric calculus

Geometric algebra takes all the different objects from exterior algebra and throws them together 
into a single pot.


The Fascinating perspective of Geometric Algebra #SoME4 #SoMEPi
https://www.youtube.com/watch?v=m5aKoQ2FTeo
-Can this be seen as a nesting of 2 geometric algebras - the 3D algebra G^(3,0,0) for the
 Maxwell's equations - but here, the scalars within it are themselves complex-valued? And the
 complex numbers are isomorphic to the geometric algebra G(0,1,0) such that we get a composed
 geometric algebra? Is such a composition of geometric algebras a thing?
 

The Geometric Truth Behind Pauli Spinors
https://www.youtube.com/watch?v=S8bATrOPgWU

Eccentric:
https://www.youtube.com/@EccentricTuber
-Channel with lots of geometric algebra content

sudgylacmoe:
https://www.youtube.com/@sudgylacmoe
-Channel with lots of geometric algebra content


The Periodic Table of Geometric Algebras - CL(3,0,1) does all 3D game math, so what does CL(p,q,r) d
https://www.youtube.com/watch?v=oXcp3gA8erQ  


Amazing Things You Can Do in Geometric Algebra Explained
https://www.youtube.com/watch?v=xGuN6KM_D18

The Fascinating perspective of Geometric Algebra #SoMEpi
https://www.youtube.com/watch?v=m5aKoQ2FTeo
-Unifies Maxwell's equations with gravity by using complex numbers for the components of
 multivectors

Exploring Geometric Algebra via Quantum Mechanics: Introduction
https://www.youtube.com/watch?v=nUhX1c8IRJs


It gets complex. Spacetime Rotations #SoME4
https://www.youtube.com/watch?v=wuHnK9_MBpg
-Lorentz trafos via rotors


\end{comment}