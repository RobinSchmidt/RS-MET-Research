\section{Geometric Algebra}
Geometric algebra, also known as Clifford algebra, can be viewed as a further generalization of exterior algebra. In exterior algebra, the k-vectors of different grades are well separated and do not mix. Geometric algebra brings them all together into a shared space by means of taking a formal sum of objects of different grades. The sum is "formal" in the sense that we do not actually perform any calculations in such a sum. It's just a way to pack several different things together into a whole but keeping them seperated by attaching some "tags" to them. This "sum" is totally analogous to the $z = a + \i b$ in complex numbers. Here, the imaginary unit $\i$ is a tag that keeps the imaginary part seperated from the real part. In fact, a plain old vector can also be written as a formal sum of the basis vectors scaled by the coordinates [verify!], so what we do here is not really something crazily new. Geometric algebra also introduces a new kind of product: the geometric product. This is simply defined as the sum of the dot product and the wedge product (which, in this context, are also called inner and outer product respectively), where the sum is to be understood in the aforementioned formal sense. Recall that the dot product between two vectors gives a scalar and their wedge product gives a bivector. That means the geometric product of two vectors will yield a formal sum of a scalar and a bivector. In general, such combinations of scalars, vectors, bivectors and trivectors will be called multivectors. Such things may seem like very strange objects that defy any intuition. It's perhaps best to think of the differently graded k-vectors inside a multivector as encoding disparate aspects of the same situation. By bringing disparate quantities together in its particular ways, geometric algebra allows to write down certain systems of equations that relate different things in a single equation. As a little spoiler, an example are Maxwell's equations of electromagnetism. In their classic formulation in the language of vector calculus, they are a system of 4 coupled partial differential equations relating the electric field, magnetic field, current density and charge density. In a formulation based on geometric algebra, these 4 equations merge into a single equation and even a very simple one. The different aspects of the situation, i.e. the different physical quantities that are involved, are encoded in the different grades of the multivector and thereby kept recoverably separated yet packed together in a way that is actually quite natural and useful.

%As an appetizer, the Maxwell equations

%A k-vector in nD space can be specified by n-choose-k numbers.

%Geometric algebra


\begin{comment}

%...

\end{comment}