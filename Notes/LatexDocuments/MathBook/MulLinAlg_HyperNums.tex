\section{Hypercomplex Numbers}
We have seen the real numbers and the complex numbers as an extension of the reals motivated by the desire to be able to solve polynomial equations. We arrived at a two dimensional number system. In general, when we call something a number system, we expect that to be some set of objects which we can treat (more or less) like the rational, real or complex numbers that we are used to. In particular: we expect from such a set that addition, subtraction, multiplication and division operations are defined between pairs of such set elements and we expect these operations to behave (more or less) in the same way that we know from our more familiar number systems. That is: we may expect things like associativity, commutativity and distributivity from these operations. ...TBC...

% Explain what we expect from a number system


\subsection{Complex Numbers}

\subsection{Hyperbolic Numbers}

\subsection{Dual Numbers}

\subsection{Quaternions}

\subsection{Octonions}

\subsection{Sedenions}







\begin{comment}

-Frame the algebras of hypercomplex numbers as particular subsets of matrices. They can
 all be represented by a constrained set of matrices

\end{comment}