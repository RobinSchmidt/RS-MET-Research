\section{Hypercomplex Numbers}
We have seen the real numbers and the complex numbers as an extension of the reals motivated by the desire to be able to solve polynomial equations. We arrived at a two dimensional number system. One might ask the question whether or not there are other number systems out there. But what do we mean by the term "number system"? What features must an object have, such that we could reasonably call it a "number"? The answer lies mostly in the kinds of operations that we expect to be able to do with these objects. In general, when we call something a number system, we expect that to be some set of objects which we can treat (more or less) like the rational, real or complex numbers that we are used to. In particular: we expect from such a set that addition, subtraction, multiplication and division operations are defined between pairs of such set elements and we expect these operations to behave (more or less) in the same way that we know from our more familiar number systems. That is: we may expect things like associativity, commutativity and distributivity from these operations. ...TBC...

% Explain what we expect from a number system

% Note that we can do these things with matrices, 

\subsection{Complex Numbers}
Let's revisit the set of complex numbers and look at them through the lens of linear algebra and Euclidean geometry. Addition of complex numbers is like vector addition in $\mathbb{R}^2$ and the multiplication of complex numbers is a combination of scaling and rotation. Such an operation in the Euclidean plane can be encoded in a $2 \times 2$ matrix $\mathbf{Z}$ of the form:
\begin{equation}
 \mathbf{Z} 
 = r \cdot
 \begin{pmatrix}
 \cos(\phi) & -\sin(\phi) \\
 \sin(\phi) &  \cos(\phi)
 \end{pmatrix}
 =
 \begin{pmatrix}
  a & -b \\
 b  &  a
 \end{pmatrix} 
\end{equation}
I have chosen the letter $\mathbf{Z}$ for a reason: we conventionally use $z$ to denote complex numbers and indeed can we use the matrix $\mathbf{Z}$ above to represent the complex number $z = a + \i b = r(\cos(\phi) + \i \sin(\phi)) = r \e^{\i \phi}$. But how does that \emph{matrix} representation fit together with the idea to represent complex numbers as \emph{vectors} in $\mathbb{R}^2$? Well, a matrix is actually nothing more then a bunch of column vectors, one next to another. We can interpret the first column of $\mathbf{Z}$ as the $\mathbb{R}^2$ vector that corresponds to $z$. Seen in this way, we see that the vector addition is apparently completely embedded in the matrix addition as well. The (redundant) second column is just there to enable multiplication - something which we can't do with vectors. So, in summary, we can use matrix addition and matrix multiplication of a very specific set of matrices to faithfully reproduce the behavior of complex addition and multiplication. It can be verified that, unlike general matrix multiplication, the multiplication of the so restricted set of matrices is commutative. Of course, the inverse of $\mathbf{Z}$ maps to the reciprocal of $z$.

...TBC...

% Explain how we usually think of matrices as operating on vectors when encoding geometric transformations...but we also have matrix multiplication...I think, the first column of the matrix Z represents the complex number as a vector. Matrix multiplication is just matrix vector multiplciation applied to multiple vectors at once, namely the columns of the right matrix factor. ...so, the vector addition is actually embedded in the matrix addition as well.


% https://en.wikipedia.org/wiki/Rotation_matrix
% https://en.wikipedia.org/wiki/Rotation_matrix#Relationship_with_complex_plane

%From the geometric behavior of the  

%We know that complex

%Complex numbers

% explain how complex numbers can be represented by a particular set of 2x2 matrices

\subsection{Hyperbolic Numbers}
To create the complex numbers, we defined the imaginary unit $\i$ as some quantity that squares to minus one. Complex numbers are then formed as a formal sum between a real number and an imaginary part which just the imaginary unit times some other real number. For the hyperbolic numbers, we do something very similar: we define a \emph{hyperbolic number} as a formal sum $a + b j$ of a real number $a$ and a hyperbolic part $b j$ made from multiple of the \emph{hyperbolic unit} $j$ which is defined to satisfy $j^2 = 1$ but is neither $+1$ nor $-1$. It is again some other thing that we envision to live on an independent axis that is orthogonal to our usual real number line. The hyperbolic numbers are also called the \emph{split-complex} numbers. ..TBC...

% mention the use in special relativity, 
% explain why they are called split-complex. I think, it has to do with splitting fields from abstract algebra

% give matrix representation - i think, cos and sin are just replaced by cosh and sinh

% Liszt important formulas

\subsection{Dual Numbers}
The complex and hyperbolic numbers are both two dimensional number systems. The dual numbers are the third in this trio. They arise from defining a quantity $\epsilon$ with the property $\epsilon^2 = 0$ but $\epsilon \neq 0$. ...TBC...

% https://en.wikipedia.org/wiki/Dual_number

% used in automatic differentiation...epsilon could perhaps be called the "infinitesimal unit" - it behaves like the dx in calculus - it vanishes when being squared


% Unsolvable System of Equations? | Dual Numbers
% https://www.youtube.com/watch?v=tuDACYvlZaY


% Multivariable Derivatives via Dual Numbers
% https://www.youtube.com/watch?v=6C2kUYFpq4s


% What is Automatic Differentiation?
% https://www.youtube.com/watch?v=wG_nF1awSSY


% Automatic Differentiation: Differentiate (almost) any function
% https://www.youtube.com/watch?v=4wgXBr7fnQg

% Automatic Differentiation Explained with Example
% https://www.youtube.com/watch?v=jS-0aAamC64


% Automatic Differentiation
% https://www.youtube.com/watch?v=R_m4kanPy6Q
% https://github.com/cg-tuwien/deep_learning_demo

% ToDo:
% -Implement a generalization of the dual numbers that also computes the 2nd derivative in addition
%  to the 1st. To implement the +,-,*,/ operators, use the suitably generalized versions of sum-
%  product- and quotient rule. (see the respective section in the Calculus part)
% -Maybe generalize to 3rd, 4th, etc. derivatives.

\subsection{Quaternions}
So far, we have seen only 2D number systems. The space we live in happens to have 3 dimensions. That made the Irish mathematician William Rowan Hamilton in the mid 1800s contemplate, if there could be something similar to the complex numbers but for 3 dimensions. After a long fruitless search, he eventually discovered the 4-dimensional quaternions. Today we know from abstract algebra - a field which did not yet exist at that time - that finding a 3D number system with the desired properties is actually impossible [VERIFY]. But 4D works. ...TBC...


% https://en.wikipedia.org/wiki/Quaternion

\subsection{Octonions}

\subsection{Sedenions}







\begin{comment}

-Frame the algebras of hypercomplex numbers as (representably by) particular subsets of
 the set of matrices. That's why we put it into the linear algebra chapter.
 
https://mathworld.wolfram.com/HypercomplexNumber.html
https://en.wikipedia.org/wiki/Hypercomplex_number

-Idea to create an n-dimenasional algebra:
 -define a bijective mapping between n-vectors and a certain subset of n-by-n-matrices
 -use matrix addition and multiplication
 -perhaps the map from matrices to vectors could be: extract 1st column and the map from
  vectors to matrices could be like in 3D (a,b,c) = a*I + b*shift(i,-1) + c*shift(I,-2)
  where shift(A,k) is a circular shift by k of the columns - negative k shift left. That
  mapping would at least ensure that the unit vectors map back and forth correctly which is
  a necessary (but not sufficient) condition for it to work out. 
 -more generally, one could map between n-vectors and m-by-m matrices with m^2 >= n
 -to show that no nD algebra with desried properties is possible, one needs to show that
  no such mapping can exist between n-vectors and m-by-m matrices for any m (it's trivial
  for m^2 < n but as soon as the matrices have enough degrees of freedom, it's nontrivial)


Climbing past the complex numbers.
https://www.youtube.com/watch?v=q3Tbf-d9sE4
-Explains the Cayley-Dickson construction
 https://en.wikipedia.org/wiki/Cayley%E2%80%93Dickson_construction



Every Hypercomplex Number Explained #SoME4
https://www.youtube.com/watch?v=DuNXn6qA2NE
-Very nice overview over the different hypercomplex number systems
-Seems like only associative systems can be represented via matrices? At 9:42. Verify!
-Video shows connections to Clifford algebras and abstract algebra (in particular ring theory)
 -> abstract algebra can be seen as bringing order into the vast zoo of possible number systems

\end{comment}