\section{Tensor Algebra}
Tensors are a generalization of scalars and vectors and are used to represent physical or geometrical quantities that are invariant with respect to a given coordinate system. Once a coodinate system has been chosen, these quantities can be \emph{represented} by multidimensional arrays. A scalar is just a number but can be seen as 0D array in this context. A vector is represented by a 1D array of numbers, usually written as column. A linear transformation of a vector is represented by a 2D array of numbers, i.e. a matrix. Now, we can see how this pattern should continue: we may have quantities that are representable by arrays with more than 2 dimensions - imagine slicing a bunch of matrices on top of each other for a 3D array, giving some sort of 3D block with small cubicles containing the numbers. A 4D array is harder to visualize, but we may imagine it as a list (i.e. a 1D array) of such blocks where the first index just selects, which of the blocks within the list we mean. And then we can imagine a list of such lists for a 5D array - or maybe a matrix of 3D blocks, etc. The dimensionality of the array is called the rank of the tensor. Be careful to not confuse this with dimensionality of the underlying vector space.





\subsection{Covariance, Contravariance and Invariance}
Our eventual goal is to develop an algebra that is suitable to represent quantities whose real-world meaning is independent from the arbitrariness of our choice of a coordinate system. However, in actual calculations, we have to make such a choice, so we can assign numbers to things. For example, a velocity can be represented by its velocity vector $\mathbf{v}$. The direction of movement in the real world does not depend on our chosen coordinate system but the components of the vector, i.e. the numbers in our 1D array, will. We could represent the \emph{same} vector in a different coordinate system as well. Then we would have different numbers, but the vector itself as an abstract entity would still be the same. Imagine a 2D vector and assume that we would obtain a new coordinate system by rotating both basis vectors of our old coordinate system by 30 degrees counterclockwise. The same vector in this new coordinate system would have components that appear rotated 30 degrees clockwise with respect to the vector's components in the old coordinate system. The clockwise rotation of the vector components compensates for the counterclockwise rotation of our basis vectors. We say that the vector is \emph{contravariant}: in the event of a change of the coordinate basis vectors, the components of the vector change in a way that is contrary to the change in the basis vectors. There can also be vector-like quantities that change in the same way as our basis vectors (todo: give an example). Such quantities are called \emph{covariant}. Finally, scalars represent quantities whose numerical value never changes at all. Their values are not dependent on our choice of coordiate system. Therefore, scalars are also called \emph{invariant} quantities in the context of tensor algebra (verify!)...are there also other invariant quantities or are scalars the only ones? ..and what about mixed variance in higher rank tensors?

% contravariant vectors: position, velocity, acceleration, force, electric and gravitational field
% covariant vectors (verify): normals to planes, angular momenetum moment of inertia,  magnetic field...i think, these correspond to covectors? ...or maybe bivectors? figure out!

..tbc...



\begin{comment}
====================================================================================================
-A tensor can be seen as a function that takes a number of vectors and covectors  as inputs and 
 produces a scalar as output. This mapping is linear in all its inputs, i.e. it is a multilinear
 map.
-It can also be seen as a multidimensional array, or rather, a geometric of physical quantity that
 can be represented by such an array, once a basis has been chosen. The tensor itself is invariant
 ...or is this the right word? i think tensors can be covariant, contravariant or a mix of both. 
 Truly invariant are the scalars that result when fully evaluating a tensor. Regardless of the 
 chosen coordinate system, they will always have the same numerical value
-When seen as a multidimensional array, it must obey certain transformation rules, when the basis
 is changed.
-how can we interpret a matrix vector-product in this context? maybe as partial evaluation? 
 multiplying a matrix from the left by a vector gives another vector. this is a tensor that 
 requires another covector as input to produce a scalar
-what about inverse tensors and tensor division? i think, with repect to the tensor product, this
 would not make much sense. only when a tensor indeed resulted from tensor-multiplying two lower
 rank tensors, we may be able to calculate one of the factors, given the other factor and the 
 product
-Examples for tensor quantities are: scalars, vectors, linear maps, quadratic (bilinear) forms(?)

Let's try to express the some known products in terms of tensors and index notation:

scalar product: c = a^i b_i = b_i a^i  (product of vector a and covector b)
matrix-vector product: c_i = a^k_i b_k  (verify)
matrix-matrix product: ...
vector-matrix product:  ...
outer product of vectors: c_{ij} = a_i b_j  ...this is actually the tensor product

Kronecker product of two matrices: c_{ij kl} = a_{ij} b_{kl}
...but wait - no - this is a rank 4 tensor - the kronecker product is again a matrix - it has the
same elements but arranged ad 2D array rather than 4D

Questions:
-What about the various exterior products that can be computed from 1-vectors, 2-vectors, 
 1-covectors, 2-covectors, etc? How can they be expressed in index notation?
-What about the geometric products of various quantities in geometric algebra?

see:
https://www.youtube.com/watch?v=TvxmkZmBa-k&list=PLJHszsWbB6hrkmmq57lX8BV-o-YIOFsiG&index=2


\end{comment}