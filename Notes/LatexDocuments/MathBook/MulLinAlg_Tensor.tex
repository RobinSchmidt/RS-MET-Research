\section{Tensor Algebra}




\begin{comment}
====================================================================================================
-A tensor can be seen as a function that takes a number of vectors and covectors  as inputs and 
 produces a scalar as output. This mapping is linear in all its inputs, i.e. it is a multilinear
 map.
-It can also be seen as a multidimensional array, or rather, a geometric of physical quantity that
 can be represented by such an array, once a basis has been chosen. The tensor itself is invariant
 ...or is this the right word? i think tensors can be covariant, contravariant or a mix of both. 
 Truly invariant are the scalars that result when fully evaluating a tensor. Regardless of the 
 chosen coordinate system, they will always have the same numerical value
-When seen as a multidimensional array, it must obey certain transformation rules, when the basis
 is changed.
-how can we interpret a matrix vector-product in this context? maybe as partial evaluation? 
 multiplying a matrix from the left by a vector gives another vector. this is a tensor that 
 requires another covector as input to produce a scalar
-what about inverse tensors and tensor division? i think, with repect to the tensor product, this
 would not make much sense. only when a tensor indeed resulted from tensor-multiplying two lower
 rank tensors, we may be able to calculate one of the factors, given the other factor and the 
 product
-Examples for tensor quantities are: scalars, vectors, linear maps, quadratic forms(?)

Let's try to express the some known products in terms of tensors and index notation:

scalar product: c = a^i b_i = b_i a^i  (product of vector a and covector b)
matrix-vector product: c_i = a^k_i b_k  (verify)
matrix-matrix product: ...
vector-matrix product:  ...
outer product of vectors: c_{ij} = a_i b_j  ...this is actually the tensor product

Kronecker product of two matrices: c_{ij kl} = a_{ij} b_{kl}
...but wait - no - this is a rank 4 tensor - the kronecker product is again a matrix - it has the
same elements but arranged ad 2D array rather than 4D

Questions:
-What about the various exterior products that can be computed from 1-vectors, 2-vectors, 
 1-covectors, 2-covectors, etc? How can they be expressed in index notation?
-What about the geometric products of various quantities in geometric algebra?


\end{comment}