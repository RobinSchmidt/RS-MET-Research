\chapter{Number Theory}
The great Carl Friedrich Gauss once said: "Mathematics is the queen of the sciences and number theory is the queen of mathematics". Number theory is mostly concerned with questions about discrete number systems such as the natural and integer numbers. When I say "discrete", I mean that the numbers in the studied systems are enumerable, i.e. have a cardinality equal to the of the naturals which means they are countably infinite. There is a branch of number theory, called "Transcendental Number Theory", that goes beyond the discrete worlds but by and large, number theory is considered a topic that firmly sits in the world of discrete mathematics. In a branch of number theory that is called "Analytic Number Theory", one makes use of continuous number systems such as the real and complex numbers. But there, these are tools and the questions that one ultimately seeks to answer are still about discrete numbers. A prime example (pun intended) of a topic in analytic number theory is the famous Riemann hypothesis which relates the distribution of the prime numbers (discrete!) to the zeros a certain analytic function (continuous!), i.e. a function that is defined over the complex plane. The other important branch of number theory is called "Algebraic Number Theory" ,...TBC...

% Algebraic number theory

\begin{comment}
	
Topics:
-Divisibility: Euclidean algo ((ext.) GCD, LCM), Primes, Factorization
-Modular Arithmetic
-Diophantine Equations
 -Fermat's Last Theorem
  -Pythagorean Triples
-Algberaic Numbers 
 -Roots of polynomials with coeffs in z, 
  -for Algebraic Integer, the polynomial must be monic


Resources:
https://en.wikipedia.org/wiki/Number_theory
https://en.wikipedia.org/wiki/Diophantine_equation
https://en.wikipedia.org/wiki/Transcendental_number_theory
	
\end{comment}