\chapter{Number Theory}
The great Carl Friedrich Gauss once said: "Mathematics is the queen of the sciences and number theory is the queen of mathematics". Number theory is mostly concerned with questions about discrete number systems such as the natural and integer numbers. When I say "discrete", I mean that the numbers in the studied systems are enumerable, i.e. have a cardinality equal to the of the naturals which means they are "only" countably infinite (and sometimes even finite). There is a branch of number theory, called "Transcendental Number Theory", that goes beyond the discrete worlds but by and large, number theory is considered to be a topic that firmly sits in the world of discrete mathematics. In a branch of number theory that is called "Analytic Number Theory", one makes use of continuous number systems such as the real and complex numbers. But there, these continuous numbers are only used as auxiliary tools and the questions that one ultimately seeks to answer are still about discrete numbers. A prime example (pun intended) of a topic in analytic number theory is the famous Riemann hypothesis which relates the distribution of the prime numbers (discrete!) to the zeros a certain analytic function (continuous!), i.e. a function that is defined over the complex plane. To this day, the question whether or not the hypothesis is true is considered to be one of the most important open questions in mathematics and many would probably even consider it \emph{the} most important. The other important branch of number theory is called "Algebraic Number Theory". This part of number theory deals with new number systems that behave in certain algebraic ways similar to the integers or rationals - like finite fields (aka Galois fields) and cyclotomic fields, where \emph{field} is a technical term from abstract algebra. This part of number theory also studies the solubility of multivariate polynomial equations, called Diophantine equations, within the integers. Questions like: has the equation $a^n + b^n = c^n$ integer solutions for a given $n > 1 \in \mathbb{N}$, i.e. are there any triples $(a,b,c)$ of integer numbers that satisfy the equation. It turns out that for $n=2$, there are infinitely many such solutions, called the Pythagorean triples, and for $n > 2$ there are no integer solutions whatsoever. This is called "Fermat's Last Theorem", which - despite its name - isn't really a theorem that Fermat has proven. The status of a theorem was only attained in 1995 when Andrew Wiles published the first proof that was accepted by the mathematical community. Fermat claimed to have a proof but didn't publish it and it is widely believed today that the proof he had was probably erroneous. A distinctive feature of typical number theoretical problems is that the questions are often easy to ask but the proofs for the answers require highly advanced mathematical techniques. Number theory was for a long time considered to a topic in pure mathematics with no practical applications. The distinguished English number theorist Godfrey Harold Hardy (1877-1947) famously considered himself lucky to work in a field of math which, in his opinion, could never be used and abused for anything related to war. How wrong he was - today, number theory is the single most important branch of mathematics that is used in cryptography. The biggest employer of mathematicians on our beautiful planet is the National Security Agency (NSA) and most of them are number theorists.

%  \footnote{I fully agree with the first part but within math, I personally like calculus most. From a perspective of aesthetics and elegance, the continuous world of smooth functions appeals to me much more. Complex analysis, differential geometry, geometric calculus etc. - that's the really cool stuff for me. But who am I to controvert Gauss? And actually, there's also analytic number theory, so maybe that disagreement can be reconciled when the higher levels of number theory are reached.}

% Say a bit more about algwebraic number theory:
% cyclotomic fields (or rings?), primes therein, finite fields, Galois fields, Fermat's last theorem
% https://en.wikipedia.org/wiki/Cyclotomic_field
% https://mathworld.wolfram.com/CyclotomicField.html

%  Wiles' proof features advanced and modern tools from algebraic number theory prominently [VERIFY!] that were not yet known in Fermat's day.

%Before that proof, it should have been called "Fermat's Hypothesis" and after the publication "Wiles' Theorem". Well - 

% Algebraic number theory

% https://en.wikipedia.org/wiki/Algebraic_number_theory

\begin{comment}
	
Topics:
-Divisibility: Euclidean algo ((ext.) GCD, LCM), Primes, Factorization
-Number Theoretic Functions
 -Prime counting function
 -Moebius function
 -Multiplicativity
-Modular Arithmetic
-Diophantine Equations
 -Fermat's Last Theorem
  -Pythagorean Triples
 -Elliptic Curves
-Algberaic Numbers 
 -Roots of polynomials with coeffs in z, 
  -for Algebraic Integer, the polynomial must be monic
-


where does the Collatz Conjecture fit in? Maybe some sort of "Algorithmic Number Theory" or
"Recursion Theory" or "Recursive Series Theory"? What about the Ackermann function? Or busy beavers? I guess, we are venturing into the realms of theorectical computer science with these.

Resources:
https://en.wikipedia.org/wiki/Number_theory
https://en.wikipedia.org/wiki/Diophantine_equation
https://en.wikipedia.org/wiki/Transcendental_number_theory
https://en.wikipedia.org/wiki/Algebraic_number_theory
https://en.wikipedia.org/wiki/Abc_conjecture
https://en.wikipedia.org/wiki/Elliptic_curve
	
	
https://www.youtube.com/watch?v=DQdgls3f2OM
if (a,b,c) is a Pythagorean triple then (a^m + b^m + c^m) / (a+b+c) is an integer for odd m



https://www.youtube.com/watch?v=M-9_rZfVQVE
 2:20  Fermat's last theorem in different number systems
 5:35  In R, the equation a^n + b^n = c^n has infintely many solutions. Just treat a,b, as free
       parameters and compute c as \sqrt[n]{a^2 + b^2}. But it's much more difficult to solve in Z
 6:20  Formula for primitive Pythagorean triples: take two integer parameters m,n with n > m >=1,
       gcd(m,n) = 1, m + n = odd and take the triples (n^2 - m^2, 2 n m, n^2 + m^2).
10:45: Over Z[i], 2 factors as (1+i)(1-i) and 5 = (1+2i)(1-2i), a^2+b^2 = (a+bi)(a-bi)
11:20: Z[1/2] gives the dyadic rationals
12:40: Z[sqrt(-3)] was used by Euler to solve a^3 + b^3 = c^3 (it has no solutions in Z+)
13:20  This involved factoring an expresion like u^2 + 3 v^2. It canbe done as
       u^2 + 3 v^2 = (u + r v)(u - r v)  with r = sqrt(-3). But his proof was wrong because for it
       to work, the factorizationof number has to be unique. but it isn't. For example: 
       4 = 2*2 = (1+r)*(1-r). The ring is not a unique factorization domain (UFD). 
14:35  Z[w] with w = cbrt(1) creates the Eisenstein integers, has units +1,-1,w,w^2,-w,-w^2. This
       ring contains sqrt(-3) as well and is a UFD. 4 still factors like in Z[sqrt(-3)] but the
       different factorizations are now associated, so they don't count as different anymore. This
       patches Euler's faulty proof
16:20  In Z[w], we have  a^3 + b^3 = (a + b) (a^2 - a b + b^2) = (a + b)(a + bw)(a + bw^2)
16:40  a^4 + b^4 = c^4 means (a^2)^2 + (b^2)^2 = (c^2)^2, so we can show that there exists no 
       Pythagorean triple in which a,b,c are all squares.
17:50  for a^5 + b^5 = c^5, we use Z[z5] where z5 = \sqrt[5]{1} - the cyclotomic integers of
       order(?) 5
18:20  Z[z23] is not a UFD
...


https://en.wikipedia.org/wiki/Lagrange%27s_four-square_theorem
https://en.wikipedia.org/wiki/Fermat_polygonal_number_theorem
 
https://en.wikipedia.org/wiki/Additive_number_theory
https://en.wikipedia.org/wiki/Partition_(number_theory)
https://en.wikipedia.org/wiki/Multiplicative_number_theory
https://en.wikipedia.org/wiki/Analytic_number_theory
  
	
A Book of Abstract Algebra, pg 349 ff has a nice overview of important facts
	
	
Why do prime numbers make these spirals? | Dirichlet’s theorem and pi approximations	
https://www.youtube.com/watch?v=EK32jo7i5LQ	



What was Fermat’s “Marvelous" Proof? | Infinite Series
https://www.youtube.com/watch?v=SsVl7_R2MvI
-Has different definition of primality that is equivalen to the usual one in the integers
 but different in other rings - in other rings, there may be a difference between irreducibility
 and primality - these two features only coincide in rings where the fundamental theorem of
 arithemtic holds, i.e. every element has a unique factorization into a product of primes

Why Number Theory is Hard
https://www.youtube.com/watch?v=i8bn0US_k84
-When representing numbers as infinitely long vectors in which the exponents of the prime
 factorization of given numbers appears, we can express the following operations:
 -multiplication of the numbers is done by vector addition
 -gcd and lcm are found by taking elemnt-wise min and max respectively.
 -coprime numbers are orthogonal
 
Algebraic number theory - an illustrated guide | Is 5 a prime number? 
https://www.youtube.com/watch?v=4m_EaWA08H0
	
https://mathworld.wolfram.com/QuadraticReciprocityTheorem.html	
	
Der Große Satz von Fermat (mit Beweis für n=3 und n=4)	
https://www.youtube.com/watch?v=vtEfvrX0Avo	

Goldbach Conjecture:
https://en.wikipedia.org/wiki/Goldbach%27s_conjecture
	
	
Why are the prime rows in Pascal's Triangle so special?	
https://www.youtube.com/watch?v=DJ1AlibmUIE	
	
	
The Unlikeliness of Numbers Sharing Factors	
https://www.youtube.com/watch?v=Cbkyarr7QRc
-uses ideas from probability theory and caclulus (infinite products)	
	
What's the next freak identity? A new deep connection with Sophie Germain primes	
https://www.youtube.com/watch?v=phqXU-1CFas	
	
https://www.youtube.com/watch?v=GY4yN4cROE8  Fermat's Last Theorem
	
	
https://www.youtube.com/watch?v=OCgv0v_HHUA  Domains with Finite Primes
-has alternative (more general)	 definition of a prime
	

https://www.youtube.com/watch?v=qiPBbdY6CRY  50 INTERESTING FACTS about PRIME NUMBERS

	
\end{comment}