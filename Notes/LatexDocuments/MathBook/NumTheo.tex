\chapter{Number Theory}
The great Carl Friedrich Gauss once said: "Mathematics is the queen of the sciences and number theory is the queen of mathematics". Number theory is mostly concerned with questions about discrete number systems such as the natural and integer numbers. When I say "discrete", I mean that the numbers in the studied systems are enumerable, i.e. have a cardinality equal to the of the naturals which means they are countably infinite. There is a branch of number theory, called "Transcendental Number Theory", that goes beyond the discrete worlds but by and large, number theory is considered a topic that firmly sits in the world of discrete mathematics. In a branch of number theory that is called "Analytic Number Theory", one makes use of continuous number systems such as the real and complex numbers. But there, these are tools and the questions that one ultimately seeks to answer are still about discrete numbers. A prime example (pun intended) of a topic in analytic number theory is the famous Riemann hypothesis which relates the distribution of the prime numbers (discrete!) to the zeros a certain analytic function (continuous!), i.e. a function that is defined over the complex plane. To this day, the question whether or not the hypothesis is true is considered to be one of the most important open questions in mathematics and many would probably even consider it \emph{the} most important. The other important branch of number theory is called "Algebraic Number Theory". Besides questions about prime numbers, number theory also studies the solubility of multivariate polynomial equations, called Diophantine equations, within the integers. Questions like: has the equation $a^n + b^n = c^n$ for a given $n > 1 \in \mathbb{N}$ integer solutions, i.e. are there any triples $(a,b,c)$ of integer numbers that satisfy the equation. It turns out that of $n=2$, there are infinitely many such solutions, called the Pythagorean triples, and for $n > 2$ there are no integer solutions whatsoever. This is called "Fermat's Last Theorem", which - despite its name - wasn't really a theorem until 1995 when Andrew Wiles published the first proof that was accepted by the mathematical community. And this proof features tools from algebraic number theory prominently [VERIFY!]. A distinctive feature of number theory is that the questions are often easy to state and understand but the (proofs for) the answers require highly advanced mathematical techniques.

% Algebraic number theory

\begin{comment}
	
Topics:
-Divisibility: Euclidean algo ((ext.) GCD, LCM), Primes, Factorization
-Modular Arithmetic
-Diophantine Equations
 -Fermat's Last Theorem
  -Pythagorean Triples
 -Elliptic Curves
-Algberaic Numbers 
 -Roots of polynomials with coeffs in z, 
  -for Algebraic Integer, the polynomial must be monic

-Number Theoretic Functions
 -Prime counting function
 -Moebius function
 -Multiplicativity


Resources:
https://en.wikipedia.org/wiki/Number_theory
https://en.wikipedia.org/wiki/Diophantine_equation
https://en.wikipedia.org/wiki/Transcendental_number_theory
https://en.wikipedia.org/wiki/Algebraic_number_theory
https://en.wikipedia.org/wiki/Abc_conjecture
https://en.wikipedia.org/wiki/Elliptic_curve
	
\end{comment}