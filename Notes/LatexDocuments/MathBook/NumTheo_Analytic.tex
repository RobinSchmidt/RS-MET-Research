\section{Analytic Number Theory}


\subsection{Euler's Product Formula}
Let $s$ be a complex number with a real part strictly greater than one. Then the following sum and product will converge to the same value:
\begin{equation}
\sum_{n=1}^{\infty} \frac{1}{n^s} = \prod_{p \in \mathbb{P}} \frac{1}{1 - \frac{1}{p^s}}
\end{equation}
This is a pretty stunning result and it is known as the Euler product formula. We need the $\Re(s) > 1$ condition for the sum to converge. For the product to converge, we could be a bit more liberal and just require $\Re(s) > 0$ [VERIFY]. What makes this formula so interesting is that on the left hand side, we have a sum over \emph{all} integers whereas on the right hand side, we have a product involving \emph{only} the primes. Thus, this formula establishes a relationship between a very simple and boring sequence of numbers which we fully understand (namely $1,2,3,4,5,6\ldots$) and a very mysterious sequence of numbers (namely the sequence of primes $2,3,5,7,11,13,\ldots$) that we are yet trying to understand. Thereby the formula establishes a nice starting point for investigations of the sequence of primes.

% A crazy result using Euler's Product Formula
% https://www.youtube.com/watch?v=tSfNNXCBGrs

% The Riemann Hypothesis (Christmas Lecture 2016) [English subtitles]
% https://www.youtube.com/watch?v=sZhl6PyTflw
% bei 1:04

\subsection{Riemann's Zeta Function}

\subsection{$L$-Functions}

\subsection{Langlands Correspondence}

% Revolutionary Math Proof No One Could Explain...Until Now
% https://www.youtube.com/watch?v=RX1tZv_Nv4Y

\subsubsection{Modular Forms}




\begin{comment}

A crazy result using Euler's Product Formula
https://www.youtube.com/watch?v=tSfNNXCBGrs

\end{comment}