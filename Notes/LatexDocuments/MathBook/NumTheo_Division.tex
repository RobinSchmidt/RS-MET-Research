\section{Division} 

\subsection{Divisibility}
A natural number like, say, $12$ can be divided by $6$ to give $2$. The result $2$ happens to be a natural number. We will denote this circumstance as $6 \mid 12$ which reads as "six divides twelve" or "six is a divisor of twelve". Conversely, it can also divided by $2$ to give $6$, so we also have $2 \mid 12$. On the other hand $12 / 5 = 2.4$ which is not a natural number, so $5$ does not divide $12$ which we write as $5 \nmid 12$. The number $12$ is also divisible by $3$ because this division results in $4$ which is also a natural number. And, of course, by symmetry, we have also that $4$ divides $12$ because that gives $3$. Looks like the set of divisors of $12$ is given by $\{2,3,4,6\}$. You may check that $12$ has no other divisors except the numbers $1$ and $12$ itself. The number itself and the number $1$ are always divisors of any number, so we call them the trivial divisors. As a matter of convention, in a list of divisors of a given number, we usually include these trivial divisors as well, so the set of divisors of $12$ is actually $\{1,2,3,4,6,12\}$. 

\subsubsection{Composite Numbers}
Any number that has nontrivial divisors is called a \emph{composite number} because we can multiplicatively "compose" that number from smaller numbers. If we count the number of divisors of $12$, we get $6$. The number of divisors of a number is an important enough number theoretic feature that it has a notation: for a natural number $n$, we denote by $d(n)$ the function that maps $n$ to its number of divisors. So, we have $d(12) = 6$. This is actually a quite high value for $d$ for such a small number as $12$. As you may easily verify or take my word for it, 12 has a $d(n)$ that is greater than the $d(n)$ of all numbers below $12$. If a number has that feature, we call it a \emph{highly composite} number. If the number of divisors of a number $n$ is not "strictly greater than" but only "greater or equal to" the number of divisors of all smaller numbers, we call it a \emph{largely composite} number. So, "highly composite" is a stronger notion that "largely composite". There is also the even stronger notion of a \emph{superiorly composite number}...TBC...

\subsubsection{Prime Numbers}
There are some numbers that have \emph{only} the trivial divisors. These very special numbers take the center stage in number theory and are called \emph{prime numbers}. By and large, prime numbers are defined to be numbers that are divisible only by themselves and by $1$. According to that definition, the number $1$ would actually qualify but by convention, we do not consider $1$ a prime number. This was a matter of debate among mathematicians until fairly recently (reminder: definitions can't be right or wrong - only more or less useful) but that debate is now settled.
%The first few
...TBC...give table of first 100 or so prime numbers.

\medskip
Because prime numbers are so special and important, it would be nice to have some sort of formula or algorithm to produce all the prime numbers. Ideally, we would like to define a function $p(n)$ that spits out the $n$th prime number. If you can find such a formula for $p(n)$ that is easy to evaluate, eternal fame is yours. It's not at all easy and actually one of the holy grails of mathematics.

\paragraph{Willans' Formula}
But what exactly do I mean by "easy to evaluate"? If anything is allowed, then it is indeed possible to write down an explicit formula for the $n$th prime. For example, this beast here:
\begin{equation}
 p(n) = 1 + \sum_{i = 1}^{2^n} 
 \floor[\Bigg]{
 \Biggl(
 \frac{n}{\sum_{j=1}^{i} \floor[\big]{ \bigl( \cos(\pi \frac{(j+1)!+1}{j}) \bigr)^2 } 
  }
 \Biggr)^{1/n}
 }
\end{equation}
is due to C.P. Willans and I will not explain here why it works because that's not my point here [TODO: insert reference to an explanation]. My point is that it certainly qualifies as an explicit formula but it is practically useless. The first thing we note is that the outer sum goes up to $2^n$ which already puts the cost of the evaluation of the formula as function of $n$ into the category of algorithms with exponential time complexity. If you don't know what that means, let me just tell you that it means: "too inefficient for practical use except (maybe) for very small inputs". And that is true even if we overly optimistically assume the thing inside the sum to be in the realm of constant complexity - which it certainly isn't. 

% Willan's formula:
% https://en.wikipedia.org/wiki/Formula_for_primes
% https://www.youtube.com/watch?v=j5s0h42GfvM
% https://mathworld.wolfram.com/WillansFormula.html
% https://www.cambridge.org/core/journals/mathematical-gazette/article/abs/on-formulae-for-the-nth-prime-number/43E49D11DFEAD3E4CBC12F17C87F5EE1

\paragraph{The Sieve of Erathostenes}
A much more practical way to produce primes is to precompute a list of primes, store it somewhere and read the $n$th entry from this list whenever the $n$th prime is needed. One algorithm to produce such a list is the sieve of Erathostenes. The algorithm works as follows: We start with a list of all numbers up to some upper limit $m$. The first prime in this list is $2$. We pick $2$ and mark all multiples of it (except $2$ itself) as non-prime. They can't be prime because they are multiples of $2$. Then we scan the list for the first number that is not yet marked as non-prime. That would be $3$. Then again we  mark all proper multiples of $3$ as non-prime. Repeat: the next is $5$. And so on. When we are done, all numbers in our list that are not marked as non-prime are primes. We can actually stop when we reach $\sqrt{m}$ because ...TBC...

% ToDo: talk about the algorithmic complexity - I think what we actually want in number theory is to find efficient formulas or algorithms to produce prime numbers or to decide whether or not a given number is prime

%We mark all numbers that are multiples of $2$ except two itself as non-prime. They are mulitples of 

\paragraph{Primality Tests}
Another question besides "what is the $n$th prime" could be that we have some number $n$ given and want to figure out whether or not $n$ is a prime number. If we have a list of primes that is long enough that $n$ would be included, if it would be prime, then we could just search in this list for $n$. Assuming to use use binary search, this would have the rather efficient complexity of $\mathcal{O}(\log(n))$. If we find it, then $n$ is prime and if don't find it, then it isn't prime. If no such list is available, a naive algorithm to test primality is by "trial division": we try to divide $n$ by all numbers less than or equal to $\sqrt{n}$ and check the remainder. If in one of these trials, we get a remainder of zero, then $n$ is divisible by that number and hence not prime. If we find no such case, then $n$ is prime. That algorithm would have a complexity of $\mathcal{O}(\sqrt{n})$ and is not really practical for large $n$.


% we say that $12$ is divisible by $2$ and $6$ and 

% https://en.wikipedia.org/wiki/Table_of_divisors

% https://en.wikipedia.org/wiki/Divisor_function
% The divisor function is multiplicative ... but not completely multiplicative
% maybe include that in the section about multiplicativity

% https://en.wikipedia.org/wiki/Highly_composite_number
% https://en.wikipedia.org/wiki/Superior_highly_composite_number

\subsubsection{Division with Remainder}
% Every natural number n can written as  n = d*q + r  for every given positive natural number d
% where q: quotient (not stardardized according to Weitz), r: remainder, d: divisor (given), 
% n: number (given, dividend) where q,r are uniqely determined and r < d
% what happens in rings where multiplication non-commutative?
% -Frame division as repated subtraction:
%  -Initialize r = n, q = 0
%  -As long as r >= d: r -= d, q += 1
% -This algorithm will eventually halt
% -When it halts, q and r will be our results
%
% -Notation for remainders, example: 16 mod 7 = 2
% 
% This is a naive algorithm with tiem complexity O(n). Give better algorithm...

% http://weitz.de/files/skript.pdf




\paragraph{Modulo for Negative Numbers}
% Take care when using the modulo operator in programming languages. It may produce different
% results!



\paragraph{Divsibility Rules}
% give rules for decimal numbers
% if a|b and a|c then a|(x*b + y*c) for any x,y in Z  (Bezout's Lemma(?))
% special cases: a|(b+c), a|(b-c), a|(c-b)

\subsubsection{Greatest Common Divisors}
% denoted as gcd(n,m) or sometimes just as (n,m) in articles about number theory.

\paragraph{The Euclidean Algorithm}
% still to this day the best known way to compute the gcd.


\paragraph{Lowest Common Multiples}
A concept that is closely related to the idea of a greatest common divisor is that of a lowest common multiple. ...TBC...

% https://www.youtube.com/watch?v=Vpt5F53D47Y
% is it true that a|bc  ->  a|b or a|c ? NOPE: Example: b=6, c=5, a=10
% Correct is: a|bc and gcd(a,b) = 1  ->  a|c




\subsubsection{Factorization}
Any natural number $n$ can be written as a product of prime numbers. That statement is an importnat theorem, called the "Fundamental Theorem of Arithmetic". In such a product, a prime number is allowed to occur multiple times so you may also hear the statement in the form: "...as a product of powers of prime numbers". The prime numbers themselves have only one single factor - the number itself. But that is also considered to be a product - namely a product over one single factor. In programmer's speak, it's a somewhat degenerate edge case (actually the even more edgy edge case is a product over zero numbers which is defined as $1$). But be that as it may - the theorem implies that we can break down any number $n$ into its prime factors. That process, or sometimes also the result of that process, is called the \emph{prime factorization} of $n$. ...TBC...


% https://en.wikipedia.org/wiki/Square-free_integer
% all prime factors are unique

\subsubsection{Coprimality}
Two numbers $n$ and $m$ are said to be \emph{coprime} or \emph{mututally prime} when they do not have a common prime factor in their factorization. That is equivalent to saying that their greatest common divisor is $1$. It is also equivalent to saying that their least common multiple is just the product of the two numbers [VERIFY!].


% maybe this should be a section rather than a subsection:
\subsection{Number Theoretic Functions}
\emph{Number theoretic functions}, sometimes also called \emph{arithmetic functions}, are functions that take a natural number $n$ as input and produce as output another number. Real or complex outputs are allowed in general but more typically, the outputs are also integers [VERIFY]. These functions encode some property about the natural numbers. We have already encountered the number of divisors function $d(n)$ as an example of such a number theoretic function. But there are many more. Before introducing them one by one, we'll first talk about some general features that these functions may or may not have.

% https://en.wikipedia.org/wiki/Arithmetic_function

\subsubsection{The Prime Counting Function}



% number of (unique) prime factors https://en.wikipedia.org/wiki/Prime_omega_function
% arithmetic derivative https://en.wikipedia.org/wiki/Arithmetic_derivative