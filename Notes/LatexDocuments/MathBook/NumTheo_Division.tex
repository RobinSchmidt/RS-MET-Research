\section{Division} 

\section{Divisibility}
A natural number like, say, $12$ can be divided by $6$ to give $2$. The result $2$ happens to be a natural number. We will denote this circumstances as $6 \mid 12$ which reads as "six divides twelve" or "six is a divisor of twelve". Conversely, it can also divided by $2$ to give $6$, so we also have $2 \mid 12$. On the other hand $12 / 5 = 2.4$ which is not a natural number, so $5$ does not divide $12$ which we write as $5 \nmid 12$. The number $12$ is also divisible by $3$ because this division results in $4$ which is also a natural number. And, of course, by symmetry, we have also that $4$ divides $12$ because that gives $3$. Looks like the set of divisors of $12$ is given by $\{2,3,4,6\}$. You may check that $12$ has no other divisors except the numbers $1$ and $12$ itself. The number itself and the number $1$ are always divisors of any number, so we call them the trivial divisors. As a matter of convention, in a list of divisors of a given number, we usually include these trivial divisors as well, so the set of divisors of $12$ is actually $\{1,2,3,4,6,12\}$. 

\subsection{Composite Numbers}
Any number that has nontrivial divisors is called a \emph{composite number} because we can multiplicatively "compose" that number from smaller numbers. If we count the number of divisors of $12$, we get $6$. The number of divisors of a number is an important enough number theoretic feature that it has a notation: for a natural number $n$, we denote by $d(n)$ the function that maps $n$ to its number of divisors. So, we have $d(12) = 6$. This is actually a quite high value for $d$ for such a small number as $12$. As you may easily verify or take my word for it, 12 has a $d(n)$ that is greater than the $d(n)$ of all numbers below $12$. If a number has that feature, we call it a \emph{highly composite} number. If the number of divisors of a number $n$ is not "strictly greater than" but only "greater or equal to" the number of divisors of all smaller numbers, we call it a \emph{largely composite} number. So, "highly composite" is a stronger notion that "largely composite". There is also the even stronger notion of a \emph{superiorly composite number}...TBC...

\subsection{Prime Numbers}
There are some numbers that have \emph{only} the trivial divisors. These very special numbers take the center stage in number theory and are called \emph{prime numbers}. By and large, prime numbers are defined to be numbers that are divisible only by themselves and by $1$. According to that definition, the number $1$ would actually qualify but by convention, we do not consider $1$ a prime number. This was a matter of debate among mathematicians until fairly recently (reminder: definitions can't be right or wrong - only more or less useful) but that debate is now settled.
%The first few
...TBC...give table of first 100 or so prime numbers.

\medskip
Because prime numbers are so special, it would be nice to have some sort of formula or algorithm to produce all the prime numbers. Ideally, we would like to define a function $p(n)$ that spits out the $n$th prime number. If you can find such a formula that is easy to evaluate, eternal fame is yours. It's not at all easy and actually one of the holy grails of mathematics.

\paragraph{Willan's Formula}
But what exactly do I mean by "easy to evaluate". If anything is allowed, than it is indeed possible to write down an explicit formula for the $n$th prime. For example, this beast here:
\begin{equation}
 p(n) = 1 + \sum_{i = 1}^{2^n} 
 \floor[\Bigg]{
 \Biggl(
 \frac{n}{\sum_{j=1}^{i} \floor[\big]{ \bigl( \cos(\pi \frac{(j+1)!+1}{j}) \bigr)^2 } 
  }
 \Biggr)^{1/n}
 }
\end{equation}
is due to C.P. Willans and I will not explain here why it works because that's not my point here [TODO: insert reference to an explanation]. My point is that it certainly qualifies an explicit formula but it is practically useless. The first thing we note is that the outer sum goes up to $2^n$ which already puts the evaluation of the formula into the category of algorithms with exponential time complexity. If you don't know what that means, let me just tell you that it means: "too inefficient for practical use except (maybe) for very small inputs". And that is true even if we assume the thing inside the sum to be in the realm of constant complexity - which it certainly isn't. 

% Willan's formula:
% https://en.wikipedia.org/wiki/Formula_for_primes
% https://www.youtube.com/watch?v=j5s0h42GfvM
% https://mathworld.wolfram.com/WillansFormula.html
% https://www.cambridge.org/core/journals/mathematical-gazette/article/abs/on-formulae-for-the-nth-prime-number/43E49D11DFEAD3E4CBC12F17C87F5EE1

\paragraph{The Sieve of Erathostenes}
A much more practical way to produce primes is to precompute a list of primes, store it somewhere and read the $n$th entry from this list whenever the $n$th prime is needed. One algorithm to produce such a list is the sieve of Erathostenes. ...TBC...


%Finding a closed form formula that spits out all the prime numbers and only prime
%how do we find all the prime numbers? 



% we say that $12$ is divisible by $2$ and $6$ and 

% https://en.wikipedia.org/wiki/Table_of_divisors

% https://en.wikipedia.org/wiki/Divisor_function
% The divisor function is multiplicative ... but not completely multiplicative
% maybe include that in the section about multiplicativity

% https://en.wikipedia.org/wiki/Highly_composite_number
% https://en.wikipedia.org/wiki/Superior_highly_composite_number

\subsection{Division with Remainder}

\subsubsection{Greatest Common Divisors}
\paragraph{The Euclidean Algorithm}
\paragraph{Lowest Common Multiples}
A concept that is closely related to the idea of a greatest common divisor is that of a lowest common multiple.



\subsection{Factorization}