\section{Division} 

%===================================================================================================
\subsection{Divisibility}
A natural number like, say, $12$ can be divided by $6$ to give $2$. The result $2$ happens to be a natural number. We will denote this circumstance as $6 \mid 12$ or $6 | 12$ which reads as "six divides twelve" or "six is a divisor of twelve". Conversely, it can also divided by $2$ to give $6$, so we also have $2 | 12$. On the other hand $12 / 5 = 2.4$ which is not a natural number, so $5$ does not divide $12$ which we write as $5 \nmid 12$. The number $12$ is also divisible by $3$ because this division results in $4$ which is also a natural number. And, of course, by symmetry, we have also that $4$ divides $12$ because that gives $3$. Looks like the set of divisors of $12$ is given by $\{2,3,4,6\}$. You may check that $12$ has no other divisors except the numbers $1$ and $12$ itself. The number itself and the number $1$ are always divisors of any number, so we call them the trivial divisors. As a matter of convention, in a list of divisors of a given number, we usually include these trivial divisors as well, so the set of divisors of $12$ is actually $\{1,2,3,4,6,12\}$. 

%---------------------------------------------------------------------------------------------------
\subsubsection{Composite Numbers}
Any number that has nontrivial divisors is called a \emph{composite number} because we can multiplicatively "compose" that number from smaller numbers. If we count the number of divisors of $12$, we get $6$. The number of divisors of a number is an important enough number theoretic feature that it has a notation: for a natural number $n$, we denote by $d(n)$ the function that maps $n$ to its number of divisors. So, we have $d(12) = 6$. This is actually a quite high value for $d$ for such a small number as $12$. As you may easily verify or take my word for it, 12 has a $d(n)$ that is greater than the $d(n)$ of all numbers below $12$. If a number has that feature, we call it a \emph{highly composite} number. If the number of divisors of a number $n$ is not "strictly greater than" but only "greater or equal to" the number of divisors of all smaller numbers, we call it a \emph{largely composite} number. So, "highly composite" is a stronger notion that "largely composite". There is also the even stronger notion of a \emph{superiorly composite number}...TBC...

%---------------------------------------------------------------------------------------------------
\subsubsection{Prime Numbers}
There are some numbers that have \emph{only} the trivial divisors. These very special numbers take the center stage in number theory and are called \emph{prime numbers}. By and large, prime numbers are defined to be numbers that are divisible only by themselves and by $1$. According to that definition, the number $1$ would actually qualify but by convention, we do not consider $1$ a prime number. This was a matter of debate among mathematicians until fairly recently (reminder: definitions can't be right or wrong - only more or less useful) but that debate is now settled.
%The first few
...TBC...give table of first 100 or so prime numbers.

% ToDo: This should be a higher level section - at the same level as division

\medskip
Because prime numbers are so special and important, it would be nice to have some sort of formula or algorithm to produce all the prime numbers. Ideally, we would like to define a function $p(n)$ that spits out the $n$th prime number. If you can find such a formula for $p(n)$ that is easy to evaluate, eternal fame is yours. It's not at all easy and actually one of the holy grails of mathematics.

% The Correct Definition of Prime Numbers | They taught you WRONG!
% https://www.youtube.com/watch?v=yu13sjeHFXo
% -Defines primes by the property: A number p is a prime iff for all integers a,b:
%  p | ab  ->  p|a or p|b
% -For the integers, this definition is equivalent to the simpler definition above but this 
%  definition here is better suited for generalizations to other number systems

\paragraph{Willans' Formula}
But what exactly do I mean by "easy to evaluate"? If anything is allowed, then it is indeed possible to write down an explicit formula for the $n$th prime. For example, this beast here:
\begin{equation}
 p(n) = 1 + \sum_{i = 1}^{2^n} 
 \floor[\Bigg]{
 \Biggl(
 \frac{n}{\sum_{j=1}^{i} \floor[\big]{ \bigl( \cos(\pi \frac{(j+1)!+1}{j}) \bigr)^2 } 
  }
 \Biggr)^{1/n}
 }
\end{equation}
is due to C.P. Willans and I will not explain here why it works because that's not my point here [TODO: insert reference to an explanation]. My point is that it certainly qualifies as an explicit formula but it is practically useless. The first thing we note is that the outer sum goes up to $2^n$ which already puts the cost of the evaluation of the formula as function of $n$ into the category of algorithms with exponential time complexity. If you don't know what that means, let me just tell you that it means: "too inefficient for practical use except (maybe) for very small inputs". And that is true even if we overly optimistically assume the thing inside the sum to be in the realm of constant complexity - which it certainly isn't. 

% Willan's formula:
% https://en.wikipedia.org/wiki/Formula_for_primes
% https://www.youtube.com/watch?v=j5s0h42GfvM
% https://mathworld.wolfram.com/WillansFormula.html
% https://www.cambridge.org/core/journals/mathematical-gazette/article/abs/on-formulae-for-the-nth-prime-number/43E49D11DFEAD3E4CBC12F17C87F5EE1

% Michael Penn presents an alternative formula here:
% https://www.youtube.com/watch?v=eQ4ozNU6Mck
% -This formula is somewhat simpler because it doesn't need the cosine.
% -These formulas are built from Wilson's theorem which implies (is actually equivalent to): 
%  p is prime iff (p-2)! = 1 (mod p)
% -The upper summation limit 2^n is needed because the n-th prime is <= 2^n. With a tighter bound,
%  we could perhaps reduce this limit to something more sane.

% how to find all of the primes
% https://www.youtube.com/watch?v=Cnr-Jvn8B74
% 6:30: Wilson's theorem: n is prime  iff  (n-1)! = -1 (mod n)

\paragraph{The Sieve of Erathostenes}
A much more practical way to produce primes is to precompute a list of primes, store it somewhere and read the $n$th entry from this list whenever the $n$th prime is needed. One algorithm to produce such a list is the sieve of Erathostenes. The algorithm works as follows: We start with a list of all numbers up to some upper limit $m$. The first prime in this list is $2$. We pick $2$ and mark all multiples of it (except $2$ itself) as non-prime. They can't be prime because they are multiples of $2$. Then we scan the list for the first number that is not yet marked as non-prime. That would be $3$. Then again we  mark all proper multiples of $3$ as non-prime. Repeat: the next is $5$. And so on. When we are done, all numbers in our list that are not marked as non-prime are primes. We can actually stop when we reach $\sqrt{m}$ because ...TBC...

% ToDo: talk about the algorithmic complexity - I think what we actually want in number theory is to find efficient formulas or algorithms to produce prime numbers or to decide whether or not a given number is prime

%We mark all numbers that are multiples of $2$ except two itself as non-prime. They are mulitples of 

\paragraph{Primality Tests}
Another question besides "what is the $n$th prime" could be that we have some number $n$ given and want to figure out whether or not $n$ is a prime number. If we have a list of primes that is long enough that $n$ would be included, if it would be prime, then we could just search in this list for $n$. Assuming to use use binary search, this would have the rather efficient complexity of $\mathcal{O}(\log(n))$. If we find it, then $n$ is prime and if don't find it, then it isn't prime. If no such list is available, a naive algorithm to test primality is by "trial division": we try to divide $n$ by all numbers less than or equal to $\sqrt{n}$ and check the remainder. If in one of these trials, we get a remainder of zero, then $n$ is divisible by that number and hence not prime. If we find no such case, then $n$ is prime. That algorithm would have a complexity of $\mathcal{O}(\sqrt{n})$ and is not really practical for large $n$.


% we say that $12$ is divisible by $2$ and $6$ and 

% https://en.wikipedia.org/wiki/Table_of_divisors

% https://en.wikipedia.org/wiki/Divisor_function
% The divisor function is multiplicative ... but not completely multiplicative
% maybe include that in the section about multiplicativity

% https://en.wikipedia.org/wiki/Highly_composite_number
% https://en.wikipedia.org/wiki/Superior_highly_composite_number

\paragraph{Euclid's Theorem} There are infinitely many prime numbers.

%\paragraph{Fundamental Theorem of Arithmetic} Every natural number $n$ can be written as a product of prime numbers. 

% https://www.youtube.com/watch?v=iaUwNuaSLUk  Cardinality of the Continuum
% Has a simple explanation for Euclid's theorem in the first two minutes

% What about primality in finite rings? For example in Z_7 we have 2*5 = 10 = 3, so 3 is not a prime
% in Z_7. Are there any primes at all in finite fields. I don't think so. But in finite rings that are not fields, e.g. Z_6, there should be primes.

%---------------------------------------------------------------------------------------------------
\subsubsection{Prime Factorization - The Fundamental Theorem of Arithmetic}
Any positive natural number $n \geq 1$ can be written as a product of prime numbers in a way that is unique up to ordering of the factors. That statement is an important theorem, called the \emph{"Fundamental Theorem of Arithmetic"}. In such a product, a prime number is allowed to occur multiple times so you may also hear the statement in the form: "...as a product of powers of prime numbers". The theorem can be expressed as a formula as:
\begin{equation}
 n = p_1^{e_1} \cdot p_2^{e_2} \cdot \ldots \cdot p_m^{e_m}
   = \prod_{k=1}^{m} p_k^{e_k}
\end{equation}
where the $p_k$ are the distinct prime factors, the $e_k$ their associated exponents and $m$ the number of distinct prime factors. The theorem says that such an expansion is always possible. It is customary to order the factors such that $p_1 < p_2 < \ldots < p_m$. This representation of the number $n$ as such a product is called the \emph{canonical representation} of $n$ or the \emph{standard form}. The prime numbers themselves have only one single factor - the number itself. But that is also considered to be a product - namely a product over one single factor. In programmer's speak, it's a somewhat degenerate edge case. The even more edgy edge case is a product over zero numbers, i.e. the empty product, which is defined as $1$. So we have $1$ also covered. You may object that the factorization of $1$ is not unique because we can multiply any number of ones into the product without changing it - but remember that one is, by definition, not a prime number. 

\medskip
If we allow $e_k = 0$ then we can actually write $n$ as a product over all primes:
\begin{equation}
 n = \prod_{k=1}^{\infty} p_k^{e_k}
\end{equation}
and it doesn't matter whether a prime $p_k$ occurs in $n$ or not. If it doesn't, we just set the corresponding exponent to zero. For the number $1$, all $e_k$ would be zero and we would have an infinite product over ones which is still one. The theorem implies that we can break down any number $n$ into its prime factors. For this reason, the prime numbers ares sometimes called the "atoms" of the natural numbers. The process of finding these prime factors, or sometimes also the result of that process, is called the \emph{prime factorization} of $n$.

% https://en.wikipedia.org/wiki/Integer_factorization
% https://en.wikipedia.org/wiki/Fundamental_theorem_of_arithmetic

% ...is called the canonical representation or standard form

\paragraph{Naive Algorithm} A naive algorithm to obtain such a factorization of a given number $n$ could proceed as follows. Let's assume, we have a list of prime numbers available and its biggest entry is greater than or equal to $\sqrt{n}$. ...TBC...

% https://en.wikipedia.org/wiki/Square-free_integer
% all prime factors are unique

\paragraph{Relevance for Cryptography} The fact that factoring numbers into their prime factors is computationally expensive for large numbers is the basis of most contemporary asymmetric encryption algorithms. In this context, "asymmetric" means that those algorithms work with a public key for encryption and a private key for decryption. The most well known example of such an algorithm is the RSA-algorithm, named after their inventors Rivest, Shamir and Adleman.

% https://en.wikipedia.org/wiki/RSA_(cryptosystem)#Operation





%---------------------------------------------------------------------------------------------------
\subsubsection{Division with Remainder}
% Every natural number n can written as  n = d*q + r  for every given positive natural number d
% where q: quotient (not stardardized according to Weitz), r: remainder, d: divisor (given), 
% n: number (given, dividend) where q,r are uniqely determined and r < d
% what happens in rings where multiplication non-commutative?
% -Frame division as repated subtraction:
%  -Initialize r = n, q = 0
%  -As long as r >= d: r -= d, q += 1
% -This algorithm will eventually halt
% -When it halts, q and r will be our results
%
% -Notation for remainders, example: 16 mod 7 = 2
% 
% This is a naive algorithm with tiem complexity O(n). Give better algorithm...

% http://weitz.de/files/skript.pdf




\paragraph{Modulo for Negative Numbers}
% Take care when using the modulo operator in programming languages. It may produce different
% results!



\paragraph{Divisibility Rules}
% give rules for decimal numbers
% if a|b and a|c then a|(x*b + y*c) for any x,y in Z  (Bezout's Lemma(?))
% special cases: a|(b+c), a|(b-c), a|(c-b)

\subsubsection{Greatest Common Divisors}
% denoted as gcd(n,m) or sometimes just as (n,m) in articles about number theory.

% gcd and lcm in terms of prime factorization:
% https://en.wikipedia.org/wiki/Fundamental_theorem_of_arithmetic#Arithmetic_operations

\paragraph{The Euclidean Algorithm}
% still to this day the best known way to compute the gcd.


% Could this be the foundation of Number Theory? The Euclidean Algorithm visualized
% https://www.youtube.com/watch?v=ZUgzeUVsMME
% -has nice explanation of Hasse diagrams
% -Euclidean algo is facster to find the GCD than doing a full factorization of the inputs

\paragraph{Lowest Common Multiples}
A concept that is closely related to the idea of a greatest common divisor is that of a lowest common multiple. ...TBC...

% https://www.youtube.com/watch?v=Vpt5F53D47Y
% is it true that a|bc  ->  a|b or a|c ? NOPE: Example: b=6, c=5, a=10
% Correct is: a|bc and gcd(a,b) = 1  ->  a|c

% lcm(a,b) * gcd(a,b) = a*b

% Can we invert gcd or lcm?. Like gcd(8,12) = 4. If we know 8 and 4, can we reproduce 12? No!
% It could also have been e.g. 20. But maybe we can reproduce a,b if we know gcd(a,b) and 
% lcm(a,b)? For a numeric experiment, we could loop through a,b (from 1 to N, say), produce
% gcd and lcm for the N^2 possible pairs and check, if they are all unique. If one of the pairs
% appears more than once, we have to chance of inverting this 2-in/2-out function. If the pairs
% are all unique, we could invert the function by tabulation. But maybe there could be a better
% algorithm?

% I think, neutral elements for gcd and lcm are 0 and 1 respectively (verify!)

% What about lcm(a,b) / gcd(a,b)? Is this an interesting function? Has it some interpretation? Can
% we state some interesting laws for that?

% Are there any interesting properties of gcd and lcm? Like - maybe a distributive rule? But no:
%
%   lcm(10,gcd(12,15)) = 30 != 10 = lcm(gcd(10,12),gcd(10,15))
%   gcd(10,lcm(12,15)) = 10 != 30 = gcd(lcm(10,12),lcm(10,15))
%
% But it looks like they are sort of anti-distributive, if that's a thing? They actually seem to
% be anti-distributive in both possible ways: lcm anti-distributes over gcd and vice versa. Maybe
% we could call that feature mutually antidistributive or bi-antidistributive? Compare that to the
% de morgan rules of logic and set theory - they are mutually dsitributive, I think. And/or maybe
% they are self distributive? And/or maybe distributive with respect to multiplication? Figure 
% that out! If that's the case, maybe it may make sense to introduce infix operators for gcd and 
% lcm. Due to commutativity, they should be horizontally symmetric. Maybe a downard arrow for gcd
% and an upward arrow for lcm could make sense. It's suggestive as: divide-down/multiply-up.

% See also:
% https://en.wikipedia.org/wiki/Least_common_multiple#Lattice-theoretic


%---------------------------------------------------------------------------------------------------
\subsubsection{Coprimality}
Two numbers $n$ and $m$ are said to be \emph{coprime} or \emph{mututally prime} when they do not have a common prime factor in their factorization. That is equivalent to saying that their greatest common divisor is $1$. It is also equivalent to saying that their least common multiple is just the product of the two numbers [VERIFY!].

% numbers are always coprime to their successor

%===================================================================================================
% maybe this should be a section rather than a subsection:
\subsection{Number Theoretic Functions}
\emph{Number theoretic functions}, sometimes also called \emph{arithmetic functions}, are functions that take a natural number $n$ as input and produce as output another number. Real or complex outputs are allowed in general but more typically, the outputs are also integers [VERIFY]. These functions encode some property about the natural numbers. We have already encountered the number of divisors function $d(n)$ as an example of such a number theoretic function. But there are many more. Before introducing them one by one, we'll first talk about some general features that these functions may or may not have. ...TBC...

% Should be a section, not a subsection

% explain multiplicativity as an important feature. I think, it means that f(m*n) = f(m) * f(n)

% https://en.wikipedia.org/wiki/Arithmetic_function

% https://en.wikipedia.org/wiki/Divisor_function
% https://en.wikipedia.org/wiki/Partition_function_(number_theory)#Recurrence_relations
% https://en.wikipedia.org/wiki/Integer_partition


% The Euler Totient Function: A MASTERCLASS! | Number Theory Lecture!
% https://www.youtube.com/watch?v=60ibhKzLzL8



%---------------------------------------------------------------------------------------------------
\subsubsection{The Prime Counting Function}
The so called \emph{prime counting function} is an important function in number theory because it encapsulates all the information that there is to know about where the prime numbers are. It is denoted by the greek letter pi as $\pi(n)$. This is not to be confused with the semicircle constant $\pi$ from geometry. This $\pi$ here is a function that takes a natural number $n$ input and it spits out the number of prime numbers that are less than or equal to $n$. For example, $\pi(10) = 4$ because there are $4$ primes that are less than or equal to $10$, namely $2,3,5,7$. At $n = 11$, it jumps up from $4$ to $5$, because $11$ is a prime. It's a stairstep function that makes a step of height $1$ up at every prime number. ...TBC...Plot the function


\paragraph{The Prime Number Theorem}
The prime number theorem makes a statement about the asymptotic behavior of the prime counting function $\pi(n)$. It says that $\pi(n)$ is asymptotically equivalent to $n / \log(n)$. That means that when we use $n / \log(n)$ as approximation for $\pi(n)$, then the relative error, i.e. the error as percentage of the correct value, will approach zero as $n$ approaches infinity. Another way to state this is that the ratio between the actual function $\pi(n)$ and its approximant $n / \log(n)$ approaches one as $n$ approaches infinity. That's the way it's usually stated as a formula:
\begin{equation}
 \lim_{n \rightarrow \infty}  \frac{\pi(n)}{n / \log(n) }  = 1
\end{equation}
It's equivalent to saying that the $n$th prime number is approximately to be found at $n \log(n)$. As said, yet another way to state the theorem is that $n / \log(n)$ is a "good" approximation to $\pi(n)$, i.e. $\pi(n) \approx n / \log(n)$ in the sense that both functions are asymptotically equivalent. 

\medskip
There are other asymptotically equivalent functions that are still better, though. With the simple tweak to the function to use $n / (\log(n) - 1)$ instead, one can already improve the approximation quite a bit. One might also try to use $n / (\log(n) - B)$ for some constant $B$ that one should pick in an optimal way\footnote{ToDo: Explain the optimality criterion. Are we doing some sort of least squares fitting here?}. In his first experiments, Legendre determined an optimal value of $B = 1.08366$ which was dubbed Legendre's constant at the time. But it appeared only to be that value because Legendre looked at a small amount of data. It later turned out that the optimal value is indeed exactly $1$.

\medskip
There is a still better way to approximate $\pi(n)$. It is in terms of the integral logarithm function $\Li(n)$ (see page \pageref{Eq:LogarithmicIntegral}). This approximation just uses $\pi(n) \approx \Li(n)$. This approximation is used in a stronger version of the prime number theorem that is still unproven - the famous Riemann hypothesis. If that hypothesis is true, then the approximation error between $\pi(n)$ and $\Li(n)$ will approach zero even faster ...TBC...

%...TBC...ToDo: Give better approximation in terms of $\Li(x)$ 

% https://en.wikipedia.org/wiki/Prime_number_theorem
% https://mathworld.wolfram.com/PrimeNumberTheorem.html
% https://www.britannica.com/science/prime-number-theorem

% https://en.wikipedia.org/wiki/Legendre%27s_constant

% number of (unique) prime factors https://en.wikipedia.org/wiki/Prime_omega_function
% arithmetic derivative https://en.wikipedia.org/wiki/Arithmetic_derivative

% How to Prove a Number is Irrational
% https://www.youtube.com/watch?v=cQYGwiXSvi8
% The part about "Legendre's Constant" is interesting. It turns out, that this constant is just
% the number 1.
% \pi(x) ~= x / (ln(x) - B)    B is Legendre's constant - turns out, it's 1.
% https://en.wikipedia.org/wiki/Legendre%27s_constant

% Factorials, prime numbers, and the Riemann Hypothesis
% https://www.youtube.com/watch?v=oVaSA_b938U
% -prime density function: \delta(x) = 1 / log(x)
% -The prime number theorem says: the order of approximation error of Li(x) in approximating \pi(x)
%  is of order less than x^1. That is: $\lim_{x \rightarrow \infty} |pi(x) - Li(x)| / x^1 = 0$
% -The Riemann hypothesis strengthens this by saying the order of the approximation error behaves 
%  even better - namely $\lim_{x \rightarrow \infty} |pi(x) - Li(x)| / x^p = 0$ when p > 0.5

% What is the Riemann Hypothesis REALLY about?
% https://www.youtube.com/watch?v=e4kOh7qlsM4

% The Key to the Riemann Hypothesis - Numberphile
% https://www.youtube.com/watch?v=VTveQ1ndH1c

\begin{comment}


Mathe-News: Die größte bekannte Primzahl
https://www.youtube.com/watch?v=SPYf1M_DqHU
-Primes of the form  a^b - 1  can only be of the form  2^p - 1  with p being itself prime. That is,
 the only basis that works is 2 and the exponent has to be a prime.
-Fermat's little theorem: a^p = p  (mod p)   for p prime an a integer. If a is no multiple of p then
 a^(p-1) = 1  (mod p)
-Lucas-Lehmer test


Idea for proof by induction for number-theorectical theorems:
-Base case: Show that the theorem holds for prime numbers 
-Induction step: Show that if the theorem holds for two numbers a and b, it follows that it also
 holds for their product a*b. 
-That shows that it holds for all natural numbers. Maybe call it induction over factorizations or
 shorter: factor induction.


What is the Moebius function? #SomePi
https://www.youtube.com/watch?v=fGbJrY75LU8


-The term "square-free" appears quite a lot all over math - in number theory but also in the theory
 of polynomials. It's not a well chosen word, though. What it intends to mean is that every factor
 is unique, i.e. has a multiplicity of 1. Having no squares (factors with multiplicity 2), of
 course, implies that there are no cubes (multiplicity 3) either or even higher order factors 
 because a cube contains a square, so to speak: x^3 = x^2 * x. But when one hears "square free", 
 one might be tempted to think only about the actual squares - not the cubic, quartic, quitic, 
etc. terms. But they are meant as well. ...I think...verify!


\end{comment}