\section{Division} 

\section{Divisibility}
A natural number like, say, $12$ can be divided by $6$ to give $2$. The result $2$ happens to be a natural number. We will denote this circumstances as $6 \mid 12$ which reads as "six divides twelve" or "6 is a divisor of twelve". Conversely, it can also divided by $2$ to give $6$, so we also have $2 \mid 12$. On the other hand $12 / 5 = 2.4$ which is not a natural number, so $5$ does not divide $12$ which we write as $5 \nmid 12$. The number $12$ is also divisible by $3$ because this division results in $4$ which is also a natural number. And, of course, by symmetry, we have also that $3$ divides $12$. The set of divisors of $12$ is given by $\{2,3,4,6\}$. You may check that there are not other divisors except the numbers $1$ and $12$ itself. The numbers itself and the number $1$ are always divisors of any number, so we call them the trivial divisors. As a matter of convention, in a list of divisors of a given number, we usually include these trivial divisors as well, so the set of divisors of $12$ is actually $\{1,2,3,4,6,12\}$.

% we say that $12$ is divisible by $2$ and $6$ and 

\subsection{Division with Remainder}

\subsubsection{Greatest Common Divisors}
\paragraph{The Euclidean Algorithm}
\paragraph{Lowest Common Multiples}
A concept that is closely related to the idea of a greatest common divisor is that of a lowest common multiple.

\section{Prime Numbers}

\subsection{Factorization}