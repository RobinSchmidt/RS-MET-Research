\section{Modular Integers}


% Some authors use the abbreviation $\mathbb{Z}_p = \mathbb{Z} / p \mathbb{Z}$ exclusively in the case where $p$ is prime. I have never seen a convincing justification for such a restriction (actually, not ever any justification at all), so I will liberally use the notation $\mathbb{Z}_n = \mathbb{Z} / n \mathbb{Z}$ also for composite numbers $n$.

% Do not make the mistake to think of the numbers 0,1,2,...,p-1 as the natural numbers. They behave very differently - at least when wrap-around occurs

%===================================================================================================
\subsection{Operations}

\subsubsection{Addition and Multiplication}
% -also include division, explain how to construct the inverse elements for addition and 
%  multiplication
% -mention that division only works when modulus m is a prime - only then will have all elements
%  multiplicative inverses

\subsubsection{Exponentiation}
% -also include roots and logarithms
% -state conditions under which roots and logarithms exist


%===================================================================================================
\subsection{Theorems}


\subsubsection{The Chinese Remainder Theorem}

% https://www.youtube.com/watch?v=KwHHYv2WCg8


%###################################################################################################
% Maybe move into its own file
\section{Other Number Systems}

\subsection{Quadratic Integers}

% Quadratic Integers and Primes
% https://www.youtube.com/playlist?list=PLRcOL8MUr-pcIbez7ZRalqnSmYvLT00Fd

\subsubsection{Gaussian Integers}

\subsubsection{Eisenstein Integers}

\subsection{Cyclotomic Integers}


\subsection{$p$-adic Numbers}

\subsubsection{Motivation}
Consider this: $9 + 1 = 10$, $99 + 1 = 100$, $999 + 1 = 1000$ and so on. What should $\ldots 999 + 1$ be? Apparently, it would have to be $\ldots 000$, i.e. a string of zeros that extends infinitely far to the left. We are not used to think about numbers like this. Usually, in our common decimal system, we write down numbers with a finite number of digits. But we sometimes allow ourselves to think about a number like $14.72$ as $14.7200000\ldots$, i.e. we may imagine to zero extend it infinitely far to the right. But to the left? No! That doesn't make any sense! Or does it? Well - if $\ldots 000$ should make any sense then it certainly appears as if it should be out good old friend, the number zero. Now consider this: $\ldots 999  = \sum_{k=0}^{\infty} 9 \cdot 10^k = 9 \cdot \sum_{k=0}^{\infty} 10^k$. If we apply our formula for the geometric series $\sum_{k=0}^{\infty} p^k = \frac{1}{1-p}$ , which isn't supposed to work in this case because $|p| \geq 1$, we would get $9 \frac{1}{1-10} = 9 \frac{1}{-9} = -1$. If we compare this with our previous "result" $\ldots 999 + 1 = 0 \Leftrightarrow \ldots 999 = -1$, that seems to be compatible. Is your mind already blown? ...TBC...

%at least  ...TBC...
%If you reca

\subsubsection{Convergence on $\mathbb{Q}$}
We know from calculus that a sequence of (rational) numbers is said to converge to a limit $L$ if the distance between $L$ and the sequence elements becomes arbitrarily small. By that, we implicitly assume that by "distance" we mean the absolute value of the difference. It turns out that by defining a different distance measure, we can re-purpose this definition of convergence to mean something different. ...TBC...

\paragraph{Distance Metrics}

% In the Topology chapter, we define the requirements for a distance function. Maybe move it into the LinAlg chapter and the refer to it from here and from topology


%As distance between two numbers, we usually take the norm of their difference where "norm", so far, was just a fancy word for "absolute value". We know from one of the very first chapters (page \ref{Tab:Norm}) that there are a couple of properties that we expect from a sensible norm: positive definiteness, multiplicativity and triangle inequality. The usual absolute value on the rational numbers satisfies these requirements. It turns out, though, that there is an alternative possibility that also satisfies them. ...TBC...

% -define what properties a metric has to satisfy
% -state Ostrowski's theorem: there are onyl two possible metrics on the rationals:
%  -the normal absolute value of difference
%  -the p-adic metric

% Mathematicians Use Numbers Differently From The Rest of Us
% https://www.youtube.com/watch?v=tRaq4aYPzCc  by Veritasium
% -about p-adic numbers

% https://www.youtube.com/watch?v=o02uipdcT7Y  How to Wrangle Infinity (an intro to p-adic numbers)
% https://www.youtube.com/watch?v=v9QTK7zBAhw  Intuition for the p-adic metric
% https://www.youtube.com/watch?v=aSxvz0NUXfc  Ostrowski's Theorem (p-adic metric continued)

% 1 Billion is Tiny in an Alternate Universe: Introduction to p-adic Numbers
% https://www.youtube.com/watch?v=3gyHKCDq1YA

% [ANT15] p-adic integers: a primer, and an application (part 1)
% https://www.youtube.com/watch?v=SwbJmKbFo7o

% -The notion of convegergence of a sequence is formally the same: the distance between successive
%  elements must get arbitraily small - but it means something very different because the distance
%  is defined in a very different way.
% -Numbers that extend have many digits may have very small absolute value
% -The numbers exntend infinitely to the left
% -the number of zeros on the right end tells us, how often the number is divisible by the base
% -Ostrowski's theorem: there are only two notrivial metrics on the rationals: the usual abs-of-dif
%  and the p-adic metric
% -p-adic numbers can only have finitely many digits to the right of the decimal point

% What does it feel like to invent math?
% https://www.youtube.com/watch?v=XFDM1ip5HdU  by 3blue1brown


%###################################################################################################
% Maybe move into its own file
\section{Open Problems}
The field of number theory is full of open problems that are well known in popular math. This is because the problems are often easy to state but have turned out to be really hard to solve. Some problems can be understood with just middle school or even elementary school math background but even the best mathematicians in the world were not yet able to solve them. That's what makes them so charming for exposition in pop-math.

\paragraph{Goldbach's Conjecture}
This conjecture states that every even natural number greater than $2$ can be written as the sum of two prime numbers (not necessarily in a unique way). At the time of this writing, the conjecture has been experimentally verified for numbers up to $4 \cdot 10^{18}$ but a proof is still elusive. Although $10^{18}$ may sound like a lot, compared with infinity, it is about just as good (i.e. bad) as $10$ itself\footnote{OK - I may be slightly exaggerating. Having a conjecture hold for up to $10^{18}$ may give us a bit more confidence than having it to hold only up to $10$. But really, compared with infinity, every finite number is kinda like the same size - the ratio $\frac{finite}{infinity}$ is always exactly zero.}. And there have been historical examples for conjectures that did hold up experimentally for very big numbers but have nonetheless proven to be false ...TBC...give example

% every even natural number greater than 2 is the sum of two prime numbers.

% How about: Every natural number can be written as sum of 2 prime-powers? I have experimentally verified it in my head for numbers up to 20 (LOL!)...maybe write a little algorithm to test some more numbers.

\paragraph{Collatz's Conjecture}
Another unsolved number theoretic problem that is quite popular is the following: Consider the following simple algorithm: Start with any natural number $n > 1$. Now repeat: If the number is even, divide it by two, If the number is odd, triple it and add one. That means, in each iteration, you apply the simple function:
\begin{equation}
f(n) = \begin{cases}
n/2    \qquad & \text{if $n$ is even}  \\
3n + 1 \qquad & \text{if $n$ is odd}
\end{cases}
\end{equation}
and take the result as input for the next iteration. The conjecture now says that for any starting value $n_0$, the iteration will eventually settle into the cycle $4-2-1-4-2-1-\ldots$. Experimentally, this was verified for numbers up to the order of $10^{20}$. The problem was brought up by Lothar Collatz in 1937 and his therefore named after him. Another name is the $3n + 1$ conjecture. There are generalizations......TBC...

% https://en.wikipedia.org/wiki/Collatz_conjecture
% https://de.wikipedia.org/wiki/Collatz-Problem
% https://de.wikipedia.org/wiki/Collatz_Conjecture
% https://www.spektrum.de/lexikon/mathematik/das-collatz-problem/1712
% https://en.wikipedia.org/wiki/Collatz_conjecture#Experimental_evidence


% The Oldest Unsolved Problem in Math
% https://www.youtube.com/watch?v=Zrv1EDIqHkY
% -Do odd perfect numbers exist?

\begin{comment}
-Modular addition and multiplication
-Modular exponentiation, logarithms and roots
-Chinese Remainder Theorem
\end{comment}