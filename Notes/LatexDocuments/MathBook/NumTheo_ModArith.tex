\section{Modular Arithmetic}


% Some authors use the abbreviation $\mathbb{Z}_p = \mathbb{Z} / p \mathbb{Z}$ exclusively in the case where $p$ is prime. I have never seen a convincing justification for such a restriction (actually, not ever any justification at all), so I will liberally use the notation $\mathbb{Z}_n = \mathbb{Z} / n \mathbb{Z}$ also for composite numbers $n$.

%===================================================================================================
\subsection{Operations}

\subsubsection{Addition and Multiplication}
% -also include division, explain how to construct the inverse elements for addition and 
%  multiplication
% -mention that division only works when modulus m is a prime - only then will have all elements
%  multiplicative inverses

\subsubsection{Exponentiation}
% -also include roots and logarithms
% -state conditions under which roots and logarithms exist


%===================================================================================================
\subsection{Theorems}


\subsubsection{The Chinese Remainder Theorem}

% https://www.youtube.com/watch?v=KwHHYv2WCg8


\begin{comment}
-Modular addition and multiplication
-Modular exponentiation, logarithms and roots
-Chinese Remainder Theorem
\end{comment}