\section{Modular Arithmetic}


% Some authors use the abbreviation $\mathbb{Z}_p = \mathbb{Z} / p \mathbb{Z}$ exclusively in the case where $p$ is prime. I have never seen a convincing justification for such a restriction (actually, not ever any justification at all), so I will liberally use the notation $\mathbb{Z}_n = \mathbb{Z} / n \mathbb{Z}$ also for composite numbers $n$.

%===================================================================================================
\subsection{Operations}

\subsubsection{Addition and Multiplication}
% -also include division, explain how to construct the inverse elements for addition and 
%  multiplication
% -mention that division only works when modulus m is a prime - only then will have all elements
%  multiplicative inverses

\subsubsection{Exponentiation}
% -also include roots and logarithms
% -state conditions under which roots and logarithms exist


%===================================================================================================
\subsection{Theorems}


\subsubsection{The Chinese Remainder Theorem}

% https://www.youtube.com/watch?v=KwHHYv2WCg8


%###################################################################################################
% Maybe move into its own file
\section{Open Problems}
The field of number theory is full of open problems that are well known in popular math. This is because the problems are often easy to state but have turned out to be really hard to solve. Some problems can be understood with just middle school or even elementary school math background but even the best mathematicians in the world were not yet able to solve them. That's what makes them so charming for exposition in pop-math.

\paragraph{Goldbach's Conjecture}
This conjecture states that every even natural number greater than $2$ can be written as the sum of two prime numbers (not necessarily in a unique way). At the time of this writing, the conjecture has been experimentally verified for numbers up to $4 \cdot 10^{18}$ but a proof is still elusive. Although $10^{18}$ may sound like a lot, compared with infinity, it is about just as good (i.e. bad) as $10$ itself\footnote{OK - I may be slightly exaggerating. Having a conjecture hold for up to $10^{18}$ may give us a bit more confidence than having it to hold only up to $10$. But really, compared with infinity, every finite number is kinda like the same size - the ratio $\frac{finite}{infinity}$ is always exactly zero.}. And there have been historical examples for conjectures that did hold up experimentally for very big numbers but have nonetheless proven to be false ...TBC...give example

% every even natural number greater than 2 is the sum of two prime numbers.

% How about: Every natural number can be written as sum of 2 prime-powers? I have experimentally verified it in my head for numbers up to 20 (LOL!)...maybe write a little algorithm to test some more numbers.

\paragraph{Collatz's Conjecture}
Another unsolved number theoretic problem that is quite popular is the following: Consider the following simple algorithm: Start with any natural number $n > 1$. Now repeat: If the number is even, divide it by two, If the number is odd, triple it and add one. That means, in each iteration, you apply the simple function:
\begin{equation}
f(n) = \begin{cases}
n/2    \qquad & \text{if $n$ is even}  \\
3n + 1 \qquad & \text{if $n$ is odd}
\end{cases}
\end{equation}
and take the result as input for the next iteration. The conjecture now says that for any starting value $n_0$, the iteration will eventually settle into the cycle $4-2-1-4-2-1-\ldots$. Experimentally, this was verified for numbers up to the order of $10^{20}$. The problem was brought up by Lothar Collatz in 1937 and his therefore named after him. Another name is the $3n + 1$ conjecture. There are generalizations......TBC...

% https://en.wikipedia.org/wiki/Collatz_conjecture
% https://de.wikipedia.org/wiki/Collatz-Problem
% https://de.wikipedia.org/wiki/Collatz_Conjecture
% https://www.spektrum.de/lexikon/mathematik/das-collatz-problem/1712
% https://en.wikipedia.org/wiki/Collatz_conjecture#Experimental_evidence

\begin{comment}
-Modular addition and multiplication
-Modular exponentiation, logarithms and roots
-Chinese Remainder Theorem
\end{comment}