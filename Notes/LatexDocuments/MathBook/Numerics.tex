\chapter{Numerics}
Numerical mathematics, or numerics, for short, is the branch of math that deals with tasks like actually producing numbers, typically floating point numbers, on a computer. Tasks like evaluating the function $f(x) = \sin(x)$ at some given $x$ or finding a definite integral of a complicated function that has no elementary antiderivative or finding the solution of some complicated differential equation that has no elementary solution or finding all the complex roots of a polynomial high degree such that no elementary formula exists. 

%evaluating

% types of error: approximation error, rounding error

\section{Numerical Linear Algebra}

\subsection{Solving Linear Systems of Equations}

\subsection{Finding Eigenvalues and Eigenvectors}

\subsection{Matrix Decompositions}





\section{Numerical Analysis}


\subsection{Interpolation}
Interpolation is the business of reconstructing a continuous function from a bunch of data points. In the simplest case, we are dealing with an unknown $1D$ function of the of the form $y = f(x)$ and have pairs of data points $(x_i, y_i)$. Given only the data, we are only able to "evaluate" our mystery function exactly at the data points. That means, if someone gives us n $x$-value, we can immediately produce the corresponding $y$-value by just searching through our data until we find a point with $x_i = x$ and then just spit out the corresponding $y_i$. But what if one gives as an $x$ that isn't any of the recorded $x_i$? If we assume that the given $x$ falls somewhere in between some recorded $x_i$ and $x_{i+1}$, then we call the task \emph{interpolation}. It could also be the case that the given $x$ falls outside the range of observed data. In such a case, the task is called \emph{extrapolation}.

% Ex.: f(x) = x^2, data: (0,0), (1,1), (2,4), (3,9), (4,16), (5,25)

% extrapolation


\subsection{Function Approximation}

\subsection{Function Evaluation}

\subsection{Root Finding}


\subsection{Optimization}

% constrained and unconstrained



\begin{comment}

\end{comment}