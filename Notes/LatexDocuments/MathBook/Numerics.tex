\chapter{Numerics}
Numerical mathematics, or numerics, for short, is the branch of math that deals with tasks like actually producing numbers, typically floating point numbers, on a computer. Tasks like evaluating the function $f(x) = \sin(x)$ at some given $x$ or finding a definite integral of a complicated function that has no elementary antiderivative or finding the solution of some complicated differential equation that has no elementary solution or finding all the complex roots of a polynomial high degree such that no elementary formula exists. Or tasks like solving a linear system of equations or finding all eigenvalues and eigenvectors of a given matrix or finding the singular value decomposition of a matrix

%evaluating

% types of error: approximation error, rounding error

%###################################################################################################
\section{Numerical Linear Algebra}


%===================================================================================================
\subsection{Solving Linear Systems of Equations}

%---------------------------------------------------------------------------------------------------
\subsubsection{Gaussian Elimination}

\paragraph{Pivoting}

\paragraph{Gauss-Jordan Method}


%===================================================================================================
\subsection{Finding Eigenvalues and Eigenvectors}


%===================================================================================================
\subsection{Matrix Decompositions}

%---------------------------------------------------------------------------------------------------
\subsubsection{$LU$-Decomposition}

% QR, eigen, SVD




%###################################################################################################
\section{Numerical Analysis}


%===================================================================================================
\subsection{Interpolation}
Interpolation is the business of reconstructing a continuous function from a bunch of data points. In the simplest case, we are dealing with an unknown $1D$ function of the of the form $y = f(x)$ and have pairs of data points $(x_i, y_i)$. Given only the data, we are only able to "evaluate" our mystery function exactly at the data points. That means, if someone gives us n $x$-value, we can immediately produce the corresponding $y$-value by just searching through our data until we find a point with $x_i = x$ and then just spit out the corresponding $y_i$. But what if one gives as an $x$ that isn't any of the recorded $x_i$? If we assume that the given $x$ falls somewhere in between some recorded $x_i$ and $x_{i+1}$, then we call the task \emph{interpolation}. It could also be the case that the given $x$ falls outside the range of observed data. In such a case, the task is called \emph{extrapolation}.

% Ex.: f(x) = x^2, data: (0,0), (1,1), (2,4), (3,9), (4,16), (5,25)

% extrapolation

%===================================================================================================
\subsection{Function Evaluation and Approximation}

%---------------------------------------------------------------------------------------------------
\subsubsection{Taylor Polynomials}
We know from calculus (see page \pageref{Eq:TaylorSeries}) that a function $f(x)$ can be expanded around a point $x_0$ via a Taylor series as follows:
\begin{equation}
f(x) = \sum_{k=0}^\infty \frac{f^{(k)}(x_0)}{k!} (x-x_0)^k
\end{equation}
If we do not take the sum up to infinity but only up to some finite value $n$, then we obtain an approximation to $f(x)$ which will tend to get better when we crank up $n$. This approximation is called the $n$th order Taylor polynomial of $f$ around $x_0$. 

% Taylor-series, Pade-Approximation, Lesat-Squares, Minimax, Chebychev-Approximation, Continued fractions, cordic


% Convolutions and Polynomial Approximation
% https://www.youtube.com/watch?v=4P4Ufumu9ms
% -Explains Weierstrass approximation theorem

%===================================================================================================
%\subsection{Function Evaluation}

%---------------------------------------------------------------------------------------------------
%\subsubsection{Taylor Polynomials}


%If we want to evaluate a function $f(x)$ and we happen  


%===================================================================================================
\subsection{Root Finding}

%===================================================================================================
\subsection{Optimization}

% constrained and unconstrained






\begin{comment}


Numerische Methoden - Angewandte Mathematik für Ingenieure
https://www.youtube.com/playlist?list=PLcQq8Z8G1veYx72P3svwG5K72WINt_tvZ


Translating formulas to code:
-A matrix equation like x = A^{-1} b means: solve A x = b. We don't want to actually compute the
 inverse of A because that's much more expensive that solveing the linear system of equation by
 e.g. Gaussian elimination.
-An equation like p = atan(y/x) usually means p = atan2(y,x) ...at least in 99.9% of all cases. 
 Beware of the argument order. atan2 usually takes the y coordinate first and the x coordinate 
 second - but  there may be exceptions. atan2(y,x) computes the angle of the point (x,y) with the x-axis. This  angle is used in the polar coordinate representation (r,p) of the point where r is 
 the radius and p the angle also known as "phase" or "argument".
 

\end{comment}