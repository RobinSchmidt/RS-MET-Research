%THIS FILE IS OBSOLETE - but maybe some comments or links or text snippets should be retained, i.e. moved elsewhere

%\chapter{Order Theory}
%Abstract algebra is a way to abstractly characterize ideas like addition and multiplication, i.e. certain binary operations between elements of a given set that produce a new element. Order theory, on the other hand, is a way to abstractly characterize certain binary relations between elements of a set such as "less", "equal", "greater", etc. ...TBC...

% What abstract algebra is to ideas of binary operations between set elements like "plus" and "times", order theory is to binary relations like "less", "equal", "greater"

% Maybe this shouldn't be a chapter in its own right but a section in the set theory chapter

\begin{comment}

https://en.wikipedia.org/wiki/Order_theory
https://en.wikipedia.org/wiki/Glossary_of_order_theory

https://en.wikipedia.org/wiki/Well-order
https://en.wikipedia.org/wiki/Well-ordering_principle
https://en.wikipedia.org/wiki/Well-ordering_theorem

https://de.wikipedia.org/wiki/Wohlordnung


-Explain what it means when one says that the complex numbers can't be ordered. Because of course 
 we can order them - for example: order by real part and in case of equal real parts, order by 
 imaginary parts. The thing is: when doing this, we will break certain relations like
 a > b and c > d  ->  ac > bd. That is: our ordering does not play nicely with the arithmetic 
 operations

\end{comment}