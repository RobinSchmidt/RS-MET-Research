\chapter{Philosophical Principles}



%###################################################################################################
\section{Elements of Math}

\subsection{Definitions}


\subsection{Theorems}

\subsection{Proofs}


\subsection{Conjectures}
A conjecture is a proposition that is seen as a potential candidate for a possible theorem. But the truth of the proposition has not yet been proven. ...TBC...

\subsubsection{Empirical Evidence}
To come up with conjectures, one can take the tried and true approach of the empirical sciences which is called the \emph{scientific method}. It consists of a cycle of (1) discovery of patterns, (2) attempt of explanation, (3) making predictions based on the explanation attempt, (4) design of experiments to verify the predictions, (5) execution of the experiments, (6) examination of the experimental results - back to (1). In mathematics, the "experiments" will typically involve a computer program. For example, it's possible to write a program that checks a proposition like the Goldbach conjecture experimentally up to some large number $N$. If the conjecture fails for some $n \leq N$, the conjecture has been falsified - it's now known to be false. If it holds up up to $N$, we may (very tacitly!) treat that as evidence, that it might hold up in general. This evidence in mathematics is sometimes called \emph{numerical evidence}. ...TBC...



%###################################################################################################
\section{Abstraction}
Mathematics in infamous for being very abstract. But was does "abstract" mean? Abstraction in mathematics is the process of recognizing, identifying and explaining common patterns in a bunch of concrete problems and then trying to find a solution for all these concrete problems at once by finding a solution to the abstract version of these problems. Solving a problem of the same type later again will then be much easier - we will just need to recognize that it again a concrete instance of an abstract problem that we have already solved. We can then just apply our general solution. ...TODO: explain generalization, permanence principle

% Generalization, Permanence, Pattern recognition/identification/explanation

%###################################################################################################
\section{Duality}

% Symmetry, Isomorphism



\begin{comment}

-Bridges (between subfields) - everything hangs together, it's a network of knowledge
-Definitions, Theorems, Proofs, Conjectures, Experimental Evidence
-Formalism vs Intution
-Finitism, Constructivism
-Automation/mechanization of thought
-Precision, Non-Ambiguity
-Complexity from simplicity - deconstruction to simpler parts...there's a name for that
 emergence?

\end{comment}