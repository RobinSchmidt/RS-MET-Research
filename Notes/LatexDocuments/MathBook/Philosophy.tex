\chapter{Philosophical Principles}
This chapter is a bit less mathematical and more philosophical and maybe meta-mathematical. We look at some general underlying principles of mathematics ...TBC...

% Meta-math


%###################################################################################################
\section{Elements of Math}

\subsection{Definitions}


\subsection{Theorems}

\subsection{Proofs}


\subsection{Conjectures}
A conjecture is a proposition that is seen as a potential candidate for a possible theorem. But the truth of the proposition has not yet been proven. Conjectures are typically formed when one plays around with some problem and discovers some apparent regularities, rules or laws which seem to be true but are not yet known facts.  ...TBC...

\subsubsection{Empirical Evidence}
To come up with conjectures, one can take the tried and true approach of the empirical sciences which is called the \emph{scientific method}. It consists of a cycle of (1) discovery of patterns, (2) attempt of explanation, (3) making predictions based on the explanation attempt, (4) design of experiments to verify the predictions, (5) execution of the experiments, (6) examination of the experimental results - back to (1). In mathematics, the "experiments" will typically involve a computer program. For example, it's possible to write a program that checks a proposition like the Goldbach conjecture experimentally up to some large number $N$. If the conjecture fails for some $n \leq N$, the conjecture has been falsified - it's now known to be false. If it holds up up to $N$, we may (very tacitly!) treat that as evidence, that it might hold up in general. This evidence in mathematics is sometimes called \emph{numerical evidence}. In contrast to the natural sciences, empirical evidence - however overwhelmingly abundant it may be - is never a substitute for a proper proof to establish the truth of a proposition. Modern mathematics has stricter requirements and higher standards - it demands an airtight, rigorous proof.

...TBC...



%###################################################################################################
\section{Abstraction}
Mathematics in infamous for being very abstract. But was does "abstract" mean? Abstraction in mathematics is the process of recognizing, identifying and explaining common patterns in a bunch of concrete problems and then trying to find a solution for all these concrete problems at once by finding a solution to the abstract, generalized version of these problems. Solving a problem of the same type later again will then be much easier - we will just need to recognize that it again a concrete instance of an abstract problem that we have already solved. We can then just apply our general solution. ...TODO: explain generalization, permanence principle

% Generalization, Permanence, Pattern recognition/identification/explanation


%###################################################################################################
\section{Reductionism and Emergence}

% Complexity from simplicity - deconstruction to simpler parts...there's a name for that
% emergence? Reductionism - I think, that was the word
% https://en.wikipedia.org/wiki/Reductionism
% -relevant also in the natural sciences

%###################################################################################################
\section{Duality}
An interesting observation across many fields of mathematics is that we sometimes find objects that in some sense are a sort of "mirror image" of other objects. ...TBC...examples: vectors/covectors, vectors/pseudovectors, points/lines, dual graphs, dual platonic solids, ...

% schlüssel/schloss prinzip
% Symmetry, Isomorphism

% What about inversion as in differentiate/integrate
% adjoint matrices?

% https://en.wikipedia.org/wiki/Duality_(mathematics)
% https://en.wikipedia.org/wiki/Dual_lattice
% https://en.wikipedia.org/wiki/Duality_(order_theory)


%###################################################################################################
\section{Unification}
The more math you learn, the more unexpected connections and bridges between seemingly unrelated fields of math become apparent. It seems like somehow everything hangs together in a vast interconnected network of mathematical knowledge. 

%as if there is a greater underlying unified theory of all (or at least much) of mathematics. ...TBC...

% example: Langlands program


\begin{comment}

-Formalism vs Intution
-Precision, Non-Ambiguity of langauge
-Automation/mechanization of thought
-Finitism, Constructivism

\end{comment}