\chapter{Physics}  


\begin{comment}
List branchs of physics and which areas of math are especially important in those:
-Classical Mechanics:
 -Differential Equations (ordinary and partial)
-Lagrangian and Hamiltonian Mechanics:
 -Calculus of Variations
-Electrodynamics:
 -Vector Calculus
-Fluid Dynamics: 
 -Vector calculus
 -Complex Analysis (Rimeann mapping, conformal mapping)
-Continuum Mechanics:
 -Tensor Calculus
-Quantum Mechanics:
 -Linear Algebra
  -Unitary and Hermitian matrices
  -Eigenvalues and -vectors
 -Group Theory:
  -Matrix Lie-groups (Pauli, Dirac, etc.)
 -Operator Theory
-Relativity:
 -Tensor calculus
 -Differential geometry
 -Group theory (Lorentz group, Poincare group, etc.)
-Thermodynamics / Statistical Mechanics: Probability theory (verify!)
-Cosmology: 
 -Differential geometry
 -Information theory (verify!), 
 -Topology (verify!)
-String Theory / M-Theory:
 -Manifolds (diff. geo.)
 -Knot Theory (?)
 -Number Theory
 
 
-Data Science:
 -Multivariable Calculus, Optimization
 -Tensor Algebra
 
 Lagrange Mechanik verstehen! - Lagrange Funktion, Euler-Lagrange Gleichung (Physik)
https://www.youtube.com/watch?v=Uk24tKZUq1M&list=PLdTL21qNWp2YiZaBF9xMb82kSpBc3YnxQ&index=71 
 
----------------------------------------------------------------------------------------------------
General Relativity:

Famous quote by physicist John Archibald Wheeler about the essence of general relativity:

"Spacetime tells matter how to move, matter tells spacetime how to curve."

Analogy in Newton's world:
Forces tell masses how to move (accelerate): F = m a, a = F/m. Masses tell (gravitational) forces how to pull: F_g = G m_1 m_2 r / |r|^3. Maybe Newton's law should be rephrased in terms of momentum to make the analogy even closer: d p / d t = F

In relativity: The change of the energy-mmomentum tensor is some function of the metric tensor - in a sloppy symbolic way: T_{\mu \nu} = f( g_{\mu \nu} ) and likewise g_{\mu \nu} = f^{-1} (T_{\mu \nu}). But actually the function $f$ also depend on spatial and temporal derivatives of the (tensor valued) quantity $g$, not just on $g$ itself - although, if we assume $g = g(t,x,y,z)$ then that function actually contains all the info abouts its (partial) derivatives anyway. But usually, when we write down a differential equation, we make such dependencies on derivatives epxlicit by writing things like: $f(g, g_t, g_x, g_y, g_z, g_{tt}, g_{tx}, g_{ty}, \ldots)$

https://en.wikipedia.org/wiki/Einstein_field_equations


https://www.youtube.com/watch?v=KW4yBSV4U38
at 7:50, she says, "they are differential equations for the metric tensor". So, the focus seems to be on the time evolution of g, not T. The equations by themselves describe both on equal footing (I think)

 
The 4 Most Fundamental Objects in ALL Physics #SoME4
https://www.youtube.com/watch?v=ggH_kIkO8jk
-Spinors, Twistors, ...
 

\end{comment}