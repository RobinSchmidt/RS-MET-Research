\chapter{Physics}  
% -should be a chapter, not a section
% -Sections should be: Classical Mechanics, Electrodynamcis, Relativity (maybe together w Electro),
%  Quantum Mechanics (including QFT, 
%  Standard Model), 
% -Emergent Physics (Fluid Dynamics, Thermodynamics, Continuum Mechanics, Magnetohydrodynamics)



%===================================================================================================
\subsection{Fluid Dynamics}
% fluid dynamics: complex calculus







%===================================================================================================
\subsection{Statistical Mechanics and Thermodynamics}
This subfield of physics deals with the application of Newtonian (and sometimes quantum) mechanics to particle systems with a number of particles that is so huge, that it's impossible and also not meaningful to try to describe what individual particles do. Instead, one considers the properties of the whole ensemble of particles as a whole. Notions like density, pressure and temperature are developed as \emph{emergent} properties of the whole ensemble. The mathematical tools that are required for this are the ideas from probability theory and statistics. ...TBC...

% emergent laws
% math: probability, statistics?

% https://en.wikipedia.org/wiki/Thermodynamics
% https://en.wikipedia.org/wiki/Laws_of_thermodynamics

% https://en.wikipedia.org/wiki/Statistical_mechanics
% https://www.britannica.com/science/statistical-mechanics

%===================================================================================================
\subsection{Continuum Mechanics}

% describes how continuous solid (elastic) media respond to forces by deformation

% math: tensor calculus, partial differential equations




%===================================================================================================
\subsection{Relativity}

% math: general: tensor calculus, differential geometry; special: hyperbolic numbers/geometry


%\subsection{Elementary Particle Physics}
% String theor, M-theory 
% math: group theory? symmetry? maybe even number theory? the zeta(-1) = -1/12 occurs somewhere in string theory, I think.


%===================================================================================================
\subsection{Quantum Field Theory}

A \emph{field theory} is a theory in which physical objects like particles are described as (scalar-, vector- or tensor-) fields that somehow interact with one another. So, in a field theory, the "particle" is not modeled as a point like object but rather as a continuous function that formally extends over all space and time. When I say it extends only "formally" over all space, you should picture in your head something like a Gaussian bell curve which also formally extends from minus to plus infinity, but tends towards zero so quickly (namely, exponentially-squared) to both sides, that it becomes negligible everywhere except for some finite region around the peak. In a field theory for particles, that is how the typical fields that model the particles behave, too [VERIFY!].

\medskip
A \emph{quantum theory} is a theory in which the values of physical quantities cannot just be anything from a continuous range but rather values from a discrete set where discrete can mean either finite or countably infinite. Which particular value from this discrete set will be observed in a particular situation will be predicted by a quantum theory only in terms of probabilities, not in a surefire deterministic sense like classical theories (field or not) would do. ...

\medskip
A quantum field theory combines both of these aspects ...TBC...

% standard model of particle physics:
% -fermions(constitute matter): electrons, quarks, neutrinos,..
% -bosons (mediate interactions): photon, gluon, Z, W, Higgs, ..

% -Higgs-field is scalar field (spin 0), other boson-fields are vector fields (spin 1),
%  fermion-fields are spinor fields (spin 1/2).
% -Spinors are made from Grassmann numbers. Their multiplication is anticommutative -> Squares of
%  such "numbers" are zero.
%  https://en.wikipedia.org/wiki/Grassmann_number
%  -this is the Pauli exclusion principle - no two fermions can be in the same state - that's why
%  electrons can't pass through each other

% Quantum Field Theory visualized
% https://www.youtube.com/watch?v=MmG2ah5Df4g

% Supersymmetry, explained visually
% https://www.youtube.com/watch?v=0GUTJQCeKBE

%===================================================================================================
\subsection{Theories of Everything}
At this time, we have no experimentally confirmed physical theory that explains all physical phenomena. Instead, the current situation is that we have not one theory of everything but rather two theories, each of which explains a certain range of phenomena. For the small scale from elementary particles over nuclei and atoms up to molecules, we have quantum field theory. For the large scale, ranging from solar systems over galaxies up to the whole universe, we have general relativity. For the intermediate scale of the world of our own limited human experiences, ranging roughly from bacteria over humans and mountains up to our planet, we can get away with various approximations of both (at least mostly - GPS actually needs general relativity to function properly). Classical Newtonian mechanics can be obtained as an asymptotically correct approximation of quantum systems when the number of particles gets huge [TODO: clarify - it's not about the "number" per se...it's more about the "wavelengths" of the combined systems...I think]. It can also be obtained as an asymptotically correct approximation of general relativity, when the spacetime gets flat. Both conditions apply to the world of our daily experiences. Both of these theories predict the actual experimental observations within their realm of applicability with impressive accuracy. Yet, they cannot both be correct at the same time because they are mathematically incompatible, so to speak. the great challenge of modern physics is to come up with one single coherent mathematical theory that models all phenomena. There are a couple of candidates....TBC...

\subsubsection{Supersymmetry}

% math: group theory ...I guess

% https://en.wikipedia.org/wiki/Superspace
% https://en.wikipedia.org/wiki/Supersymmetry

% Supersymmetry, explained visually
% https://www.youtube.com/watch?v=0GUTJQCeKBE

% The laws of nature are invariant with respect to certain transformations (spatial or temporal 
% translation, rotation,swicthing from one frame of reference to another). These classical 
% symmetries are the Poincare symmetries
%
% -Coleman-mandula theorem: the Poincare symmetries plus the symmetries of the quark-field (3 
%  interchangable (?) versions of each quark), charged particle field (complex phase doesn't matter) 
%  and  (...some more quantum field symmetries...)  are the only possible ones. But there's a 
%  loophole: they assumed that the fields are descirbed by ordinary numbers. When Grassmann numbers
%  are allowed, the theorem does not apply
% -SuSy is though to be the only possible extensin to our current model
% -Space is described as pair of two complemantary spaces - one for numbers, one for Grassmann 
%  numbers
%
% SuSy is a proposed new set of symmetries which has not been observed. Speculates about the existence of new types of particles not ever observed, so far. Each fermion and boson has a partner particle in the other category

% hypothetical symmetry between particles of matter and interaction. may explain dark matter

% ..I think, it has been experimentally disproved already ?

% -SuSy is used in Superstring theory. ..string theory with strings that have a supersymmetry along
%  their surface
% -Its also a model for supergravity - predicts a superpartner for the graviton

\subsubsection{String Theory}

\paragraph{M Theory}

% Quantum Loop Gravity

%each of which is very well confirmed by experiments

% Chemists have their periodic table. For a couple of decades, group theorists have their "periodic table", too

% Maybe make a section about the principle of least action, refering to the Hossenfelder video.

\begin{comment}
List branchs of physics and which areas of math are especially important in those:
-Classical Mechanics:
 -Differential Equations (ordinary and partial)
-Lagrangian and Hamiltonian Mechanics:
 -Calculus of Variations
-Electrodynamics:
 -Vector Calculus
-Fluid Dynamics: 
 -Vector calculus
 -Complex Analysis (Rimeann mapping, conformal mapping)
-Continuum Mechanics:
 -Tensor Calculus
-Quantum Mechanics:
 -Linear Algebra
  -Unitary and Hermitian matrices
  -Eigenvalues and -vectors
 -Group Theory:
  -Matrix Lie-groups (Pauli, Dirac, etc.)
 -Operator Theory
-Relativity:
 -Tensor calculus
 -Differential geometry
 -Group theory (Lorentz group, Poincare group, etc.)
-Thermodynamics / Statistical Mechanics: Probability theory (verify!)
-Cosmology: 
 -Differential geometry
 -Information theory (verify!), 
 -Topology (verify!)
-String Theory / M-Theory:
 -Manifolds (diff. geo.)
 -Knot Theory (?)
 -Number Theory
 
 
-Data Science:
 -Multivariable Calculus, Optimization
 -Tensor Algebra
 
 Lagrange Mechanik verstehen! - Lagrange Funktion, Euler-Lagrange Gleichung (Physik)
https://www.youtube.com/watch?v=Uk24tKZUq1M&list=PLdTL21qNWp2YiZaBF9xMb82kSpBc3YnxQ&index=71 
 
----------------------------------------------------------------------------------------------------
General Relativity:

Famous quote by physicist John Archibald Wheeler about the essence of general relativity:

"Spacetime tells matter how to move, matter tells spacetime how to curve."

Analogy in Newton's world:
Forces tell masses how to move (accelerate): F = m a, a = F/m. Masses tell (gravitational) forces how to pull: F_g = G m_1 m_2 r / |r|^3. Maybe Newton's law should be rephrased in terms of momentum to make the analogy even closer: d p / d t = F

In relativity: The change of the energy-mmomentum tensor is some function of the metric tensor - in a sloppy symbolic way: T_{\mu \nu} = f( g_{\mu \nu} ) and likewise g_{\mu \nu} = f^{-1} (T_{\mu \nu}). But actually the function $f$ also depend on spatial and temporal derivatives of the (tensor valued) quantity $g$, not just on $g$ itself - although, if we assume $g = g(t,x,y,z)$ then that function actually contains all the info abouts its (partial) derivatives anyway. But usually, when we write down a differential equation, we make such dependencies on derivatives epxlicit by writing things like: $f(g, g_t, g_x, g_y, g_z, g_{tt}, g_{tx}, g_{ty}, \ldots)$

https://en.wikipedia.org/wiki/Einstein_field_equations


https://www.youtube.com/watch?v=KW4yBSV4U38
at 7:50, she says, "they are differential equations for the metric tensor". So, the focus seems to be on the time evolution of g, not T. The equations by themselves describe both on equal footing (I think)

 
 

\end{comment}