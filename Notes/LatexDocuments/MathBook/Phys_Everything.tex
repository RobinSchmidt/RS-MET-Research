
\section{Theories of Everything}
At this time, we have no experimentally confirmed physical theory that explains all physical phenomena. Instead, the current situation is that we have not one theory of everything but rather two theories, each of which explains a certain range of phenomena. For the small scale from elementary particles over nuclei and atoms up to molecules, we have quantum field theory. For the large scale, ranging from solar systems over galaxies up to the whole universe, we have general relativity. For the intermediate scale of the world of our own limited human experiences, ranging roughly from bacteria over humans and mountains up to our planet, we can get away with various approximations of both (at least mostly - GPS actually needs general relativity to function properly). Classical Newtonian mechanics can be obtained as an asymptotically correct approximation of quantum systems when the number of particles gets huge [TODO: clarify - it's not about the "number" per se...it's more about the "wavelengths" of the combined systems...I think]. It can also be obtained as an asymptotically correct approximation of general relativity, when the spacetime gets flat. Both conditions apply to the world of our daily experiences. Both of these theories predict the actual experimental observations within their realm of applicability with impressive accuracy. Yet, they cannot both be correct at the same time because they are mathematically incompatible, so to speak. the great challenge of modern physics is to come up with one single coherent mathematical theory that models all phenomena. There are a couple of candidates. They are all speculative and some of them are already at least partially refuted. They are nevertheless interesting enough to take a look at them. ...TBC...


%===================================================================================================
%\subsection{Supersymmetry}

% math: group theory ...I guess

% https://en.wikipedia.org/wiki/Superspace
% https://en.wikipedia.org/wiki/Supersymmetry

% Supersymmetry, explained visually
% https://www.youtube.com/watch?v=0GUTJQCeKBE



% The laws of nature are invariant with respect to certain transformations (spatial or temporal 
% translation, rotation,swicthing from one frame of reference to another). These classical 
% symmetries are the Poincare symmetries
%
% -Coleman-mandula theorem: the Poincare symmetries plus the symmetries of the quark-field (3 
%  interchangable (?) versions of each quark), charged particle field (complex phase doesn't matter) 
%  and  (...some more quantum field symmetries...)  are the only possible ones. But there's a 
%  loophole: they assumed that the fields are descirbed by ordinary numbers. When Grassmann numbers
%  are allowed, the theorem does not apply
% -SuSy is though to be the only possible extensin to our current model
% -Space is described as pair of two complemantary spaces - one for numbers, one for Grassmann 
%  numbers
%
% SuSy is a proposed new set of symmetries which has not been observed. Speculates about the existence of new types of particles not ever observed, so far. Each fermion and boson has a partner particle in the other category

% hypothetical symmetry between particles of matter and interaction. may explain dark matter

% ..I think, it has been experimentally disproved already ?

% -SuSy is used in Superstring theory. ..string theory with strings that have a supersymmetry along
%  their surface
% -Its also a model for supergravity - predicts a superpartner for the graviton


%===================================================================================================
%\subsection{String Theory}

%\paragraph{M Theory}

% Quantum Loop Gravity

%each of which is very well confirmed by experiments

% Chemists have their periodic table. For a couple of decades, group theorists have their "periodic table", too

% Maybe make a section about the principle of least action, refering to the Hossenfelder video.