%===================================================================================================
\section{Quantum Mechanics}
Quantum mechanics deals with the world of atomic and subatomic particles. It turns out that their behavior is not adequately described by classical mechanics. Classical mechanics describes the behavior of point like particles in a deterministic way. This is not how such small particles behave. Quantum mechanics describes the behavior of spatially smeared out particles in a probabilistic way. The main tool is the so called \emph{quantum wavefunction}. This is a complex valued (!) function of space and time and typically denoted by the greek letter psi as $\psi = \psi(t,x,y,z)$. There is a law called the \emph{Born rule} that allows us to convert the complex values of the wavefunction into probabilities of observing the particle in a particular \emph{state}, e.g. at a particular position. The Schroedinger equation is a partial differential equation that determines the time evolution of the wavefunction $\psi$. And "determines" is to be taken literally here - the time evolution of the wavefunction is indeed entirely deterministic. It is just the \emph{observation} that is probabilistic, i.e. the probabilistic aspect kicks in only at the point of making an observation, i.e. at the moment when we "apply" the Born rule. There are some very deep philosophical questions with regard to what constitutes such a moment, but I won't get into these here (search for "measurement problem" if you want to read more about this). The "quantum" in quantum mechanics comes from the fact that the states in which a quantum system can be observed do not lie on a continuum. In quantum mechanics, any given physical system has a finite or countably infinite set of states that it can be found in. The observable states do not lie on a continuum as in classical mechanics but are discrete or \emph{quantized}. The main mathematical toolkit for quantum mechanics is the machinery of linear algebra - but it will often be the infinite dimensional version of it, i.e. the theory of linear operators that we have met in the section about functional analysis. Our \emph{state vectors} will often be continuous functions and our matrices will be operators that act on on these functions. 

%That, together with the fact that we must deal with complex valued vectors or functions and that there is a very special notation in use, makes it all a bit intimidating at first - but essentially it's mostly just linear algebra.
...TBC...

% We will see a lot of Hermitian and unitary matrices or operators. The unitary one dicatate the time evolution whereas the Hermitian ones occur in measurements


% https://en.wikipedia.org/wiki/Introduction_to_quantum_mechanics

%===================================================================================================
\subsection{Notation and Terminology}
On first encounter, quantum mechanics can be very intimidating. They will talk about superpositions of eigenfunctions of self-adjoint operators in infinite dimensional complex Hilbert spaces of square-integrable functions and they will use a very peculiar notation with lots of greek letters and adornments (hats, tildes, bars, asterisks) for all of that. Don't let that scare you off! Essentially, it all boils down to linear algebra. Granted, we'll have to use vector spaces over complex numbers and these vector spaces will sometimes be infinite dimensional (i.e. spaces of continuous functions). It's linear algebra nonetheless. ...TBC...

% operating on sets over the complex numbers

%Although the vector spaces in quantum mechanics are always complex and often infinite-dimensional, the mathematical framework is for the most part just linear algebra (in the infinite dimensional case operator algebra). Quantum mechanics does, however, use its own peculiar notation. ...TBC...

%(n complex vector spaces has its own notation
% Hilbert space
% self-adjoint = Hermitian = complex analogon of symmetric
% unitary

% https://en.wikipedia.org/wiki/Separable_space

\paragraph{Bras and Kets}
%The form of the Schroedinger equation uses the notation for partial differential equations that we have introduced in the section about them using subscript $t$ for a derivative with respect to $t$. This looks different from what you will typically find in physics textbooks. There, the preferred notation for quantum mechanics is the so called bra-ket notation invented by Paul Dirac. In this notation, normal vectors are denoted as so called "ket" vectors

% write it in bra-ket notation

% https://en.wikipedia.org/wiki/Bra%E2%80%93ket_notation


%===================================================================================================
\subsection{Observables and Measurements}
% maybe place before Time Evolution


\subsubsection{The Born Rule}
% rename to measurements or observations - born rule could be a sub-point


% time independent schroedinger equation


%The do not behave in a deterministic way like classical particles do but rather in a probabilistic way. This behavior is captured in a mathematical object called the quantum wavefunction which is the primary tool to describe the particle

% they are also not point-like but rather spread out in space

% https://en.wikipedia.org/wiki/Wave_function

% https://en.wikipedia.org/wiki/Measurement_problem

% Schroedinger eqaution descirbes time evolution of psi
% Born rule describes how probabilities are derived from the wavefunction
% https://en.wikipedia.org/wiki/Born_rule

% The function can be observed only in one of the eigenstates. I think, we must form the scalar product of the state with the eigenstate to get the probability to find the system in that eigenstate?

% math: Theory of linear operators, probability, Hamiltonian is also important.
% wave-particle duality







%===================================================================================================
\subsection{Time Evolution}

\subsubsection{The Schroedinger Equation}
The  Schroedinger equation is to quantum mechanics what Newton's second law $F = m a$ is to Newtonian mechanics. It is the main governing equation that determines how the system will evolve over time. Written out in full, it can be stated as:
\begin{equation}
 \i \hbar \frac{\partial}{\partial t} \Psi(\mathbf{r}, t) =
 - \frac{\hbar^2}{2m} \left(
 \frac{\partial^2 \Psi(\mathbf{r}, t)}{\partial x^2} + 
 \frac{\partial^2 \Psi(\mathbf{r}, t)}{\partial y^2} + 
 \frac{\partial^2 \Psi(\mathbf{r}, t)}{\partial z^2}
 \right)
 +
 V(\mathbf{r}, t) \Psi(\mathbf{r}, t)
\end{equation}
which is quite a mouthful, but we will now dissect it and introduce some notation and then write it down in a much more convenient form. The main character in the equation is $\Psi(\mathbf{r}, t)$ which is called the wavefunction. It is a continuous function of the spatial position vector $\mathbf{r} = (x,y,z)$ and of the time instant $t$. We are dealing with a time varying scalar field. The somewhat unusual twist is that it is complex valued - but it's a scalar field nonetheless - so nothing too crazy, so far. The left hand side is, up to a factor which we could easily move to the right and side, the time derivative of that wavefunction. So the equation tells us, how the scalar field changes over time. The right hand side will allow us to calculate that temporal derivative, just like we are used to from the section about partial differential equations. It consists of two terms. The first term is the (scaled) sum of the second partial derivatives with respect to our three spatial coordinates. We recognize this as the Laplacian of the scalar field $\Psi$. The second term reads, with suppressed function arguments, as $V \Psi$. This is just a product between the wavefunction itself and some other given function. This function $V$ is the description of our concrete physical situation in terms of a potential field. When we bring the factor $\i \hbar$ over to the right hand side, use the Laplacian operator, suppress the function arguments and use the subscript notation for the derivative with respect to $t$, we obtain a form that is a lot prettier and easier to digest:
\begin{equation}
\Psi_t = \frac{\i \hbar}{2m} \; \Delta \Psi + \frac{V}{\i \hbar} \; \Psi
       = \left(\frac{\i \hbar}{2m} \Delta  + \frac{V}{\i \hbar} \right)  \Psi
\end{equation}
In the form in the middle, it really looks like the partial differential equations that we already met. Specifically, the term with the Laplacian closely resembles what we have seen in the diffusion equation. Just with the twist that the diffusion coefficient is imaginary. The form on the right is quite instructive as well. It has an operator that is applied to $\Psi$ and the result of that application gives us the time derivative $\Psi_t$. This is exactly the format in which we need the equation when we want to implement it in a numerical solver (like forward Euler) [VERIFY this form of the formulas - I derived them myself from the written out form above].

% a post where i explain why the delta function is eigenfunction of teh position operator
% https://www.facebook.com/MusicEngineer/posts/pfbid02F2LacsTjTVjhzXCV29XwyAsBQnb8gFUmUNczjVfbEfVZgHnpg2pPh3BSKTPNPdVFl

%\begin{equation}
% \Psi_{t} = \frac{1}{\i \hbar} \hat{H} \Psi
% \qquad \text{where} \qquad
% \hat{H} = \hat{T} + \hat{V} = 
%\end{equation}


%\begin{equation}
% \boldsymbol{\psi}_{t} = \frac{1}{\i \hbar} \mathbf{H} \boldsymbol{\psi}
% \qquad \text{where} \qquad
% \mathbf{H} = T + V = 
%\end{equation}
%Here I have used lowercase boldface for the wavefunction $\boldsymbol{\psi}$ to indicate that we think about it as a vector and uppercase boldface for $\mathbf{H}$ because we think about it as a linear operator (like a matrix, but more abstractly) that acts on a vector. This is an uncommon way to write it down. I like to put the time derivative unadorned to the left hand side such that the right hand side is an explicit prescription for "what comes next". This is how we would "implement" the equation in a computer program for a numerical solver and this is the way I think about things. The operator $\mathbf{H}$ in this equation is called the "Hamiltonian" and it encodes the energy of the system. Inside it, it will typically encapsulate some spatial derivatives of $\boldsymbol{\psi}$ such that the whole equation indeed follows our usual format of writing down a partial differential equation in space and time [VERIFY!].

...TBC...

% The function psi is a time varying, complex valued scalar field, i.e. when we plugin in a quadruple t,x,y,z, it produces a scalar. But: the whole function psi is seen as vector, i.e. an element of a space of functions. The value of psi at some particular spacetime point is a (complex) scalar. That is confusing

% How does the Hamlitonian "encode" the energy? I think, it's an operator which, when being applied to the wavefunction, computes the total energy of the system. But wouldn't that rather be a functional than an operator

% susskind eq 4.10 pg 103  and 119

% https://en.wikipedia.org/wiki/Schr%C3%B6dinger_equation
% https://en.wikipedia.org/wiki/Probability_current

% https://en.wikipedia.org/wiki/Schr%C3%B6dinger_equation#Changes_of_basis

% https://en.wikipedia.org/wiki/Schr%C3%B6dinger_equation#Time-dependent_equation
% https://en.wikipedia.org/wiki/Hamiltonian_(quantum_mechanics)

% maybe use vector notation for psi and matrix for H
% 

% have the matrices also a name? transition matrices? evolution matrcies ? tiem evolution operator?
% https://en.wikipedia.org/wiki/Time_evolution




% Alternatives to Schoedinger equation

% https://en.wikipedia.org/wiki/Matrix_mechanics
% " Its account of quantum jumps supplanted the Bohr model's electron orbits. It did so by interpreting the physical properties of particles as matrices that evolve in time. It is equivalent to the Schrödinger wave formulation of quantum mechanics"

% https://en.wikipedia.org/wiki/Heisenberg%27s_entryway_to_matrix_mechanics

% https://en.wikipedia.org/wiki/Path_integral_formulation
% "The Schrödinger equation is a diffusion equation with an imaginary diffusion constant, and the path integral is an analytic continuation of a method for summing up all possible random walks.[2]"


% Imaginäre Zahlen - Sind sie real?
% https://www.youtube.com/watch?v=zsZqhoAQdj4

%===================================================================================================
\subsection{Quantum Field Theory}

A \emph{field theory} is a theory in which physical objects like particles are described as (scalar-, vector- or tensor-) fields that somehow interact with one another. So, in a field theory, the "particle" is not modeled as a point like object but rather as a continuous function that formally extends over all space and time. When I say it extends only "formally" over all space, you should picture in your head something like a Gaussian bell curve which also formally extends from minus to plus infinity, but tends towards zero so quickly (namely, exponentially-squared) to both sides, that it becomes negligible everywhere except for some finite region around the peak. In a field theory for particles, that is how the typical fields that model the particles behave, too [VERIFY!].

\medskip
A \emph{quantum theory} is a theory in which the values of physical quantities cannot just be anything from a continuous range but rather values from a discrete set where discrete can mean either finite or countably infinite. Which particular value from this discrete set will be observed in a particular situation will be predicted by a quantum theory only in terms of probabilities, not in a surefire deterministic sense like classical theories (field or not) would do. ...

\medskip
A quantum field theory combines both of these aspects ...TBC...

% standard model of particle physics:
% -fermions(constitute matter): electrons, quarks, neutrinos,..
% -bosons (mediate interactions): photon, gluon, Z, W, Higgs, ..

% -Higgs-field is scalar field (spin 0), other boson-fields are vector fields (spin 1),
%  fermion-fields are spinor fields (spin 1/2).
% -Spinors are made from Grassmann numbers. Their multiplication is anticommutative -> Squares of
%  such "numbers" are zero.
%  https://en.wikipedia.org/wiki/Grassmann_number
%  -this is the Pauli exclusion principle - no two fermions can be in the same state - that's why
%  electrons can't pass through each other

% Quantum Field Theory visualized
% https://www.youtube.com/watch?v=MmG2ah5Df4g


\subsubsection{The Standard Model of Particle Physics}


\begin{comment}




https://en.wikipedia.org/wiki/Matrix_mechanics#Matrix_basics

"If an observable is measured and the result is a certain eigenvalue, the corresponding eigenvector is the state of the system immediately after the measurement. The act of measurement in matrix mechanics 'collapses' the state of the system. If one measures two observables simultaneously, the state of the system collapses to a common eigenvector of the two observables. Since most matrices don't have any eigenvectors in common, most observables can never be measured precisely at the same time. This is the uncertainty principle.

If two matrices share their eigenvectors, they can be simultaneously diagonalized. In the basis where they are both diagonal, it is clear that their product does not depend on their order because multiplication of diagonal matrices is just multiplication of numbers. The uncertainty principle, by contrast, is an expression of the fact that often two matrices A and B do not always commute, i.e., that AB − BA does not necessarily equal 0. "



Inner Products | Pivotal Math of Quantum Mechanics | Quantum Theory
https://www.youtube.com/watch?v=dLP6goswxFI
https://www.youtube.com/watch?v=fMi6NUA85uY&list=PLVvnOdsCze6HnYJfCE9v1PHq1FfYTX99X
-5 Axioms (general principles):
 (1) A system is described in terms of a state-vector

Bloch's Theorem Explained Simply #SoMEpi
https://www.youtube.com/watch?v=85bdcSgdXIc

I Solved Schrodinger Equation Numerically and Finally Understood Quantum Mechanics
https://www.youtube.com/watch?v=E8N1yskPN28
-Postulates:
-The state of a system is described a the wave function p(x,t)
-Observables are represented by Hermitian operators
-The values measured of a given observable can only be the eigenvalues of the associated operator
-The average value of a system in state p of an observable represented by operator A is given by:
 <A> = \int p^H A p
-The state p evolves according to the Schroedinger equation
-The wavefunction must be antisymmetric with respect to the interchange of all coordinates of
 one fermion with those of another. Electronic spin must be included in this set of coordinates


The math of how atomic nuclei stay together is surprisingly beautiful | Full movie #SoME2
https://www.youtube.com/watch?v=FL3ImtGcHqQ


A Brief Guide to Quantum Model of Atom | Quantum Numbers
https://www.youtube.com/watch?v=-4-TJMhGlX8


\end{comment}