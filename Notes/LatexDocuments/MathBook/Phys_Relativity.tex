\section{Relativity}

%===================================================================================================

% math: general: tensor calculus, differential geometry; special: hyperbolic numbers/geometry


% Explain sign conventions ("mostly minus metric", "mostly plus metric")
% https://en.wikipedia.org/wiki/Sign_convention
% https://physics.stackexchange.com/questions/172229/sign-convention-for-the-minkowski-metric-eta-mu-nu
% -explain how the mostly plus metric can be interpreted in terms of making the time component 
%  imaginary. We generally use: D = dx^2 + dy^2 + dz^2 - dt^2. But if we want to use the "normal" 
%  formula euclidean distance D = dx^2 + dy^2 + dz^2 + dt^2 the we can achieve this by replacing t 
%  with i*t. The squaring of the i that is attached to the time component will introduce the minus
%  sign.

%\subsection{Elementary Particle Physics}
% String theor, M-theory 
% math: group theory? symmetry? maybe even number theory? the zeta(-1) = -1/12 occurs somewhere in string theory, I think.
