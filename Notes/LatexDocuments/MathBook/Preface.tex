\chapter{Preface}
Mathematics is a broad subject and I would be way out of my depth to try to give a definition what it actually is. I like to think of it as the study of structures, patterns and equivalences and as a way to organize, systematize, formalize and eventually automate thought processes. One big theme is to figure out, under which circumstances one thing is equal to another. Often, there are multiple ways to compute the same thing and a big sub-theme of finding equivalences is to find computational shortcuts that allow to do a computation more efficiently than was previously possible. The body of mathematical knowledge today is an impressive tower of known facts about abstract constructs of the human mind. The first of these abstract constructs that one usually encounters in elementary school is the idea of a natural number and one learns how to add, subtract, multiply and divide them. Building on that, one later encounters negative numbers, rational numbers, real numbers and complex numbers. I like to think of numbers as the ground floor of the tower. Numbers usually set the stage for doing mathematics on first encounter, but the foundations of mathematics can go down to even lower levels: If numbers are the ground floor, then it may be appropriate to think of set theory as the first and mathematical logic as the second basement floor. Having numbers in place, one can go up one floor and look at functions that map numbers to other numbers. One then may realize that addition, multiplication, etc. are actually functions, too: they take two inputs and map them to a single output. Such generalization with hindsight is also a common theme in math. Another floor higher, you can look at mappings that map functions to other functions. And it goes ever higher up. Well, actually, it also kind of branches out while going up, so maybe a tree of knowledge might be better analogy than a tower - but then the roots (levels below ground) do not branch out as much as the upper levels. But there certainly is some sort of trunk that everybody needs to know. This trunk contains numbers themselves (from the natural to the complex ones), equations (and solution techniques for them when they contain unknowns), elementary functions (polynomial, rational, exponential, trigonometric and their inverses), linear algebra (vectors, matrices, linear systems of equations) and single variable calculus (derivatives, integrals, differential equations). With these basics in place, math branches out into various directions. Some of these are: multivariable calculus: the study of functions of several variables, abstract algebra: generalizes ideas such as addition, multiplication, etc. to other sorts of objects and identifies the common structure, number theory: studies natural numbers and especially prime numbers in depth, functional analysis: studies functions of functions, topology: studies qualitative properties of shapes, etc. 
% graph theory: studies connectivity in networks

% "One big theme is to figure out, under which circumstances one thing is equal to another"
% Another big theme is to extend/generalize definitions - for example exponents from natural to
% integer, rational, real and even complex numbers. Yet another big them is abstraction.

\subsection{Goals and Audience}
This document is an attempt to give a condensed high level overview about what's going on in a particular subject and to give a comprehensive catalog of formulas, recipes and algorithms to actually get the work done. There is little regard to derivation or justification and no regard whatsoever to proof or mathematical rigor. It's meant to be a collection of recipes for the practitioner who needs to use math in applications. In focus are the questions: What is it? What can I do with it? How can I do it? The focus is deliberately not: Why does it work the way it does? That would fill volumes (and has done so) and thereby just distract from getting the work done. The scope is broad but shallow. I don't want to drown the reader in details of derivations. If you are looking for a detailed, in depth understanding, you will need to consult actual math textbooks. This book here should serve more as a launchpad and give you the right keywords to search for.  Despite being shallow with regard to derivations (and therefore understanding), I strive to be comprehensive with regard to listing potentially important facts, formulas and algorithms. It's more like a cheat-sheet rather than a textbook. I try to minimize using forward references, i.e. references to material that is only covered in later chapters. But in some cases, these are inavoidable. The body of math knowledge just isn't a linear bottom-to-top sequence. It's more like a tree but actually even more complicated than that. Past the bottom layers, it's more like a vast interconnected network where everything hangs together, so putting it in a strict linear order from bottom to top is difficult. I will try nonetheless. Whenever a section is marked with a star symbol $\star$, it means that this section may contain references to ideas that are introduced only later in the book and can be skipped on a first reading without losing the general flow. The material is organized into 4 parts devoted to continuous mathematics, discrete mathematics, structural mathematics and applications of mathematics. 



%The material is organized as follows: Part 1 deals with basics, linear algebra, calculus and geometry - roughly speaking, the world of continuous mathematics. Part 2 is devoted to discrete mathematics... Part 3 is devoted to applications.... tbc...

% My motivations to write this are: I tend to forget things easily when I don't use them on a regular basis - and this applies in particular to math knowledge. So the book serves as a personal notebook about things that I have once learned, so I can look them up later. A memory extension, so to speak. It also helps me to organize the pile of facts for myself. I found during writing that having to think about how to organize the chapters really helps me to structure the knowledge in my own head. By laying out book's structure, I have to think about how the all the different math things (of which there are a lot) fit together. Also, trying to explain things really helps to understand them better.


\subsection{About the Author}
I'm professionally a programmer of audio and music software and the most fun part of the activities in this field is for me the fact that I can put mathematics and algorithms to use for creative and artistic purposes. The intersection of math and art is generally a very fascinating topic for me. My special focus is of course sound, i.e. digital signal processing in one dimension, but I also find the math involved in visual art interesting which involves topics like computer graphics and image processing. I graduated in 2007 from the Technical University Berlin with a master\footnote{The actual official term is "Magister Artium" which was how this degree was called before the German university system switched to master and bachelor.} degree in communication science with focus on audio signal processing as major subject with technical acoustics and computer science as minor subjects. Over time, it became a habit of mine to learn math from books and YouTube videos and I tend to get dragged into various rabbit holes. There is so much to learn and it so easy to forget all these things again when one doesn't use them on a regular basis. This book here shall also serve as a sort of written down digest of everything that I have learned over this time, so I have a place to look it up when I forget it - which invariably \emph{will} happen. This self description should also be taken as a disclaimer: I'm not a professional mathematician - just a layman with some interest in math. The book may contain mistakes. I always only explain things at my current level of understanding and I may have gotten some things wrong. This is not an authoritative textbook.

% If you find mistakes you can ...insert link to github research repo here - correction can be proposed in the issues tracker

%I dug deeper and deeper into the rabbit hole


\subsection{Notation and Terminology}
Just like any other domain of science, art or profession, mathematics uses its own, domain specific language. This special language includes a vocabulary of words as well as a lot of special characters and symbols. Unfortunately, in some cases, there is no universal consensus in the mathematics community about what the basic definitions and meaning of certain symbols should be. Some authors may use one definition of a certain term while other authors use another. Every author sticks to their definitions - often without mentioning the existence of other definitions. This can make it really confusing to bring together knowledge from various sources. In such cases, I will try to facilitate the comprehension of these sources by clearly stating alternative definitions, where they exist. For the book, I will pick those definitions and notations that I personally prefer or which are most common but I will try to point it out and raise a figurative warning flag when other alternative definitions exist.

% Babylonische Sprachverwirrung?

% ToDo: make an overview over the notation used, make an "About the author section"

% Bad names are uninformative, really bad names are misleading
% ex of 1: Young's modulus (elasticity), relativity theory (spacetime theory)

% ex. of 2: ring (algebra), cyclic coordinates,


% Der Mathematik EISBERG – Wie tief geht er?
% https://www.youtube.com/watch?v=vYzicsuQeQk

% Mathematicians Can't Agree Even On Basics
% https://www.youtube.com/watch?v=rBz5k2T7p4A