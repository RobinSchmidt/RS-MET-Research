\chapter{Randomness}
To model certain processes in the real world, we often need to model their inherent randomness. Historically, the mathematical ideas behind these modeling approaches arose in the context of analyzing are gambling games such as rolling dice. The goal was to develop winning strategies\footnote{Yeah - it was about making money, I guess.} for such games that involve randomness. Today, the applications of this mathematical subfield are vast and include data science, social sciences, information theory (data compression, error detection/correction, etc.), statistical physics, signal processing, time series prediction (finance, etc.) and more. The mathematics of randomness includes the subfields of probability theory and statistics where the former provides the theoretical underpinnings of the latter and the latter can be seen as the practical application of the former. In German, this branch of mathematics is sometimes called "Stochastik" (literally: "stochastics") but such an umbrella term isn't common in English, so I had to resort to the somewhat clunky phrase "mathematics of randomness". ...TBC...
%The


% Applications: 
% -analysis of gambling games (lottery, coin flips, rolling dice)
% -social sciences
% -information theory, coding, data compression, error detection/correction
% -statistical physics
% -data science, machine learning
% -signal processing (noise reduction, etc)
% -time series prediction (stock markets, etc.)

\begin{comment}
-In German, we use the term "stochastics" as umbrella term for the mathematics of randomness. The
 field includes probability theory and statistics

\end{comment}