\chapter{Randomness}
To model certain processes in the real world, we often need to model their inherent randomness. Historically, the mathematical ideas behind these modeling approaches arose in the context of analyzing are gambling games such as rolling dice. The goal was to develop winning strategies\footnote{Yeah - it was about making money, I guess.} for such games that involve randomness. Today, the applications of this mathematical subfield are vast and include data science, social sciences, information theory (data compression, error detection/correction, etc.), statistical physics, signal processing, time series prediction (finance, etc.) and more. The mathematics of randomness includes the subfields of probability theory and statistics where the former provides the theoretical underpinnings of the latter and the latter can be seen as the practical application of the former to real world data. In German, this branch of mathematics is sometimes called "Stochastik" (literally: "stochastics") but such an umbrella term isn't common in English, so I had to resort to the somewhat clunky phrase "mathematics of randomness". Or - well - screw that. I'll call it stochastics from now on. I consider the field of stochastics as a sort of inner-mathematical application of mathematics. In some way, it is a subfield of mathematics in and of itself but it is also a good example of the application of lower level mathematical ideas to an area of practical relevance. Some of the lower level ideas that will be featured prominently are set theory, combinatorics and calculus - mostly the integration side of it. To make it all airtight in a formal sense, one needs to use measure theory and the Lebesgue integral in certain contexts - for example, if you would ask about the probability of hitting a rational number when picking a number randomly from some interval of real numbers. But don't worry too much about that. The intuition of the Riemann integral is usually perfectly fine if one keeps some minor caveats in the back of the head. We'll first look into probability theory, then into statistics and then into some concrete applications from science, engineering and management.

% ...and maybe art? for example computer graphics - terrain generation, plant/organism generatioon
% discuss (pseudo) random number generators

% https://en.wikipedia.org/wiki/Operations_research
% https://en.wikipedia.org/wiki/Management_science
% https://en.wikipedia.org/wiki/Production_planning



\begin{comment}
-In German, we use the term "stochastics" as umbrella term for the mathematics of randomness. The
 field includes probability theory and statistics

ToDo:
- Random Processes (Markov chains, etc.)

\end{comment}