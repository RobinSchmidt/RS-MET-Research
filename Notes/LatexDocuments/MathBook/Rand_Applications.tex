\section{Applications}
The mathematics developed so far has a lot of practical use cases. We need it to analyze data that we have collected from the real world. The goal is usually to discover underlying patterns in order to make predictions or just to understand what is going on. There are also situations in which we want to synthesize artificial random data according to certain specifications like a given probability distribution. Such data can be used for artistic purposes or for driving algorithms that incorporate some sort of randomness to achieve some higher goal like optimizing some process or computing some quantity. The results may be suboptimal or inaccurate but will with high probability be close to the desired optimum or value. Such randomized algorithms are generally known as "Monte Carlo methods" alluding to a place that is famous for its casinos.

% https://en.wikipedia.org/wiki/Monte_Carlo_method


%===================================================================================================
\subsection{Data Analysis}



%===================================================================================================
\subsection{Random Generators}
%Analyzing data collected form the real world is one 
%There are many applications in which we need some means to produce some artificial random data

%---------------------------------------------------------------------------------------------------
\subsubsection{Pseudo Random Number Generators}
% https://en.wikipedia.org/wiki/List_of_random_number_generators

\paragraph{Linear Congruential Method}

\paragraph{Linear Feedback Shift Registers}
% https://en.wikipedia.org/wiki/Linear-feedback_shift_register

\paragraph{Mersenne Twister}
% https://en.wikipedia.org/wiki/Mersenne_Twister

\paragraph{Quality Measures}

% -Cryptographic security, period length, statistical independence

%cryptographically secure pseudorandom number generators

\paragraph{Arbitrary Distributions}
% mention how to approximate the Gaussian distribution with piecewise polynomials (see codebase)



\paragraph{Cellular Automata}
% Fur - Turing's method - reaction-diffusion




%---------------------------------------------------------------------------------------------------
\subsubsection{Artistic Uses of Randomness}

\paragraph{Noise Generators for Sound Synthesis}

\paragraph{Surface Texture and Terrain in Graphics}

\paragraph{Plants, Animals and other Organisms}




%===================================================================================================
\subsection{Monte Carlo Methods}


%===================================================================================================
\subsection{Information Theory}

% - explain entropy of a source/sender

%---------------------------------------------------------------------------------------------------
\subsubsection{Data Transmission}
% sender - channel - receiver, the channel may introduce errors

\paragraph{Error Detection}

\paragraph{Error Correction}

%---------------------------------------------------------------------------------------------------
\subsubsection{Data Compression}
% - mention zip, flac, png, etc.

\paragraph{Huffman Encoding}

\paragraph{Run-Length Encoding}






\begin{comment}

- Monte Carlo Methods

\end{comment}