\section{Probability}


\subsection{Probability Spaces}
A probability space consists of 3 ingredients: (1) A set $\Omega$, called the \emph{sample space}, that represents the set of all possible outcomes of some sort of random experiment. The "random experiment" in this context could be, for example, something tossing a coin or rolling a die. In the former case, we would have $\Omega = \{heads, tails\}$ and in the latter case $\Omega = \{1,2,3,4,5,6\}$. (2) A set $\mathcal{F}$ of events that could occur in such an experiment. For example: "the die lands on a 3" or "the die lands on an even number" or "the die lands on a number less than 5", etc. This so called \emph{event space} is a particular\footnote{To be precise, the set of subsets must be a $\sigma$-algebra (as defined in measure theory), but that's not particularly important here} set of subsets of $\Omega$. It's the set of all events that we want to assign a probability to. (3) A \emph{probability function} $P: \mathcal{F} \rightarrow [0,1]$ that assigns to each event $A \in \mathcal{F}$ a probability of occurrence. The function $P$ should satisfy the following conditions: An event that is certain should get a probability of one. Formally, that means $P(\Omega) = 1$. An event that is impossible should get a probability of zero. That translates to $P(\emptyset) = 0$.

...TBC...explain interpretation of the probability in the frequentist and Bayesian sense. Explain how probability zero means (almost) impossible and one means (almost) certain ...the "almost" part relevant only for infinite sample spaces, I think. I think $P$ should also satisfy additivity: for disjoint sets $A,B$, we should have $P(A \cup B) = P(A) + P(B)$. $P(\Omega) = 1$, $P(\emptyset) = 0$ ...verify!

% maybe use itemize or enumerate latex environment or somrthing
% discuss frequenties vs Bayesian interpretation of probabilities

%\paragraph{Remakr}


% https://en.wikipedia.org/wiki/Probability_space

% https://www.its.caltech.edu/~mshum/stats/lect1.pdf

% https://ethz.ch/content/dam/ethz/special-interest/mavt/dynamic-systems-n-control/idsc-dam/Lectures/Stochastic-Systems/Probability.pdf

% https://www.statlect.com/glossary/probability-space

\begin{comment}

Probability is just...really weird
https://www.youtube.com/watch?v=zczGnnM05TQ

\end{comment}

