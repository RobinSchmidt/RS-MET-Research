\section{Biology}

%===================================================================================================
\subsection{Morphogenesis}

%---------------------------------------------------------------------------------------------------
\subsubsection{Diffusion-Reaction Equations}

\begin{eqnarray}
u_{tt} &= D_u (u_{xx} + u_{yy}) + f(u,v)  \\
v_{tt} &= D_v (v_{xx} + v_{yy}) + g(u,v) 
\end{eqnarray}

% Alan Turing Lecture 2021 'Modelling Pattern Formation in Developmental Biology'
% https://www.youtube.com/watch?v=Rv9NKugal3g  
% has at 24 min examples for what f,g could be (Gierer-Meinhardt- , Thomas-, Schnakenberg- models)
% at 12 min, has equation for conditions for the parameters

% Morphogenesis
% https://www.youtube.com/watch?v=dMkvezo9t2c

% Cellular Automata

% Lindenmayer Systems

\begin{comment}

-Evolution
-Genetics
 -DNA matching
-Microbiology:
 -Protein structure (knot theory?)
-Ecology (Predator-prey model)
-Medicine
 -Diagnostics (Radon-Trafo)
 -Drug-Testing (statistics)
 -Epidemiology (SIR-model, ...)



Reaction-Diffusion Equations:

https://en.wikipedia.org/wiki/Turing_pattern
https://en.wikipedia.org/wiki/Reaction%E2%80%93diffusion_system
https://en.wikipedia.org/wiki/The_Chemical_Basis_of_Morphogenesis
https://www.youtube.com/watch?v=JLkCaBwRrVo   The Mathematical Code Hidden In Nature

https://www.youtube.com/watch?v=alH3yc6tX98   Can Math Explain How Animals Get Their Patterns?
  -> shows equations in the nicest form (without explaining them):
  da / dt = d_a * Lap(a) + F(a,b)
  db / dt = d_b * Lap(b) + G(a,b)
-I think, a is the concentration of the activator, b the concentration of the inhibitor, d_a and
 d_b their respective diffusion coeffs and F,G encode the reactions. Lap(..) means the Laplacian.
 The inhibitor spreads/diffuses faster than the activator. The initial condition is a random 
 distribution of both.
-What should go into F and G? Maybe F(a,b) = p*a - q*a*b, G(a,b) = r*a - s*a*b
 p,q,r,s are all positive, p controls how the activator reproduces itself, r controls how the
 activator produces inhibitor, q controls how the inhibitor inhibits reproduction of activator and
 s controls how the inhibitor inhibits its own reproduction (by reducing the amount of actiavtor)?
 Or maybe it should be  G(a,b) = r*a - s*b*b  or   G(a,b) = r*a - s*b  the last seems most plausible
 for the bunnies-and-foxes situation. Isn't that a bit similar to how I obtained the SIRP model from
 the SIR model - just the same idea applied to the Volterra-Lotka model?
-We could also write these equations in 2D as
 a_t = d_a (a_{xx} + a_{yy}) + F(a,b)
 b_t = d_b (b_{xx} + b_{yy}) + G(a,b)


https://www.dna.caltech.edu/courses/cs191/paperscs191/turing.pdf  Turing's original paper


https://en.wikipedia.org/wiki/Belousov%E2%80%93Zhabotinsky_reaction
https://en.wikipedia.org/wiki/Fisher%27s_equation

https://www.youtube.com/watch?v=Tpg-wfU_3qw  Anwendung - Zelluläre Automaten zur Simulation von Morphogenese

https://www.researchgate.net/publication/234144100_Morphogenesis_Origins_of_Patterns_and_Shapes

...should go to biology. there we can also insert the predator-prey equations, the SIRP model
for pandemics, lindenmayer systems, knot-theory for protein or DNA analysis, string-matching



\end{comment}