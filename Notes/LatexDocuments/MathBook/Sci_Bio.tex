\section{Biology}

%===================================================================================================
\subsection{Morphogenesis}
Biologists call the process by which nature lets living structures take their shape \emph{morphogenesis} which literally translates to "shape creation". 



% https://en.wikipedia.org/wiki/Morphogenesis

% Morphogenesis
% https://www.youtube.com/watch?v=dMkvezo9t2c

%---------------------------------------------------------------------------------------------------
\subsubsection{Diffusion-Reaction Equations} In 1952, British genius Alan Turing published a paper called "The Chemical Basis of Morphogenesis" in which he proposed a mathematical model which could potentially explain such processes of morphogenesis. The model is based on so called \emph{reaction-diffusion equations} which, in their simplest form, are a system of two partial differential equations (in more complicated variants, there can be more than two equations [VERIFY]). The model proposes that we should consider two time varying scalar fields. In 2D, that would be two functions $u(x,y,t), v(x,y,t)$ where $x,y$ are our two spatial coordinates and $t$ is time as always. These functions model the local concentration of two chemicals, in this context called \emph{morphogens}, where $u$ plays the role of an \emph{activator} and $v$ plays the role of an \emph{inhibitor}. The presence of the activator stimulates production of more activator and more inhibitor. The presence of the inhibitor inhibits the production of both substances. The exact way in which this activation and inhibition takes place is captured in two functions $f(u,v)$ and $g(u,v)$ respectively. The function $f(u,v)$ determines how much activator $u$ is produced as the result of the presence of some amount of activator $u$ and inhibitor $v$ and the function $g(u,v)$ determines how much inhibitor is produced. Both of these functions should be increasing functions of $u$ and decreasing functions of $v$ [VERIFY]. In addition to this production process which takes place at any point in the $xy$-plane, there is also a diffusion process that lets the substances tend to spread out spatially over time. This diffusion is captured in a diffusion term of the same form that we already know from the heat equation, i.e. a diffusion constant times the Laplacian. Mathematically, the model is a system of two coupled partial differential equations and can be written down as follows:
\begin{eqnarray}
u_t &= D_u (u_{xx} + u_{yy}) + f(u,v)  \\
v_t &= D_v (v_{xx} + v_{yy}) + g(u,v) 
\end{eqnarray}



% Alan Turing Lecture 2021 'Modelling Pattern Formation in Developmental Biology'
% https://www.youtube.com/watch?v=Rv9NKugal3g  
% has at 24 min examples for what f,g could be (Gierer-Meinhardt- , Thomas-, Schnakenberg- models)
% at 12 min, has equation for conditions for the parameters

% https://en.wikipedia.org/wiki/Turing_pattern
% https://en.wikipedia.org/wiki/Reaction%E2%80%93diffusion_system




% Cellular Automata

% Lindenmayer Systems

\begin{comment}

-Evolution
-Genetics
 -DNA matching
-Microbiology:
 -Protein structure (knot theory?)
-Ecology (Predator-prey model)
-Medicine
 -Diagnostics (Radon-Trafo)
 -Drug-Testing (statistics)
 -Epidemiology (SIR-model, ...)



Reaction-Diffusion Equations:

https://en.wikipedia.org/wiki/Turing_pattern
https://en.wikipedia.org/wiki/Reaction%E2%80%93diffusion_system
https://en.wikipedia.org/wiki/The_Chemical_Basis_of_Morphogenesis
https://www.youtube.com/watch?v=JLkCaBwRrVo   The Mathematical Code Hidden In Nature

https://www.youtube.com/watch?v=alH3yc6tX98   Can Math Explain How Animals Get Their Patterns?
  -> shows equations in the nicest form (without explaining them):
  da / dt = d_a * Lap(a) + F(a,b)
  db / dt = d_b * Lap(b) + G(a,b)
-I think, a is the concentration of the activator, b the concentration of the inhibitor, d_a and
 d_b their respective diffusion coeffs and F,G encode the reactions. Lap(..) means the Laplacian.
 The inhibitor spreads/diffuses faster than the activator. The initial condition is a random 
 distribution of both.
-What should go into F and G? Maybe F(a,b) = p*a - q*a*b, G(a,b) = r*a - s*a*b
 p,q,r,s are all positive, p controls how the activator reproduces itself, r controls how the
 activator produces inhibitor, q controls how the inhibitor inhibits reproduction of activator and
 s controls how the inhibitor inhibits its own reproduction (by reducing the amount of actiavtor)?
 Or maybe it should be  G(a,b) = r*a - s*b*b  or   G(a,b) = r*a - s*b  the last seems most plausible
 for the bunnies-and-foxes situation. Isn't that a bit similar to how I obtained the SIRP model from
 the SIR model - just the same idea applied to the Volterra-Lotka model?
-We could also write these equations in 2D as
 a_t = d_a (a_{xx} + a_{yy}) + F(a,b)
 b_t = d_b (b_{xx} + b_{yy}) + G(a,b)


https://www.dna.caltech.edu/courses/cs191/paperscs191/turing.pdf  Turing's original paper


https://en.wikipedia.org/wiki/Belousov%E2%80%93Zhabotinsky_reaction
https://en.wikipedia.org/wiki/Fisher%27s_equation

https://www.youtube.com/watch?v=Tpg-wfU_3qw  Anwendung - Zelluläre Automaten zur Simulation von Morphogenese

https://www.researchgate.net/publication/234144100_Morphogenesis_Origins_of_Patterns_and_Shapes


Can One Mathematical Model Explain All Patterns In Nature?
https://www.youtube.com/watch?v=F1hX_nzTlgU


How To Create Seamless Turing Patterns (Photoshop & Illustrator Tutorial)
https://www.youtube.com/watch?v=eKLki6-gynU

...should go to biology. there we can also insert the predator-prey equations, the SIRP model
for pandemics, lindenmayer systems, knot-theory for protein or DNA analysis, string-matching



\end{comment}