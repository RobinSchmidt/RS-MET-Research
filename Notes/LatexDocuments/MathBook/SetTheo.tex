\chapter{Set Theory}
[this is still very drafty and needs a lot of proofreading]

\section{Naive and Axiomatic Set Theory}



% ToDo: 
% -give Cantor'S original defition of a set
% -explain the naive/unrestricted comprehension axiom

\subsection{Unrestricted Comprehension Axiom [VERIFY!]}
In naive set theory, there is one simple and very general axiom that allows us to construct sets. Let $x$ be any kind of object and $P(x)$ a predicate, i.e. a function into which we can plug in $x$ and then $P$ will return a boolean value that tells us whether or not $x$ has some property. With $P$ in hand, we can build the set $S = \{x: P(x)\}$ of all objects that satisfy our predicate. That we can always do this, for any $P$ whatsoever, is called the \emph{unrestricted comprehension} axiom, sometimes also \emph{naive comprehension} axiom. 

\subsection{Paradoxes}
It turns out that this is a bit too liberal. With unrestricted comprehension, we can build paradoxical sets...TBC...
% Maybe rename to paradoxes
% mentiona the term antinomy



\subsubsection{Russel's Paradox}
Define the Russel set\footnote{I think, it's actually not a \emph{set} but a \emph{proper class}? Verify!} $R$ as the set of all sets that do not contain themselves as an element. That means $R = \{x: x \notin x\}$. That means $x \in R \Leftrightarrow x \notin x$, i.e. $x$ is an element of $R$ if and only if $x$ is not an element of $x$. Here, $x$ can be \emph{any} set - so for any set $x$ whatsoever, we can ask the question whether or not it is an element of $R$. Now ask what happens when we let $x = R$, i.e. we ask whether or not $R$ is an element of $R$. From $x \in R \Leftrightarrow x \notin x$ with $x = R$, we immediately get the contradiction $R \in R \Leftrightarrow R \notin R$. That is obviously nonsensical. So what went wrong?

\paragraph{Solution} Apparently, when allowing to construct sets in such a completely unrestricted way, we can create paradoxical situations. The culprit lies in the assumption that sets can contain themselves as elements in the first place. Maybe we should disallow situations where $x \in x$ altogether. That seems to be a weird and counterintuitive situation anyway. We should perhaps even disallow a bit more - things like $x \in y \wedge y \in x$ or $x \in y \wedge y \in z \wedge z \in x$, etc. - i.e. disallow all recursive element relations that loop back on themselves (of which $x \in x$ is the simplest one). Ruling out such looped recursive element inclusions will later be formalized as the axiom of foundation VERIFY!...TBC...maybe we need also the axiom of restricted comprehension and/or the axiom of pairing?

% Axiom schema of specification (or of separation, or of restricted comprehension)

%Now ask: Is $R$ itself an element of $R$ or not? There are two possible answers to this question: either $R \in R$ or $R \notin R$. Let's assume that $R \notin R$, i.e. R does not contain itself as element. Then, by definition of $R$, $R$ should be an element of $R$. If, on the other hand, we assume that $R \in R$, then $R$ should not be an element of $R$.

%Let $R$ be the set of all sets that does not contain itself as element. Now ask: "is $R$ and element of $R$ or not?

%...TBC...ToDo: explain how the paradox follows from unrestricted (naive) comprehension and how it can be fixed (I think, we need the axiom of foundation which (together with the axiom of pairing) forbids that sets can be elements of themselves)


% https://en.wikipedia.org/wiki/Zermelo%E2%80%93Fraenkel_set_theory#2._Axiom_of_regularity_(also_called_the_axiom_of_foundation)
 
% https://en.wikipedia.org/wiki/Axiom_schema_of_specification#Unrestricted_comprehension 
% https://aleph1.info/?call=Puc&permalink=mengenlehre1_1_1_Z10
% https://aleph1.info/?call=Puc&permalink=mengenlehre1_1_13

% Naive comprehension: if P(x) is a predicate, then we can build the set S = {x:  P(x)}, i.e. the set of all objects that sastify our predicate

% I think, we need the Fundierungsaxiom and need to disallow sets that can contain themselves
% https://de.wikipedia.org/wiki/Fundierungsaxiom
% Es gibt somit auch keine Menge, die sich selbst als Elemente enthält


% R = (...) - whatever in R is, R does not appear in its list of elements - by definition

% Maybe use also R-complement as the set of all sets that contain themselves

% https://de.wikipedia.org/wiki/Russellsche_Antinomie
% https://en.wikipedia.org/wiki/Russell%27s_paradox


% Note: the Russel set is actually not a set but a class, I think.

% https://en.wikipedia.org/wiki/Urelement


\subsubsection{Cantor's Paradox}
Calling Cantor's initial approach to set theory "naive" might be a bit unfair to him. He was, in fact, aware that his initial approach has some problems and came up with a paradoxon himself. It is actually a generalized version of Russel's paradox (VERIFY!)...TBC...
%We should not get the impression that Cantor himself was totally "naive"


\subsection{Systems of Axioms}
Set theoretic systems of axioms are like the atomic building blocks that we can use for the construction of arbitrary sets. An axiom may either directly state that some kind of set exists or it may tell us, how we can build new sets from existing sets. The initial freedom, chaos and anarchy of the unrestricted comprehension axiom which leads us to all these paradoxa is gone. We only have some well defined ways to build sets.

% -axioms describe
%  -what sets exist
%  -how to build new sets from existing sets - what is and what isn't allowed


\subsubsection{Zermelo, Fraenkel}
The most common system of axioms that mathematicians use today is the one proposed by Zermelo and Frankel, usually augmented with the so called axiom of choice. This system of axioms is usually called the ZFC system (for Zermelo, Fraenkel, Choice). ...TBC...

% https://en.wikipedia.org/wiki/Zermelo%E2%80%93Fraenkel_set_theory
% https://de.wikipedia.org/wiki/Zermelo-Fraenkel-Mengenlehre

% https://de.wikipedia.org/wiki/Zermelo-Mengenlehre

\paragraph{Axiom of Extensionality}
This axiom states that two sets are equal iff they contain the same elements. VERIFY
% It follows that the order doesn't matter and element duplication doesn't matter
% ToDo: explain the fancy name.

% https://en.wikipedia.org/wiki/Axiom_of_extensionality


\paragraph{Axiom of Pairing}
This axioms states that whenever we have two sets $A$ and $B$, there is a set $C$ whose elements are precisely $A$ and $B$. This set is given by $C = \{A, B\}$ and is it unique by virtue of the axiom of extensionality.

%Given any object A and any object B, there is a set C such that, given any object D, D is a member of C if and only if D is equal to A or D is equal to B (from wikipedia - I don't understand it).

% given two objects A and B, we can find a set C whose members are exactly A and B.  We call the set C the pair of A and B, and denote it {A,B}

% https://en.wikipedia.org/wiki/Axiom_of_pairing


\paragraph{Axiom of Regularity}
This axiom states that every non-empty set $A$ contains an element that is disjoint from $A$. It is also known as the axiom of foundation. ...TBC...

% The axiom of regularity together with the axiom of pairing implies that no set is an element of itself, and that there is no infinite sequence (an) such that ai+1 is an element of ai for all i. 

% https://en.wikipedia.org/wiki/Axiom_of_regularity


\paragraph{Axiom of Union}
The axiom states that for each set $A$ there is a set $B$ whose elements are precisely the elements of the elements of $A$.

% Imagine a set $A = \{ A_1, A_2, A_3, \ldots \} $ of sets. From every set $A_i$, pour all elements into the set $B$. The set $B$ is the union of all the sets $A_i$.


%https://en.wikipedia.org/wiki/Axiom_of_union


\paragraph{Axiom of power set}
% https://en.wikipedia.org/wiki/Axiom_of_power_set


\paragraph{Axiom of Infinity}
% https://en.wikipedia.org/wiki/Axiom_of_infinity


\paragraph{Axiom Schema of Specification}
% aka restricted comprehension

% https://en.wikipedia.org/wiki/Axiom_schema_of_specification


% The Weitz video on ZF explains it quite well. I think, the essence is that, in order to specify a set via a predicate, we first need to say, from which superset (or universal set) we form our set. A set builder notation $S = \{x: P(x)\}$ is not enough. We need to say something like  $S = \{x \in U: P(x)\}$ where $U$ is the universal set or superset from which we draw an x.

%A predictate like P(x) is not enough. We must say 

\paragraph{Axiom Schema of Replacement}
% https://en.wikipedia.org/wiki/Axiom_schema_of_replacement







\subsubsection{The Axiom of Choice}
The axiom of choice states that for every set of sets $A = \{ A_1, A_2, A_3, \ldots \}$, we can define a family of choice functions $f_i$ such that each $f_i$ picks an element from each of the sets $A_i$ and we can pour all these chosen elements into a new set.

% Plays a role similar to Euclid's 5th postulate?

%\paragraph{Axiom of Choice}
% https://en.wikipedia.org/wiki/Axiom_of_choice

%The Axiom of Choice | Epic Math Time
%https://www.youtube.com/watch?v=Nnt4hyJYfGA
%-Shows how axiom of choice is used to show that every surjection f has a right inverse g
% such that f(g(y)) = y for all y in Y without explicitly specifying what g does. The proof leaves
% the task of picking an x in X for each y in Y (such that g(y) = x, f(x) = y) to the reader. Take 
% as example f(x) = x^2, compare the proof to the simpler proof for the bijective f(x) = x^3

% The axiom of choice is equivalent to:
% https://en.wikipedia.org/wiki/Well-ordering_theorem
% https://en.wikipedia.org/wiki/Zorn%27s_lemma


\subsubsection{Neumann, Bernays, Gödel}

% https://en.wikipedia.org/wiki/Von_Neumann%E2%80%93Bernays%E2%80%93G%C3%B6del_set_theory
% https://de.wikipedia.org/wiki/Neumann-Bernays-G%C3%B6del-Mengenlehre


% https://en.wikipedia.org/wiki/Kripke%E2%80%93Platek_set_theory_with_urelements

\subsubsection{Axiom Candidates}
In this section, we will mention few other possible axioms that could be added to an existing system of axioms.

\paragraph{Axiom of Constructibility}

% https://en.wikipedia.org/wiki/Axiom_of_constructibility
% https://en.wikipedia.org/wiki/Constructible_universe



\section{Numbers as Sets}
In set theory, sets are the only thing that exists. Every mathematical object must somehow be viewed as some sort of set. To do all the cool math things that we love (or hate) so much, we obviously need numbers. So, we somehow need to build numbers from sets. Set theorists will talk about things like subsets of a number. This will at first sound totally nonsensical - what the heck is a subset of $10$ or $3/7$ or $\pi$ supposed to mean? But in set theory, numbers indeed \emph{are} sets (and therefore can have subsets) and we need to get used to this point of view.

%because we usually do not envision a number as a set. 
...TBC... 

\subsection{Ordinal Numbers}

% https://en.wikipedia.org/wiki/Ordinal_number

\subsection{Cardinal Numbers}
Cardinal numbers are special ordinal numbers. That means the set of cardinal numbers is a subset of the set of ordinal numbers. ...TBC...

% https://en.wikipedia.org/wiki/Cardinal_number
% https://en.wikipedia.org/wiki/Successor_cardinal
% https://en.wikipedia.org/wiki/Limit_cardinal
% https://en.wikipedia.org/wiki/Regular_cardinal
% https://en.wikipedia.org/wiki/Cofinality
% https://en.wikipedia.org/wiki/Cardinal_assignment

% https://en.wikipedia.org/wiki/Aleph_number
% https://en.wikipedia.org/wiki/Beth_number
% Limeskardinalzahl?

\subsubsection{Cardinal Arithmetic}
The cardinal numbers are a superset of the natural numbers. With natural numbers, we can do certain arithmetic operations, namely addition, multiplication and exponentiation. It is possible to extend these operations to the infinite cardinals...TBC...

% https://en.wikipedia.org/wiki/Cardinal_number#Cardinal_arithmetic


%\subsection{title}

\section{Theorems and Hypotheses}

\subsection{The Cantor-Bernstein Theorem}
To show that two sets $A$ and $B$ have the same cardinality, we must, by definition, show that a bijection between $A$ and $B$ exists. The Cantor-Bernstein theorem asserts that it suffices to show that and injection from $A$ to $B$ and an injection from $B$ to $A$ exists. If these two injections exist, then the theorem assures that also a bijection between $A$ and $B$ exists. (VERIFY) Constructing these two injections (or indirectly showing that they exist) can, in some situations, be easier than showing directly that a bijection exists, so the theorem may help us to show that two sets have the same cardinality. ...TBC...

% I think, the theorem can be derived by (VERIFY!):
% -If there exists an injective function f: A to B  then  |A| <= |B|
% -If there exists an injective function f: B to A  then  |B| <= |A|
% -If both exist, then |A| <= |B| and |B| <= |A| and therefore  |A| = |B|

\subsection{The Continuum Hypothesis}
We know that $|\mathbb{N}| < |\mathbb{R}|$. The cardinality of the real (i.e. continuous) numbers is strictly greater than the cardinality of the natural numbers. Cantor observed that any infinite subset of the real numbers that he could construct had either the same cardinality as $\mathbb{R}$ itself or the same cardinality as $\mathbb{N}$, so he hypothesized that no cardinality in between these two exists (VERIFY!). This is called the \emph{continuum hypothesis} which is sometimes abbreviated as CH. We have already introduced the notation $\aleph_0 = |\mathbb{N}|$ for the cardinality of the natural numbers. The cardinality of the real numbers, i.e. the continuum, is denoted as $\mathfrak{c} = |\mathbb{R}|$ and we also already know that $|\mathbb{R}| = 2^{\aleph_0}$. With these notations, the continuum hypothesis can be written down as: $\aleph_1 = 2^{\aleph_0}$. The generalized CH, also called Cantor's aleph hypothesis, states: $\aleph_{\alpha+1} = 2^{\aleph_{\alpha}}$ for every ordinal number $\alpha$. ....TBC...

% Q: Does the index of a cardinal number have to be an ordinal number?
% Q: How many cardinal numbers are there? Countably many or uncountably many? Is this question related to the CH?


%or as $\mathfrak{c} = 2^{\aleph_0}$



%$|\mathbb{R}| = \mathfrak{c} = 2^{\aleph_0}$. CH: $\aleph_1 = 2^{\aleph_0}$?

% ToDo: introduce notation \mathfrak{c}

\subsubsection{Independence from ZFC}
The quest was now to figure out whether or not the continuum hypothesis is true or false. As usual in axiomatic set theory, in order to prove a hypothesis true or false, one has only the axioms and the logical inference rules to work with (along with, of course, all the lower level theorems that have already been proven). So, the question in more precise terms asks: can we prove the continuum hypothesis true or false, given the ZFC axioms (or any other set of axioms). Kurt Gödel proved in 1940 that from ZFC, it cannot be proved that the continuum hypothesis is false. Paul Cohen proved in 1963 that from ZFC, it cannot be proved that the continuum hypothesis is true. These two results together mean that the continuum hypothesis is actually \emph{independent} from the ZFC axioms. This is sometimes also phrased as being \emph{undecidable} in ZFC. ...TBC...

% https://en.wikipedia.org/wiki/Continuum_hypothesis

\subsubsection{The Mathematical Multiverse}
The undecidability of the CH from ZFC means that, within the system of ZFC, we can create at least two different mathematical universes - one in which the CH is true and one in which it is false. Both of these universes are equally valid in the sense that all the theorems that follow from ZFC alone, i.e. all the mathematical theorems that we use in our usual day-to-day applied math (and many more), are true in both of them VERIFY!...TBC...

% https://en.wikipedia.org/wiki/Von_Neumann_universe


% (Provably) Unprovable and Undisprovable... How??
% https://www.youtube.com/watch?v=1RRpC7FDfEQ&lc=UgwGg_JvrNpHhgO1Vo14AaABAg.A3Uanwul3oeA3_rQWkBsOx
% see my comment there and the replies to it

% https://en.wikipedia.org/wiki/List_of_statements_independent_of_ZFC

% https://en.wikipedia.org/wiki/List_of_statements_independent_of_ZFC#Measure_theory
% CH implies that there exists a function on the unit square whose iterated integrals are not equal. The function is simply the indicator function of an ordering of [0, 1] 

% ToDo:

%\subsubsection{The Generalized Continuum Hypothesis}


%  Cantor's aleph hypothesis

% https://en.wikipedia.org/wiki/Continuum_hypothesis#Generalized_continuum_hypothesis


\begin{comment}

ToDo:
-Infinite Sets 
 -Construction of natural, integer, rational and real numbers
 -Cardinal Numbers
 -Ordinal Numbers (as generalization of cardinal numbers)
 -Surreal Numbers
 -Cantor-Bernstein Theorem
 -Continuum Hypothesis
 -Notation: looks like in axiomatic set theory, it's more common to use lowercase letters like x,y for sets (rather than A,B, ...)
 -maybe use \varphi for predicates



Set Containment:
-If for all x in A, x in B, then A \subseteq B

Lebesgue-Measure:
 https://www.youtube.com/watch?v=0VD3BWDLmU0

...set theory sometimes appears like a "write-only-language

% See:
% https://aleph1.info/?call=Puc&permalink=mengenlehre1



Das Zermelo-Fraenkel-Axiomensystem der Mengenlehre (ZF)
https://www.youtube.com/watch?v=U10UYyXv5gM&list=PLb0zKSynM2PAuxxtMK1bxYPV_bUoPtpTB&index=1&t=0s

Bourbaki:
-Collective of authors writing a series of books in the early 1900s that aimed to build up all of math axiomatically. The first volume was about set theory

Was sind Kardinalzahlen? Was besagt die Kontinuumshypothese?
https://www.youtube.com/watch?v=qijXa3U4Nag&list=PLb0zKSynM2PCrgebQsfrzEsUIuA0I_wdG&index=1&t=0s




\end{comment}