\chapter{Axiomatic Set Theory}
[this chapter is still very sketchy and incomplete and needs a lot of proofreading]
% See also SetTheory.txt in the Scratch folder below the book folder


Set theory serves a couple of purposes in mathematics. It may serve as the foundational layer upon which all of other mathematics is built. One might argue that the real foundational layer is logic and set theory is already one layer above that, but it's usually still considered foundational. In the course of providing the foundations, set theory also defines the basic language in which many of the higher level mathematical constructs are expressed. As a mathematical field all by itself, it is concerned mostly with the investigation of the idea of infinity. The so called \emph{transfinite numbers} are "numbers" that go beyond our usual finite numbers. They are numbers in the sense that you can do the usual arithmetic operations, (i.e. addition, multiplication and exponentiation\footnote{As we will learn in the section about abstract algebra, subtraction and division as well as roots and logarithms do not really count as operations in their own right - they will be defined in terms so called inverse elements.}) with them. The most important types of transfinite numbers are the \emph{ordinal numbers} and the \emph{cardinal numbers}. The ordinals are an extension of naturals and the cardinals are a certain subset of the ordinals. In this wider context, the natural numbers are ordinal as well as cardinal numbers. The ordinals are used to put things in order (as the name suggests). The cardinals are used to quantify, how big a set is, i.e. how many elements it has (as the name doesn't suggest - instead, it's a weirdo reference to a catholic cleric). In mathematics, we encounter infinite sets such as $\mathbb{N}, \mathbb{Z}, \mathbb{Q}, \mathbb{R}, \mathbb{C}$, i.e. sets with infinitely many elements. Upon closer investigation, it turns out that some infinite sets are bigger than others\footnote{Mind = Blown} and cardinal numbers make that idea precise. Spoiler: $\mathbb{N}, \mathbb{Z}$ and $\mathbb{Q}$ have all the same \emph{cardinality}, that is: the same size. We call that size $\aleph_0$. That's the Hebrew letter "aleph" with subscript zero. It stands for a \emph{countable infinity}. But $\mathbb{R}$ and $\mathbb{C}$ are bigger than $\aleph_0$ while also being of equal cardinality when compared to one another. They have an \emph{uncountably infinite} size. This size is called the size of the continuum and often written as $\mathfrak{c}$. It might be tempting to assume that this is the next bigger infinity and therefore think that it might be appropriate to call it $\aleph_1$, but as we will see, it's not that simple. Ordinal numbers, on the other hand, make precise the idea of: what comes next after all ordering positions with finite index are exhausted. Imagine the infinitely long list of natural numbers written down as an infinitely long row. We could write it as an infinite tuple like $(0,1,2,3,4,\ldots)$. What comes after this row is exhausted? Well, it never can be exhausted - but what would come next if it could be exhausted? The first element of the next row, of course! In the context of ordinals, we call that first element $\omega$ and the whole second row, written in infinite tuple notation would look like $(\omega, \omega+1, \omega+2, \ldots)$. This chapter assumes familiarity with the material about naive set theory from the introductory chapter on page \pageref{Sec:NaiveSetTheory}.

%\medskip


% There are more types of transfinite numbers: Beth-numbers are those that arise form repeatedly taking power sets, starting with the set of naturals. The generalized continuum hypothesis states that the beth-numbers are equal to the aleph-numbers [VERIFY!]. 

% -The two most important concepts/definitions in set theory are: cardinality, well ordering
% -From a set-theoretical point of view, the most important number systems in math are the naturals
%  and the reals. The smallest uncountable ordinal omega_1 is also important
% 

%===================================================================================================
\section{Naive and Axiomatic Set Theory}

% ToDo: 
% -give Cantor's original defition of a set

\subsection{Cantor's Definition}
In the introductory chapter of this book, we already encountered Georg Cantor's following definition of a set: "A set is a gathering together into a whole of definite, distinct objects of our perception or of our thought which are called elements of the set.". As we noted there, a set is just a bunch of things. But what are these "things" then? In mathematics, these are typically mathematical objects like numbers or vectors or functions or operators etc.. But talking about those things requires that we already know what these things are. There are variants of set theory that postulate the existence of atomic elements, called \emph{urlements}, \emph{atoms} or \emph{individuals} and the elements of a set can then be either other sets or such urelements, which themselves have no further structure. In such a theory, numbers would be such urelements. But modern versions of set theory can get away without postulating the existence of urelements. They build up all mathematical objects purely from the notion of a set. No other sorts of things (such as urelements) are required. 

%When we want to build math really from the ground up, we cannot assume the existence of such things. We have to think about what a number even is in the first place. 

%[TODO: mention set theories with urelements/atoms/individuals]

% https://plato.stanford.edu/entries/set-theory/
% https://de.wikipedia.org/wiki/Naive_Mengenlehre
% Unter einer „Menge“ verstehen wir jede Zusammenfassung M von bestimmten wohlunterschiedenen Objekten m unserer Anschauung oder unseres Denkens (welche die „Elemente“ von M genannt werden) zu einem Ganzen.

% Mention "Urelement", "atomic element"?
% https://en.wikipedia.org/wiki/Urelement
%  urelements are in some sense dual to proper classes: urelements cannot have members whereas proper classes cannot be members. 

% Explain the Peano axioms as an example of just postulating ceratin things about natural numbers
% that are needed for basic arithmetic. Real numbers can also just be axiomatically postulated.
% Figure out if that axiom also has a name

\subsection{Intuitions}
So, let's take a step back and forget about the existence of numbers and ask ourselves, what kinds of objects a set could possibly contain. So far, we have only postulated the existence of objects that we want to call sets. So, with nothing else in hand, how about letting sets to contain other sets? Using some terminology from computer science, maybe we could build a sort of recursive data structure pretty much like a tree. A tree consists of nodes and each node can have child nodes. Let's apply the idea to sets: a set may consist of sets which we call \emph{elements}. The elements play the role of the child nodes. Let's further postulate that there also exists a set that has no elements whatsoever and let's call that special set the \emph{empty set}. Let's also assume that the identity of a set is determined by its content: if two sets have the same elements, they are the same set. There is no notion of order and it doesn't matter, if an element occurs more than once. It immediately follows that the empty set is unique. It doesn't make sense to talk about \emph{an} empty set. It really is \emph{the} empty set. So far, we have the following: (1) A set can contain sets as elements. (2) There is a special set without any elements. (3) The identity of a set is determined by its elements.

% https://www.youtube.com/watch?v=lNM6S2UUHcQ  at 25:12

%\medskip
%---------------------------------------------------------------------------------------------------
%\subsubsection{Sets of Sets}
\subsubsection{Construction of Sets}
So far, we know about one set: the empty set. It has no elements. It plays a fundamental role, so we shall give it a symbol. We'll denote it by $\emptyset$. Let's furthermore stipulate that we write sets as lists of elements inside curly braces. For example, a set that contains only the empty set would be denoted by $\{ \emptyset \}$. Let's also stipulate that we can give names to sets via the notation $A = \ldots$ where on the right side, some set in the curly brace notation can be assigned to the name $A$. Let's make an experiment: take $A = \emptyset$ and $B = \{ \emptyset \}$. That means, the set $A$ contains no elements whatsoever and the set $B$ has one single element, namely the empty set. We have said that two sets are equal if and only if they have the same elements. $A$ has no elements and $B$ has one element. We conclude that $A$ and $B$ are different. Hey! We now know already two sets which are indeed different from one another! Sets that contain only a single element do also have a special name. They are called \emph{singleton set}s or just \emph{singleton}s. Let's try to create some more sets from what we already have. How about $C = \{ B \} = \{ \{ \emptyset \} \}$. We can do that because we have said that inside the curly braces, we can write down a list of comma separated sets and $B$ is a indeed a set. But is it yet another set or is it one of those two that we already know? It's certainly not equal to $A$ since it contains an element, namely $B$, whereas $A$ contains no elements. Could it be equal to $B$? Well, $B$ contains $\emptyset$ and $C$ contains $B$. So, since $B \neq \emptyset$, we see that $B$ and $C$ have different contents, so they are different sets. We have produced yet another set that is different from those we had before. You can easily convince yourself that $D = \{ A, B\} = \{ \emptyset, \{\emptyset\} \}$ is yet another set. After all, it has two elements. Now it should be apparent that we can recursively build an infinite number of different sets. Just by putting a set into curly braces, i.e. turning a set $A$ into the singleton $\{A\}$ that contains $A$ and doing that recursively, we can already build an infinite number of different sets. But by putting several sets into one we have even many more possibilities to construct many different sets. We can build arbitrarily many distinguishable objects in various ways. It turns out that by structuring our way to construct sets, we can build arbitrarily complex structures that are rich enough to represent all the objects from our mathematical world. We have already seen one important way to structure our construction of sets. By defining $D = \{ A, B \}$, we have created a \emph{pair} $D$ from two given sets $A$ and $B$. As we will see, this pair creation is one of a few important mechanisms to build new sets from existing sets. By the way, if in this pair creation, we take both sets to be the same, we get something like $D = \{ A, A \} = \{ A \}$ because duplication doesn't matter, i.e. $ \{ A, A \} = \{ A \}$. Thus, with our pair creation operation, we can also build singletons as a special case.

% To create the set {A,B,C}, we could first create {A,B}, then {C} and then form the union. So, with pairing and union, we can recursively put sets with an arbitrary number of elements together

%---------------------------------------------------------------------------------------------------
%\medskip
\subsubsection{Relations between Sets}
We have informally stated that sets can have other sets as elements. Elements may also be called \emph{members}. The two terms are synonymous in the context of set theory \footnote{In object oriented programming, the term "member" has a different meaning, though. As we explain some things in terms of an object oriented implementation, some care has to be taken which meaning is currently meant.}. We will use the notation $x \in A$ to indicate that the set $x$ is an element of the set $A$. The "is element of" condition establishes a \emph{relation} between sets. The \emph{converse} of that relation could be called  "has element". We may use both depending on which one is more convenient in a given situation. If $x$ is not an element of $A$, we'll write $x \notin A$ which is yet another relation. The idea of a relation itself can be formalized in terms of sets. That seems cyclic and kinda is (I think), but for the time being, let's think about the "is element of" and "has element" relations in informal terms.  When implementing a data structure to represent sets\footnote{Some programming languages have a data type for sets. These usually refer to sets with urelements which are typically of another data type. Here, we really talk about sets of sets only.} in its set theoretical sense, we would expect the data structure to have a functionality that allows us to test whether or not one set is an element of another set. In object oriented languages, we could perhaps have a class \verb|Set| that has a member function \verb|hasElement(Set x)| that returns a boolean value - \verb|true|, if the set we are calling it on has the set \verb|x| as element and \verb|false| otherwise. That is basically all a set can do. Such a call could look like \verb|A.hasElement(x)| when we assume that \verb|A| is the variable name of the set that we want to ask if it has \verb|x| as element. Any other functionality must be built on top of that. But how would we implement \verb|hasElement|? Our supposed recursive data structure \verb|Set| could have a list of \verb|Set|s as data member\footnote{For technical reasons, it would probably be a list of pointers to sets because that's the way, recursive data structures are usually built. But that's an implementation detail that changes nothing about the conceptual idea. I also use the term "list" here in the Python sense as a general (dynamic) array. In C++, it would be a std::vector and not a linked list like std::list.}. Lets assume that list to be called \verb|elements|. With such an implementation, to test if the set \verb|x| is a member of \verb|A|, we would iterate through the list and for each item, compare it to \verb|x| and if we have a match, return \verb|true|. If we reach the end of the loop without finding \verb|x|, return \verb|false|. But: how would the required comparison operation be implemented? Again, recursion is the magic word. But we need one intermediate operation: the subset relation. We define that $B$ is a subset of $A$, iff all elements of $B$ are also elements of $A$. Let's assume that we would call it like \verb|A.hasSubset(B)|. That function would take a set $B$ as argument. It would iterate through all elements of $B$ and for each one, check, if it is present in $A$ via the \verb|hasElement| function. Let's denote the subset relation via the $\subseteq$ symbol such that $B \subseteq A$ means that $B$ is a subset of $A$. Having the subset relation in hand, we can define the equality relation for sets in terms of it as follows: $A = B$, iff $A \subseteq B$ and $B \subseteq A$. We could implement this as a member function that we call like \verb|A.equals(B)| or, more typically, as \verb|==| operator. Now we have come full circle in a mutual recursion implementation structure. We have defined the equality comparison that we need inside \verb|hasElement|. All in all, we are now equipped with 3 important relations between sets: "has element", "has subset" and "is equal to" and some less important derived relations like "is element of", "is not element of". I explain this stuff in terms of an actual implementation of a set data structure because I have actually implemented it that way myself to play around with these concepts. In particular, I was interested to really see Peano arithmetic "in action" by using nothing but sets. It works....

[TODO: introduce $\ni$ notation, explain strict subset relation and its implementation, give reference to the implementation (the code is here in the research repo but needs to be cleaned up)]

% the strict subset relation could be implemented via the regular subset relation and a test that checks if A has at least one element that is disjoint from the elements of B. Maybe let's call that operation hasDisjointElement

% ToDo: Maybe talk a bit about the different possible textual representations of sets like { 1,2,3 }, { 2, 3, 1 }, { 1, 2, 3, 1 }, ... - this is important for an implementation - equlatiy comparisons should not depend on the representation. Maybe link my implementation. Maybe append an "Implementation" section at the end. Maybe do this also for other math topics - like the algorithms for polynomial multiplication (convolution), etc. - make some accompanying material - Code for C++, Python, Sage, etc.

% https://en.wikipedia.org/wiki/Converse_relation

%---------------------------------------------------------------------------------------------------
\subsection{Notational Conventions}  In the context of predicate logic, variables are typically denoted by lowercase letters from the end of the alphabet such as $x,y,z$. In general mathematics, sets are usually denoted by uppercase letters from the beginning of the alphabet such as $A,B,C$ and set elements usually by lowercase $a,b,c$. In axiomatic set theory, sets are the only thing that exists - the elements of sets are also sets - and sets appear as variables in predicate logical formulas. So we have a conflict here - which of the notational conventions should we adopt? Here, I adopt the following convention: In contexts where a lot of predicate-logical formulas appear, I will use $x,y,z, \ldots$ because using the other convention tends to look ugly there. These contexts are typically the lower levels of the theory. In higher level contexts, where predicate logical formulas become less common, I'll prefer  $A,B,C, \ldots$ for sets when I want to emphasize their set nature and $a,b,c$ when I want to emphasize their element nature - so, the convention used depends on the role of the variable that I want to emphasize in a particular situation.

%---------------------------------------------------------------------------------------------------
\subsection{Unrestricted Comprehension Principle}
In naive set theory, there is one simple and very general principle that allows us to construct sets. Let $x$ be any kind of (mathematical) object and $\varphi(x)$ a predicate, i.e. a function into which we can plug in a variable $x$ and then $\varphi$ will return a boolean value that tells us whether or not $x$ has the property modeled by our predicate. With $\varphi$ in hand and using the set builder notation from the introductory chapter, we could build the set $S = \{x: \varphi(x)\}$ of all objects that satisfy our predicate. But that was what we did in "naive" theory.
%That we can always do this, for any $\varphi$ whatsoever, is called the \emph{unrestricted comprehension} axiom, sometimes also \emph{naive comprehension} axiom. 

% https://en.wikipedia.org/wiki/Universal_set
% https://en.wikipedia.org/wiki/Axiom_schema_of_specification#Unrestricted_comprehension
% https://de.wikipedia.org/wiki/Naive_Mengenlehre
% https://aleph1.info/?call=Puc&permalink=mengenlehre1_1_1_Z10
% In Weitz's video über Ordinalzahlen bei 26:45 kommt diese Definition aber wieder - diesmal für Klassen

%---------------------------------------------------------------------------------------------------
\subsection{Paradoxes}
For "axiomatic" set theory, it turns out that this unrestricted comprehension principle is a bit too liberal. With unrestricted comprehension, we can build paradoxical sets. This is obviously undesirable for a theory that aims to be the foundation of mathematics. These paradoxes are sometimes also called antinomies. Let's now have a look at some of the most famous ones.
% https://en.wikipedia.org/wiki/Antinomy

\subsubsection{Russel's Paradox}
Define the Russel set\footnote{I think, it's actually not a \emph{set} but a \emph{proper class}? Verify!} $R$ as the set of all sets that do not contain themselves as an element. That means $R = \{x: x \notin x\}$. That means $x \in R \Leftrightarrow x \notin x$, i.e. $x$ is an element of $R$ if and only if $x$ is not an element of $x$. Here, $x$ can be \emph{any} set - so for any set $x$ whatsoever, we can ask the question whether or not it is an element of $R$. Now ask what happens when we let $x = R$, i.e. we ask whether or not $R$ is an element of $R$. From $x \in R \Leftrightarrow x \notin x$ with $x = R$, we immediately get the contradiction $R \in R \Leftrightarrow R \notin R$. That is obviously nonsensical. So what went wrong?

% -compare the Russel-set to the set B that is constructed in the proof of Cantors' theorem. It's
%  the same construction

\paragraph{Solution} Apparently, when allowing to construct sets in such a completely unrestricted way, we can create paradoxical situations. The culprit lies in the assumption that sets can contain themselves as elements in the first place. Maybe we should disallow situations where $x \in x$ altogether. That seems to be a weird and counterintuitive situation anyway - why would anyone assume that a set could contain itself? That clearly immediately triggers an infinite regress\footnote{In itself, infinite regress might not be an insurmountable problem, though. In an implementation, we may actually be able to do it in terms of a graph where nodes represent sets and edges represent the is-element relation. It's no problem for a graph to have a node that has an edge to itself.}. Maybe we should disallow such situations. We should perhaps even disallow a bit more - things like $x \in y \wedge y \in x$ or $x \in y \wedge y \in z \wedge z \in x$, etc. - i.e. disallow all recursive element relations that are cyclic, i.e. loop back on themselves. Of these, $x \in x$ is just the simplest one but the more complicated ones do also seem weird. Away with them!

% I think, in terms of graph theory, we allow only (directed) acyclic graphs - verify!

% https://de.wikipedia.org/wiki/Russellsche_Antinomie
% https://en.wikipedia.org/wiki/Russell%27s_paradox


%\subsubsection{Cantor's Paradox}
%Calling Cantor's initial approach to set theory "naive" might be a bit unfair to him. He was, in fact, aware that his initial approach has some problems and came up with a paradoxon himself. It is actually a generalized version of Russel's paradox (VERIFY!)...TBC...

% https://aleph1.info/?call=Puc&permalink=mengenlehre1_1_13

% Nah, let's not metion Cantor's paradox here - it requires Cantor's theorem that the power set of a set is larger than the set itself - but this theorem actually needs a lot of set theoretic preliminaries. Or maybe mention it but say that we can understand it only later

% Empty sets and vacuous truths:
% Example: All odd numbers that are divisible by 8 are also divisible by 4. That's true - but not because every number that is divisible by 8 must have a factor of 4 but because there aren't any odd numbers that are divisible by 8 in the first place. If you make a statement about all elements of an empty set, then that statement is vacuously true.

% the empty set is a subset of {1,2,3} because every element in the empty set is also in {1,2,3}. there are no elements in the empty set - but that doesn't change that

%---------------------------------------------------------------------------------------------------
\subsection{Systems of Axioms}
Set theoretic systems of axioms are like the atomic building blocks that we can use for the construction of arbitrary sets. An axiom may either directly state that some kind of set exists or it may tell us, how we can build new sets from existing sets. The possible construction steps can then be applied recursively to build sets of arbitrary complexity. The initial freedom, chaos and anarchy of the unrestricted comprehension axiom which leads us to all these paradoxa will be gone. We will only have some well defined ways to build sets.

\subsubsection{Zermelo, Fraenkel}
The most common system of axioms that mathematicians use today is the one proposed by Zermelo and Fraenkel and it is abbreviated by ZF. It is usually augmented with the axiom of choice in which case it is abbreviated as ZFC. There are multiple equivalent ways to choose the set of axioms for ZF and some of them may be redundant in the sense that they contain axioms that are not strictly necessary, i.e. could be proved as theorems from some of the other axioms. If redundant axioms are present, then their purpose make the system more intuitive and convenient [VERIFY!].

%...TBC...

% https://en.wikipedia.org/wiki/Zermelo%E2%80%93Fraenkel_set_theory
% https://de.wikipedia.org/wiki/Zermelo-Fraenkel-Mengenlehre

% https://de.wikipedia.org/wiki/Zermelo-Mengenlehre

% https://de.wikipedia.org/wiki/Zermelo-Fraenkel-Mengenlehre#Die_Axiome_von_ZF_und_ZFC
% -This actually has an axiom for the empty set and the axiom for infinite sets is replaced by the axiom that there exists a set x such that when y \in x then also y \cup \{ y \} \in x
% -The axioms as stated there seem to make more intuitive sense than on the english wikipedia page

% Set Theory Part 2: The axioms of ZFC
% https://www.youtube.com/watch?v=m-Lv1Tf8D04&list=PLzr1oJDUNa9JInlHHH3xVH29Ovzw8YoM2&index=3

\paragraph{Axiom of Empty Set}
This axiom states that there is an empty set. We can write it formally as:
\begin{equation}
\exists x \forall y: \neg (y \in x)
\end{equation}
This is actually one of the redundant axioms and not explicitly listed in the English wikipedia article on ZFC but in the German article, it is. It is redundant because in any set theory which has the axiom schema of specification (which we do have in ZF) and axiomatically postulates existence of any set (as the axiom of infinity in ZF does), the existence of the empty set can be derived as a theorem and therefore doesn't need to be an axiom. The empty set is unique due to the axiom of extensionality. It is usually denoted by $\emptyset$ or, less often, by $\{ \}$.

% Maybe use the \notin symbol

% https://en.wikipedia.org/wiki/Axiom_of_empty_set
% I think the way of deriving the axiom of the empty set from the axiom of specification is to just use an unsatisfiable formula or just use the formula that x is not in A - then we woul have the set of all things that are in A and are not in A which is, of course, the empty set

\paragraph{Axiom of Extensionality}
This axiom states that two sets are equal iff they contain the same elements. Formally, this means that
\begin{equation}
\forall x \forall y \forall w: 
(w \in x \Leftrightarrow w \in y) \Rightarrow (x = y)
\end{equation}
From the axiom, it follows that the order of the elements in a set doesn't matter and element duplication also doesn't matter. [VERIFY! TODO: give the statement of the axiom without the equals sign, explain the fancy name - what does "extensionality" mean?]

% I think, "extension" means something like "content" or "extent"

% https://en.wikipedia.org/wiki/Axiom_of_extensionality

\paragraph{Axiom of Pairing}
This axiom states that whenever we have two sets $x$ and $y$, then there is a set $z$ whose elements are precisely $x$ and $y$. This set is given by $z = \{x, y\}$ and called the \emph{pair} of $x$ and $y$. We can write it formally as:
\begin{equation}
\forall x \forall y \exists z \forall w: (w \in z) \Leftrightarrow (w = x \vee w = y)
\end{equation}
The pair is unique by virtue of the axiom of extensionality. The axiom of pairing also allows us to build \emph{singletons} by simply letting $x = y$ such that  $z = \{x, y\} = \{x, x\} = \{x\}$. It also lets us build the \emph{ordered pair} $(x,y) = \{ \{x\}, \{x, y\} \}$ and from ordered pairs we can recursively build ordered $n$-tuples as $(x_1, x_2, \ldots, x_n) = ((x_1, x_2, \ldots, x_{n-1}), x_n)$. TODO: warn about potential confusion between ordered and unordered pairs when "pair" appears unqualified.

% In the standard formulation of the Zermelo–Fraenkel set theory, the axiom of pairing follows from the axiom schema of replacement applied to any given set with two or more elements, and thus it is sometimes omitted. 

% Explain that we use the term pair in the following text to mean the ordered pair

% https://en.wikipedia.org/wiki/Axiom_of_pairing

\paragraph{Axiom of Union}
The axiom states that for each set $x$ there is a set $y$ whose elements are precisely the elements of the elements of $x$. Formally, it can be written down as:
\begin{equation}
\forall x \exists y \forall w:
(w \in y \Leftrightarrow \exists u:(w \in u \wedge u \in x )  )
\end{equation}
Here, $x$ plays the role of the input set over whose elements we want to form the union, the union itself is $y$, $w$ is an element of this union and it is also an element of an element $u$ of $x$. Imagine a set $x = \{ x_1, x_2, x_3, \ldots \} $ of sets. From every set $x_i$, pour all elements into the set $y$. The set $y$ is the union of all the sets $x_i$. This big union over all elements of a set $x$ is also denoted as $\bigcup x$. TODO: explain how to construct a big intersection from union + specification

% https://en.wikipedia.org/wiki/Axiom_of_union
% https://en.wikipedia.org/wiki/Axiom_of_union#Relation_to_Intersection
% https://math-garden.com/unit/nst-unions/
% https://en.wikipedia.org/wiki/Axiom_of_union#Relation_to_Intersection

\paragraph{Axiom of Regularity}
This axiom states that every non-empty set $x$ contains an element that is disjoint from $x$. It is also known as the \emph{axiom of foundation}. ...TBC...

% The axiom of regularity together with the axiom of pairing implies that no set is an element of itself, and that there is no infinite sequence (an) such that ai+1 is an element of ai for all i. 

% https://en.wikipedia.org/wiki/Axiom_of_regularity

% https://en.wikipedia.org/wiki/Universal_set#Regularity_and_pairing
% -Regularity together with pairing frobids the existence of sets that contain themselves.

% I think, we need the Fundierungsaxiom and need to disallow sets that can contain themselves
% https://de.wikipedia.org/wiki/Fundierungsaxiom
% Es gibt somit auch keine Menge, die sich selbst als Elemente enthält

% https://en.wikipedia.org/wiki/Zermelo%E2%80%93Fraenkel_set_theory#2._Axiom_of_regularity_(also_called_the_axiom_of_foundation)

% -a set is called founded, if

% Das Zermelo-Fraenkel-Axiomensystem der Mengenlehre (ZF)
% https://www.youtube.com/watch?v=U10UYyXv5gM
% -it is not needed to avoid contradictions
% 

\paragraph{Axiom of Power Set}
This axiom asserts that we can construct the set of all subsets of a given set $x$. If we denote this set of all subsets of $x$ as $y$, we can write the axiom as:
\begin{equation}
\forall x \exists y \forall w: (w \in y \Leftrightarrow w \subseteq x)
\end{equation}
where we used the abbreviation $w \subseteq x$ for the formula $\forall u: (u \in w \Rightarrow u \in x)$. We interpret this as: For all sets $x$, there exists a set $y$ such that: when $w$ is an element of $y$ then $w$ is a subset of $x$. We denote the power set of a set $x$ as $\mathcal{P}(x)$ or as $2^x$. The reason for the latter notation is that for finite sets $x$, the number of elements of the power set is indeed given by $2$ to the power of the number of elements of $x$.

% https://en.wikipedia.org/wiki/Axiom_of_power_set


\paragraph{Axiom of Infinity}
The axiom of infinity asserts that there exists a set $I$ such that: (1) $I$ contains the empty set: $\emptyset \in I$ and (2) for every element $x \in I$, there exists another element $y \in I$ that is equal to $y = x \cup \{ x\}$. This so defined set $y$ is called the \emph{successor} of $x$.
\begin{equation}
\exists I: ( \emptyset\in I \wedge \forall x: (x \in I \Rightarrow (x \cup \{x\}) \in I ))
\end{equation}
This set must necessarily be infinite. To see this, start with the empty set and put it into $I$ to satisfy (1): $I = \{\emptyset\}$. But now our set $I$ violates (2). To satisfy it, we must also put $\{ \emptyset \} = \emptyset \cup \{ \emptyset \} $ into $I$, so now we have: $I = \{\emptyset, \{ \emptyset \} \}$. But now, it again violates (2) unless we also put $\{\emptyset,  \{ \emptyset \} \}$ into it, so: $I = \{\emptyset, \{ \emptyset \}, \{\emptyset,  \{ \emptyset \} \}  \}$ and so on. Every element we add to satisfy (2) once again for the previously added element will again break (2) for the newly added element. This will force us to add yet another element to satisfy (2) again and so on. For each element that we have added in construction step $n$, we will be forced to also add its successor in step $n+1$. Sets that are defined with such a rule are also called \emph{inductive sets}. The process of adding more and more elements never stops and thereby creates an infinite set. This infinite set gets the special symbol $\omega$ and plays an important role in the theory of ordinal numbers. 

% Maybe use w instead of I. It looks like omega which is a good fit

\medskip
Now this axiom may appear a little strange. While for the others, most readers will probably just happily nod along and think things like "yea...sure...of course...duh...", this axiom might be more controversial.
In mathematics, axioms are supposed to be noncontroversial things. Things that are so blatantly obviously true, that we are willing to accept them without requiring a proof. Now this axiom of infinity certainly does not fall into this "yes, of course - that's obviously true" category. Worse than not being obviously true, some people might even argue that it's actually obviously false. In fact, there is a school of thought in the philosophy of mathematics that rejects the axiom of infinity. This is aptly called \emph{finitism}. ...TBC...

% https://en.wikipedia.org/wiki/Axiom_of_infinity
% https://en.wikipedia.org/wiki/Finitism
% https://en.wikipedia.org/wiki/Primitive_recursive_arithmetic

\paragraph{Axiom Schema of Specification}
This axiom says that for any given predicate $\varphi(x)$ and any given set $A$, we can create the subset $S$ of $A$ that contains only those elements from $A$ which satisfy the predicate $\varphi$. This axiom lets us filter out elements from a given set $A$ by means of a formula of predicate logic. It's called an \emph{axiom schema} because it formally represents an infinite number of axioms - one for each imaginable formula $\varphi$. It can be formally stated as:
\begin{equation}
\forall A \exists S \forall x: (x \in S \Leftrightarrow (x \in A \wedge \varphi(x)) )
\end{equation}
The formula $\varphi$ specifies which elements we should pick from $A$. The axiom is also known as axiom (schema) of separation or axiom (schema) of restricted comprehension. If we compare it with the principle of unrestricted comprehension, we note that here, we can only select the objects from a given set $A$ whereas in unrestricted comprehension, there was some imagined underlying universe of "all things" to which the formula $\varphi$ could be applied. Now we restrict ourselves to apply it only to the elements of some given set $A$ that we already have in hand. Notation wise, we will abbreviate the set $S = \{ x : x \in A \wedge \varphi(x) \}$ as $S = \{ x \in A :  \varphi(x) \}$. ...TODO: use variable naming consistent with the axioms above

% aka restricted comprehension

% https://en.wikipedia.org/wiki/Axiom_schema_of_specification


% The Weitz video on ZF explains it quite well. I think, the essence is that, in order to specify a set via a predicate, we first need to say, from which superset (or universal set) we form our set. A set builder notation $S = \{x: P(x)\}$ is not enough. We need to say something like  $S = \{x \in U: P(x)\}$ where $U$ is the universal set or superset from which we draw an x.

%A predictate like P(x) is not enough. We must say 

% One important difference to unrestricted comprehension is that the predicate gets applied to objects that already are members of some set - the objects are not plucked out of thin air (i.e. a hypothesized universal set of all possible things) a

% the predicate can only depend on the object itself, not on what other objects are present in the set. it's context independent. An example for a context dependent predicate for a set of numbers could be "larger than the average" - such things are not possible. But it could perpahse depend on a fixed other set like: x from A belongs to B if its (not) in C - that would form the intersection of A and C

\paragraph{Axiom Schema of Replacement} The axiom schema of replacement says that the image of a set under any definable mapping is again a set. [...TBC...explain what "mapping" means in this context. It's a formal term. The formal statement of this axiom is complicated]

% https://en.wikipedia.org/wiki/Axiom_schema_of_replacement
% https://en.wikipedia.org/wiki/Functional_predicate

\subsubsection{The Axiom of Choice}
The Zermelo-Fraenkel (ZF) axioms listed so far are the backbone of modern set theory. There is one axiom that is often used together with the ZF axioms - the axiom of choice. When we add it to the ZF axioms, we call the resulting axiom system ZFC. The axiom of choice states that for every set of nonempty sets $x = \{ x_1, x_2, x_3, \ldots \}$, we can define a family of choice functions $f_i$ such that each $f_i$ picks an element from each of the sets $x_i$ and we can pour all these chosen elements into a new set. This sounds like a quite unremarkable observation. Probably many more people will nod along with this one rather than with the axiom of infinity. Nevertheless, it's the choice axiom that has a rather special status...TBC...

% Has proven to be independent from the other ZF axioms.
% Plays a role similar to Euclid's 5th postulate.

% The axiom of choice lets us select elements from a given family sets and then build new sets from these selected elements

% It is controversial because it'S non-constructive. it doesn't show us explicitly how to define such a choice function in a given situation. It also has some very strange and counterintuitive consequences such as the Banch-Tarski paradox


% https://en.wikipedia.org/wiki/Axiom_of_choice
% https://de.wikipedia.org/wiki/Auswahlaxiom
% https://www.spektrum.de/lexikon/mathematik/auswahlaxiom/390
% https://aleph1.info/?call=Puc&permalink=mengenlehre1_3_1_Z13

% Das (berühmt-berüchtigte) Auswahlaxiom
% https://www.youtube.com/watch?v=-qPpiHZ82rw
% 12:20 For every infinite set A, there's a bijection between A and A x A
% Maybe add to the Theorems section

%The Axiom of Choice | Epic Math Time
%https://www.youtube.com/watch?v=Nnt4hyJYfGA
%-Shows how axiom of choice is used to show that every surjection f has a right inverse g
% such that f(g(y)) = y for all y in Y without explicitly specifying what g does. The proof leaves
% the task of picking an x in X for each y in Y (such that g(y) = x, f(x) = y) to the reader. Take 
% as example f(x) = x^2, compare the proof to the simpler proof for the bijective f(x) = x^3

% The axiom of choice is equivalent to:
% https://en.wikipedia.org/wiki/Well-ordering_theorem
% https://en.wikipedia.org/wiki/Zorn%27s_lemma
% Jerry Bona made the joke: "The axiom of choice is obviously true, the well ordering principle obviously false and who can tell about Zorn's lemma?" 
% Tarski's theorem is also equivalent
% also: trichotomy for cardinalities of sets
% see: https://www.youtube.com/watch?v=-qPpiHZ82rw  at 23 min
% there are even more equivalent statements


% https://www.youtube.com/watch?v=5UJWIwKa8vk
% The Axiom of Choice: History, Intuition, and Conflict

% https://www.aleph1.info/Resource?call=Puc&permalink=mengenlehre1
% https://www.aleph1.info/Resource?method=get&obj=Pdf&name=ema22.pdf&pagestart=279&pageend=298

% Formulierung halb natürlich, halb prädikatenlogisch:
% https://link.springer.com/chapter/10.1007/978-3-662-68094-0_7


% https://en.wikipedia.org/wiki/Von_Neumann%E2%80%93Bernays%E2%80%93G%C3%B6del_set_theory
% https://de.wikipedia.org/wiki/Neumann-Bernays-G%C3%B6del-Mengenlehre
% https://en.wikipedia.org/wiki/Kripke%E2%80%93Platek_set_theory_with_urelements



\subsubsection{Other Axioms and Systems}
In this section, we will mention a couple of other possible axioms that could be added to an existing system of the axioms such as of ZF(C) or could be used to build new, alternative axiomatic systems. We'll also briefly look at some of these alternative systems.

\paragraph{Axiom of Constructibility} ...TBC...

% -Is a natural axiom but didn't make it into the standard canon
% -Says that the ordinal numbers form the core of the set theoretic universe 
% -it makes the continuum hypothesis provable
% -It may be written a V = L
% -It is incompatible with axioms of large cardinals

% https://en.wikipedia.org/wiki/Axiom_of_constructibility
% https://en.wikipedia.org/wiki/Constructible_universe

% https://de.wikipedia.org/wiki/Scottsches_Axiomensystem
% https://de.wikipedia.org/wiki/New_Foundations

%\paragraph{Axiom of Large Cardinals} 
% https://en.wikipedia.org/wiki/Large_cardinal
% -accepting the large cardinals axiom does not settle the continuum hypothesis (verify!)


%\paragraph{Axiom Schema of Adjunction}
%https://en.wikipedia.org/wiki/Axiom_of_adjunction


%\paragraph{Axiom Schema of Induction}
% https://en.wikipedia.org/wiki/Epsilon-induction

% https://en.wikipedia.org/wiki/Axiom_of_determinacy
% https://en.wikipedia.org/wiki/Axiom_of_countable_choice

\paragraph{The NBG System by Neumann, Bernays, Gödel} ...TBC...

% Treat also:

% https://en.wikipedia.org/wiki/General_set_theory
%  sufficient for all mathematics not requiring infinite sets, and is the weakest known set theory whose theorems include the Peano axioms.

% https://en.wikipedia.org/wiki/Kripke%E2%80%93Platek_set_theory

% https://en.wikipedia.org/wiki/Tarski%E2%80%93Grothendieck_set_theory

% https://de.wikipedia.org/wiki/Ackermann-Mengenlehre

% Questions: 
% -How does the empty set arise from the axioms?
% -How do we define intersection, complement and product of two sets?


% https://de.wikipedia.org/wiki/Scottsches_Axiomensystem
% https://en.wikipedia.org/wiki/New_Foundations
%---------------------------------------------------------------------------------------------------
\subsection{Terminology, Notation, Definitions, Conventions}
In the following we will define a couple of important set theoretical terms that we will need later. In set theory, variable names stand for sets as such but they may also stand for elements of other sets because the elements of sets are, of course, yet again sets. We will tend to use capital letters like $A,B,C,\ldots$ if we want to emphasize the set nature of an object and lowercase letters like $a,b,c,\ldots$ when we want to emphasize the element nature of an object. We may also use greek letters from the beginning of the alphabet like $\alpha,\beta,\gamma, \ldots$ for sets that represent ordinal numbers and greek letters like $\kappa,\lambda,\mu,\ldots$ for cardinal numbers.


% https://en.wikipedia.org/wiki/Urelement
% -Urelements a dual to proper classes: proper classes cannot *be* elements whereas urelements cannot *have* elements. urelements are minimal objects while proper classes are maximal objects by the membership relation (which, of course, is not an order relation, so this analogy is not to be taken literally).

% ToDo: eplxain usage of alpha, beta,...kappa,lambda for ordinal and cardinal numbers

%\paragraph{Hereditary Property}
%A set $A$ is said to have the hereditarily the property $\varphi(A)$, if

% https://en.wikipedia.org/wiki/Hereditary_property
% https://en.wikipedia.org/wiki/Hereditary_property#In_set_theory
% https://en.wikipedia.org/wiki/Hereditary_set

% Ah - no - I misunderstood it - the concept seems to be concerned with inheriting properties in a bottom-up way when building sets using the transitive closure. I first thought that means a property that the set itself and all its elemnts have...and their elements...and so on

%\paragraph{Relation Chains}
% I don't know if there's a term for this - I want to define the idea of using something like 
% 1 < 2 < 3 < 4  but with an arbitrary transitive relation - could be the subset relation, the element relation, etc.

% https://proofwiki.org/wiki/Definition:Chain_(Order_Theory)
% https://mathworld.wolfram.com/Chain.html
% https://en.wikipedia.org/wiki/Total_order#Chains



\paragraph{Classes}
Some collections of mathematical objects that are of interest turn out to be not constructible via the construction steps that our respective axiom system provides (typically ZFC). We want some freedom back to describe certain collections of objects. Basically, for some collections of objects, we want to allow ourselves to use the unrestricted comprehension principle from naive set theory again. To define a \emph{class} $C$, we write $C = \{x: \varphi(x)\}$ where $\varphi(x)$ is a predicate in the sense of predicate logic. Sometimes it's even just a property that we informally express in natural language [VERIFY!]. So, classes are a broader concept and allow for more liberal constructions. That means, we should be careful when defining classes in order to not run into paradoxes. All sets are classes but not all classes are sets. A class that is not a set is called a \emph{proper class}. For example, the collection of ordinal numbers is a proper class. The same is true for the collection of cardinal numbers. We will denote these two classes by $\mathsf{ON}$ and $\mathsf{CN}$ respectively. The class of all sets is denoted by $\mathsf{SET}$.

% Or maybe call the subsection "Other Things in Set Theory" and also lits Types, Categories, Urelements, etc

% explain Cesare Burali-Forti paradox

% https://en.wikipedia.org/wiki/Class_(set_theory)
% https://de.wikipedia.org/wiki/Klasse_(Mengenlehre)
% https://de.wikipedia.org/wiki/Klassenlogik

% https://en.wikipedia.org/wiki/Class_(set_theory)#Classes_in_formal_set_theories
% Class-function?


\paragraph{Chains of Inclusions}
To convey the fact that a set $A$ is a subset of another set $B$ which is, in turn, a subset of a third set $C$, we could write $A \subseteq B \wedge B \subseteq C$. But that is tedious, so we will instead write this as $A \subseteq B \subseteq C$. We call such a statement a chain of subset inclusions\footnote{I don't know, if that's an official term, though.}. Likewise, for $A \in B \wedge B \in C$ we may write $A \in B \in C$ and call that a chain of element inclusions. TODO: explain the term chain in the context of partial orders, Hasse diagrams, etc.

% -A chain is a subset of a poset that is totally ordered (aka lineraly ordered), i.e. each pair of
%  elements is comparable. That means, a chain is a linearly ordered set

% https://www.youtube.com/watch?v=szfsGJ_PGQ0
% talks about chains at around 18:40

\paragraph{Big Unions}
The union of two sets $A,B$ is written as $A \cup B$. If $A$ is a set of sets\footnote{Which it always is because modern set theory, all sets are sets of sets. I just wanted to emphasize it again.} such that $A = \{a_1, a_2, a_2, \ldots \}$, then we write the union $U = a_1 \cup a_2 \cup a_3 \cup \ldots$ over all the elements of A as $U = \bigcup_{a_i \in A} a_i$ or shorter $U = \bigcup_{a \in A} a$ or even shorter as $U = \bigcup A$. Analog definitions exist for other associative binary set operations such as the intersection $I = \bigcap A$ or the disjoint union $D = \bigsqcup A$ or even the product $P = \bigtimes A$. However, it's the union that turns out to be important. 

\paragraph{Hereditary Property}
In set theory, a \emph{hereditary property} is a property of a set that is inherited by all of its elements, by their elements and so on - recursively all the way down until we eventually hit the empty set. [VERIFY]

%n object that is inherited by all of its subobjects. The meaning of the term subobject depends on the context. In set theory, the subobjects of a set are taken to be its elements.

% Q: does it have to recursively hereditary?

% https://en.wikipedia.org/wiki/Hereditary_property
% https://mathworld.wolfram.com/HereditaryProperty.html

% examples of set properties that can be hereditary: being finite, countable, transitive, well-ordered



\paragraph{Transitivity}
A set $A$ is called \emph{transitive} when its element relation is transitive, i.e. $b \in a \in A \Rightarrow b \in A$. This is equivalent to the requirement that each element must also occur as subset: $a \in A \Rightarrow a \subseteq A$. Transitive sets have the following properties (1) A set $A$ is transitive if and only if the union over all its elements gives a subset of the original set: $A \subseteq \bigcup A$. (2) If $A$ is transitive, then $\bigcup A$ is transitive. ..TBC...

\paragraph{Transitive Closure}
The transitive closure of a set $A$, denoted as $\tc(A)$ is given by:
\begin{equation}
\label{Eq:TransitiveClosure}
\tc(A) = \{A, \; 
           \bigcup A, \; 
           \bigcup \bigcup A, \;
           \bigcup \bigcup \bigcup A, \;
           \ldots  \}
\end{equation}


%TODO: explain why the requirements are equivalent - they look very different

% I think, this makes the element relation qualify as strict order. For those, we require transitivity, irreflexivity and asymmetry. I think, the latter two do not need to be explictly required here because the element relation satisfies these anyway according to the axioms.

% I think, if a set is transitive and this also holds for all of its elements, it automatically holds for their elements too - and recursively all the way down. See the defintion of ordinal numbers. I think, it becomes a hereditary property

% $a \in A \wedge b \in a \Rightarrow b \in A$. 

%https://en.wikipedia.org/wiki/Transitive_set


%\paragraph{Transitive Closure}





% https://en.wikipedia.org/wiki/Transitive_set#Transitive_closure
% Is w the transitive closure of any finite natural number?
% https://en.wikipedia.org/wiki/Transitive_closure


\paragraph{Upper Bound}

% partial order, poset, maximal element

% https://en.wikipedia.org/wiki/Baire_space_(set_theory)
% Baire Space w^w, Cantor Space 2^w

% Special Subsets of the Power Set:
% -Topology
% -Sigma-Algebra



%---------------------------------------------------------------------------------------------------
\subsection{A Graph Theoretical Perspective}
ToDo: Explain how the world of sets can be visualized as a directed acyclic graph. 

%The nodes are the sets and the edges indicate the element relation: a directed edge from node A to node B means that the set represented by node A has the set represented by node B as element. So and arrow (directed edge) represents the "is-element-of" relation. One could use the other convention of representing the "has-element" relation by reversing the directions of all arrows.

% Explain the construction process by starting at "day" 0 with the empty set and on each follwong day, take all the sets that are currently available and form all sets that can be formed by having these as elements. This can also be formulated as iterated power set operation. It is not a priori obvious, that the previously created sets will also be members of the produced power set at each iteration - but by looking more closely at the proces, it turns out that they are (I think)

% Draw the (partial) graph of all finite sets. maybe up to layer 4 (which has 12 elements). See SetTheory.txt for the construction rule and Experiments.cpp for some code to compute the number of existing sets at each stage of the construction.

% Draw the (partial) graph of all Neumann naturals. Draw also a node for w (omega). It has infinitely many edges - one to each node in the graph of the Neumann naturals. Draw also the w+1 node

% Maybe it makes sense to adopt a new visualization convention for graphs: if we draw a blob around a number of nodes and give that blob a name, then that blob can also be seen as a node - i.e. a node can "contain" other nodes - which translates to: has edges to all nodes that it encircles.


% Set Theory Part 0: An informal introduction to sets - by cmacypre
% https://www.youtube.com/watch?v=c2mYWNShzTA&list=PLzr1oJDUNa9JInlHHH3xVH29Ovzw8YoM2



%===================================================================================================
\section{Order Theory}

Order theory is the part of set theory that is concerned with \emph{order relations}. Such order relations were already briefly introduced on page \pageref{Par:Orders}. Order relations formally model situations like "this is less than that" or "this comes before that". The order relations can be classified according various properties like being partial or total, strict or non-strict, etc. and the sets on which they operate can also be classified according to the existence of lower and upper bounds, existence of least and greatest elements, etc. All of that is the business of order theory and it also provides the foundation for the definition of the ordinal numbers. The central structure in order theory is that of a \emph{partially ordered set} which is abbreviated as \emph{poset}. Formally, a poset $P = (A, \leq)$ is an ordered pair of a set $A$, called the ground set of $P$, and \emph{partial order} relation $\leq$ defined on $A$. A partial order on a set $A$ is a relation on $A$ that is reflexive, antisymmetric and transitive, see page \pageref{Tab:RelationFeatures}. Partial orders can also be total orders\footnote{This is typical math-speak where, for example, groups are special semigroups, the whole set is a special kind of subset, etc. The words "partial", "semi", "sub", etc. just take away certain stricter requirements from a thing but meeting those requirements anyway does not disqualify the thing.}. Total orders are just very special partial orders, namely those for which every pair $a,b$ of elements is comparable. In some contexts, it's more convenient to work with strict orders, denoted by $<$. On any given poset with order $\leq$, it's always possible to define a corresponding strict order $<$ via $a < b \Leftrightarrow (a \leq b) \wedge (a \neq b)$ and we can also turn this around to get $\leq$ back via $a \leq b \Leftrightarrow (a < b) \vee (a = b)$. So, with this conversion prescription between strict and non-strict partial orders, we can equivalently define the poset in terms of a strict partial order as $P = (A, <)$. This is indeed more convenient for our purposes here, so we will do that. ...TBC...

%https://en.wikipedia.org/wiki/Order_theory
%https://en.wikipedia.org/wiki/Glossary_of_order_theory
% https://en.wikipedia.org/wiki/Partially_ordered_set

% I think, an example for partial order is the subset relation. We have {1,2} < {1,2,3}, 
% {2,3} < {1,2,3} but neither {1,2} < {2,3} nor {2,3} < {1,2}. We say that {1,2} and {2,3} are 
% incompatible

% In a poset, we can have "chains". These are subsets inside which every pair of elements is
% comparable. There are also antichains which are subsets where no pair of elements is comparable.

\subsection{Order Isomorphisms and Order Types}
An \emph{order isomorphism} between the two posets $(A, <_A)$ and $(B, <_B)$ is a bijective function $f: A \rightarrow B$ that preserves the ordering of the elements. That means, the bijection $f$ must satisfy: $x <_A y \Leftrightarrow f(x) <_B f(y)$ for all $x,y \in A$. In words: $x$ is less then $y$ in $A$ if and only if $f(x)$ is less than $f(y)$ in $B$. As an example, take $A=\{1,2,3,4\}$ with the natural order $1 <_A 2 <_A 3 <_A 4$ and the set $B=\{one,two,three,four\}$ with the lexicographical order $four <_B one <_B three <_B two$. We can indeed find an order isomorphism $f$ between the two posets, namely the map: $f: 1-four, 2-one, 3-three, 4-two$. It's worth to point out that finding an order isomorphism between two posets is "more difficult" than just finding any old bijection between the underlying sets. An order isomorphism has stricter requirements to qualify as such. Order isomorphisms have to respect the additional poset structures of their domain and codomain. If such an order isomorphism can be found between two posets, then these two posets are said to be \emph{order isomorphic} and they belong to the same \emph{order type}.

\medskip
In an informal sense, we can say that bijections partition the class of all sets into equivalence classes which can be interpreted as being "size types". Whenever two sets can be bijectively mapped onto each other, they belong into the same equivalence class. These equivalence classes are characterized by their \emph{cardinal number} or cardinality, i.e. size type. Order isomorphisms partition these rather broad equivalence classes further into different order types. They capture a finer structure on the class of all sets. Sets can have the same size type but different order types\footnote{I think, it would be linguistically much nicer if one would either consistently talk either about "size types" and "order types" or about "cardinalities" and "ordinalities" - however, the latter term is not a thing and size type is rarely used either. So we are stuck with the terms "cardinality" and "order type" which does not really convey the analogy between the two concepts.} - but not the other way around. Formally, talking about equivalence and order "relations" is not quite right because the underlying collections are not necessarily sets but classes - but informally, it's a good intuition to think about cardinalities as classifying sets according to their size type and orders as classifying sets according to their order type - where the latter classification is a refinement of the former. VERIFY

% cardinality = size type
% ordinallity = order type
% ordinal numbers are more complicated than cardinal numbers - that's why it's appropriate to introduce cardinals first. Introducing ordinals first would be a bit like introducing the rationals before the naturals

%there may be sets between ...TBC...

% cardinality and order type (ordinality)

% Example: the set A={1,2,3,4} with the order 1<2<3<4  and the set B={one,two,three,four} with the order
% four,one,three,two  with the bijection 1-four,2-one,3-three,2-four

% https://en.wikipedia.org/wiki/Order_type
% -the order type of the set of natural numbers is w

% https://en.wikipedia.org/wiki/Order_isomorphism
% -The class of partially ordered sets can be partitioned by it into equivalence classes, families of partially ordered sets that are all isomorphic to each other. These equivalence classes are called order types.

% https://en.wikipedia.org/wiki/Partially_ordered_set
% https://mathworld.wolfram.com/OrderType.html
% https://math.stackexchange.com/questions/49361/order-type-and-its-reverse

% -the order type of a set is the ordinal number assigned to it
% -so, it's analogous to the cardinality
% -an ordinal number represents two thinsg:
%  -the index in a well ordered set
%  -the order type of a well ordered set

% Order-isomorphism
% -bijection that preserves order relations? yeah, i think so
% -this is s strciter requirement than just requiring the existence of a bijection
% -therefore, ordinals allow a finer differentiation of the inifniteness of a set than cardinals (?)

% https://www.youtube.com/watch?v=As8rTENUOy0&list=PL2m0OzES6Uw9zK-F8BX8HuGq7HAx9KhQb&index=6

% https://risingentropy.com/the-order-type-of-the-rational-numbers/
% https://math.stackexchange.com/questions/463024/the-order-type-of-the-rationals

% -the number of all possible functions from a set of size n to itself is n^n
% -the number of bijective functions, i.e. the number of possible reorderings, is n!
% -what about the number of reorderings of w? It should be somwhere below w^w but above
%  w^n for any finite n? Or does this question make no sense because we need an aleph number for 
%  this?

% https://www.quora.com/Can-you-give-an-intuitive-explanation-of-what-an-order-type-is-in-set-theory-I-looked-at-the-Wikipedia-article-but-it-didnt-explain-very-well
% "Two orderings have the same order type if there is a way to change the names of the elements, to turn one ordering into another."

% Book Chapter - looks good - has examples:
% https://www.math.wustl.edu/~freiwald/ch8.pdf
%
% Here is the full book:
% https://www.math.wustl.edu/~freiwald/
% https://openscholarship.wustl.edu/books/20/

% The 1/3–2/3 Conjecture
% https://www.youtube.com/watch?v=pTQHAZG0r3M
% -linear extenstion of a partial order is a way to put the elements into a linear order in a way
%  that respects the original order of already comparable elements

\subsubsection{Well Ordered Sets} A well ordered set is a set with a total order defined on it and the additional property that every non-empty subset has a least element with respect to the given order. The natural numbers $\mathbb{N}$ with our usual $\leq$ relation are such a well ordered set. 

%Recall that given a set and an order relation $\leq$ on it, we can always define the corresponding strict order relation $<$ via $a < b \Leftrightarrow (a \leq b) \wedge \neg (a = b)$ and we can also turn this around to get $\leq$ back via $a \leq b \Leftrightarrow (a < b) \vee (a = b)$.

%The integers with the same usual $\leq$ relation are not well ordered because they don't have a least element. We can define a well ordering on the integers, though, by defining the following non-standard order: $0 < -1 < 1 < -2 < 2 < -3 < \ldots$.

...TBC...explain that there may be multiple elements that don't have a direct predecessor. 


% More proerties of well ordred sets
% -all proper initial segments have a successor
% -there are no infinite decreasing sequences - that also means, the set is founded (verify!)
% -every element except maybe the last has a unique successor
% -there is at most one element without successor - namely, the last element, if it exists
% -there is at least one element without a predecessor, namely the smallest element
%  -there can be arbitraily many more elements without predecessor, though
% -the successor of an element a is always uniquel determined. It is the unique minimum of all
%  elements that are greater than a
% -every subset with an upper bound contains a least upper bound


% example of a nonstandard well ordering of the naturals:
% (1) even numbers are always greater than odd numbers
%  -even and odd numbers are in their natural order
%  -it looks like: 1 < 3 < 5 < 7 < .....  < 0 < 2 < 4 < 6 < ...
%  -it has two elements without a direct predecessor: 1 and 0
%  -the ordinal number corresponding to that order is w + w = 2 * w
% (2) order numbers according to their greatest prime divisor and for numbers with equal greatest 
%     prime factors, use their natural order
%  -it looks like: 0 | 1 | 2,4,8,16,32,.. | 3,6,9,12,18,.. | 5,10,15,20,.. | 7,14,21,.. | 11,22,.. 
%  -I think, we get an order type of w*w
% (3) order numbers according to greatest divisor (not necessarily prime) - if they are the same,
%     use the second greatest divisor, etc.
%  -I think, we obtain an order type w^w?
% -I think, we can interpret "order types" as nested orderings? we order according ot one criterium,
%% if it's the same for two elments, ordre them according to the next criterium, etc.

% I think, the rationals can be well ordered by sorting by numerator and in case of equal numerator, look at the denominator. The reals may be well ordered  by ordering them by the decimal expansions lexicographically ...i guess.

% Other examples for well ordered sets:
% -N U \cup {-1}. when putting the -1 before 0 or somewhere between two naturals, we just get w as
%  order type. But if we put it after all positive numbers, we'll obtain w+1. That means 1+w = w but
%  w+1 is something new. We could also use: N U w = {N,w}. It has also order type w+1
% -N U {-1,-2,-3} with negatives after positives: w+3
% -N U {-1,-2,-3} with negatives before positives: 3+w = w
% -N with evens before odds: w*2
% -N x N, odered by 1st component first and by 2nd in case of equal 1st elements: w*w = w^2
% -N U NxN U NxNxN U... with elems from N 1st, then elems from NxN, then elems from NxNxN, etc.
%  I think, this would produce order type w^w...not sure, though. wite this set as \bigcup_{n \in N} N^n
% -But: isn't w^w also the set of all function f: w -> w and therefore uncountable? Or does the notation
%  w^w mean something different here? Like, maybe, ordinal exponentiation vs set exponentiation?
% -{...,-2,-1,0,+1,+2,...}  with 0 < 1< 2 < 3 <...< -1 < -2 < -3,...  ->   w*2
% -{x in Q: x = n + sum_{k=0}^m 1/(2^k)   n,m in N } = {0,0.5,0.75,...1,1.5,1.75,...2,2.5,2.75,...} with
%  nomral order on Q -> w^2
% -set of all possible functions from w to w  ->   w^w

% Some examples:
% https://math24.net/well-orders-page-2.html
% 



%https://en.wikipedia.org/wiki/Well-order
%https://de.wikipedia.org/wiki/Wohlordnung
% https://de.wikipedia.org/wiki/Fundierte_Menge

% About large countable ordinals:
% https://en.wikipedia.org/wiki/Ordinal_number#Some_%22large%22_countable_ordinals
% https://en.wikipedia.org/wiki/Large_countable_ordinal
% https://en.wikipedia.org/wiki/Nonrecursive_ordinal#The_Church%E2%80%93Kleene_ordinal_and_variants
% http://www.madore.org/~david/math/ordinal-zoo.pdf
% https://johncarlosbaez.wordpress.com/2016/06/29/large-countable-ordinals-part-1/
% https://johncarlosbaez.wordpress.com/2016/07/04/large-countable-ordinals-part-2/

%https://en.wikipedia.org/wiki/Well-ordering_principle
%https://en.wikipedia.org/wiki/Well-ordering_theorem

% https://www.youtube.com/watch?v=mgta9lVoF24&list=PL2m0OzES6Uw9zK-F8BX8HuGq7HAx9KhQb&index=5
% at 8:30

% https://www.youtube.com/watch?v=UxhFy4deLQA  Unendlich plus eins - Was sind Ordinalzahlen?

% How To Count Past Infinity
% https://www.youtube.com/watch?v=SrU9YDoXE88
% -7:21: "The order type of a set is just the first ordinal number not needed to label 
%  everything in the set in order" - what a beast of a sentence. The order type of the naturals
%  is omega - the first ordinal that has not been used in labeling the naturals
%  -for the naturals, cardinality and order type are the same


% https://en.wikipedia.org/wiki/Burali-Forti_paradox
% -constructing "the set of all ordinal numbers" leads to a contradiction

\subsubsection{The Well Ordering Theorem} 
The well ordering theorem states that any set $A$ can be well ordered. Now that seems to be a bold statement! Why should this be true? The proof of the theorem is very indirect and not constructive, i.e. it doesn't show you \emph{how} to well-order any given set. It's not easy to construct a well ordering for the reals, for example. The standard ordering doesn't work because not every subset of the reals has a least element. Just take the interval $(0,1]$ for example - it has no least element according to the standard order. Also, didn't we say at some point, that the complex numbers cannot be ordered? But this apparent contradiction can be resolved: What that statement means is that the complex numbers cannot be ordered in such a way to get an \emph{ordered field}. An ordered field is a formal term that requires an order relation that interacts in a certain nice way with the arithmetic operations. Of course, you can invent an order for the complex numbers, for example: Order according to real part and in case of equality, order according to the imaginary part. But such an ordering will not respect some features that we expect from an order on an ordered field.

\medskip
OK - with that potential cause of confusion out of the way - what does the theorem actually say? It says that it is \emph{somehow} always be possible to find a well ordering on any given set whatsoever. That somehow follows from the ZFC axioms but for certain sets we may not have the slightest clue for how such a well ordering may look like. ...TBC...

% I'll leave it at that

% such as $a < c, b < d \Rightarrow ab < cd$ and which are required for an ordered field.
% ..oh - no - that formula is wrong - it doesn'T work for negatives

% https://en.wikipedia.org/wiki/Ordered_field
% https://en.wikipedia.org/wiki/Well-ordering_theorem
% https://math.stackexchange.com/questions/6501/is-there-a-known-well-ordering-of-the-reals


\subsection{Transfinite Processes}
%...are processes that operate on well-ordered sets...I think

\subsubsection{Transfinite Induction}
In the chapter about logic, we have learned about the proof technique of induction. It is a technique to prove statements $\varphi(n)$ that should hold for any natural number $n$. The technique can be generalized to work not only over the set $\mathbb{N}$ but over any well ordered set - for example, over set of all ordinal or all cardinal numbers which we will encounter later. The generalized technique is called \emph{transfinite induction}. The basic technique that works over $\mathbb{N}$ had to distinguish two cases: (1) the base case and (2) the successor case. In transfinite induction, we need to consider a third case: (3) the limit case. It is called by that name because... ...TBC...

% https://en.wikipedia.org/wiki/Transfinite_induction
% https://de.wikipedia.org/wiki/Transfinite_Induktion
% https://mathworld.wolfram.com/TransfiniteInduction.html
% https://en.wikipedia.org/wiki/Mathematical_induction

% I think, transfinite induction can run over all well-ordered sets? ...verify!

\subsubsection{Transfinite Recursion}
Transfinite recursion is a method of constructing a sequence of objects - one for each ordinal number.

% Is it a way to construct more than countably many objects?

% https://en.wikipedia.org/wiki/Transfinite_induction#Transfinite_recursion
% https://math.stackexchange.com/questions/1934694/transfinite-recursion

% Example:
% https://en.wikipedia.org/wiki/Beth_number


% https://de.wikipedia.org/wiki/Transfinite_Induktion
% hat unten definition für ordinale addition

% Ordinal arithmetic
% https://jjsch.github.io/output/oa.pdf

% https://en.wikipedia.org/wiki/First_uncountable_ordinal


% https://jdh.hamkins.org/transfinite-recursion-as-a-fundamental-principle-in-set-theory/

% https://planetmath.org/transfiniterecursion



%===================================================================================================
\section{Numbers as Sets}
In set theory, sets are the only thing that exists. Every mathematical object must somehow be viewed as some sort of set. To do all the cool math things that we love (or hate) so much, we obviously need numbers. So, we somehow need to build numbers from sets. Set theorists will talk about things like subsets of a number. This will at first sound totally nonsensical - what the heck is a subset of $10$ or $3/7$ or $\pi$ supposed to mean? We are used to think of numbers as atomic entities. But in set theory, numbers indeed \emph{are} sets (and therefore can have subsets) and we need to get used to this point of view.

% 

%because we usually do not envision a number as a set. 

% https://de.wikipedia.org/wiki/Zermelo-Fraenkel-Mengenlehre#ZF_mit_Urelementen
% https://de.wikipedia.org/wiki/Urelement

% Crisis in the Foundation of Mathematics | Infinite Series
% https://www.youtube.com/watch?v=KTUVdXI2vng

%---------------------------------------------------------------------------------------------------
\subsection{Construction of the Number System}
In everyday math, we usually take numbers for granted and do not really think much about them. In set theory, all the number systems that we commonly use must first be \emph{constructed} using only sets. We'll show here briefly, how this could be done. For more details, see \cite{PDF_MathFAQ}. The construction of the natural numbers we will show here is due to John von Neumann. Therefore, these numbers are called \emph{von Neumann ordinals}. Actually, the von Neumann ordinals are a superset of the natural numbers that we construct here. This superset also includes "infinite" numbers - but for their construction, we'll need more tools. For the natural von Neumann numbers, I'll occasionally use the term \emph{Neumann natural}, although that is not an official term (I think.).

% namely: transfinite recursion

%number\footnote{This is not an official term, though. There is an official notion of the "von Neumann ordinals"}.

% https://en.wikipedia.org/wiki/Set-theoretic_definition_of_natural_numbers
% https://en.wikipedia.org/wiki/Ordinal_number#Von_Neumann_definition_of_ordinals

% https://cs.uwaterloo.ca/~alopez-o/math-faq/math-faq.pdf  pg 9 ff
% https://cs.uwaterloo.ca/~alopez-o/math-faq/node11.html

\subsubsection{Natural Numbers}
The most basic of all number systems is the system of natural numbers. We can construct the set of natural number from ZF as follows: Given any set $x$, we define the successor function as $s(x) = x \cup \{ x \}$. Note that to be able to do this, we had to use the axiom of union and the axiom of pairing (which implies the constructibility of singletons). Then we give the names $0,1,2,3,\ldots$ to the following sets:

\medskip
\begin{tabular}{l l l l l}
$0=$ &         & 
               & $\emptyset =$                                               
               & $ \{ \} $                                         \\
$1=$ & $s(0)=$ & $\emptyset \cup \{ \emptyset \} =$             
               & $ \{ \emptyset \} =$ 
               & $ \{ 0 \}$                                         \\
$2=$ & $s(1)=$ & $ \{ \emptyset \} \cup \{  \{ \emptyset \} \}=$ 
               & $ \{\emptyset,  \{ \emptyset \} \} =$       
               & $ \{ 0,  1 \} $                                    \\
$3=$ & $s(2)=$ & $\{\emptyset,  \{ \emptyset \} \} \cup \{  \{\emptyset,  \{ \emptyset \} \} \}=$ 
               & $ \{\emptyset,  \{ \emptyset \},  \{\emptyset,  \{ \emptyset \} \} \} =$       
               & $ \{ 0, 1, 2 \} $                                    \\   
\vdots \\
$n=$ & $s(n-1)=$  & & & $\{ 0, 1, 2, \ldots, n-1 \} $       
\end{tabular}
\medskip

[VERIFY! Try to format it better] That means we first define the number $0$ to be the empty set and starting from there, we use the successor function $s(x)$ repeatedly to define all other natural numbers. Numbers would be boring if we couldn't do computations with them. So, the next step is to construct our usual arithmetic operations from set theoretic operations.


\paragraph{Successor}
%The simplest operation that we want to define is the successor function $s(x)$. Here $s$ denotes a function or unary operation and $x$ denotes a set constructed like above. The table above has already shown by way of the first few examples, how we will define that function in general. 
As we have already mentioned we define: $s(x) = x \cup \{x\}$. The successor of a set $x$ is just that same set $x$ with $x$ itself added to it as new element. For example, we have $3 = s(2) = s(\{0,1\}) = \{0,1\} \cup \{2\} = \{0,1,2\}$. Instead of $s(x)$, we may also write $x+1$ when this is more convenient. So, we define:
\begin{equation}
x + 1 = s(x) = x \cup \{x\}
\end{equation}
With that construction, each natural number $x$ is just the set of all natural numbers that are less than $x$. Note that the universe of all possible sets that we can construct is much larger than those which we can construct with our successor function when starting with the empty set. For example, the set $\{\{\emptyset\}\}$ is not among them i.e. it is not a Neumann natural.

\paragraph{Order}
On our so constructed natural numbers, we can use the strict subset relation $\subset$ or the element relation $\in$ to model our usual less-than relation of the natural numbers. The subset relation works because a smaller number is indeed always represented by a strict subset of the set that represents the greater number. The element relation works because the set representing the greater number do indeed always have the set representing the smaller number as element but it's never the other way around. We define the $<$ relation between two von Neumann numbers $x,y$ as:
\begin{equation}
x < y  \quad \Leftrightarrow \quad x \in y \Leftrightarrow x \subset y
\end{equation}
Thus, we don't need to invent anything new to model our $<$ relation on the natural numbers. Our off the shelf standard set theoretic operations equip us even with two ways to do it. ...VERIFY..

\paragraph{Maximum}
The maximum of two so constructed natural numbers $a,b$ can easily be found by a standard set theoretic operation: just take the union of the sets that represent $a$ and $b$. So we define $\max(a,b) = a \cup b$. To see why this works, observe that thanks to our construction, the sets that will represent greater natural numbers will contain the sets that represent smaller numbers as subsets. In the union, the set representing the smaller number will behave neutrally. For example $2 \cup 5$ means $\{0,1\} \cup \{0,1,2,3,4\}$ when written out as sets. That union just produces $\{0,1,2,3,4\}$ which represents the number $5$ which is indeed the maximum of $2$ and $5$. This works also for unions of more than two sets. So, for a finite set $A$ of natural numbers, we define their maximum as:
\begin{equation}
\max(A) = \bigcup_{a_i \in A} a_i = \bigcup A
\end{equation}
Later, we will generalize this to infinite sets of natural numbers. In the (possibly) infinite case, we'll have to replace the maximum with the supremum [VERIFY!]. TODO: minimum via intersection

\paragraph{Predecessor}
With the so defined maximum operation, we have a way to implement a predecessor function on our von Neumann numbers. To find the predecessor $p(x)$ of a number $x$, we can just use $p(x) = \max(x) = \bigcup x$. Instead of $p(x)$, we may also write $x-1$. We have:
\begin{equation}
x - 1 = p(x) = \max(x) = \bigcup x
\end{equation}
This definition works only when $x > 0$ because $0$ itself has no predecessor. The formula would produce $0$ as the predecessor of $0$ - but that is wrong, so we need to make sure to use it only for $x > 0$.

\paragraph{Addition}
We define addition of two natural numbers $x,y$ recursively in terms of the successor function $s(x)$ as follows: $x + 0 = x$, $x + s(y) = s(x + y)$. The first formula says that when the right operand is $0$, the result is just the left operand. The second formula says that when the right operand is not zero, we should consider it as the successor of some other number $y$. The right hand side of the formula then says, we should apply addition to $x$ and $y$ and then take the successor of that. If we apply this rule recursively, we will eventually end up in the base case $x + 0 = x$ which we can then use directly. For an actual implementation, we would have to explicitly operationalize the rather implicit "consider it as the successor of" idea somehow to figure out, what that $y$ is. Fortunately, that can be done in terms of the predecessor function we have just defined. So, in summary, we define the sum $x+y$ as:
\begin{equation}
x+y =\begin{cases}
 x            & \qquad \text{if } y   =  0 \\
 (x+(y-1))+1  & \qquad \text{if } y \neq 0 
\end{cases}
\end{equation}
We could also express the second line as $x + y = s(x + p(y))$ using our successor function $s(x)$ and predecssor function $p(x)$. The important observation is that in an recursive implementation\footnote{TODO: link to the C++ implementation}, the addition function will call itself recursively but with a smaller second argument such that the recursion will eventually hit the base case where the second argument is zero.

\paragraph{Multiplication}
To define multiplication of natural numbers, we observe that $x \cdot y$ can be seen as $y$ copies of $x$ added together which in turn can be seen as $y-1$ copies of $x$ added together plus another $x$. This works as long as $y-1$ is defined. For $y=0$, it isn't - but that is our base case in which $x \cdot y$ is just $x \cdot 0 = 0$. So, we can build the following recursively defined multiplication algorithm from what we already have:
\begin{equation}
x \cdot y =\begin{cases}
 0                    & \qquad \text{if } y   =  0 \\
 (x \cdot (y-1)) + x  & \qquad \text{if } y \neq 0 
\end{cases}
\end{equation}

\paragraph{Exponentiation} Just like multiplication can be expressed in terms of repeated addition, exponentiation $x^y$ can be expressed as repeated multiplication with the base case $x^0 = 1$:
\begin{equation}
x^y =\begin{cases}
 1                  & \qquad \text{if } y   =  0 \\
 x ^{y-1}  \cdot x  & \qquad \text{if } y \neq 0 
\end{cases}
\end{equation}

\paragraph{Summary} We have constructed a representation or encoding the of natural numbers purely from sets and have expressed the basic arithmetic operations in terms of set theoretic operations. We started with the most basic operation of them all - the successor function - and then worked our way up all the way to exponentiation. TODO: define subtraction, (floor-)division, remainder, roots, logarithms - see C++ implementation - I've already implemented this stuff partially

% ToDo: maybe show some properties of the operations like asscoicativity, distribuitivity, etc.
% ..or at least meantion them

% ToDo:
% - Explain how these "natural numbers" have a cardinality that actually matches the represented
%   number. Therefore, a Neumann natural can serve as prototype set for all sets with that same
%   cardinatlity by observing that a bijection between the Neumann number and the other set with the
%   same cardinality is possible. This establishes an order on (finite) sets because we can use the
%   order defined on the Neumann naturals. 

%The structure we have created has a special name - it's called a \emph{monoid}



% these have to be partial functions. for subtraction x-y, we need y < x, for division x/y we need y != 0,
% for roots...well - we can perhaps use a floor-root definition just like with floor-division

% subtraction:    x-y = x when y=0, (x-(y-1))-1 when y != 0
% floor-divsion:  x/y = 0 when y > x, (x-y) + 1 when y >= x  ..naah...that makes no sense!
%                                     (x-y) / y + 1 when y >= x  ...try it!
% 17/5 = (17-5)/5 + 1 = 12/5 + 1 = (12-5)/5 + 1 + 1 = 7/5 + 2 = (7-5)/5 + 1 + 2 = 2/5 + 3
%      = 0 + 3 = 3
% remainder:      x - (x/y)*x   ?
% logarithm:
% root:
% Maybe for roots and logarithms, we could just use a brute force trial-and-error algorithm. For the
% n-th root, try all possible bases, take them to the n-th power until the result is >= x

% A semigroup is set with associative operation and if it also has an identity element, it's called a monoid. N is a monoid with respect to addition and multiplication (also exponentiation?)

% The so constructed natural numbers satisfy the Peano axioms (VERIFY)

% https://en.wikipedia.org/wiki/Set-theoretic_definition_of_natural_numbers
% https://en.wikipedia.org/wiki/Peano_axioms

% What is a Number? The Story of Math Starting from Zero
% https://www.youtube.com/watch?v=rhJt5AUXsl8


\subsubsection{Integer Numbers} The integers can be constructed from the naturals as equivalence classes of (ordered) pairs of naturals. Let $x = (a,b)$, i.e. our set that represents the integer $x$ is a pair of two Neumann naturals $a,b$. I will call such numbers "Neumann integers". For two such Neumann integers $x = (a,b)$ and $y = (c,d)$, we define the equivalence relation $\sim$ as follows:
\begin{equation}
(a,b) \sim (c,d) \quad \Leftrightarrow  \quad a+d = c+b
\end{equation}
Formally, we could define the set that we construct as $\mathbb{N}^2 / \sim$ which we may read as: "N-squared modulo equivalence" or "N-squared modulo tilde". The pair $(a,b)$ represents the integer $a-b$. The nonnegative integers are recovered by letting $b = 0$. For example, $3$ would be represented by $(3,0)$. However, it could also be represented by $(5,2)$ or any other pair of numbers $(a,b)$ which satisfy $a-b = 3$. It may be natural to consider $(3,0)$ as the canonical representation of the number $3$, though

\paragraph{Negation}
A negative number like $-3$ would be represented by $(0,3)$ or by $(2,5)$ etc. where the former could be seen as canonical representation. The negative counterpart to any given number is obtained by swapping the components of the pair. We could use the notation:
\begin{equation}
 -(a,b) = (b,a)
\end{equation}
for expressing the unary operation of negation applied to the Neumann integer $(a,b)$. A Neumann integer $(a,b)$ is negative iff $a < b$ according to the $<$ relation on Neumann naturals. A Neumann integer is in canonical representation if $a = 0$ or $b = 0$.

\paragraph{Addition}
We add such Neumann integers via the formula:
\begin{equation}
 (a,b)+(c,d) = (a+c,b+d)
\end{equation}
That means, addition of Neumann integers just works element-wise by making use of the previously defined addition of the Neumann naturals. ...VERIFY....TBC...explain multiplication, order, exponentiation, canonical representation (one of the components is zero - it's analogous to represent a fraction in lowest terms)...TODO: explain embedding functions
% Z is an integral domain (commutative ring with no zero divisors - maybe we need to demand a multiplicative identity? ..see chapter about rings and be consistent with it - there are different convetions for this)

% explain addition, multiplication and exponentiation (but maybe for the latter, we need rational numbers because negative exponents will lead to division)

% (a,b) represents the integer $a-b$

% https://www.youtube.com/watch?v=HKbCtoSTalk&list=PLpzmRsG7u_gpZyi0w80IBk51k6ClSJRWK&index=23


% How to Construct Infinite Sets
% https://www.youtube.com/watch?v=dz7j38sCUkI
% -1 = {(0,1),(1,2),(2,3),(3,4),...} = {(a,b) \in N^2: (b-a) = 1 } = [(0,1)]_{\sim}

\subsubsection{Rational Numbers} The rationals can be constructed from the integers as equivalence classes of pairs of integers in a way that is analogous to how we defined the integers as equivalence classes of pairs of naturals. Again, we define an equivalence relation $\sim$. This time, it's defined as:
\begin{equation}
(a,b) \sim (c,d) \quad \Leftrightarrow  \quad a \cdot d = c \cdot b
\end{equation}

..TBC...
% Q is a field

% What about exponentiaion? first with integers as exponents but then also with other
% rationals as exponents...radical numbers? What about algebraic numbers?

\subsubsection{Real Numbers} The reals can be constructed from the rationals by a technique called \emph{Dedekind cuts}. That's the most complicated step in the construction of the number system. Let's approach it by using the irrational number $\sqrt{2}$ as example. As a Dedekind cut, we represent it as follows:
\begin{equation}
\sqrt{2} = (A, B) \qquad \text{where} \qquad
A = \{a \in \mathbb{Q} : a^2  <   2 \vee   a   <  0\}, \quad
B = \{b \in \mathbb{Q} : b^2 \geq 2 \wedge b \geq 0\}
\end{equation}
That is: the real number $\sqrt{2}$ is represented as a pair of sets of rational numbers $A$ and $B$. Note that these sets are infinite. This is what makes Dedekind cuts qualitatively so much more complicated than the construction steps that we took up to now. $A$ is the set of all rational numbers that are strictly less than $\sqrt{2}$ and $B$ is the set of all rational numbers that are greater or equal to $\sqrt{2}$. The general idea is that a real number $x$ cuts the number line of the rational numbers into two parts. The left part contains all rational numbers less than $x$ and the right part contains all rational numbers greater or equal to $x$. The number $x$ itself does not need to be rational. If the real number $x$ happens to be rational, then the rational number representing $x$ becomes a member of the right set $B$. In this case, $x$ is the minimum of $B$. If $x$ is irrational, $B$ has no minimum. In this $x$ case is only the infimum (i.e. greatest lower bound) of $B$ but its not a member of $B$. ...TBC...explain addition and multiplication (interval arithmetic)

% explain how more complicated numbers like pi can be represented. Maybe by the limit of some sequence of numbers?

% in R, the order is complete  ...what does that mean? It's from the math-faq.pdf above


% https://www.youtube.com/watch?v=3FeIv08nrLk&list=PLlgGdNP9Tq4uzy2uu1D4zBg3W1pRuEUGK&index=7
% -A real number x is defined by the (infinite) set of rational numbers that are less than x. 
%  -the real number x = sqrt(2) is the least upper bound of a set of rationals:
%    S = { x \in Q : x^2 < 2  }
%   we use that set itself to represent sqrt(2)
%  -the ordering is just a subset relation
% -alternative construction: equivalence class of all Cauchy sequences that converge to x

% https://en.wikipedia.org/wiki/Dedekind_cut
% https://en.wikipedia.org/wiki/Interval_arithmetic

% https://de.wikipedia.org/wiki/Dedekindscher_Schnitt



\subsubsection{Complex Numbers} Having the real numbers in hand, constructing the complex numbers is actually very simple. We just define a complex number as a pair of real numbers. That's it. This time, we don't even need equivalence classes or infinite sets. Of course, we still need to define the operations of addition and multiplication. Addition, once again, just works element wise.  ...TBC...explain multiplication
% C is algebraically complete

% https://en.wikipedia.org/wiki/Computable_number

% All the Numbers - Numberphile
% https://www.youtube.com/watch?v=5TkIe60y2GI
% -Computable numbers are those for which we can give a finite(?) algorithm that gives us however
%  many digits of the number we want
%  -There are countably many of them
%  -Other numbers are inaccessible to us via computer technology


% Modeling more complex functions with sets:
% -Relations between two sets A,B: subsets of A x B
% -Functions: special relations
% - M x N Matrices: functions from {1,2,...,M} x {1,2,...,N}  to  the real numbers (for example)

%---------------------------------------------------------------------------------------------------
%\subsection{Some Definitions}

%\paragraph{Upper Bound}

%\paragraph{Maximal Element}
% https://en.wikipedia.org/wiki/Maximal_and_minimal_elements

%\paragraph{Supremum of a Set of Ordinals}

%\paragraph{Chain}

% https://en.wikipedia.org/wiki/Total_order#Chains

% https://en.wikipedia.org/wiki/Upper_and_lower_bounds



%---------------------------------------------------------------------------------------------------
\subsection{Transfinite Numbers}
We constructed the integers, rationals, etc. by extending the set of natural numbers $\mathbb{N}$. Here, we will consider a different extension of $\mathbb{N}$ that will also contain distinguishable infinite numbers which are called \emph{transfinite} in this context. We'll arrive at a set called the \emph{ordinal numbers} a subset of which are the \emph{cardinal numbers}. In practice, natural numbers can be used for different two things: (1) counting, how many elements there are in a set, (2) ordering, i.e. putting a bunch of things into a particular order such that we can say: this is the first, this the second, etc. The difference between these two aspects of natural numbers is rather subtle for finite numbers but will become important for infinite numbers. The ordering aspect is captured by the ordinal numbers and the counting aspect will be captured by the cardinal numbers.

%, which are, as we will see, actually a certain subset of the ordinal numbers. We will look at both kinds of transfinite numbers in turn.

% https://en.wikipedia.org/wiki/Transfinite_number

%---------------------------------------------------------------------------------------------------
\subsubsection{Beth Numbers}
The beth numbers are another type of transfinite numbers that are not so commonly used. But they do correspond best to what we might expect intuitively from differently sized infinite sets and serve as a good example for constructing sets via transfinite recursion, so that's why we will introduce them first. They are a good warm-up for the progressively more complicated topics of cardinal and ordinal numbers. The beth numbers are denoted by the Hebrew letter $\beth$ ("beth") with a subscript. We know that for any given finite set $A$, we can construct a bigger set by simply taking its power set $B = \mathcal{P}(A)$ where "bigger" is to be understood in the sense of "greater cardinality". To create an even bigger set, we can create the power set of that: $C = \mathcal{P}(B) = \mathcal{P}(\mathcal{P}(A))$. And we can continue applying the power set operation indefinitely to create ever bigger sets. As we will see later, Cantor's theorem shows that this construction of ever bigger sets using the iterated power set operation still works for infinite sets, so via the iterated power set operation, we have a way of producing a sequence of infinitely large sets with ever increasing size (cardinality). The beth numbers correspond to these sizes produced by the iterated power set operation: $\beth_n = |\mathcal{P}^n (\mathbb{N})|$, such that $\beth_0 = |\mathbb{N}|, \beth_1 = |\mathcal{P} (\mathbb{N})|, \beth_2 = |\mathcal{P}(\mathcal{P} (\mathbb{N}))|, \ldots$. Here, for the time being, we may think of $n$ as a natural number. It can actually also be an ordinal number though. That's why the index is usually a greek letter in textbooks. The beth numbers are formally defined via transfinite recursion as:
\begin{equation}
\beth_0 = \aleph_0 
        = \omega 
        = |\mathbb{N}|, \quad
\beth_{\alpha + 1} = \mathcal{P}(\beth_{\alpha}) 
                   = 2^{\beth_{\alpha}}, \quad
\beth_{\lambda} = \sup \{ \beth_{\alpha} : \alpha < \lambda \} 
                = \bigcup_{\alpha \in \lambda} \beth_{\alpha} 
\end{equation}
The leftmost definition is the base case in various notations. The middle equation is the successor step. The $(\alpha+1)$th beth number is computed from the $\alpha$th beth number by taking its power set. Remember that numbers \emph{are} sets, so yes - we can indeed take the power set of a number, even if that feels strange. VERIFY!...TBC...Explain limit step
%but to understand this definition, we first need some more definitions an tools in place. ...TBC...VERIFY

% https://en.wikipedia.org/wiki/Beth_number

% https://neugierde.github.io/cantors-attic/Beth
% -beth_omega has the property: whenever a set X has a cardinality less then beth_omega, then
%  its power set will also have a cardinality less than beth_omega

% https://planetmath.org/bethnumbers

% Maybe introduce the beth numbers after ordinals and transfinite recursion

% https://de.wikipedia.org/wiki/Beth-Funktion
% -It talks about beth numbers as cardinal numbers - so is cardinal number an umbrella term for
%  the aleph and beth numbers? figure out!

% How To Count Past Infinity
% https://www.youtube.com/watch?v=SrU9YDoXE88
% 6:02 - in the realm of the infinite, labeling things in order is different from counting them

%---------------------------------------------------------------------------------------------------
\subsubsection{Cardinal Numbers}
The purpose of cardinal numbers is to quantify the size of a set - finite or infinite. I'll split the treatment of cardinal numbers into a "preliminary" part before the section about ordinals and a "proper" part after the section about ordinals. The proper part will then really define them in terms of ordinals. I opted to do it this way because I think that conceptually, the cardinals are easier to understand. They require only the notion of bijections whereas the ordinals require the more complicated notion of order isomorphisms, so it's appropriate to treat the simpler sort of numbers first. 

%So far for our preliminary treatment of cardinal numbers. We'll now look at ordinal numbers and when we have the ordinals in hand, we'll revisit the cardinals and define them properly. I opted to split the treatment of cardinals into a "preliminary" and "proper" section because conceptually, the cardinals are easier to understand
%...TBC...

\paragraph{Size Comparisons}
For finite sets, size comparisons are easy: We just count the number of elements, which results in a natural number, and whichever set has the greater count, has the greater size. Simple enough. But what if we can't quantify the size of a set with a natural number? The idea is to observe that for finite sets, two sets of equal size can always be put into a bijection with one another. That means that for sets of equal size, we can find a function that maps one set to the other bijectively. Usually many such functions exist, but finding one is enough to demonstrate that the two sets have equal size. The idea is now to generalize this to infinite sets: Two sets are considered to be of equal size whenever we can find a bijective function between them.

\medskip
We do not only want to answer the question whether or not two sets are of equal size. We also want to know, which of the two sets is bigger in the case of non equal sizes. To this end, we observe that for finite sets, whenever we can find an injective but not bijective function $f$ from one set $A$ to another set $B$, then the set $A$ is smaller. Injective means that every element from $A$ is mapped to a different element in $B$. If an injective map is not bijective, i.e. it fails to be surjective, then that means that there are leftover elements in $B$ which are not reached by $f$. We interpret that circumstance as $B$ having a greater size than $A$ such that we cannot establish a one-to-one correspondence. 

\medskip
We now formalize these ideas as follows: We want to define a new kind of "number" that measures the size of sets. We want the class\footnote{It turns out that the cardinal numbers do not fit into a set. Just like the ordinals, they form a proper class. In a sense, there are "too many" of them. } of these new numbers to be comparable in the sense of a greater or smaller "size". So, the task is now to define a new class of numbers and an equivalence relation and a strict total order relation on the class of these numbers. These relations should capture sizes of sets in terms of existence or non-existence of bijections and injections between the sets.

\paragraph{Cantor's First Diagonal Argument} 
To get a first impression for how one can prove the existence of bijections, we will now show why $|\mathbb{Q}| = |\mathbb{N}|$ by a proof technique invented by Cantor. By using this technique, one can demonstrate the general possibility to create a bijection between a countably infinite set and its cartesian product with itself. We'll take the set $\mathbb{N}$ as the prototypical example of such a set, but the technique readily generalizes to any countably infinite set $A$ by just pointing out that, by definition of the term "countably infinite", the set $A$ can be put into a bijection with $\mathbb{N}$. To demonstrate the existence of a bijection between $\mathbb{N}$ and $\mathbb{N} \times \mathbb{N}$, we will explicitly construct one in the following way: We create a 2D table of the pairs $(n, m)$ where $n,m \in \mathbb{N}$ and then traverse the 2D table in diagonals. It looks like this:
\[
\setlength{\extrarowheight}{5pt}% local setting
\begin{array}{l|*{6}{c}}
	       &  0     &  1     &  2     &  3      &  4     & \cdots \\
	\hline
	0      &  0     &  2     &  5     &  9      & 14     & \cdots \\
	1      &  1     &  4     &  8     & 13      & 19     & \cdots \\
	2      &  3     &  7     & 12     & 18      & 25     & \cdots \\
	3      &  6     & 11     & 17     & 24      & 32     & \cdots \\
	4      & 10     & 16     & 23     & 31      & 40     & \cdots \\
	\vdots & \vdots & \vdots & \vdots & \vdots  & \vdots & \ddots \\
\end{array} 
\qquad \qquad
\begin{aligned}
&k = f(m,n) = \frac{(m+n)(m+n+1)}{2} + n \\
\\
&n = k-t\\ 
&m = w-n\\
& \text{ where } w = \left\lfloor \frac{\sqrt{8k+1}-1}{2} \right\rfloor, 
  \quad          t = \frac{w^2+w}{2} 
\end{aligned}
\]
To figure out the number $k$ to which the pair $n,m$ gets mapped, one uses a so called \emph{pairing function} to compute $k = f(m,n)$. Depending on the path taken through the 2D array, there may be different possible pairing functions. As said, Cantor's original one traverses the diagonals one by one and the single number index $k$ can be computed from 2D index $(m,n)$ by the formula given next to the table. The formula can be uniquely inverted, i.e given a $k \in \mathbb{N}$ we can uniquely compute the pair $(m,n)$ that corresponds to it. The formulas for this are also given. For more details about that, see \cite{WK_PairingFunction} or search the internet for "pairing function". Pairing functions are also of practical interest in the realm of programming because they allow to uniquely map back and forth between two-dimensional and one-dimensional indices, i.e. the allow a unique and reversible encoding of a pair $(m,n)$ into a single number $k$. Extensions exist to map back and forth between 3D and 1D, 4D and 1D, etc. The takeaway is that we have explicitly constructed a bijection between $\mathbb{N}$ and $\mathbb{N}^2$. We could easily extend the idea to bijections between $\mathbb{Z}$ and $\mathbb{Z}^2$ and by noting that $\mathbb{Q}$ is just a subset of $\mathbb{Z}^2$, we can see that $|\mathbb{Q}| \leq |\mathbb{Z}^2| = |\mathbb{N}|$. I hope it's obvious that $|\mathbb{Q}|$ can't be strictly less than $|\mathbb{N}|$, so we can conclude that $|\mathbb{Q}| = |\mathbb{N}|$. [TODO: verify formulas - implement and test them - maybe give alternative un/pairing formulas that are better in practice]

% https://en.wikipedia.org/wiki/Pairing_function
% https://mathworld.wolfram.com/PairingFunction.html
% https://www.cantorsparadise.com/cantor-pairing-function-e213a8a89c2b
% https://www.cs.upc.edu/~alvarez/calculabilitat/enumerabilitat.pdf
% https://de.wikipedia.org/wiki/Cantorsche_Paarungsfunktion
% https://de.wikipedia.org/wiki/Dreieckszahl#Dreieckswurzel
% https://en.wikipedia.org/wiki/Triangular_number#Triangular_roots_and_tests_for_triangular_numbers
% http://www.szudzik.com/ElegantPairing.pdf
% https://github.com/drhagen/pairing
% https://drhagen.com/blog/superior-pairing-function/
% https://drhagen.com/blog/multidimensional-pairing-functions/

\paragraph{Cantor's Second Diagonal Argument} 
To convince ourselves that it actually makes sense to talk about different sizes of infinity, we will now show why $|\mathcal{P}(\mathbb{N})| > |\mathbb{N}|$ by a different kind of diagonal argument. This technique is an example of the proof by contradiction method. Let's assume that we have list of subsets $S_n$ of the natural numbers $\mathbb{N}$. That means, each natural number $n \in \mathbb{N}$ is mapped to a subset $S_n \subseteq \mathbb{N}$ of the natural numbers. We could visualize such a list as an infinite table where the $(i,j)$th entry tells us whether or not the number $j$ (i.e. the column header) is an element of the subset with index $i$ (i.e. the $i$ in the $S_i$ in the row header). An example for how this this could look like is depicted in the left table of the figure below (ignore the tables in the middle and on the right for the moment).
\[
\setlength{\extrarowheight}{5pt}% local setting
\begin{array}{l|*{6}{c}}
	       &  0     &  1     &  2     &  3      &  4     & \cdots \\
	\hline
	S_0    & \in    &  \in   & \notin & \in     & \notin & \cdots \\
	S_1    & \notin &  \in   & \in    & \in     & \notin & \cdots \\
	S_2    & \notin &  \in   & \notin & \in     & \notin & \cdots \\
	S_3    & \in    & \in    & \notin & \notin  & \in    & \cdots \\
	S_4    & \in    & \notin & \notin & \in     & \in    & \cdots \\
	\vdots & \vdots & \vdots & \vdots & \vdots  & \vdots & \ddots \\
\end{array}
\qquad \qquad
\setlength{\extrarowheight}{5pt}
\begin{array}{l|*{6}{c}}
	       &  b_0   &  b_1   &  b_2   &  b_3    &  b_4   & \cdots \\
	\hline
	  0    &  1     &  1     &  0     &  1      &  0     & \cdots \\
	  1    &  0     &  1     &  1     &  1      &  0     & \cdots \\
	  2    &  0     &  1     &  0     &  1      &  0     & \cdots \\
	  3    &  1     &  1     &  0     &  0      &  1     & \cdots \\
	  4    &  1     &  0     &  0     &  1      &  1     & \cdots \\
	\vdots & \vdots & \vdots & \vdots & \vdots  & \vdots & \ddots \\
\end{array}
\qquad \qquad
\setlength{\extrarowheight}{5pt}
\begin{array}{l|*{6}{c}}
	       &  d_0   &  d_1   &  d_2   &  d_3    &  d_4   & \cdots \\
	\hline
	0      &  3     &  1     &  4     &  1      &  5     & \cdots \\
	1      &  2     &  7     &  1     &  8      &  2     & \cdots \\
	2      &  1     &  6     &  1     &  8      &  0     & \cdots \\
	3      &  0     &  5     &  7     &  7      &  2     & \cdots \\
	4      &  1     &  4     &  1     &  4      &  2     & \cdots \\
	\vdots & \vdots & \vdots & \vdots & \vdots  & \vdots & \ddots \\
\end{array}
\]
Let's now assume that the list of subsets is complete, i.e. every possible subset of $\mathbb{N}$ is represented by some row of our table. We will now show that this assumption must be false by explicitly constructing another subset $T$ of $\mathbb{N}$ that is guaranteed to be not in our table, i.e. is none of the $S_i$. We build $T$ as follows: We scan our existing table along its diagonal and whatever symbol we find there, we'll do the opposite: If we find an $\in$ at position $(i,i)$, then we will not include $i$ into our set $T$. If we find a $\notin$ at position $(i,i)$, then we will include $i$ into our set $T$. The so constructed set $T$ cannot be any of the $S_i$ because it will differ at at least one position from any of the $S_i$. Therefore, our table cannot possibly have been complete in the first place. At least one subset of $\mathbb{N}$, namely $T$, was not reached by our supposed bijection\footnote{Question: Couldn't we add our new subset $T$ at the $\omega$th row? And a list with $\omega + 1$ rows has still a countable number of rows? ...idk...}. So it isn't a bijection after all because it fails to be surjective. Because bijectivity was our only assumption about $f$, this shows that such a bijectivity assumption can never hold up. Therefore, we conclude that the cardinality of the power set of $\mathbb{N}$ must be strictly greater than the cardinality of $\mathbb{N}$ itself: $|\mathcal{P}(\mathbb{N})| > |\mathbb{N}|$.

% https://en.wikipedia.org/wiki/List_of_mathematical_constants
% pi      = 3.1415926535897932384626433832795028841971693993751058209749445923...
% e       = 2.7182818284590452353602874713526624977572470936999595749669676277...
% phi     = 1.6180339887498948482045868343656381177203091798057628621354486227...
% gamma   = 0.5772156649015328606065120900824024310421593359399235988057672348...
% sqrt(2) = 1.4142135623730950488016887242096980785696718753769480731766797379...

\medskip
The table in the middle is actually the same table as the one on the left, just with some renaming. For the table entries, we replaced $\in$ with $1$ and $\notin$ with $0$ and adapted the table headers a little. The column headers are now called $b_i$ to indicate that this could be the $i$th bit in an infinitely long sequence of $0$s and $1$s, i.e. bits. Writing the table in this form shows us that we can map every natural number to an infinite sequence of bits. We can interpret the sequence of bits $b_n$ in each row as indicators for whether or not the number $n$ is a member of the subset with the given row index. But we can also interpret the sequence of bits as representing a real number in the half open interval $[0,1)$ in binary notation. The list of infinitely many strings of bits where each string is of infinite length can therefore be used to encode real numbers. Can our list ever contain all real numbers, i.e. have a row for every possible real number in $[0,1)$? No - because by the very same construction, we can construct a binary representation of a number that is guaranteed to be not in our list. To construct this number $r$, scan again along the diagonal and whenever we find a $0$ in position $(i,i)$, put a $1$ into our string at position $i$ - and vice versa. Always do the opposite.

\medskip
The rightmost table shows that the argument works almost in the same way for real numbers in their decimal expansion: We again assume that we have a table with a countably infinite number of rows and columns and for each row, we write into it the decimal representation of a real number in $[0,1)$ without the leading $0$. To construct a new number that is guaranteed to be not in our list, we scan the diagonal and make sure that we put a digit at the $i$th position of our number that is different from the $(i,i)$th table entry. That means the "do the opposite" prescription for the case of binary representation just generalizes to "do something different" in base $10$ and, in fact, in any base whatsoever. This is actually a foreshadowing\footnote{So don't worry, if the statement seems still confusing} for some interesting result for cardinal exponentiation: when the exponent is infinite, the base doesn't matter. In our example, we found $|2^{\mathbb{N}}| = |10^{\mathbb{N}}|$. More on this later. [VERIFY!]

\medskip
...TBC... introduce the symbol $\mathfrak{c}$ for the size of the continuum, explain that the result $|\mathcal{P}(A)| > |A|$ holds for any set and its power set (Cantor's theorem), explain how to map $[0,1)$ bijectively to $\mathbb{R}$, explain the complication of non-unique representations of e.g. $0.010000... = 0.0099999...$

% OLD:
%To convince ourselves that it actually makes sense to talk about different sizes of infinity, we will now show why $|\mathbb{R}| > |\mathbb{N}|$ by a different flavor of Cantor's diagonalization technique. This technique is an example of the proof by contradiction method and it is a method for showing the impossibility of the existence of a bijection between two sets. The general mode of operation of the technique is: We start by assuming that we do have a bijection $f: A \rightarrow B$ from a set $A$ to set $B$. Then we will explicitly construct an element $b \in B$ which we can show to not be reached by our supposed bijection $f$. Therefore $f$ cannot be surjective and therefore not bijective. That proves that our initial assumption of $f$ being bijective must have been a false assumption. Because bijectivity was our only assumption about $f$, this shows that such a bijectivity assumption can never hold up [VERIFY]. 

%OK - now from the abstract to the concrete. We will need 3 steps: (1) Show why a bijection between $\mathbb{N}$ and $2^{\mathbb{N}}$ is impossible, (2) show how the unit interval $[0,1)$ of real numbers can be put into bijection with $2^{\mathbb{N}}$, (3) Show how  $[0,1)$ can be put into bijection with $\mathbb{R}$. .....TBC...TODO: show how with a minor tweak, the argument can be used to show that $|A| < |2^A|$ for any set $A$ - that's more useful in the following material

% Idea: if we produce the $b \in B$ this way, couldn't we just add the line with the so produced b to the table? Maybe at the w-th position? It seems like postulating that we can always add such a line feels exactly like postulating the existence of an infinite set as in the axiom of infinity. Couldn't we put this new line of the table at the w-th position?

%TODO: (1) show how the unit interval $[0,1)$ of real number can be put into bijection with $2^{\mathbb{N}}$, (2) show how a bijection between $\mathbb{N}$ and $2^{\mathbb{N}}$ is impossible, (3) then show how  $[0,1)$ can be put into bijection with $\mathbb{R}$

%  show that in general the power set of any given countably infinite set cannot be put into bijection with the original set, then 


%https://www.youtube.com/watch?v=iaUwNuaSLUk  7:18

%\subsubsection{Cardinal Arithmetic}

% https://de.wikipedia.org/wiki/Kardinalzahlarithmetik#Kontinuumsfunktion
% -> this page is the target of the Gimel-Funktion link here:
% https://de.wikipedia.org/wiki/Aleph-Funktion
% ...so Gimel-Funktion is a synonym for Kontinuumsfunktion?
% ..its just the function k -> 2^k. I guess the name can be explained by observing that it gives
% the size of the continuum when we plug in the first infinite cardinal

% $\aleph_0 + 1 = \aleph_0$ but $\omega + 1 \neq \omega$. I think $\aleph_0$ and $\omega$ are the same thing in a similar way as the natural number 0 and the real number 0 are the same. They represent the same idea but in a different data type - and the data types respond differently to certain arithmetic operations (for example, to division)

% What A General Diagonal Argument Looks Like (Category Theory)
% https://www.youtube.com/watch?v=dwNxVpbEVcc

\paragraph{Capturing Infinite Sizes in Numbers} So, we have seen that when we compare the sizes of infinite sets in terms of im/possibility of a bijection between them, then at least two different sizes of infinity must exist, namely the size of $\mathbb{N}$ and the size of $\mathbb{R}$. In fact, there's an infinite hierarchy of sizes of infinity and $|\mathbb{N}|$ and $|\mathbb{R}|$ are just two very humble members of that hierarchy. This should be very plausible after having seen the beth numbers. This, possibly infinite, size is captured in another kind of transfinite numbers - the so called \emph{cardinal numbers}. The finite cardinal numbers are just the natural numbers. The infinite ones are denoted by the Hebrew letter $\aleph$ with an index, such as $\aleph_0$, pronounced "aleph-zero" or "aleph-null" or "aleph-nought". This $0$th aleph number is, by definition, the cardinality of $\mathbb{N}$. So you might ask: what is then $\aleph_1$? Is it the cardinality of $\mathbb{R}$? The answer to that question is a bit more complicated - asserting that this is the case is the famous \emph{continuum hypothesis}. What we can say, however, is that the beth numbers constitute an indefinitely increasing sequence of infinite sizes. We can say this because we have seen by Cantor's 2nd diagonal argument that no bijection can exist between a set and its power set - and the beth numbers are defined by iterating the power set operation. There's actually a generalization of the continuum hypothesis that states that $\aleph_n = \beth_n$ for any $n$. For the time being, you can imagine $n$ as natural but the generalization goes even further and allows $n$ to be ordinal. But whatever the case may be, whether $|\mathbb{R}| = \aleph_1$ or not, we do have an increasing sequence of infinite set sizes. At this point, it is still conceivable that the aleph numbers (defined by existence of bijections) have a finer granularity that the beth numbers (defined by iterating the power set operation). VERIFY!.....TBC...


\paragraph{Cardinal Arithmetic}
The cardinal numbers are a superset of the natural numbers. With natural numbers, we can do certain arithmetic operations, namely addition, multiplication and exponentiation. It is possible to extend these operations to the infinite cardinals...TBC...


% https://en.wikipedia.org/wiki/Cardinal_number#Cardinal_arithmetic
% https://en.wikipedia.org/wiki/Inaccessible_cardinal

\paragraph{Addition} For two sets $A,B$ with cardinalities $|A|,|B|$ repectively, we define the sum $|A| + |B|$ as the cardinality of the disjoint union of $A$ and $B$: $|A| + |B| = |A \sqcup B|$. If $A$ and $B$ are already disjoint, one could also take the cardinality of the ordinary union $|A \cup B|$. The definition makes intuitive sense when one considers what happens to the total number of objects when we throw the content of to finite bags of real world objects (say, apples\footnote{Because, of course, it's always about apples in math.}) together. Sets/bags of real world objects are always disjoint by nature because no physical object can be in two places at the same time. In the world of math where two sets can in fact contain the same element (mathematical sets are not physical bags), we need this little complication of making them artificially distinguishable - that's why we need the disjoint union in general. If $A$ and $B$ are both finite, the cardinal addition just becomes the normal addition of natural numbers. If at least one of the sets $A, B$ is infinite, then the cardinal sum of the two is just the maximum: $|A| + |B| = \max(|A|, |B|)$ iff $|A| \geq \aleph_0 \vee |B| \geq \aleph_0$.
\begin{equation}
|A| + |B| = |A \sqcup B| =
\begin{cases}
|A| + |B|       \qquad &\text{if $A$ and $B$ are finite} \\
\max(|A|,|B|)   \qquad &\text{if at least one of $A, B$ is infinite}
\end{cases}
\end{equation}
We can interpret this maximum as: adding finite sets to infinite ones doesn't change their cardinality. The cardinality of the finite set gets just absorbed in the infinite one. A similar thing happens when sets with smaller sizes of infinity get absorbed into bigger infinities. A set of natural or even rational numbers is just a completely insignificant dust in the vastly bigger continuum of the real numbers, for example. Adding $\mathbb{Q}$ to $\mathbb{R} \setminus \mathbb{Q}$ doesn't really matter in terms of cardinality.

%...TBC...this notation seem kinda atypical, I think - more typical is to use $\kappa, \lambda$ or something

% https://aleph1.info/?call=Puc&permalink=mengenlehre1_1_12_Z2
% https://en.wikipedia.org/wiki/Cardinal_number#Cardinal_addition

\paragraph{Multiplication} The product of two cardinal numbers $|A|,|B|$ is defined as cardinality of the set product: $|A| \cdot |B| = |A \times B|$. The situation here is the same as for addition. For finite cardinals, the multiplication is the same as for natural numbers. If at least one of the sets is infinite, the product of their cardinalities is just the maximum.
\begin{equation}
|A| \cdot |B| = |A \times B| =
\begin{cases}
|A| \cdot |B|    \qquad &\text{if $A$ and $B$ are finite} \\
\max(|A|,|B|)   \qquad &\text{if at least one of $A, B$ is infinite}
\end{cases}
\end{equation}
This may be a bit harder to swallow. But we have already seen how $\mathbb{N}$ can be bijectively mapped to $\mathbb{N}^2$, so it might be somewhat plausible to believe that the set product of infinite sets does not yield larger sets in terms of cardinality. Although, in the case of mapping $\mathbb{R}$ bijectively to $\mathbb{R}^2$, Cantor himself couldn't believe it at first. But he found a bijection: To bijectively map a real number $z$ to a pair of real numbers $(x,y)$, just take the decimal expansion\footnote{Actually, the decimal expansion of many numbers is not unique - consider $1 = 0.999\ldots$. And one perhaps also has to take some care about how to treat the position of the decimal dot. But these are minor complications that can be solved.} of $z$ and alternatingly put digits into $x$ and $y$, i.e. "unzip" the decimal expansion into two. The reverse mapping is just a "zip" operation. So - yeah - it works out to find bijections between $\mathbb{N}$ and $\mathbb{N}^2$ and it works also for  $\mathbb{R}$ and $\mathbb{R}^2$. And it works in general. Of course, these examples are no proof - but they are a plausibility check.


% https://en.wikipedia.org/wiki/Cardinal_number#Cardinal_multiplication

\paragraph{Exponentiation} OK - so addition and multiplication of infinite cardinal numbers is actually pretty boring. It's just a maximum operation. The most interesting operation on cardinal numbers is the exponentiation. As usual, the definition of the operation is based on the corresponding set operation: $|A|^{|B|} = |A^B|$. The set $A^B$ is the set of all functions from $B$ to $A$. If $A$ and $B$ are finite, we observe, that the number of possible different functions from $B$ to $A$ is indeed given by $|A|^{|B|}$.




...TBC...ToDo: subtraction, division, roots, logarithms - look at the code

% https://en.wikipedia.org/wiki/Cardinal_number#Cardinal_exponentiation




%\subsection{title}


%\paragraph{Note}
%



\subsubsection{Ordinal Numbers}
Ordinal numbers are, as the name suggests, used to order things, i.e. to put them into an ascending sequence. The role that bijections play in the definition of cardinal numbers will now be played by order isomorphisms which are bijections with more specific properties. By using order isomorphisms rather than general bijections, one partitions the class of all sets into \emph{order types} rather than size types. This is a finer classification. Sets of the same size type, such as $\mathbb{N}, \mathbb{Z}, \mathbb{Q}$ which are indistiguishable by cardinality will become distiguishible because they belong to different order types. ...TBC...

%It turns out that the tool need to distinguish between them is the notion of their respective order types. ...TBC...

\paragraph{Intuition} ...TODO: explain how different order types can be visualized and how addition and multiplication work in terms of that visualization

\paragraph{Definition of Ordinal Numbers}
The formal set theoretical definition of an ordinal number is simple: A set $\alpha$ is called an \emph{ordinal number} if it is transitive. Using greek letters from the beginning of the alphabet is common practice in the theory of ordinal numbers, so we will adopt that practice here, too. We first observe that our natural numbers constructed via the von Neumann construction do indeed qualify as ordinal numbers according to this definition. We conclude that natural numbers are ordinal numbers. They are the finite ordinal numbers. There are more ordinal numbers, though. 

\paragraph{Infinite Ordinals}
...TBC...

% Let u stand for union. Then:
% - w = { 0,1,2,3,... } = N
% - w+1 = w u {w} = { 0,1,2,3,...,w } = N u {N}
% - w+2 = w+1 u {w+1} = { 0,1,2,3,...,w,w+1 }
% ...etc.


% https://en.wikipedia.org/wiki/Ordinal_number
% https://de.wikipedia.org/wiki/Ordinalzahl#Limes-_und_Nachfolgerzahlen

% https://de.wikipedia.org/wiki/Ordinalzahl

% https://www.youtube.com/watch?v=Ay0151PkgMA&list=PL2m0OzES6Uw9zK-F8BX8HuGq7HAx9KhQb&index=13
% -ordinal numbres are transivtive sets that are well-ordered by the element relation

% Set Theory | Lesson 11: Ordinals!!! [CC]
% https://www.youtube.com/watch?v=efTeurdX__A&list=PLlgGdNP9Tq4uzy2uu1D4zBg3W1pRuEUGK&index=11

% https://aleph1.info/?call=Puc&permalink=mengenlehre1_2_1
% % https://aleph1.info/Resource?method=get&obj=Pdf&name=mengenlehre1.pdf
% -We can imagine the ordinal numbers (or order numbers) as follows:
%  0,1,2,...,w,w+1,w+2,...,w+w,w+w+1,w+w+2,...
% -The idea is that we can pick any ordinal number and from there travel to the right by single
%  steps (with a successor function) or in larger steps by adding another ordinal to our current
%  position
% -Each such journey has a uniquely determined destination - the limit or supremum of all such 
%  steps
% -pg 284 of pdf book: for ordinal numbers, the element-of relation serves as the order relation?
%  It's also here:  https://aleph1.info/?call=Puc&permalink=mengenlehre1_2_6_Z4

% https://www.youtube.com/watch?v=5NSg_wJkEMc&list=PLHW2zF4VTA-dP091KTn3VbWcHKssMkSq4&index=5
% -all the ordinals less than w_1 are countable
% -the "order-type" is the index of thw w-number?

% https://www.youtube.com/watch?v=As8rTENUOy0&list=PL2m0OzES6Uw9zK-F8BX8HuGq7HAx9KhQb&index=6
% -Introduces the "ordinal union" at around 5:20. I'm not sure, if that is a special sort of
%  operation or just the regular union applied to ordinals. He talks about first algining them
%  before taking the union...but I guess, that alignment is just necessarry for the visualization?
%  ...ahh...ok...yes - he says: "it simply returns the bigger ordinal". In chapter 7, he also says,
%  it's the supremum
% -so: w1 is the ordinal union (i.e. supremum) of all the countable ordinals

% It's important to note that the order type of a poset is not an intrinsic feature of the underlying set. It really comes into existence only after an order relation is defined on the set. We can, for example, define various different order relations on the set of rationals (todo: mention calkin-wilf trees, order by denominator, then by numerator, using Cantor digonalization)

% https://en.wikipedia.org/wiki/Calkin%E2%80%93Wilf_tree


% "infinite" arithmetic is strange
% https://www.youtube.com/watch?v=ZfBT5KhZFQs

% Make a section about "Infinite Ordinals". Start expalining w, then w+1 etc. Then w^2, w^3,...,w^w.
% Explain what all the inifinite ordinals "are" in terms of sets. Write them down in formal set
% builder notation. Maybe allow the slightly informal ... notation. Start from the Neumann naturals.
% Explain what axioms are needed to get to w. I think, we need the axiom of union and pairing
% (implying singletons) to get from one finite ordinal to the next. I think, to get to w, we
% additionally need the axiom of infinity?




\paragraph{Cantor Normal Form}
Every ordinal number $\alpha$ can be written in the form:
\begin{equation}
\alpha = \omega^{\beta_1} c_1 + \omega^{\beta_2} c_2 + \ldots + \omega^{\beta_k} c_k
\end{equation}
where $k$ is a natural number, the $c_i$ are also natural numbers and the $\beta_i$ are ordinal numbers and are sorted from largest to smallest, i.e. $\beta_1 > \beta_2 > \ldots > \beta_k$. The greatest exponent $\beta_1$ is called the degree of $\alpha$. Compare that to the way in which usually interpret decimal numbers as a weighted sum of powers of $10$. It's very similar here - just that the basis is not $10$ but $\omega$ and the weights are multiplied with the powers from the right. This is really necessary because ordinal multiplication is not commutative and multiplying an infinite ordinal from the left by any natural number changes nothing. The $\beta_i$ can, in turn, also be expressed in Cantor normal form and so on until one eventually arrives at an expression that contains only natural numbers and $\omega$s. This is similar to the hereditary base-$n$ notation of natural numbers. ...TODO: give alternative form without the $c_i$ coeffs

% https://de.wikipedia.org/wiki/Cantorsche_Normalform
% https://en.wikipedia.org/wiki/Ordinal_arithmetic#Cantor_normal_form
% https://en.wikipedia.org/wiki/Goodstein%27s_theorem#Hereditary_base-n_notation

% https://en.wikipedia.org/wiki/Ordinal_arithmetic#Natural_operations


% https://www.youtube.com/watch?v=0XneC6Iz9N0
% eps_0 = tetrate(w,w)

\subsubsection{Cardinals as Special Ordinals}
We have formerly introduced the cardinal numbers without reference to ordinal numbers. We explained their properties and what we can do with them but we actually didn't really define what they \emph{are}. It turns out that they can seen as a specific subset of the ordinal numbers.

%It will turn out that we can use special ordinal numbers for that purpose. Not all ordinal numbers are suitable for that, but some are. That means the set of cardinal numbers is a subset of the set of ordinal numbers. 

\paragraph{The Cardinality Function [VERIFY!]}
What we eventually want is a function into which we can plug in an arbitrary set $A$. This function should produce as output an object that is comparable according to a suitable order relation which should capture the "size" of the set. We will denote this function as $|A|$ like an absolute value. We will sometimes also use the notation $\card(A)$ especially when we want to refer to the function $\card$ itself without giving it an argument. Because we are are concerned with ordering sets according to their sizes, it makes sense to investigate the ordinal numbers because ordering things is their job. To compute the output of $\card$ for a given set $A$, we may first apply a suitable well-ordering function $w(A)$ to it to obtain an ordinal number $\gamma = w(A)$. According to the well ordering theorem, such a function $w$ exists for any set whatsoever. It's an order isomorphism and therefore in particular a bijection, so it doesn't change the property that we want to interpret as size. Starting from this so found ordinal $\gamma$, we now scan the ordinals leftward to find the leftmost\footnote{This leftward scanning process is guaranteed to terminate because the ordinals are well ordered [VERIFY]} ordinal $\alpha$ that allows a bijective map between $\gamma$ and $\alpha$. This ordinal number is the cardinal number that measures the size of $A$. With this definition, cardinal numbers are specific ordinal numbers. They are those ordinal numbers $\alpha$ that have the property that any ordinal number $\beta$ to their left, i.e. any ordinal $\beta < \alpha$ according to the order relation $<$ on the ordinals, allows an injective but not bijective map from $\beta$ to $\alpha$. We can actually also formulate the action of the $\card$ function without the intermediate step of applying the well-ordering function $w$ to $A$ first. In a semi-formal language, we could define the desired function as:
\begin{equation}
\card = 
\begin{cases}
\mathsf{SET} \rightarrow \mathsf{ON} \\
\card(A) = \text{leftmost } \alpha \in \mathsf{ON} \text{ that allows bijection } 
           f: A \rightarrow \alpha
\end{cases}
\end{equation}
It's a function from the class of all sets into the class of the ordinal numbers. Such a definition allows us to plug any set into the function. If we specifically plug ordinal numbers into it, we could plot it as a function mapping ordinals to ordinals. For the finite ordinals, it would be the identity function. For the infinite ordinals, it would be piecewise constant and make an upward jump at every limit ordinal that represents a higher cardinality. For $\omega, \omega+1, \omega^2, \omega^\omega, \varepsilon_0, \ldots$ it would stay constant and just output $\omega$. The first jump would be at $\omega_1$ where it also jumps up to $\omega_1$, etc. The jumps are indeed rather coarsely spaced out. Not even every limit ordinal has a jump. [VERIFY! Much of it is my personal interpretation] ...TBC...

% A good resource is this:
%https://www.youtube.com/watch?v=qijXa3U4Nag
% in SetTheo.txt in the scratch folder, I have taken some notes from this

% https://en.wikipedia.org/wiki/Cardinal_number
% https://en.wikipedia.org/wiki/Successor_cardinal
% https://en.wikipedia.org/wiki/Limit_cardinal
% https://en.wikipedia.org/wiki/Regular_cardinal
% https://en.wikipedia.org/wiki/Cofinality
% https://en.wikipedia.org/wiki/Cardinal_assignment
% https://en.wikipedia.org/wiki/Cardinality
% https://en.wikipedia.org/wiki/Cardinality#Cardinal_numbers
% "for each ordinal a, aleph_{a+1} is the least cardinal number greater than aleph_a"

\paragraph{The Aleph Function} 
The outputs of the $\card$ function are specific ordinal numbers. We are now interested in its ascending sequence of outputs for infinite sets...TBC...is $\aleph_{\alpha} = |\omega_{\alpha}|$? yeah - I think so.

% see:
% https://www.youtube.com/watch?v=hj9hPx23vz0&list=PL2m0OzES6Uw9zK-F8BX8HuGq7HAx9KhQb&index=9


% https://de.wikipedia.org/wiki/Aleph-Funktion
% https://www.mathreference.com/set-card,aleph.html
% https://en.wikipedia.org/wiki/Aleph_number
% https://en.wikipedia.org/wiki/Aleph_number
% https://en.wikipedia.org/wiki/Beth_number
% Limeskardinalzahl?

% Don't confuse things like $\aleph_{\omega}$ and $\card(\omega)$. They are very different: $\card(\omega) = \aleph_0$ and $\alpeh_{\omega}$ is...I don't know...unfathomably big, I guess. It's a kind of "aleph-infinity"

% aleph is a function from the ordinals to the ordinals
% card is used to measure set sizes, aleph is used to order infinities
% Aleph is not surjective: many ordinals are not reached by it
% Aleph is not injective:  many ordinals are mapped to the same ordinal
% It behaves a bit like a floor function on the infinite ordinals, I think. On the finite ordinals,
% it's the identity on the finite numbers
% Aleph_0 is the smallest infinite size because every infinite part of it is still of the same size

% Cantor's Theorem, the Aleph Function, and Beth Numbers
% https://www.youtube.com/watch?v=ejcoVuPXihk

% - Maybe use latex smybols \simeq, \prec, and \preceq  for the cardinality comparison relation
%   see: https://www.math.uci.edu/~xiangwen/pdf/LaTeX-Math-Symbols.pdf
%   down below in the ams list, there are also negated variants of them. But maybe we should use
%%  the regular less than < for size comparison and the prec, succ stuff for ordinal comparison
%   ...the names suggest that this is the intended use. Or maybe use the same symbol, i.e.
%   overload it, with the convention that when both numbers are cardinals, we mean the cardinal
%   version. Whenever one of the numbers is an ordinal, we mean the ordinal version. I mean, we
%   also overload the +, \cdot, etc that way so it seems consistent to also overload the <,=,...
%   





%---------------------------------------------------------------------------------------------------
\subsubsection{Surreal Numbers}
So far, we have constructed the ordinal numbers from the natural numbers and, on a completely different route, we also have constructed the real numbers from the natural numbers. As it stands, these two number systems do not mix. It does not make sense to ask what $\omega \cdot \sqrt{2} + \pi$ is, for example. Real numbers and transfinite ordinal numbers are incompatible. But there is a number system is which such a thing actually does make sense. This system embeds both the ordinal and the real numbers - it even includes the hyperreal numbers from non-standard analysis. That system was discovered by John Conway and the name \emph{surreal numbers} is due to Donald Knuth. Conway himself just called them "numbers" indicating that the set is an all encompassing set of numbers. Indeed, the surreal numbers form the largest possible set of numbers that is a totally ordered field and contain the reals [VERIFY]. The construction of the surreal numbers using sets starts from scratch - it can be seen as an alternative to the von Neumann construction of the naturals (with subsequent construction of the integers, rationals and reals). The numbers are constructed recursively in terms of (ordered) pairs of previously constructed numbers. The construction starts with the number zero which defined as the pair $0 = (\emptyset, \emptyset)$ [TODO: I think it should be $(\{\emptyset\},\{\emptyset\})$ - see comments]. The positive integer $n$ is represented by $n = (n-1,\emptyset)$ and a negative integer $-n$ by $-n = (\emptyset, -(n-1))$. For example, $1=(0,\emptyset), 2=(1,\emptyset), 3=(2,\emptyset), \ldots$ and $-1=(\emptyset,0), -2=(\emptyset,-1), -3=(\emptyset,-2), \ldots$. Conway has introduced a special notation for this - he writes $\{L|R\}$ for the ordered pair $(L,R)$ where the $L,R$ variables are interpreted as "left" and "right" side of the number. When $L$ or $R$ or both are the empty set, the position can be left blank. So, for example, in the Conway notation, the numbers could be written as $3=\{2|\}$, $-3=\{|-2\}$,  $0=\{|\}$ etc.. So far, in our example, we have left one of the sides blank. The interesting thing happens when we do not leave anything blank. For example the pair $\{2|3\}$ represents the number $5/2$. The general rule is that the pair $\{L|R\}$ represents the number $(L+R)/2$, i.e. the average of $L$ and $R$ [VERIFY!]. Geometrically speaking, it represents the midpoint between $L$ and $R$ on the real number line. The construction of the surreal numbers can be visualized as a process in which at each step, a bunch of new numbers are born. Each number has a birthday - that metaphorical terminology is indeed used in the theory of surreal numbers. At the $n$th stage of the construction, we produce the integers $n$ and $-n$ as well as all the midpoints between the numbers that we already had before (not all of them will be new though - but some will). Continuing this recursive construction indefinitely, we will eventually produce all the dyadic rationals, i.e. rational numbers whose denominator is a power of two. At each construction step $n$, the spacing between our existing inner numbers gets smaller and new numbers will be created at the outer frontiers left and right, i.e. at $\pm n$. In the limit of reaching the $\omega$th construction step, we will have produced all the real numbers because we'll have squeezed every possible number between two dyadic rationals with "infinite" denominators, so to speak. We will also have produced the infinite ordinals $\omega$, $-\omega$ and also their reciprocals\footnote{Technically, at the $\omega$th step, the denominators will be $2^\omega$ rather than just $\omega$ but that doesn't matter because $2^\omega = \omega$, I think...VERIFY} The number $1/\omega$ is the infinitesimally small number $\varepsilon$ from the hyperreal numbers ...VERIFY! ....TBC...explain surreal arithmetic  wait - I think, my explanation is wrong: the $L,R$ parts are not previously constructed numbers but sets of previously constructed numbers - so $\{2|3\}$ does not mean $(2,3)$ but rather $(\{2\},\{3\})$?

% https://www.youtube.com/watch?v=_s9J0AQ7wW0&list=PLEAaeELS_iVEwIeQL-BSCmeldZLM5yID1
% -in the context of games, L and R are the left and right player's options
% -the video suggests at 2:27 that even zero is actually $(\{\emptyset\},\{\emptyset\})$
% -another presentation of 0 is {1|1} or {2|2}
% -numbers are special games. the sign determines who wins, the value how many move the win is away




% https://en.wikipedia.org/wiki/Surreal_number
% -Each number is formed from an ordered pair of subsets of numbers already constructed
% -given subsets L and R of numbers such that all the members of L are strictly less than all the members of R, then the pair { L | R } represents a number intermediate in value between all the members of L and all the members of R.

% https://en.wikipedia.org/wiki/Superreal_number
% https://en.wikipedia.org/wiki/Hyperreal_number

% https://de.wikipedia.org/wiki/Surreale_Zahl

% explain also surcomplex numbers

% https://www.ams.org/journals/tran/1985-287-01/S0002-9947-1985-0766225-7/S0002-9947-1985-0766225-7.pdf

% On Numbers and Games
% https://books.google.de/books?id=tXiVo8qA5PQC&redir_esc=y

% THE ABSOLUTE ARITHMETIC CONTINUUM AND THE UNIFICATION OF ALL NUMBERS GREAT AND SMALL
% https://web.archive.org/web/20170703064055/http://www.ohio.edu/people/ehrlich/Unification.pdf

% https://www.scientificamerican.com/article/surreal-numbers-are-a-real-thing-heres-how-to-make-them/

% Surreale Zahlen - reell, infinitesimal, transfinit und noch viel mehr
% https://www.youtube.com/watch?v=Q9gGVUiMo04

% https://www.cut-the-knot.org/WhatIs/Infinity/SurrealNumbers.shtml

% https://en.wikipedia.org/wiki/Ordinal_arithmetic#Exponentiation
% 2^w = w

% Surreal Numbers
% https://www.youtube.com/watch?v=WBYMZawy8a4

%===================================================================================================
\section{Theorems and Hypotheses}

%---------------------------------------------------------------------------------------------------
\subsection{Cantor's Theorem}
Cantor's theorem says that for any set $A$, the power set $\mathcal{P}(A)$ has a strictly greater cardinality than $A$ itself. That is: $|A| < |\mathcal{P}(A)|$. For finite sets, these two cardinalities are related by $|\mathcal{P}(A)| = 2^{|A|}$. [Q: what about infinite sets using cardinal exponentiation?]

% https://en.wikipedia.org/wiki/Cantor%27s_theorem

% Endless Sizes of Infinity, Explained in 5 Levels
% https://www.youtube.com/watch?v=V0VVYsbP0WU&list=PLYteaSMVa_xbyUxGuq456V7wOjt4OFNav&index=1
% -has an easy to follow proof of Cantor's theorem. It's basically a variation of a diagonal argument:
%  -Take a set A (for example, the natural numbers)
%  -Suppose that there exists a bijection f between A and its power set P(A): f: A -> P(A)
%  -We construct another subset B of A that is guaranteed to be not reached by f as follows:
%   -For every x in A, look at what x is mapped to by f, i.e. consider f(x) which is a subset of A
%   -If x is not in f(x), include x in B. If x is in f(x), do not include x in B.
%  -B is guaranteed to be different from any right hand side of f(x) because there is at least one
%   element in f(x) that is not in B or vice versa. B and f(x) have at least one element not in 
%   common - namely, the element x - it occurs only either in f(x) or in B but never in both.
%  -So, B is different from f(x) - for any x in A whatsoever. We have constructed a subset B of A 
%   which is never reached by f.
%  -Therefore, f cannot be bijection. It fails to be surjective.

% https://www.youtube.com/watch?v=iaUwNuaSLUk
% proves the theorem at aroun 7:00. It's essentially the diagonal argument. But I think the explanantion is not quite right. I think, we must negate the (n,n)th entry - i.e. Y becomes N and N becomes Y. He seems to only to the N becomes Y thing but omits the Y becomes N thing. Ah - no - it's OK - n gets added to S iff n not in S_n - so yeah, it is indeed like negating all the Y/N entries. But the explanation would be more clear, if he had just said: negate the (n,n)th entry


% https://www.youtube.com/watch?v=TbeA1rhV0D0
% -explains why Cantor's theorem holds for infinite sets
% -also says that the cardinality of the Aleph-numbers is not Aleph_0

% Cantor's Theorem | Explanation
% https://www.youtube.com/watch?v=caeHVXnLrbU


% Defining Infinity | Infinite Series
% https://www.youtube.com/watch?v=VCksQ6g2yh0
% -Introduces the Beth numbers - they are the numbers resulting from repeatedly applying the power
%  set operations starting at the naturals

% A Hierarchy of Infinities | Infinite Series | PBS Digital Studios
% https://www.youtube.com/watch?v=i7c2qz7sO0I

% How Infinity Explains the Finite | Infinite Series
% https://www.youtube.com/watch?v=oBOZ2WroiVY&t=161s
% -explains how ordinal numbers are used to prove that Goodstein sequences eventually hit zero. It
%  also introduces a theorem that says that the ordinals are really needed for the proof. That may
%  serve as motivation why bother with ordinals

%---------------------------------------------------------------------------------------------------
\subsection{The Cantor-Bernstein Theorem}
To show that two sets $A$ and $B$ have the same cardinality, we must, by definition, show that a bijection between $A$ and $B$ exists. The Cantor-Bernstein theorem asserts that it suffices to show that and injection from $A$ to $B$ and an injection from $B$ to $A$ exists. If these two injections exist, then the theorem assures that also a bijection between $A$ and $B$ exists. Constructing these two injections (or indirectly showing that they exist) can, in some situations, be easier than showing directly that a bijection exists, so the theorem may help us to show that two sets have the same cardinality.

% I think, the theorem can be derived by (VERIFY!):
% -If there exists an injective function f: A to B  then  |A| <= |B|
% -If there exists an injective function f: B to A  then  |B| <= |A|
% -If both exist, then |A| <= |B| and |B| <= |A| and therefore  |A| = |B|

% An "obvious" theorem about infinite sets (The Cantor-Bernstein-Schröder Theorem)
% https://www.youtube.com/watch?v=Nv90aDcwzv0

%---------------------------------------------------------------------------------------------------
\subsection{The Continuum Hypothesis}
We know that $|\mathbb{N}| < |\mathbb{R}|$. The cardinality of the real (i.e. continuous) numbers is strictly greater than the cardinality of the natural numbers. Cantor observed that any infinite subset of the real numbers that he could construct had either the same cardinality as $\mathbb{R}$ itself or the same cardinality as $\mathbb{N}$, so he hypothesized that no cardinality in between these two exists. This is called the \emph{continuum hypothesis} which is sometimes abbreviated as CH. We have already introduced the notation $\aleph_0 = |\mathbb{N}|$ for the cardinality of the natural numbers. The cardinality of the real numbers, i.e. the continuum, is denoted as $\mathfrak{c} = |\mathbb{R}|$ and we also already know that $|\mathbb{R}| = 2^{\aleph_0}$. With these notations, the continuum hypothesis can be written down as: $\aleph_1 = 2^{\aleph_0}$. The generalized CH, also called Cantor's aleph hypothesis, states: $\aleph_{\alpha+1} = 2^{\aleph_{\alpha}}$ for every ordinal number $\alpha$. ....TBC...[todo: explain how GCH means that the aleph numbers equal the beth numbers]

% Q: Does the index of a cardinal number have to be an ordinal number?
% Q: How many cardinal numbers are there? Countably many or uncountably many? Is this question related to the CH?
% -if GCH is false, the beth numbers are sparser than the aleph numbers, i.e. there are aleph 
%  numbers tha don't have a beth partner (Q: why couldn't it be the other way around? or even both
%  ways?)
% -Historically, the undecidability of CH is the first example a mathematically relevant undecidable
%  questions (which were known to exist previously due to Gödel's incompleteness theorems)


%or as $\mathfrak{c} = 2^{\aleph_0}$



%$|\mathbb{R}| = \mathfrak{c} = 2^{\aleph_0}$. CH: $\aleph_1 = 2^{\aleph_0}$?

% ToDo: introduce notation \mathfrak{c}

% https://www.youtube.com/watch?v=iaUwNuaSLUk  Cardinality of the Continuum

\subsubsection{Independence from ZFC}
The quest was now to figure out whether or not the continuum hypothesis is true or false. As usual in axiomatic set theory, in order to prove a hypothesis true or false, one has only the axioms and the logical inference rules to work with (along with, of course, all the lower level theorems that have already been proven). So, the question in more precise terms asks: can we prove the continuum hypothesis true or false, given the ZFC axioms (or any other set of axioms). Kurt Gödel proved in 1940 that from ZFC, it cannot be proved that the continuum hypothesis is false. Paul Cohen proved in 1963 that from ZFC, it cannot be proved that the continuum hypothesis is true. These two results together mean that the continuum hypothesis is actually \emph{independent} from the ZFC axioms. This is sometimes also phrased as being \emph{undecidable} in ZFC. ...TBC...

% https://en.wikipedia.org/wiki/Continuum_hypothesis

\subsubsection{The Mathematical Multiverse}
The undecidability of the CH from ZFC means that, within the system of ZFC, we can create at least two different mathematical universes - one in which the CH is true and one in which it is false. Both of these universes are equally valid in the sense that all the theorems that follow from ZFC alone, i.e. all the mathematical theorems that we use in our usual day-to-day applied math (and many more), are true in both of them VERIFY!...TBC...

% https://en.wikipedia.org/wiki/Von_Neumann_universe


% (Provably) Unprovable and Undisprovable... How??
% https://www.youtube.com/watch?v=1RRpC7FDfEQ&lc=UgwGg_JvrNpHhgO1Vo14AaABAg.A3Uanwul3oeA3_rQWkBsOx
% see my comment there and the replies to it

% https://en.wikipedia.org/wiki/List_of_statements_independent_of_ZFC

% https://en.wikipedia.org/wiki/List_of_statements_independent_of_ZFC#Measure_theory
% CH implies that there exists a function on the unit square whose iterated integrals are not equal. The function is simply the indicator function of an ordering of [0, 1] 

% ToDo:

%\subsubsection{The Generalized Continuum Hypothesis}


%  Cantor's aleph hypothesis

% https://en.wikipedia.org/wiki/Continuum_hypothesis#Generalized_continuum_hypothesis


%---------------------------------------------------------------------------------------------------
\subsection{Consequences of the Axiom of Choice}
The axiom of choice, abbreviated as AC, is the most controversial of all axioms in ZFC. It has a couple of very interesting and partially counterintuitive consequences. Its status in set theory is somewhat comparable to that of Euclid's parallel axiom in geometry. One often tries to avoid using the AC in proofs if possible because then the proofs will hold up in an even bigger mathematical universe, i.e. one in which the AC is not assumed. The AC has been proven to be independent of ZF. That means, it cannot be proven true or false from the ZF axioms. What is most interesting about the AC is that there are a whole bunch of mathematical statements that turn out to be equivalent to the AC in ZF. That means, when we assume the rest of the ZF axioms to be true, then AC implies the given statement $\varphi$ and vice versa. Informally: $(ZF \wedge AC) \Rightarrow \varphi$ and: $(ZF \wedge \varphi) \Rightarrow AC$   ...TBC...



\paragraph{Well Ordering Theorem} 
From the AC, along with the other axioms of ZF, the well ordering theorem can be proven. That means: The AC implies the well ordering theorem. On the other hand, the well ordering theorem also implies the axiom of choice. Trivially, if we have a well ordering of a set in hand, we can utilize that to construct a choice function by saying: just always choose the minimum [VERIFY!]. So, this direction of the mutual implication is easy. The other way is less easy but it also works out. At the end, what that means is that the AC and the well ordering theorem are equivalent.

% The Axiom of Choice
% https://www.youtube.com/watch?v=szfsGJ_PGQ0
% -has more equivalent statements around 28

% Das (berühmt-berüchtigte) Auswahlaxiom
% https://www.youtube.com/watch?v=-qPpiHZ82rw
% -Jede unendliche Menge A läßt sich ijektiv auf A x A abbilden

% https://en.wikipedia.org/wiki/Well-ordering_theorem#History
% "...in first-order logic the well-ordering theorem is equivalent to the axiom of choice...In second-order logic, however, the well-ordering theorem is strictly stronger than the axiom of choice..."
% ...WTF?! How can 2nd order logic "undo" truths that have been proven by 1st order logic?

\paragraph{Zorn's Lemma}
Zorns lemma is another mathemetical statement that turns out to be equivalent to the continuum hypothesis. It states that\footnote{Taken from Wikipedia}: "Suppose a partially ordered set $P$ has the property that every chain in $P$ has an upper bound in $P$. Then the set $P$ contains at least one maximal element.". So - this statement is quite a bit more difficult to understand than the axiom of choice or the well ordering theorem. There is a famous quote of Jerry Bona who put it that way: "The axiom of choice is obviously true, the well ordering principle is obviously false and who can tell about Zorn's lemma?". The thing is: all 3 statements are provably equivalent when ZF is assumed. The quote refers to what we intuitively tend to think. The joke nicely demonstrates an important aspect of the power of mathematical thinking: We can use it to start with things that we can easily nod along with intuitively. From there, we can prove true facts that are totally counterintuitive and we can also prove facts that are not so easily accessible to our intuition.

% If we have a chain, i.e. a set of sets that has a total order, we can use its union as an upper bound

%Set theory can sometimes lead to counterintuitive

% "The axiom of choice is obviously true, the well ordering theorem is obviously false and who can tell about Zorn's lemma?" 

% Zorn's Lemma, The Well-Ordering Theorem, and Undefinability (Version 2.0)
% https://www.youtube.com/watch?v=W8J4eEiAtIk
% -A well-ordering on R is not definable (at around 14:00)

% https://en.wikipedia.org/wiki/Zorn%27s_lemma
% https://de.wikipedia.org/wiki/Lemma_von_Zorn

% Zorn's Lemma Demystified
% https://www.youtube.com/watch?v=oRH-HTrAvxw  by sudgylacmoe
% -has nice intro about order theory at the beginning

\paragraph{Tarski's Theorem}
Tarski proved 1924 that in ZF, the theorem "For every infinite set $A$, there exists a bijection between $A$ and $A \times A$" implies the axiom of choice. The converse is also true and that was already known at that time.  TBC

%That means in ZF, the theorem and the axiom of choice are equivalent. ...? What did I mean by that?
% ..it seems rather that the axiom of choice and the existence of the bijections are equivalent?

% https://en.wikipedia.org/wiki/Tarski%27s_theorem_about_choice

\paragraph{The Banach-Tarski Paradox}

% or rather Banach-Tarski's Paradox?

% Das Banach-Tarski-Paradoxon (Weihnachtsvorlesung 2023)
% https://www.youtube.com/watch?v=fHE94q2g3hs

%---------------------------------------------------------------------------------------------------
\subsection{Gödel's Incompleteness Theorems}
By picking a particular set of axioms, one selects a particular mathematical universe. The hope was to find a set of axioms that is consistent and complete....TBC...

% Consistency means that we cannot derive contradictions, i.e. cannot prove things that are false. Completeness means that all true propositions can be proven

% https://en.wikipedia.org/wiki/G%C3%B6del%27s_incompleteness_theorems

% https://www.youtube.com/watch?v=d0__uZE_x1k
% A theory that is consistent and expressive enough to talk about itself cannot be complete

% https://www.youtube.com/watch?v=qSiLjXlFlYE





%===================================================================================================
\section{Applications}

%---------------------------------------------------------------------------------------------------
\subsection{Foundation of Math}
Set theory can serve as the foundational layer of all of mathematics. We have already seen how the natural numbers can be modeled via sets using the von Neumann construction. In an earlier chapter of the book (page \pageref{Sec:NaiveSetTheory} ff), we have also seen how sets can be used to model ordered pairs, tuples and in a further step also equivalence relations and equivalence classes. Taking that, together with the von Neumann construction of the natural numbers, we were able to model the integer and rational numbers, too. With further set theoretical constructions, namely Dedekind cuts, we could even model the real numbers. By using ordered pairs again, we could also model complex numbers. By using the notion of a function as a special kind of relation, which in turn, is a special kind of set, we can model functions. Let's take pause and look at what we have done so far: We modeled a quite important chunk of the mathematical universe - namely everything from natural numbers up to complex functions - with nothing but sets. We can also model yet more mathematical ideas - like functionals, operators, points, vectors, matrices, etc. ...TBC...

% It's quite astonishing that the simple idea of a set can lead to such a rich and diverse universe of mathematical objects

% It's tempting to construct a sort of graph where the nodes are the mathematical concepts (like number, function, matrix, operator, manifold) and we use edges to mean "can be modeled via". Maybe we need different kinds of edges or maybe we need a multigraph - we'll see.

% ToDo: draw a partial graph


%Building on that, we could also


%---------------------------------------------------------------------------------------------------
\subsection{Proofs}
Set theory is used extensively in formal proofs of mathematical theorems...TBC...

\subsubsection{Goodstein's Theorem}
Goodstein's theorem is a good example for a mathematical theorem that makes a statement about sequences of natural numbers but whose proof needs the machinery of ordinal numbers. ...TBC...


\begin{comment}

ToDo:

-Re-organize the matrial about transfinite numbers
 -Pull the order theory and transfinite recursion stuff out of the ordinal numbers section
  and put it before the "Numbers as Sets" section
 -start with Beth numbers (they are the easiest to understand)
 -then introduce cardinal numbers in a perliminary/informal way, i.e. without introducing 
  them as special ordinal numbers
 -then intrpduce ordinal numbers
 -then revisit cardinal numbers and define them properly as special ordinal numbers
 -the rationale in splitting the material about cardinals into two parts is that they are
  easier to understand

-cardinality = size type, "ordinality" = order type (not an official term, I think)

\section{Naive and Axiomatic Set Theory}

\section{Order Theory}
...
\subsection{Transfinite Processes}

\section{Numbers as Sets}
...
\subsection{Transfinite Numbers}

\subsubsection{Beth Numbers}
\subsubsection{Cardinal Numbers Informally}
\subsubsection{Ordinal Numbers}
\subsubsection{Cardinal Numbers Formally}

\section{Theorems and Hypotheses}



Lebesgue-Measure:
 https://www.youtube.com/watch?v=0VD3BWDLmU0

...set theory sometimes appears like a "write-only-language

% See:
% https://aleph1.info/?call=Puc&permalink=mengenlehre1
% Full online book about set theory (in German)
% pdf version:
% https://aleph1.info/Resource?method=get&obj=Pdf&name=mengenlehre1.pdf

Das Zermelo-Fraenkel-Axiomensystem der Mengenlehre (ZF)
https://www.youtube.com/watch?v=U10UYyXv5gM&list=PLb0zKSynM2PAuxxtMK1bxYPV_bUoPtpTB&index=1&t=0s

Bourbaki:
-Collective of authors writing a series of books in the early 1900s that aimed to build up all of math axiomatically. The first volume was about set theory

Was sind Kardinalzahlen? Was besagt die Kontinuumshypothese?
https://www.youtube.com/watch?v=qijXa3U4Nag&list=PLb0zKSynM2PCrgebQsfrzEsUIuA0I_wdG&index=1&t=0s


For the ZFC axioms, see:

https://en.wikipedia.org/wiki/Zermelo%E2%80%93Fraenkel_set_theory#Axioms
1. Axiom of extensionality
2. Axiom of regularity (also called the axiom of foundation)
3. Axiom schema of specification (or of separation, or of restricted comprehension)
4. Axiom of pairing
5. Axiom of union
6. Axiom schema of replacement
7. Axiom of infinity
8. Axiom of power set
9. Axiom of choice

https://de.wikipedia.org/wiki/Zermelo-Fraenkel-Mengenlehre#Die_Axiome_von_ZF_und_ZFC
 1. Extensionalitätsaxiom
 2. Leermengenaxiom     (not included in the English wiki page - it is redundant)
 3. Paarmengenaxiom
 4. Vereinigungsaxiom
 5. Unendlichkeitsaxiom
 6. Potenzmengenaxiom
 7. Fundierungsaxiom
 8. Aussonderungsaxiom
 9. Ersetzungsaxiom
10. Auswahlaxiom




https://en.wikipedia.org/wiki/Foundations_of_mathematics
https://en.wikipedia.org/wiki/Controversy_over_Cantor%27s_theory


https://en.wikipedia.org/wiki/Type_theory

https://www.youtube.com/watch?v=szfsGJ_PGQ0  The Axiom of Choice


https://www.youtube.com/watch?v=JUVYBBlxw7E
Math PhD Student reacts to "Proof" of Continuum Hypothesis
at 9:09, he explains Aleph 1
-the explanation seems to imply that the aleph numbers themselves are countable - verify!

How to Divide by "Zero" | Infinite Series
https://www.youtube.com/watch?v=uxpowBoPieQ
-What they mean here by "divide by zero" is to form a quotient set wrt an equivalence relation.
 The modular integers mod 5: Z/5Z are in a sense defined by a division by zero because 5Z is the
 equivalence class of zero.

Essence of Set Theory:
https://www.youtube.com/playlist?list=PL2m0OzES6Uw9zK-F8BX8HuGq7HAx9KhQb
-Set theory is also the branch of math that closer investigates the idea of infinity

Ordinals vs Cardinals (and how many algebraic numbers are there?)
https://www.youtube.com/watch?v=pSSsZLTMDq0
-at 4:55 it seems to imply that a countable set is always orderable?
-explains nicely why the algebraic numbers are countable

Other potentially interesting stuff:
https://en.wikipedia.org/wiki/Supertransitive_class
https://en.wikipedia.org/wiki/Von_Neumann_universe
https://de.wikipedia.org/wiki/Von-Neumann-Hierarchie

https://en.wikipedia.org/wiki/Veblen_function
https://en.wikipedia.org/wiki/Ordinal_arithmetic#Cantor_normal_form

https://en.wikipedia.org/wiki/Primitive_recursive_set_function
https://en.wikipedia.org/wiki/Constructible_universe

https://en.wikipedia.org/wiki/Filter_(set_theory)
https://en.wikipedia.org/wiki/Ultrafilter_on_a_set


https://en.wikipedia.org/wiki/Derived_set_(mathematics)
-Deiser's book says, we obtain the derivative (on R) by first adding the boundary to the set and
 then removing all isolated points - not sure, if that's the same notion, though.


How to Construct Infinite Sets
https://www.youtube.com/watch?v=dz7j38sCUkI


https://www.youtube.com/watch?v=sQ_znt_03Ws&list=PLTpJiP-hRaV2ZKBQipaeWp7Yxk_IK5X-5&index=8

-The world of sets can be organized as a directed graph where each set is represented by a
 node and the element relation is represented by a directed edge. We can organize the sets into
 levels:
 0: E
 1: {E}
 2: {{E}}, {{E},E}
 3: {{{E}}}, ....
 The sets on each level may have incoming edges from all the sets in layers above. Maybe implement
 it. i think, on each level, we just take all the sets of the previous level and add to them each
 set of the previus level as element? Is that right? In order to get a new set in the level n that
 has not been created before, we need to include at least one set from the level n-1. Ah - just
 adding the sets from the previousl level as elements is not enough. We must add all non-empty
 collections of sets from level n-1. 

-To modeling the mathematical uinverse with sets, let's take an inventory of of objects that
 we encounter:
 numbers, functions, binary operations (add, mul,...), sets
-We could imagine the world of math a a sort of graph where the objects (numbers, sets,) are nodes
 and the could be different kinds of directed edges that mean: "can be modeled as", 
 "is special case of"
-For each type of object, we could 
 truth values:
 propositions:
 sets: fundamental
 numbers: can be modeled as sets (von Neumann construction, Dedekind cuts, etc.)
 pairs: can be modeled as sets (Kuratowski pair)
 tuples: can be modeled by nested pairs
 relations: special case of sets (namely, sets of pairs)
 functions: special case of relations
 equivalences: special case of relations
 binary operations on a set A: special case of functions, namly f: A x A  ->  A
 arrangements (like vectors, matrices): can be modeled using nested tuples
 points/vectors: can be modeled as tuples
 manifolds: can be modeled as sets of points



https://www.youtube.com/watch?v=c4oKdtUtJOk  The Cantor Set
- Shows how the Cantor set (which is uncountable but has length/measure zero) can be put into
  bijection with the real number line. It actually shows how it could be put in bijection with the
  unit interval, but we already know how to put that into bijection with the real number line.
  The elements of the Cantor set are those whose ternary expansion contains no 1s. We could just 
  re-interpret the 2s as 1s and have a mapping between the Cantor set and unit interval in binary
  expansion (I think, the video gives another bijection, though - verify).


The Most Controversal PARADOX In Mathematics
https://www.youtube.com/watch?v=Pb3QImXXPcY


Zermelo-Fraenkel Set Theory
https://www.youtube.com/watch?v=IfYP37x6bN4


\end{comment}