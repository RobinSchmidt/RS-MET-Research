\chapter{Set Theory}


\section{Naive and Axiomatic Set Theory}

\subsection{Antinomies}

\subsection{Systems of Axioms}


\subsubsection{Zermelo-Fraenkel}

\paragraph{The Axiom of Choice}

%The Axiom of Choice | Epic Math Time
%https://www.youtube.com/watch?v=Nnt4hyJYfGA
%-Shows how axiom of choice is used to show that every surjection f has a right inverse g
% such that f(g(y)) = y for all y in Y without explicitly specifying what g does. The proof leaves
% the task of picking an x in X for each y in Y (such that g(y) = x, f(x) = y) to the reader. Take 
% as example f(x) = x^2, compare the proof to the simpler proof for the bijective f(x) = x^3


\begin{comment}

ToDo:
-Infinite Sets 
 -Construction of natural, integer, rational and real numbers
 -Cardinal Numbers
 -Ordinal Numbers (as generalization of cardinal numbers)
 -Surreal Numbers

Cardilality:
-If there exists an injective function f: A to B  then  |A| <= |B|

Set Containment:
-If for all x in A, x in B, then A \subseteq B

Lebesgue-Measure:
 https://www.youtube.com/watch?v=0VD3BWDLmU0

...set theory sometimes appears like a "write-only-language"

\end{comment}