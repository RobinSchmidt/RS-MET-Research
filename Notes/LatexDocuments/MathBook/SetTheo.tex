\chapter{Set Theory}
[this chapter is still very sketchy and incomplete and needs a lot of proofreading]

% For the logic chapter that should come before this chapter:
%\begin{eqnarray}
%\forall x : \varphi(x) \Leftrightarrow \neg \exists x : \neg \varphi(x) \\
%\exists x : \varphi(x) \Leftrightarrow \neg \forall x : \neg \varphi(x)
%\end{eqnarray}

% https://en.wikipedia.org/wiki/Theory_(mathematical_logic)#First-order_theories
% https://en.wikipedia.org/wiki/Method_of_analytic_tableaux
% https://en.wikipedia.org/wiki/Resolution_(logic)
% https://en.wikipedia.org/wiki/G%C3%B6del%27s_completeness_theorem
% https://en.wikipedia.org/wiki/List_of_first-order_theories

% https://en.wikipedia.org/wiki/Second-order_logic
% https://en.wikipedia.org/wiki/Higher-order_logic

%===================================================================================================
\section{Naive and Axiomatic Set Theory}

% ToDo: 
% -give Cantor's original defition of a set

\subsection{Cantor's Definition}
In the introductory chapter of this book, we already encountered Georg Cantor's following definition of a set: "A set is a gathering together into a whole of definite, distinct objects of our perception or of our thought which are called elements of the set.". As we noted there, a set is just a bunch of things. But what are these "things" then? In mathematics, these are typically mathematical objects like numbers or vectors or functions or operators etc.. But talking about those things requires that we already know what these things are. When we want to build math really from the ground up, we cannot assume the existence of such things. We have to think about what a number even is in the first place.

% https://plato.stanford.edu/entries/set-theory/
% https://de.wikipedia.org/wiki/Naive_Mengenlehre
% Unter einer „Menge“ verstehen wir jede Zusammenfassung M von bestimmten wohlunterschiedenen Objekten m unserer Anschauung oder unseres Denkens (welche die „Elemente“ von M genannt werden) zu einem Ganzen.

% Mention "Urelement", "atomic element"?
% https://en.wikipedia.org/wiki/Urelement
%  urelements are in some sense dual to proper classes: urelements cannot have members whereas proper classes cannot be members. 

\subsection{Intuitions}
So, let's take a step back and forget about the existence of numbers and ask ourselves, what kinds of objects a set could possibly contain. So far, we have only postulated the existence of objects that we want to call sets. So, with nothing else in hand, how about letting sets to contain other sets? Using some terminology from computer science, maybe we could build a sort of recursive data structure pretty much like a tree. A tree consists of nodes and each node can have child nodes. Let's apply the idea to sets: a set may consist of sets which we call \emph{elements}. The elements play the role of the child nodes. Let's further postulate that there also exists a set that has no elements whatsoever and let's call that special set the \emph{empty set}. Let's also assume that the identity of a set is determined by its content: if two sets have the same elements, they are the same set. There is no notion of order and it doesn't matter, if an element occurs more than once. From that it immediately follows that the empty set is unique. It doesn't make sense to talk about \emph{an} empty set. It really is \emph{the} empty set. So far, we have the following: (1) A set can contain sets as elements. (2) There is a special set without any elements. (3) The identity of a set is determined by its elements.

% https://www.youtube.com/watch?v=lNM6S2UUHcQ  at 25:12

%\medskip
%---------------------------------------------------------------------------------------------------
%\subsubsection{Sets of Sets}
\subsubsection{Construction of Sets}
OK, so far, we know about one set: the empty set. It has no elements. It plays a fundamental role, so we shall give it a symbol. We'll denote it by $\emptyset$. Let's furthermore stipulate that we write sets as lists of elements inside curly braces. For example, a set that contains only the empty set would be denoted by $\{ \emptyset \}$. Let's also stipulate that we can give names to sets via the notation $A = \ldots$ where on the right side, some set in the curly brace notation can be assigned to the name $A$. Let's make an experiment: take $A = \emptyset$ and $B = \{ \emptyset \}$. That means, the set $A$ contains no elements whatsoever and the set $B$ has one single element, namely the empty set. We have said that two sets are equal if and only if they have the same elements. $A$ has no elements and $B$ has one element. We conclude that $A$ and $B$ are different. Hey! We now know already two sets which are indeed different from one another! Sets that contain only a single element do also have a special name. They are called \emph{singleton set}s or just \emph{singleton}s. Let's try to create some more sets from what we already have. How about $C = \{ B \} = \{ \{ \emptyset \} \}$. We can do that because we have said that inside the curly braces, we can write down a list of comma separated sets and $B$ is a indeed a set. But is it yet another set or is it one of those two that we already know? It's certainly not equal to $A$ since it contains an element, namely $B$, whereas $A$ contains no elements. Could it be equal to $B$? Well, $B$ contains $\emptyset$ and $C$ contains $B$. So, since $B \neq \emptyset$, we see that $B$ and $C$ have different contents, so they are different sets. We have produced yet another set that is different from those we had before. You can easily convince yourself that $D = \{ A, B\} = \{ \emptyset, \{\emptyset\} \}$ is yet another set. After all, it has two elements. Now it should be apparent that we can recursively build an infinite number of different sets. Just by putting a set into curly braces, i.e. turning a set $A$ into the singleton $\{A\}$ that contains $A$ and doing that recursively, we can already build an infinite number of different sets. But by putting several sets into one we have even many more possibilities to construct many different sets. We can build arbitrarily many distinguishable objects in various ways. It turns out that by structuring our way to construct sets, we can build arbitrarily complex structures that are rich enough to represent all the objects from our mathematical world. We have already seen one important way to structure our construction of sets. By defining $D = \{ A, B \}$, we have created a \emph{pair} $D$ from two given sets $A$ and $B$. As we will see, this pair creation is one of a few important mechanisms to build new sets from existing sets. By the way, if in this pair creation, we take both sets to be the same, we get something like $D = \{ A, A \} = \{ A \}$ because duplication doesn't matter, i.e. $ \{ A, A \} = \{ A \}$. Thus, with our pair creation operation, we can also build singletons as a special case.

% To create the set {A,B,C}, we could first create {A,B}, then {C} and then form the union. So, with pairing and union, we can recursively put sets with an arbitrary number of elements together

%---------------------------------------------------------------------------------------------------
%\medskip
\subsubsection{The Element Relation}
We have informally stated that sets can have other sets as elements. As we will see later, that establishes a \emph{relation} between sets. The idea of a relation itself can be formalized in terms of sets. That seems cyclic and kinda is (I think), but for the time being, let's think about the is-element-of relation in informal terms.  When implementing a data structure to represent sets in its set theoretical sense, we would expect the data structure to have a functionality that allows us to test, whether or not one set is an element of another set. In object oriented languages, we could perhaps have a class \verb|Set| that has a member function \verb|hasElement(Set x)| that returns a boolean value - true, if the object we are calling it on has the set \verb|x| as element and false otherwise. That is basically all a set can do. Any other functionality must be built on top of that.

% ToDo: explain how to implement the subset relation and equality comparisons based on hasElement


% ToDo: Maybe talk a bit about the different possible textual representations of sets like { 1,2,3 }, { 2, 3, 1 }, { 1, 2, 3, 1 }, ... - this is important for an implementation - equlatiy comparisons should not depend on the representation. Maybe link my implementation. Maybe append an "Implementation" section at the end. Maybe do this also for other math topics - like the algorithms for polynomial multiplication (convolution), etc. - make some accompanying material - Code for C++, Python, Sage, etc.




%---------------------------------------------------------------------------------------------------
\subsection{Notational Conventions}  In the context of predicate logic, variables are typically denoted by lowercase letters from the end of the alphabet such as $x,y,z$. In general mathematics, sets are usually denoted by uppercase letters from the begin of the alphabet such as $A,B,C$ and set elements usually by lowercase $a,b,c$. In axiomatic set theory, sets are the only thing that exists - the elements of sets are also sets - and sets appear as variables in predicate logical formulas. So we have a conflict here - which of the notational conventions should we adopt? Here, I adopt the following convention: In contexts where a lot of predicate-logical formulas appear, I will use $x,y,z, \ldots$ because using the other convention tends to look ugly there. These contexts are typically the lower levels of the theory. In higher level contexts, where predicate logical formulas become less common, I'll prefer  $A,B,C, \ldots$ for sets. [TODO: introduce $\in$, $\notin$, $\ni$]

%---------------------------------------------------------------------------------------------------
\subsection{Unrestricted Comprehension Principle}
In naive set theory, there is one simple and very general principle that allows us to construct sets. Let $x$ be any kind of (mathematical) object and $\varphi(x)$ a predicate, i.e. a function into which we can plug in a variable $x$ and then $\varphi$ will return a boolean value that tells us whether or not $x$ has the property modeled by our predicate. With $\varphi$ in hand and using the set builder notation from the introductory chapter, we could build the set $S = \{x: \varphi(x)\}$ of all objects that satisfy our predicate. But that was what we did in "naive" theory.
%That we can always do this, for any $\varphi$ whatsoever, is called the \emph{unrestricted comprehension} axiom, sometimes also \emph{naive comprehension} axiom. 

% https://en.wikipedia.org/wiki/Universal_set
% https://en.wikipedia.org/wiki/Axiom_schema_of_specification#Unrestricted_comprehension
% https://de.wikipedia.org/wiki/Naive_Mengenlehre
% https://aleph1.info/?call=Puc&permalink=mengenlehre1_1_1_Z10
% In Weitz's video über Ordinalzahlen bei 26:45 kommt diese Definition aber wieder - diesmal für Klassen

%---------------------------------------------------------------------------------------------------
\subsection{Paradoxes}
For "axiomatic" set theory, it turns out that this unrestricted comprehension principle is a bit too liberal. With unrestricted comprehension, we can build paradoxical sets. This is obviously undesirable for a theory that aims to be the foundation of mathematics. These paradoxes are sometimes also called antinomies. Let's now have a look at some of the most famous ones.
% https://en.wikipedia.org/wiki/Antinomy

\subsubsection{Russel's Paradox}
Define the Russel set\footnote{I think, it's actually not a \emph{set} but a \emph{proper class}? Verify!} $R$ as the set of all sets that do not contain themselves as an element. That means $R = \{x: x \notin x\}$. That means $x \in R \Leftrightarrow x \notin x$, i.e. $x$ is an element of $R$ if and only if $x$ is not an element of $x$. Here, $x$ can be \emph{any} set - so for any set $x$ whatsoever, we can ask the question whether or not it is an element of $R$. Now ask what happens when we let $x = R$, i.e. we ask whether or not $R$ is an element of $R$. From $x \in R \Leftrightarrow x \notin x$ with $x = R$, we immediately get the contradiction $R \in R \Leftrightarrow R \notin R$. That is obviously nonsensical. So what went wrong?

\paragraph{Solution} Apparently, when allowing to construct sets in such a completely unrestricted way, we can create paradoxical situations. The culprit lies in the assumption that sets can contain themselves as elements in the first place. Maybe we should disallow situations where $x \in x$ altogether. That seems to be a weird and counterintuitive situation anyway - why would anyone assume that a set could contain itself? We should perhaps even disallow a bit more - things like $x \in y \wedge y \in x$ or $x \in y \wedge y \in z \wedge z \in x$, etc. - i.e. disallow all recursive element relations that are cyclic, i.e. loop back on themselves. Of these, $x \in x$ is just the simplest one but the more complicated ones do also seem weird. Away with them!

% https://de.wikipedia.org/wiki/Russellsche_Antinomie
% https://en.wikipedia.org/wiki/Russell%27s_paradox


%\subsubsection{Cantor's Paradox}
%Calling Cantor's initial approach to set theory "naive" might be a bit unfair to him. He was, in fact, aware that his initial approach has some problems and came up with a paradoxon himself. It is actually a generalized version of Russel's paradox (VERIFY!)...TBC...

% https://aleph1.info/?call=Puc&permalink=mengenlehre1_1_13

% Nah, let's not metion Cantor's paradox here - it requires Cantor's theorem that the power set of a set is larger than the set itself - but this theorem actually needs a lot of set theoretic preliminaries. Or maybe mention it but say that we can understand it only later

%---------------------------------------------------------------------------------------------------
\subsection{Systems of Axioms}
Set theoretic systems of axioms are like the atomic building blocks that we can use for the construction of arbitrary sets. An axiom may either directly state that some kind of set exists or it may tell us, how we can build new sets from existing sets. The possible construction steps can then be applied recursively to build sets of arbitrary complexity. The initial freedom, chaos and anarchy of the unrestricted comprehension axiom which leads us to all these paradoxa will be gone. We will only have some well defined ways to build sets.

\subsubsection{Zermelo, Fraenkel}
The most common system of axioms that mathematicians use today is the one proposed by Zermelo and Frankel and it is abbreviated by ZF. It is usually augmented with the axiom of choice in which case it is abbreviated as ZFC. There are multiple equivalent ways to choose the set of axioms for ZF and some of them may be redundant in the sense that they contain axioms that are not strictly necessary, i.e. could be proved as theorems from some of the other axioms. If redundant axioms are present, then their purpose make the system more intuitive and convenient [VERIFY!].

%...TBC...

% https://en.wikipedia.org/wiki/Zermelo%E2%80%93Fraenkel_set_theory
% https://de.wikipedia.org/wiki/Zermelo-Fraenkel-Mengenlehre

% https://de.wikipedia.org/wiki/Zermelo-Mengenlehre

% https://de.wikipedia.org/wiki/Zermelo-Fraenkel-Mengenlehre#Die_Axiome_von_ZF_und_ZFC
% -This actually has an axiom for the empty set and the axiom for infinite sets is replaced by the axiom that there exists a set x such that when y \in x then also y \cup \{ y \} \in x
% -The axioms as stated there seem to make more intuitive sense than on the english wikipedia page


\paragraph{Axiom of Empty Set}
This axiom states that there is an empty set. We can write it formally as:
\begin{equation}
\exists x \forall y: \neg (y \in x)
\end{equation}
This is actually one of the redundant axioms and not explicitly listed in the English wikipedia article on ZFC but in the German article, it is. It is redundant because in any set theory which has the axiom schema of specification (which we do have in ZF) and axiomatically postulates existence of any set (as the axiom of infinity in ZF does), the existence of the empty set can be derived as a theorem and therefore doesn't need to be an axiom. The empty set is unique due to the axiom of extensionality. It is usually denoted by $\emptyset$ or, less often, by $\{ \}$.

% https://en.wikipedia.org/wiki/Axiom_of_empty_set

\paragraph{Axiom of Extensionality}
This axiom states that two sets are equal iff they contain the same elements. Formally, this means that
\begin{equation}
\forall x \forall y \forall w: 
(w \in x \Leftrightarrow w \in y) \Rightarrow (x = y)
\end{equation}
From the axiom, it follows that the order of the elements in a set doesn't matter and element duplication also doesn't matter. [VERIFY! TODO: give the statement of the axiom without the equals sign, explain the fancy name - what does "extensionality" mean?]

% https://en.wikipedia.org/wiki/Axiom_of_extensionality

\paragraph{Axiom of Pairing}
This axiom states that whenever we have two sets $x$ and $y$, then there is a set $z$ whose elements are precisely $x$ and $y$. This set is given by $z = \{x, y\}$ and called the \emph{pair} of $x$ and $y$. We can write it formally as:
\begin{equation}
\forall x \forall y \exists z \forall w: (w \in z) \Leftrightarrow (w = x \vee w = y)
\end{equation}
The pair is unique by virtue of the axiom of extensionality. The axiom of pairing also allows us to build \emph{singletons} by simply letting $x = y$ such that  $z = \{x, y\} = \{x, x\} = \{x\}$. It also lets us build the \emph{ordered pair} $(x,y) = \{ \{x\}, \{x, y\} \}$ and from ordered pairs we can recursively build ordered $n$-tuples as $(x_1, x_2, \ldots, x_n) = ((x_1, x_2, \ldots, x_{n-1}), x_n)$. 

% in the standard formulation of the Zermelo–Fraenkel set theory, the axiom of pairing follows from the axiom schema of replacement applied to any given set with two or more elements, and thus it is sometimes omitted. 

% https://en.wikipedia.org/wiki/Axiom_of_pairing

\paragraph{Axiom of Union}
The axiom states that for each set $x$ there is a set $y$ whose elements are precisely the elements of the elements of $x$.

% Can we write it like this?:
%\begin{equation}
%\forall x \forall y \exists z \forall w: w \in z \Leftrightarrow w \in x \vee w \in y
%\end{equation}

% Imagine a set $x = \{ x_1, x_2, x_3, \ldots \} $ of sets. From every set $x_i$, pour all elements into the set $y$. The set $y$ is the union of all the sets $x_i$.


% https://en.wikipedia.org/wiki/Axiom_of_union
% https://en.wikipedia.org/wiki/Axiom_of_union#Relation_to_Intersection
% https://math-garden.com/unit/nst-unions/


\paragraph{Axiom of Regularity}
This axiom states that every non-empty set $x$ contains an element that is disjoint from $x$. It is also known as the axiom of foundation. ...TBC...

% The axiom of regularity together with the axiom of pairing implies that no set is an element of itself, and that there is no infinite sequence (an) such that ai+1 is an element of ai for all i. 

% https://en.wikipedia.org/wiki/Axiom_of_regularity

% https://en.wikipedia.org/wiki/Universal_set#Regularity_and_pairing
% -Regularity together with pairing frobids the existence of sets that contain themselves.

% I think, we need the Fundierungsaxiom and need to disallow sets that can contain themselves
% https://de.wikipedia.org/wiki/Fundierungsaxiom
% Es gibt somit auch keine Menge, die sich selbst als Elemente enthält

% https://en.wikipedia.org/wiki/Zermelo%E2%80%93Fraenkel_set_theory#2._Axiom_of_regularity_(also_called_the_axiom_of_foundation)

\paragraph{Axiom of power set}
% https://en.wikipedia.org/wiki/Axiom_of_power_set


\paragraph{Axiom of Infinity}
% https://en.wikipedia.org/wiki/Axiom_of_infinity


\paragraph{Axiom Schema of Specification}
% aka restricted comprehension

% https://en.wikipedia.org/wiki/Axiom_schema_of_specification


% The Weitz video on ZF explains it quite well. I think, the essence is that, in order to specify a set via a predicate, we first need to say, from which superset (or universal set) we form our set. A set builder notation $S = \{x: P(x)\}$ is not enough. We need to say something like  $S = \{x \in U: P(x)\}$ where $U$ is the universal set or superset from which we draw an x.

%A predictate like P(x) is not enough. We must say 

% One important difference to unrestricted comprehension is that the predicate gets applied to objects that already are members of some set - the objects are not plucked out of thin air (i.e. a hypothesized universal set of all possible things) a

% the predicate can only depend on the object itself, not on what other objects are present in the set. it's context independent. An example for a context dependent predicate for a set of numbers could be "larger than the average" - such things are not possible. But it could perpahse depend on a fixed other set like: x from A belongs to B if its (not) in C - that would form the intersection of A and C

\paragraph{Axiom Schema of Replacement}
% https://en.wikipedia.org/wiki/Axiom_schema_of_replacement


\subsubsection{The Axiom of Choice}
The axiom of choice states that for every set of nonempty sets $x = \{ x_1, x_2, x_3, \ldots \}$, we can define a family of choice functions $f_i$ such that each $f_i$ picks an element from each of the sets $x_i$ and we can pour all these chosen elements into a new set.

% Has proven to be independent from the other ZF axioms.
% Plays a role similar to Euclid's 5th postulate.

% The axiom of choice lets us select elements from a given family sets and then build new sets from these selected elements

% https://en.wikipedia.org/wiki/Axiom_of_choice

% https://de.wikipedia.org/wiki/Auswahlaxiom

% https://www.spektrum.de/lexikon/mathematik/auswahlaxiom/390

% https://aleph1.info/?call=Puc&permalink=mengenlehre1_3_1_Z13

% Das (berühmt-berüchtigte) Auswahlaxiom
% https://www.youtube.com/watch?v=-qPpiHZ82rw
% 12:20 For every infinite set A, there's a bijection between A and A x A
% Maybe add to the Theorems section

%The Axiom of Choice | Epic Math Time
%https://www.youtube.com/watch?v=Nnt4hyJYfGA
%-Shows how axiom of choice is used to show that every surjection f has a right inverse g
% such that f(g(y)) = y for all y in Y without explicitly specifying what g does. The proof leaves
% the task of picking an x in X for each y in Y (such that g(y) = x, f(x) = y) to the reader. Take 
% as example f(x) = x^2, compare the proof to the simpler proof for the bijective f(x) = x^3

% The axiom of choice is equivalent to:
% https://en.wikipedia.org/wiki/Well-ordering_theorem
% https://en.wikipedia.org/wiki/Zorn%27s_lemma
% Jerry Bona made the joke: "The axiom of choice is obviously true, the well ordering principle obviously false and who can tell about Zorn's lemma?" 
% Tarski's theorem is also equivalent
% also: trichotomy for cardinalities of sets
% see: https://www.youtube.com/watch?v=-qPpiHZ82rw  at 23 min
% there are even more equivalent statements


% https://www.aleph1.info/Resource?call=Puc&permalink=mengenlehre1
% https://www.aleph1.info/Resource?method=get&obj=Pdf&name=ema22.pdf&pagestart=279&pageend=298

% Formulierung halb natürlich, halb prädikatenlogisch:
% https://link.springer.com/chapter/10.1007/978-3-662-68094-0_7


% https://en.wikipedia.org/wiki/Von_Neumann%E2%80%93Bernays%E2%80%93G%C3%B6del_set_theory
% https://de.wikipedia.org/wiki/Neumann-Bernays-G%C3%B6del-Mengenlehre


% https://en.wikipedia.org/wiki/Kripke%E2%80%93Platek_set_theory_with_urelements

% https://de.wikipedia.org/wiki/Scottsches_Axiomensystem
% https://de.wikipedia.org/wiki/New_Foundations

\subsubsection{Other Axioms and Systems}
In this section, we will mention a couple of other possible axioms that could be added to an existing system of the axioms such as of ZF(C) or could be used to build new, alternative axiomatic systems. We'll also briefly look at some of these alternative systems.


\paragraph{Axiom of Constructibility} ...TBC...

% https://en.wikipedia.org/wiki/Axiom_of_constructibility
% https://en.wikipedia.org/wiki/Constructible_universe


%\paragraph{Axiom Schema of Adjunction}
%https://en.wikipedia.org/wiki/Axiom_of_adjunction


%\paragraph{Axiom Schema of Induction}
% https://en.wikipedia.org/wiki/Epsilon-induction

\paragraph{The NBG System by Neumann, Bernays, Gödel} ...TBC...

% Treat also:

% https://en.wikipedia.org/wiki/General_set_theory
%  sufficient for all mathematics not requiring infinite sets, and is the weakest known set theory whose theorems include the Peano axioms.

% https://en.wikipedia.org/wiki/Kripke%E2%80%93Platek_set_theory

% https://en.wikipedia.org/wiki/Tarski%E2%80%93Grothendieck_set_theory

% https://de.wikipedia.org/wiki/Ackermann-Mengenlehre

% Questions: 
% -How does the empty set arise from the axioms?
% -How do we define intersection, complement and product of two sets?

%---------------------------------------------------------------------------------------------------
\subsection{Classes}
% Or maybe call the subsection "Other Things in Set Theory" and also lits Types, Categories, Urelements, etc

% https://en.wikipedia.org/wiki/Class_(set_theory)


%===================================================================================================
\section{Numbers as Sets}
In set theory, sets are the only thing that exists. Every mathematical object must somehow be viewed as some sort of set. To do all the cool math things that we love (or hate) so much, we obviously need numbers. So, we somehow need to build numbers from sets. Set theorists will talk about things like subsets of a number. This will at first sound totally nonsensical - what the heck is a subset of $10$ or $3/7$ or $\pi$ supposed to mean? But in set theory, numbers indeed \emph{are} sets (and therefore can have subsets) and we need to get used to this point of view.

% 

%because we usually do not envision a number as a set. 
...TBC... 

% https://de.wikipedia.org/wiki/Zermelo-Fraenkel-Mengenlehre#ZF_mit_Urelementen
% https://de.wikipedia.org/wiki/Urelement

%---------------------------------------------------------------------------------------------------
\subsection{Construction of the Number System}
In everyday math, we usually take numbers for granted and do not really think much about them. In set theory, all the number systems that we commonly use must first be \emph{constructed} using only sets. We'll show here, how this could be done.

% https://cs.uwaterloo.ca/~alopez-o/math-faq/math-faq.pdf  pg 9 ff
% https://cs.uwaterloo.ca/~alopez-o/math-faq/node11.html

\subsubsection{Natural Numbers}
The most basic of all number systems is the system of natural numbers. We can construct the set of natural number from ZF as follows. Given any set $x$, define the successor function as $s(x) = x \cup \{ x \}$. Then we give the names $1,2,3,\ldots$ to the following sets [VERIFY! Maybe format it better]: 

\medskip
\begin{tabular}{l l l l l}
$0=$ &         & 
               & $\emptyset =$                                               
               & $ \{ \} $                                         \\
$1=$ & $s(0)=$ & $\emptyset \cup \{ \emptyset \} =$             
               & $ \{ \emptyset \} =$ 
               & $ \{ 0 \}$                                         \\
$2=$ & $s(1)=$ & $ \{ \emptyset \} \cup \{  \{ \emptyset \} \}=$ 
               & $ \{\emptyset,  \{ \emptyset \} \} =$       
               & $ \{ 0,  1 \} $                                    \\
$3=$ & $s(2)=$ & $\{\emptyset,  \{ \emptyset \} \} \cup \{  \{\emptyset,  \{ \emptyset \} \} \}=$ 
               & $ \{\emptyset,  \{ \emptyset \},  \{\emptyset,  \{ \emptyset \} \} \} =$       
               & $ \{ 0, 1, 2 \} $                                    \\   
\vdots \\
$n=$ & $s(n-1)=$  & & & $\{ 0, 1, 2, \ldots, n-1 \} $       
\end{tabular}
\medskip

We define addition of two natural numbers $x,y$ recursively as follows: $x + 0 = x$, $x + s(y) = s(x + y)$. The first formula says that when the right operand is $0$, the result is just the left operand. The second formula says that when the right operand is not zero, we should consider it as the successor of some other number $y$. The right hand side of the formula then says, we should apply addition to $x$ and $y$ and then take the successor of that. If we apply this rule recursively, we will eventually end up in the base case $x + 0 = x$ which we can then use directly.

\medskip
To define multiplication, we use the rules: $x \cdot 0 = 0$, $x \cdot s(y) = (x \cdot y) + x$. ...TBC...

% N is a monoid (a semigroup is set with associative operation and if it also has an identity element, it's calle a monoid)
% The so cosntructed natural numbers satisfy the Peano axioms (VERIFY)

% https://en.wikipedia.org/wiki/Set-theoretic_definition_of_natural_numbers
% https://en.wikipedia.org/wiki/Peano_axioms

\subsubsection{Integer Numbers}
...TBC...
% Z is an integral domain (commutative ring with no zero divisors - maybe we need to demand a multiplicative identity? ..see chapter about rings and be consistent with it - there are different convetions for this)

\subsubsection{Rational Numbers}
...TBC...
% Q is a field

\subsubsection{Real Numbers}
...TBC...
% in R, the order is complete  ...what does that mean? It's from the math-faq.pdf above

\subsubsection{Complex Numbers}
...TBC...
% C is algebraically complete


%---------------------------------------------------------------------------------------------------
\subsection{Transfinite Numbers}
We constructed the integers, rationals, etc. by extending the set of natural numbers $\mathbb{N}$. Here, we will consider a different extension of $\mathbb{N}$ that will also contain distinguishable infinite numbers which are called \emph{transfinite} in this context. The resulting set of numbers is called the \emph{ordinal numbers}. In practice, natural numbers can be used for different two things: (1) counting, how many elements there are in a set, (2) ordering, i.e. putting a bunch of things into a particular order such that we can say: this is the first, this the second, etc. The difference between these two aspects of natural numbers is rather subtle for finite numbers but will become important for infinite numbers. The ordering aspect is captured by the ordinal numbers and the counting aspect will be captured by the \emph{cardinal numbers}, which are, as we will see, actually a certain subset of the ordinal numbers. We will look at both kinds of transfinite numbers in turn.

% https://en.wikipedia.org/wiki/Transfinite_number


\subsubsection{Ordinal Numbers}


% https://en.wikipedia.org/wiki/Ordinal_number

% https://www.youtube.com/watch?v=UxhFy4deLQA  Unendlich plus eins - Was sind Ordinalzahlen?

\subsubsection{Cardinal Numbers}
Cardinal numbers are special ordinal numbers. That means the set of cardinal numbers is a subset of the set of ordinal numbers. ...TBC...

% https://en.wikipedia.org/wiki/Cardinal_number
% https://en.wikipedia.org/wiki/Successor_cardinal
% https://en.wikipedia.org/wiki/Limit_cardinal
% https://en.wikipedia.org/wiki/Regular_cardinal
% https://en.wikipedia.org/wiki/Cofinality
% https://en.wikipedia.org/wiki/Cardinal_assignment

% https://en.wikipedia.org/wiki/Aleph_number
% https://en.wikipedia.org/wiki/Beth_number
% Limeskardinalzahl?

%\subsubsection{Cardinal Arithmetic}



The cardinal numbers are a superset of the natural numbers. With natural numbers, we can do certain arithmetic operations, namely addition, multiplication and exponentiation. It is possible to extend these operations to the infinite cardinals...TBC...

% https://en.wikipedia.org/wiki/Cardinal_number#Cardinal_arithmetic

\paragraph{Addition} For two sets $A,B$ with cardinalities $|A|,|B|$ repectively, we define the sum $|A| + |B|$ as the cardinality of the disjoint union of $A$ and $B$: $|A| + |B| = |A \sqcup B|$. If $A$ and $B$ are both finite, the cardinal addition just becomes the normal addition of natural numbers. If at least one of the sets $A, B$ is infinite, then the cardinal sum of the two is just the maximum: $|A| + |B| = \max(|A|, |B|)$ iff $|A| \geq \aleph_0 \vee |B| \geq \aleph_0$. 

%...TBC...this notation seem kinda atypical, I think - more typical is to use $\kappa, \lambda$ or something

% https://aleph1.info/?call=Puc&permalink=mengenlehre1_1_12_Z2
% https://en.wikipedia.org/wiki/Cardinal_number#Cardinal_addition

\paragraph{Multiplication} The product of two cardinal numbers $|A|,|B|$ is defined as cardinality of the set product: $|A| \cdot |B| = |A \times B|$. The situation here is the same as for addition. For finite cardinals, the multiplication is the same as for natural numbers. If at least one of the sets is infinite, the product of their cardinalities is just the maximum.

% https://en.wikipedia.org/wiki/Cardinal_number#Cardinal_multiplication

\paragraph{Exponentiation} The most interesting operation on cardinal numbers is the exponentiation. As usual, the definition of the operation is based on the corresponding set operation: $|A|^{|B|} = |A^B|$. The set $A^B$ is the set of all functions from $B$ to $A$. If $A$ and $B$ are finite, we observe, that the number of possible different functions from $B$ to $A$ is indeed given by $|A|^{|B|}$.

...TBC...ToDo: subtraction, division, roots, logarithms

% https://en.wikipedia.org/wiki/Cardinal_number#Cardinal_exponentiation




%\subsection{title}


\subsubsection{Transfinite Induction}

% https://en.wikipedia.org/wiki/Transfinite_induction
% https://de.wikipedia.org/wiki/Transfinite_Induktion
% https://mathworld.wolfram.com/TransfiniteInduction.html


%\subsubsection{Transfinite Recursion}



%===================================================================================================
\section{Theorems and Hypotheses}

%---------------------------------------------------------------------------------------------------
\subsection{Cantor's Theorem}
Cantor's theorem says that for any set $A$, the power set set $\mathcal{P}(A)$ has a strictly greater cardinality than $A$ itself. That is: $|A| < |\mathcal{P}(A)|$. For finite sets, these two cardinalities are related by $|\mathcal{P}(A)| = 2^{|A|}$. [Q: what about infinite sets using cardinal exponentiation?]

% https://en.wikipedia.org/wiki/Cantor%27s_theorem

%---------------------------------------------------------------------------------------------------
\subsection{The Cantor-Bernstein Theorem}
To show that two sets $A$ and $B$ have the same cardinality, we must, by definition, show that a bijection between $A$ and $B$ exists. The Cantor-Bernstein theorem asserts that it suffices to show that and injection from $A$ to $B$ and an injection from $B$ to $A$ exists. If these two injections exist, then the theorem assures that also a bijection between $A$ and $B$ exists. (VERIFY) Constructing these two injections (or indirectly showing that they exist) can, in some situations, be easier than showing directly that a bijection exists, so the theorem may help us to show that two sets have the same cardinality. ...TBC...

% I think, the theorem can be derived by (VERIFY!):
% -If there exists an injective function f: A to B  then  |A| <= |B|
% -If there exists an injective function f: B to A  then  |B| <= |A|
% -If both exist, then |A| <= |B| and |B| <= |A| and therefore  |A| = |B|

%---------------------------------------------------------------------------------------------------
\subsection{Tarski's Theorem}
Tarski proved 1924 that in ZF, the theorem "For every infinite set $A$, there exists a bijection between $A$ and $A \times A$" implies the axiom of choice. The converse is also true and that was already known at that time. That means in ZF, the theorem and the axiom of choice are equivalent. 

% https://en.wikipedia.org/wiki/Tarski%27s_theorem_about_choice

%---------------------------------------------------------------------------------------------------
\subsection{The Continuum Hypothesis}
We know that $|\mathbb{N}| < |\mathbb{R}|$. The cardinality of the real (i.e. continuous) numbers is strictly greater than the cardinality of the natural numbers. Cantor observed that any infinite subset of the real numbers that he could construct had either the same cardinality as $\mathbb{R}$ itself or the same cardinality as $\mathbb{N}$, so he hypothesized that no cardinality in between these two exists (VERIFY!). This is called the \emph{continuum hypothesis} which is sometimes abbreviated as CH. We have already introduced the notation $\aleph_0 = |\mathbb{N}|$ for the cardinality of the natural numbers. The cardinality of the real numbers, i.e. the continuum, is denoted as $\mathfrak{c} = |\mathbb{R}|$ and we also already know that $|\mathbb{R}| = 2^{\aleph_0}$. With these notations, the continuum hypothesis can be written down as: $\aleph_1 = 2^{\aleph_0}$. The generalized CH, also called Cantor's aleph hypothesis, states: $\aleph_{\alpha+1} = 2^{\aleph_{\alpha}}$ for every ordinal number $\alpha$. ....TBC...

% Q: Does the index of a cardinal number have to be an ordinal number?
% Q: How many cardinal numbers are there? Countably many or uncountably many? Is this question related to the CH?


%or as $\mathfrak{c} = 2^{\aleph_0}$



%$|\mathbb{R}| = \mathfrak{c} = 2^{\aleph_0}$. CH: $\aleph_1 = 2^{\aleph_0}$?

% ToDo: introduce notation \mathfrak{c}

\subsubsection{Independence from ZFC}
The quest was now to figure out whether or not the continuum hypothesis is true or false. As usual in axiomatic set theory, in order to prove a hypothesis true or false, one has only the axioms and the logical inference rules to work with (along with, of course, all the lower level theorems that have already been proven). So, the question in more precise terms asks: can we prove the continuum hypothesis true or false, given the ZFC axioms (or any other set of axioms). Kurt Gödel proved in 1940 that from ZFC, it cannot be proved that the continuum hypothesis is false. Paul Cohen proved in 1963 that from ZFC, it cannot be proved that the continuum hypothesis is true. These two results together mean that the continuum hypothesis is actually \emph{independent} from the ZFC axioms. This is sometimes also phrased as being \emph{undecidable} in ZFC. ...TBC...

% https://en.wikipedia.org/wiki/Continuum_hypothesis

\subsubsection{The Mathematical Multiverse}
The undecidability of the CH from ZFC means that, within the system of ZFC, we can create at least two different mathematical universes - one in which the CH is true and one in which it is false. Both of these universes are equally valid in the sense that all the theorems that follow from ZFC alone, i.e. all the mathematical theorems that we use in our usual day-to-day applied math (and many more), are true in both of them VERIFY!...TBC...

% https://en.wikipedia.org/wiki/Von_Neumann_universe


% (Provably) Unprovable and Undisprovable... How??
% https://www.youtube.com/watch?v=1RRpC7FDfEQ&lc=UgwGg_JvrNpHhgO1Vo14AaABAg.A3Uanwul3oeA3_rQWkBsOx
% see my comment there and the replies to it

% https://en.wikipedia.org/wiki/List_of_statements_independent_of_ZFC

% https://en.wikipedia.org/wiki/List_of_statements_independent_of_ZFC#Measure_theory
% CH implies that there exists a function on the unit square whose iterated integrals are not equal. The function is simply the indicator function of an ordering of [0, 1] 

% ToDo:

%\subsubsection{The Generalized Continuum Hypothesis}


%  Cantor's aleph hypothesis

% https://en.wikipedia.org/wiki/Continuum_hypothesis#Generalized_continuum_hypothesis


%---------------------------------------------------------------------------------------------------
\subsection{Gödel's Incompleteness Theorems}
By picking a particular set of axioms, one selects a particular mathematical universe. The hope was to find a set of axioms that is consistent and complete....TBC...

% Consistency means that we cannot derive contradictions, i.e. cannot prove things that are false. Completeness means that all true propositions can be proven

% https://en.wikipedia.org/wiki/G%C3%B6del%27s_incompleteness_theorems

% https://www.youtube.com/watch?v=d0__uZE_x1k
% A theory that is consistent and expressive enough to talk about itself cannot be complete

% https://www.youtube.com/watch?v=qSiLjXlFlYE

\begin{comment}

ToDo:
-Infinite Sets 
 -Construction of natural, integer, rational and real numbers
 -Cardinal Numbers
 -Ordinal Numbers (as generalization of cardinal numbers)
 -Surreal Numbers
 -Cantor-Bernstein Theorem
 -Continuum Hypothesis
 -Notation: looks like in axiomatic set theory, it's more common to use lowercase letters like x,y for sets (rather than A,B, ...). That looks better in predicate logic formulas
 -maybe use \varphi for predicates



Set Containment:
-If for all x in A, x in B, then A \subseteq B

Lebesgue-Measure:
 https://www.youtube.com/watch?v=0VD3BWDLmU0

...set theory sometimes appears like a "write-only-language

% See:
% https://aleph1.info/?call=Puc&permalink=mengenlehre1



Das Zermelo-Fraenkel-Axiomensystem der Mengenlehre (ZF)
https://www.youtube.com/watch?v=U10UYyXv5gM&list=PLb0zKSynM2PAuxxtMK1bxYPV_bUoPtpTB&index=1&t=0s

Bourbaki:
-Collective of authors writing a series of books in the early 1900s that aimed to build up all of math axiomatically. The first volume was about set theory

Was sind Kardinalzahlen? Was besagt die Kontinuumshypothese?
https://www.youtube.com/watch?v=qijXa3U4Nag&list=PLb0zKSynM2PCrgebQsfrzEsUIuA0I_wdG&index=1&t=0s


For the ZFC axioms, see:

https://en.wikipedia.org/wiki/Zermelo%E2%80%93Fraenkel_set_theory#Axioms
1. Axiom of extensionality
2. Axiom of regularity (also called the axiom of foundation)
3. Axiom schema of specification (or of separation, or of restricted comprehension)
4. Axiom of pairing
5. Axiom of union
6. Axiom schema of replacement
7. Axiom of infinity
8. Axiom of power set
9. Axiom of choice

https://de.wikipedia.org/wiki/Zermelo-Fraenkel-Mengenlehre#Die_Axiome_von_ZF_und_ZFC
 1. Extensionalitätsaxiom
 2. Leermengenaxiom     (not included in the English wiki page - it is redundant)
 3. Paarmengenaxiom
 4. Vereinigungsaxiom
 5. Unendlichkeitsaxiom
 6. Potenzmengenaxiom
 7. Fundierungsaxiom
 8. Aussonderungsaxiom
 9. Ersetzungsaxiom
10. Auswahlaxiom





How to implement the idea of sets in software:
-Make a (base?)class "Set" or maybe SetZFC
-It should have virtual functions like hasElement(const Set& elem), an equality comparison that takes two pointers to sets, etc.
-It should have an array of pointers (or maybe direct memebrs? but maybe that's not possible) for the elements - the elements of a set are also sets
-The empty set is the one where the elements array is empty
-There should be functions for union and intersection and factory functions for creating certain special sets like those representing numbers, ordered pairs, etc.
-Implement functions for addition and multiplication according to how these functions are defined set theoretically
-Generally, implement all functions to create sets that should exist accrding to the axioms
-Maybe have different sorts of Set like SetZFC, SetNBG, etc. - each implementing a different set of primitive axioms
-Maybe the axioms should be implemented as static member functions and all the higher level set building functions as free functions


https://en.wikipedia.org/wiki/Foundations_of_mathematics
https://en.wikipedia.org/wiki/Controversy_over_Cantor%27s_theory


https://en.wikipedia.org/wiki/Type_theory

https://www.youtube.com/watch?v=szfsGJ_PGQ0  The Axiom of Choice


\end{comment}