%\documentclass[12pt, twocolumn]{article}
%\documentclass[12pt, openany]{book}
\documentclass[12pt, oneside]{book}
%\usepackage{fullpage}           % makes all margins 1 inch?
\topmargin=-1.0cm
\textheight=23cm
\evensidemargin=-1.0cm
\oddsidemargin=-1.0cm
\textwidth=19cm
\setcounter{secnumdepth}{-1}  % suppress numbering of sections
\usepackage{amsmath}
\usepackage{amssymb}          % for mathbb
\usepackage{hyperref}
\usepackage{array}            % For Cayley tables
\usepackage{stmaryrd}         % for \llbracket, \rrbracket
%\usepackage{cancel}           % \cancel to strike out math symbols - nah - it's ugly

\usepackage{comment}          
% to comment out larger sections via \begin{comment} ... \end{comment} 
% see:
% https://tex.stackexchange.com/questions/17816/commenting-out-large-sections
% https://tex.stackexchange.com/questions/11177/how-to-write-hidden-notes-in-a-latex-file/73418


\usepackage{color}               % colored text
\usepackage{listings}            % source code formatting 
%\lstset{language=python}
\definecolor{mygreen}{rgb}{0,0.6,0}
\definecolor{mygray}{rgb}{0.5,0.5,0.5}
\definecolor{mymauve}{rgb}{0.58,0,0.82}
\lstset{ %
  backgroundcolor=\color{white},   
  %basicstyle=\footnotesize\ttfamily,  % the size of the fonts that are used for the code
  basicstyle=\ttfamily,               % the size of the fonts that are used for the code
  captionpos=none,                    % no captions (and no empty space either)
  commentstyle=\color{mygreen},       % comment style
  frame=single,	                      % adds a frame around the code
  keywordstyle=\color{blue},          % keyword style
  language=Python,
  stringstyle=\color{mymauve},        % string literal style
  columns=flexible,                   %
  keepspaces=true,                    % keeps spaces in text
  tabsize=4,
}


\usepackage{tikz}
%\usetikzlibrary{calc} % maybe later
\usetikzlibrary{positioning}


%\DeclareMathOperator{\d}{d}                  % exterior derivative
\DeclareMathOperator{\grad}{\mathbf{grad}}
\DeclareMathOperator{\curl}{\mathbf{curl}}
\DeclareMathOperator{\dive}{div}
\DeclareMathOperator{\atan2}{atan2}
\DeclareMathOperator{\rank}{rank}
\DeclareMathOperator{\im}{im}                 % image of a function/map
\DeclareMathOperator{\vectorize}{vec}
\DeclareMathOperator{\geo}{geo}               % geometric multiplicity
\DeclareMathOperator{\alg}{alg}               % algebraic multiplicity 
\DeclareMathOperator{\sign}{sign}    
\DeclareMathOperator{\card}{card}             % cardinality        
\DeclareMathOperator{\tc}{tc}                 % transitive closure of a set
%\DeclareMathOperator{\Eig}{Eig} 

\DeclareMathOperator{\e}{\mathrm{e}}          % for Euler's number - ToDo: use \e consistently!
%\newcommand{\e}{\operatorname{e}}            % ...alternative definition (possibly)

\usepackage{mathtools}                        % for "\DeclarePairedDelimiter" macro
\DeclarePairedDelimiter{\floor}{\lfloor}{\rfloor}

\newcommand{\norm}[1]{\left\lVert#1\right\rVert}

% There are multiple conventions to express a logical exclusive or - we make the choice for the
% whole text here:
\newcommand*\xor{\mathbin{\veebar}}              % exclusive or - alternatives: \oplus, \dot{\vee}
\newcommand*\nand{\mathbin{\barwedge}}
\newcommand*\then{\mathbin{\rightarrow}}         % \implies is already defined
\newcommand*\mequiv{\mathbin{\leftrightarrow}}   % material equivalence
% We follow wolfram:
% https://mathworld.wolfram.com/XOR.html
% https://mathworld.wolfram.com/NAND.html

%\let\cleardoublepage\clearpage

% Maybe move the stuff up to here into a _Setup.tex file that can be included from
% _FullBook.tex and _SingleChapter.tex