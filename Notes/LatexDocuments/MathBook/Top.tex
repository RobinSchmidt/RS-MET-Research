\chapter{Topology}
Topology can be described as the study of those properties of geometric objects that are preserved under continuous deformations of the object. An example of such a property is the number of holes that it has: a sphere has none, a torus has one. Intuitively, for such a deformation to qualify as "continuous", it shall not rip or cut apart material, poke holes into it, glue things together that formerly were apart, close holes, etc. Mathematically, the class of allowed deformations is defined in terms of \emph{a topology} which is a specific set of subsets of a set $S$. Note how the term "topology" is overloaded as referring to two things: to the mathematical field itself and to a particular mathematical definition of a set of a certain kind that is used within that field.

%Topology has been called "rubbersheet geometry" by some authors which captures the idea quite nicely. ...TBC...


\begin{comment}

% knot theory, moebis strips

https://en.wikipedia.org/wiki/Topology

https://en.wikipedia.org/wiki/Topological_space

\end{comment}