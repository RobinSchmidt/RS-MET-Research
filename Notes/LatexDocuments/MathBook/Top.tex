\chapter{Topology}
Topology can be described as the study of those properties of geometric objects that are preserved under continuous deformations of the object. An example of such a property is the number of holes that it has: a sphere has none, a torus has one. There are also shapes that have two, three, etc.. That number is such a topological property. You may continuously deform the object as much as you like - the number of holes will remain the same. Intuitively, for such a deformation to qualify as "continuous", it shall not rip or cut apart material, poke holes into it, glue things together that formerly were apart, close holes, etc. Mathematically, the class of allowed deformations is defined in terms of \emph{a topology} which is a specific set of subsets of a set $S$. Note how the term "topology" is overloaded as referring to two things: to the mathematical field itself and to a particular mathematical definition of a set of a certain kind that is used within that field. Such a topology will allow us to define continuity in a way that doesn't rely on a distance measure. Central to the field of topology are homeomorphisms which are invertible functions between two topological spaces which are continuous in both directions. In this context, the "topological spaces" are interpreted as geometric shapes and the function transforms one shape into another.

% Topology revolves around the notion of "closeness". When are two elements close to each other? What are th maps that preserve closeness?

%such as the surface of

%homotopies, which are continuous transformations of one function into 

%Topology has been called "rubbersheet geometry" by some authors which captures the idea quite nicely. ...TBC...


\begin{comment}

% knot theory, moebius strips

https://en.wikipedia.org/wiki/Topology
https://en.wikipedia.org/wiki/Topological_space

https://en.wikipedia.org/wiki/Topological_property

https://en.wikipedia.org/wiki/Homeomorphism
https://en.wikipedia.org/wiki/Homotopy

Grundbegriffe der Topologie: offene Mengen und Umgebungen
https://www.youtube.com/watch?v=55jUgqwFnec&list=PLb0zKSynM2PD3i3xMuWrUF9_txMrJMGEZ&index=23
in the playlist about differential geometry


https://de.wikipedia.org/wiki/Satz_von_Stokes
https://de.wikipedia.org/wiki/De-Rham-Kohomologie#Satz_von_de_Rham


What is a hole?
https://www.youtube.com/watch?v=IDcw33YRgpY
-about algebraic topology
-sequel of "What is algebraic geometry?" https://www.youtube.com/watch?v=MflpyJwhMhQ
-A space has "no holes" iff every loop in it is homotopic to (i.e. can be contracted to) a point
-one loop is homotopic to another, if it can be continuosuly be transformed into it

Examples of topological invariants (verify!):
-number of holes (sphere vs torus)
-number of sides (loop vs Moebius strip)
-minimal number of crossings in a knot
-winding numbers (of a given path with respect to a given point/hole?)?
-dimensionality of an objcet and/or hole?


https://www.youtube.com/watch?v=unikWDPBY8M  Topology is Impossible Without These 7 Things


\end{comment}