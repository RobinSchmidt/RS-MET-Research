\section{Deformations}


\subsection{Homeomorphisms}
A \emph{homeomorphism} between two topological spaces $X$ and $Y$ is a function or map $f: X \rightarrow Y$ that is continuous, bijective (and therefore invertible) and its inverse $f^{-1}: Y \rightarrow X$ is also continuous. Maps with that feature are also called \emph{bicontinuous}. Homeomorphisms play a role in topology that is similar to the role of homomorphisms\footnote{Mind the missing "e": In topology, it's about homEOmorphisms, in group theory abou homOmorphisms} in group theory. In group theory, the inverse of a bijective and structure preserving map is automatically again a structure preserving map. Therefore, in the definition of a homomorphism in group theory, we do not explicitly need to require the inverse to be structure presevring. We get that feature for free. In topology, the inverse of a bijective continuous map is not necessarily continuous, so we must explicitly require the continuity of the inverse in the definition of a homemorphism. ...TBC...

%https://en.wikipedia.org/wiki/Homeomorphism

% https://www.youtube.com/watch?v=U6Wv7SeoJdk
% Was ist ein Homöomorphismus? Was bedeutet "homöomorph"?
% For algebraic structures, the inverse of every bijective homomorphism is automatically also a (bijective) homomorphism. In topology with homeomrphisms, that is not necessarily the case. We have to explicitly require that the inverse map is also continuous.

% In topology, the maps we are interested in should preserve closeness. The maps that do that are the continuous maps. Closeness can also be defined in terms of the closure(?) of a set X denoted as $\overline{X}$: if $x \in \overline{X}$ then $x$ is close to $X$

\subsection{Homotopies}

% https://en.wikipedia.org/wiki/Homotopy


\subsection{Connectedness}

% path-connected, simply commnected, ..


\begin{comment}

I think, the importnat thing is the notion of continuity defined as mapping compact sets to compact sets. Maybe consider discontinuous function (like the setp function) and show how their images are necessarily non-compact. What about 1/x ...on R and R \cup \infty. I think, in the latter set 1/x is continuous? It goes to +inf and re-emerges from -inf (which are identified). As the radius of the ball that ecloses the set, we can now take infinity itself because it is in the set?


\end{comment}