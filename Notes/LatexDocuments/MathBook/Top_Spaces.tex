\section{Toplogical Spaces}
Our goal here is define a notion of closeness or distance between two elements of a set. But this notion shall be abstract enough to apply to sets that may be wildly different from the Euclidean spaces in which we have a good intuition for what distance means. It shall be a very general notion of distance. We will do this by starting with our familiar Euclidean space and defining more and more general kinds of spaces which will culminate in the definition of a topological space - the fundamental mathematical structure in which we will do topology.

%===================================================================================================
\subsection{Inner Product Space}
An inner product space is a vector space $V$ where an inner product between any pair of vectors $\mathbf{v,w}$ is defined. We denote that inner product as $\langle \mathbf{v,w} \rangle$. For example, in our familiar Euclidean vector spaces $\mathbb{R}^n$, that inner product is defined as $\langle \mathbf{v,w} \rangle = \sum_{i=1}^n v_i w_i$. To qualify as an inner product, the function $\langle \mathbf{v,w} \rangle$ must satisfy the following properties:



%We can also define such an inner product on  $\mathbb{C}^n$ as $\langle \mathbf{v,w} \rangle = \sum_{i=1}^n v_i \overline{w_i}$ where $\overline{w_i}$ denotes complex conjugation.
% bilinear, homogeneous,

...TBC...bilinear, homogeneous,...


% https://en.wikipedia.org/wiki/Inner_product_space




%===================================================================================================
\subsection{Normed Space}
A normed space is a vector space $V$ on which a norm function is defined. That means, for each element $\mathbf{v} \in V$, we can compute the norm of $\mathbf{v}$, denoted as $|\mathbf{v}|$. The output of that function will be a non-negative real number which we interpret as a measure of the size or length of the vector. To qualify as a norm, our norm function must satisfy a couple of more properties. We have already met them in slightly different form in the section about elementary functions. Let $\mathbf{v} \in V$ and let $\lambda$ be in the set of scalars. Then:

\medskip
\begin{tabular}{l l}
Non-negative:         & $|\mathbf{v}| \geq 0$  \\
Positive definite:    & $|\mathbf{v}| = 0 \Leftrightarrow \mathbf{v} = \mathbf{0}$  \\
Absolute homogeneity: & $|\lambda \mathbf{v}| = |\lambda| \cdot |\mathbf{v}|$  \\
Triangle inequality:  & $|\mathbf{v} + \mathbf{w}| \leq |\mathbf{v}| + |\mathbf{w}|$
\end{tabular}
\medskip

[VERIFY]...TBC..

% https://en.wikipedia.org/wiki/Normed_vector_space

\paragraph{Induced Norm}
If our vector space $V$ happens to be an inner product space, then we can use the inner product to define a norm as $|\mathbf{v}| = \sqrt{\langle \mathbf{v,v}  \rangle}$. The so defined function does indeed meet all the requirements for a norm. For any norm that was induced by an inner product, the parallelogram equation holds: $|x+y|^2 + |x-y|^2 = 2(|x|^2 + |y|^2)$. With this induced norm, we see that every inner product space is also a normed space. But the set of normed spaces is more general. To define a norm, we do not necessarily need to have an inner product. Other norms are possible, too. That means, the idea of a normed space is more general than the idea of a topological space.

% https://www.youtube.com/watch?v=qwsvXa7nIGg
% at 16:44 - for norms that are induced by an inner product, the parallelogram equation holds


%\subsubsection{Examples} Let's now look at a couple of norms that are not induced by an inner product.

%\paragraph{Sum norm} [sum of absolute values]

\paragraph{$L^p$-Norm} Let's consider vectors $\mathbf{x} = (x_1, x_2, \ldots, x_n)$ in $\mathbb{R}^n$. For such a vector, the $L^p$ norm for some given real number $0 < p \leq \infty$ is defined as: $ |\mathbf{x}|_p = (|x_1|^p + |x_2|^p + \ldots + |x_n|^p)^{1/p}$. For $p=2$ we get the $L^2$ norm which is just our usual Euclidean norm. It can be verified that for $p=1$, the parallelogram equation does not hold (in fact, it holds only for $p = 2$, I think - VERIFY!). That means we have found at least one norm that is not induced by an inner product. We cannot find an inner product that would induce this norm when we believe that any such induced norm is bound to satisfy the parallelogram equation. This confirms that the notion of a normed space is more general than the notion of an inner product space. Another important special case for the $L^p$ norm is the case where $p = \infty$. This norm just is just equal to the maximum of the absolute values of the components of $\mathbf{x}$.

%[TODO: explain the $L^\infty$ norm]

%If we assume the parallelogram equation to be true for any norm induced by an inner product, this confirms that the

% https://en.wikipedia.org/wiki/Lp_space
% https://de.wikipedia.org/wiki/Lp-Raum} 



%===================================================================================================
\subsection{Metric Space}
A metric space is a set $S$ in which for every pair of elements $x,y \in S$ we can compute a number $d(x,y)$ which we want to interpret as "distance" between $x$ and $y$. Note that we do not require $S$ to be a vector space - for the time being, it's just a set. The distance function $d(x,y)$ must satisfy the following properties for all $x,y \in S$:

\medskip
\begin{tabular}{l l}
Zero distance to itself:            & $d(x, x) = 0$  \\
Positive between distinct elements: & $x \neq y \Rightarrow d(x,y) > 0$  \\
Symmetry:                           & $d(x, y) = d(y, x)$  \\
Triangle inequality:                & $d(x,z) \leq d(x,y) + d(y,z)$
\end{tabular}
\medskip

where the positivity axiom can actually be weakened to only require the distance between distinct elements to be nonzero. [TODO: explain how positivity then follows and how the implication is actually both ways]. Our distance function $d = d(x,y)$ is also called \emph{a metric} on the set $S$. 

% https://www.youtube.com/watch?v=qwsvXa7nIGg
% at 15:35 - calls the first property jsut "definite"

% https://www.youtube.com/watch?v=WjuKCrys_Ig&list=PLHi0WgifODX19zsJhvCrizYEXiHY5qc9n&index=2

% https://en.wikipedia.org/wiki/Pseudometric_space
% https://en.wikipedia.org/wiki/Metric_space#Semimetrics
% https://de.wikipedia.org/wiki/Pseudometrik

% https://en.wikipedia.org/wiki/Minkowski_space#Metric_signature
% https://mathworld.wolfram.com/MinkowskiMetric.html

\paragraph{Induced Metric}
The requirements for a norm and for a metric are sufficiently similar that we can make a natural connection between norms and metrics. If our set $S$ happens to be a normed vector space $V$, then the norm defined on $V$ induces a metric on that space by defining the \emph{norm induced metric} via the norm of the difference: $d(\mathbf{x, y}) = |\mathbf{x-y}|$. Forming a difference is always possible because a normed space is required to be a vector space. With the norm induced metric, we see that every normed space can easily be turned into a metric space. So, if we assume that we construct a metric via this induction, every normed space is automatically also a metric space. But not vice versa. The definition of a metric space is less restrictive than the definition of a normed space. The set of metric spaces is more general. We lifted our idea of distance to a more abstract level. Thereby we enable ourselves to talk about distances in more general sets than normed vector spaces.

\medskip
[Q: Can we turn this around by saying that a metric induces a norm by defining the norm as distance from the origin $|\mathbf{x}| = d (\mathbf{x, 0})$ whenever our set happens to have some special element $\mathbf{0}$ that we call the origin?]

\subsubsection{Examples} To get a feeling for how the so defined notion of metric is more general than our familiar Euclidean metric, let's now look at a couple of examples of metrics other than the Euclidean. 

\paragraph{Minkowski and Manhattan Distance}
The Minkowski distance is to metrics what the $L^p$-norm is to norms. Let $\mathbf{x,y}$ be two vectors in $\mathbb{R}^n$ and $p \in \mathbb{Z}$. Then, the Minkowski distance of order $p$ between the vectors is defined as $d_M(\mathbf{x,y}) = (\sum_{i=1}^n |x_i - y_i|^p )^{1/p}$. The special case for $p=1$ is also called Manhattan distance or city block distance. It is the total distance that you would have to walk if you can't take the direct diagonal route but instead have to piece your path together from purely horizontal and vertical segments (or, more generally, take partial paths on an orthogonal grid). This is how the streets are laid out in Manhattan, hence the name. 

\medskip
By the way: Do not confuse the Minkowski distance with the Minkowski metric from special relativity. The latter is something else and it actually wouldn't even qualify as a metric according to our definition here because it can become negative.

\paragraph{Hamming Distance} 
It is interesting to see a distance measure defined on a set that is very uncommon in mathematics: the set of strings - in the computer science or programmer sense. The Hamming distance between two strings is defined to be number of positions in which the two strings differ. The Hamming distance between strings of zeros and ones is an important tool in the design and analysis of error detection and correction schemes in noisy digital communication channels. 

\paragraph{Levenshtein Distance} 
The Levenshtein distance between two strings is the minimum number of editing operations (deletions, replacements or insertions of characters) that one would need to transform one string into the other. It is also a bona fide metric according to our definition. It is important in applications like automatic correction of human inputs from a computer keyboard.

\medskip
Both Hamming and Levenshtein distance show that the "space" of all strings can indeed be made into a metric space. We also observe that it can be done in different ways and how distance measures can be tailored to a particular application. The errors that can occur in a noisy binary communication channel is the replacement of a zero by a one or the other way around. This is precisely what the Hamming distance measures. The kinds of errors that humans can make when entering text on a computer keyboard cannot only be replecement of a character but also the omssion or adddition of one. The Levenshtein distance was designed to capture that. [Q: Can a norm on strings be defined that induces the Levenshtein distance? I guess not - but can it be proven?] 

%[TODO: figure out, if the Levenshtein metric can be shown to not be induced by a metric - we need to show that no metric on strings is possible that induces the Levenstein distance]



%https://en.wikipedia.org/wiki/Minkowski_distance
% ...is basically for metrics what the L^p norm is for norms

%TODO:  Explain equivalence of metrics
% https://www.youtube.com/watch?v=qwsvXa7nIGg  24:30
% Equivalent metrics lead to the same notions of continuiuty and covergence and also induce the saem topology

% Continuity: let f: R^m -> R^n. f is continuous at x_0 if for every eps > 0 we can find a delta > 0 such when the output is inside an epsilon-ball around x_0, the the output is around a delta-ball around f(x_0)? or som

\subsubsection{A Taxonomy of Sets}
We can use the metric $d$ to define a certain classification of sets that is important in the field of topology. Eventually, the goal is to define all these features without resorting to a metric but let's first see how it's been done with a metric.

%the important notion of an open set. Our eventual goal will be to define an open set without referring to a metric, but let's first see how it's been done with a metric.

\paragraph{Definition: Open Ball} Let $\varepsilon > 0$ be a real number and $x$ be an element of our metric space.  We define the open $\varepsilon$-ball around $x$ as the following set: $B_\varepsilon(x) = \{ y : d(x,y) < \varepsilon \}$.

\medskip
The open ball $B_\varepsilon(x)$ is the set of points, whose distance to $x$ is less than $\varepsilon$. Using $\varepsilon$ indicates that we intend to allow this value to become arbitrarily small as long as it is greater than zero. Intuitively, we may envision $B_\varepsilon(x)$ as a (possibly very small) halo around $x$. In $\mathbb{R}$, the "open balls" are open intervals, in $\mathbb{R}^2$ they are open circular discs, in $\mathbb{R}^3$ they are actual balls, i.e. filled spheres (but excluding the surface - that exclusion is what makes them "open") and in higher dimensions $\mathbb{R}^n$, they are filled $n$D hyperspheres (again excluding their $(n-1)$-dimensional boundary).

\paragraph{Definition: Interior Point} A point $x \in X$ is said to be an inner point of $x$, if there exists an $\varepsilon > 0$ such that $B_\varepsilon(x) \subseteq X$. 

%https://en.wikipedia.org/wiki/Interior_(topology)

\medskip
Intuitively, an interior point of a set $X$ is a point $x$ that admits us to define a (possibly small) halo around it that is fully contained in the set $X$.

\paragraph{Definition: Open Set} A set $X$ is said to be open, if all of its elements are interior points. 

\medskip
Intuitively, an open set does not have any boundary points as elements of the set. A boundary point of a set $X$ is usually defined to be a point $x$ where every $\varepsilon$-ball around the point contains points that are inside the set $X$ as well as points that are outside the set $X$. When a set is open, we can always find some (possibly tiny) open ball around it that is fully inside the set.

\paragraph{Definition: Closed Set} A set $X$ is closed if its complement is open [VERIFY]

\medskip
Do not make the mistake of thinking that "closed" is just the opposite of "open" as in "not open". That would be wrong. The half-open interval $[0,1)$ is neither open nor closed, for example. [TODO: give an example of a set that is open and closed]

\paragraph{Definition: Bounded Set} A set $X$ is said to be bounded if we can find an $x \in X$ and an $\varepsilon > 0$ such that $X \subseteq B_\varepsilon(x)$ [VERIFY]. 

\medskip
The intuition behind this definition is that the set $X$ has a finite extent. We pick some element $x$ inside the set $X$ and draw a ball with radius $\varepsilon$ around it. It must be possible to find a finite radius $\varepsilon$ such that this ball contains the whole set $X$. In this case, we actually imagine $\varepsilon$ to be potentially rather large as long as it is a finite number.

\paragraph{Definition: Compact Set} A set $X$ is said to be compact if it is bounded and closed.

\medskip
Compactness is important because ...TBC...

% Topologische Räume als Abstraktion von "Nähe"
% https://www.youtube.com/watch?v=qwsvXa7nIGg  at around 25 min

% https://en.wikipedia.org/wiki/Metric_space#Norm_induced_metric
% https://en.wikipedia.org/wiki/Metric_space



%\subsection{Uniform Space}
% is in between metric and topological


%===================================================================================================
\subsection{Topological Space}
In a topological space, we can talk about "distances" and "closeness" between of elements and "neighborhoods" of elements without even needing to refer to a metric. The only notion that we will need to refer to is that of an open set. Obviously, we will need an alternative but equivalent definition for what "open set" is supposed to mean if we want to do away with the metric.



\subsubsection{A Taxonomy of Sets Revisited}
We now want to revisit our taxonomy of sets. The goal is to define all those properties without using a metric. The central idea to make that work is that of an open set which we now want to define without these "open balls" which themselves were founded on a metric. ...TBC...

% Our eventual goal is to abstract away the idea of a metric. 

%We now want to define what an open set is without referring to a metric. This will be the key abstraction that will allow us to define closeness without resorting to the numerical quantitative notion of distance that a metric is. It will be a purely qualitative notion of closeness ...TBC...


%[TODO: explain how a metric induces a topology and therefore metric spaces are topological spaces but the latter notion is more general.]
% Different metrics can induce the same topology. If two different metric induce the same topology, the these metrics are calle equivalent.

% https://www.youtube.com/watch?v=3j8GIJd5-3A&list=PLHi0WgifODX19zsJhvCrizYEXiHY5qc9n
% 10:25


\subsubsection{Topologies on a Set}


\paragraph{Definition: Topology} 
Let $X$ be a set. A set $T \subseteq \mathcal{P}(X)$ of subsets of $X$ is called a topology on $X$ if it satisfies: (1) The empty set and the complete set are in $T$: $\emptyset \in T$ and $X \in T$. (2) Any finite intersection of elements $t_k \in T$ is again in $T$: $\bigcap_{k=1}^n t_k \in T$ for $n \in \mathbb{N}$. (3) Any (finite or infinite) union of elements $t_k \in T$ is again in $T$: $\bigcup_{k=1}^\infty t_k \in T$. 

\medskip
The finite case for the union can be thought of as special case of the infinite case where most (actually almost all) of the $t_k$ are empty. So it doesn't need to be stated separately [VERIFY!]. With that definition, the elements of a topology $T$ are indeed precisely the open sets in $X$. [TODO: VERIFY and explain why] ...TBC...

\paragraph{Definition: Topological Space} A topological space is a set $X$ on which a topology $T$ is defined.

% https://en.wikipedia.org/wiki/Topological_space
% https://mathworld.wolfram.com/TopologicalSpace.html

\paragraph{Induced Topology} 
We have seen how we can define an open subset of a set $X$ when we have a metric $d$ available. If we collect all the so defined open subsets of a set into a set $T$, then this set $T$ will indeed qualify as a topology on $X$. It is the topology that is induced by the metric $d$. [VERIFY] That means that every metric space is automatically also a topological space. But - by now you know the drill - the notion of a topological space is more general. There are topologies which are not (and cannot be) induced by a metric. [TODO: give an example]


\medskip
Note how our subsequent generalizations led to the following onion like structure of subset inclusions: inner product spaces $\subset$ normed spaces $\subset$ metric spaces $\subset$ topological spaces. The topological space is now finally the abstraction that we want to work with and upon which the mathematical field of topology revolves around. To build intuition, we will nevertheless usually use a more concrete space like $\mathbb{R}^n$ that we can intuitively understand (although even that can get hard when $n > 3$).



...TBC...


%\subsubsection{Compactness}
%\paragraph{Continuity}
%\paragraph{Convergence}



% Topologische Räume als Abstraktion von "Nähe"
% https://www.youtube.com/watch?v=qwsvXa7nIGg

% https://en.wikipedia.org/wiki/Topological_space
% https://en.wikipedia.org/wiki/Topology

\subsubsection{Some Special Topological Spaces}
Some kinds of topological spaces occur often enough that mathematicians found it appropriate to assign names to them. We'll now list a few of them.

\paragraph{Hausdorff Spaces} [VERIFY!] A topological space is called a Hausdorff space, when any two elements $x,y$ of that space can be separated by disjoint open sets. That means we can always find two disjoint open sets $X,Y$ such that $x$ is in $X$ but not in $Y$ and $y$ is in $Y$ but not in $X$. The set $X$ surrounds $x$ and the set $Y$ surrounds $y$ and the sets $X,Y$ don't meet or even overlap. Every metric space is a Hausdorff space.

...TBC...

% Banach, Hilbert, Sylow, Krylov?

% banach-

\begin{comment}

Zusammenhang zwischen normierten, metrischen und topologischen Räumen
https://www.youtube.com/watch?v=3j8GIJd5-3A&list=PLHi0WgifODX19zsJhvCrizYEXiHY5qc9n

Definition of closeness or neighborhood: two elements $x,y$ of $X$ are close to each other iff there exists an open set that contains both $x$ and $y$

A topology is a collection of open sets.



\end{comment}