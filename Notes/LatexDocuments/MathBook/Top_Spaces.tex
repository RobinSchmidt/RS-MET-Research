\section{Toplogical Spaces}

\subsection{Normed Space}
A normed space is a vector space $V$ on which a norm function is defined. That means, for each element $\mathbf{v} \in V$, we can compute the norm of $\mathbf{v}$, denoted as $|\mathbf{v}|$. The output of that function will be a non-negative real number. In fact, the norm function must satisfy a couple of more properties - those that we have already met in the section about elementary functions. They are:

\medskip
\begin{tabular}{l l}
Non-negative:         & $|\mathbf{v}| \geq 0$  \\
Positive definite:    & $|\mathbf{v}| = 0 \Leftrightarrow \mathbf{v} = \mathbf{0}$  \\
Multiplicative:       & $|\mathbf{v} \cdot \mathbf{w}| = |\mathbf{v}| \cdot |\mathbf{w}| $  \\
Triangle inequality:  & $|\mathbf{v} + \mathbf{w}| \leq |\mathbf{v}| + |\mathbf{w}| $
\end{tabular}
\medskip

[VERIFY]...TBC..

\subsection{Metric Space}
A metric space is a set $S$ in which for every pair of elements $x,y \in S$ we can compute a number $d(x,y)$ which we want to interpret as "distance" between $x$ and $y$. Note that we do not require $S$ to be a vector space - for the time being, it's just a set. The distance function $d(x,y)$ must satisfy the following properties:

\medskip
\begin{tabular}{l l}
Non-negative:         & $d(\mathbf{x, y}) \geq 0$  \\
Positive definite:    & $d(\mathbf{x,y }) = 0 \Leftrightarrow \mathbf{x} = \mathbf{y}$  \\
Triangle inequality:  & $d(\mathbf{x,z}) \leq d(\mathbf{x,y}) + d(\mathbf{y,z})$
\end{tabular}
\medskip

[VERIFY] ...TBC...


% induced norm:
% If we have a distance function and an origin, we can use these two to define a norm as a distance to the origin. If we have a norm and a notion of subtraction, we can use these two to define a distance as norm of a difference.

\subsection{Topological Space}

\begin{comment}



\end{comment}