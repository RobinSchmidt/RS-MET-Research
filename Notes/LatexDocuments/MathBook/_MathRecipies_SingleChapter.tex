% This file is meant for working on a chapter. During writing, I don't want to always build the whole book after each small edit because that takes too long. So during writing, I use this file that includes only the chapter that I currently work on to speed up the edit-build-preview cycle. ...TBC...


%\documentclass[12pt, twocolumn]{article}
%\documentclass[12pt, openany]{book}
\documentclass[12pt, oneside]{book}
%\usepackage{fullpage}           % makes all margins 1 inch?
\topmargin=-1.0cm
\textheight=23cm
\evensidemargin=-1.0cm
\oddsidemargin=-1.0cm
\textwidth=19cm
\setcounter{secnumdepth}{-1}  % suppress numbering of sections
\usepackage{amsmath}
\usepackage{amssymb}          % for mathbb
\usepackage{hyperref}
\usepackage{array}            % For Cayley tables
\usepackage{stmaryrd}         % for \llbracket, \rrbracket

\usepackage{comment}          
% to comment out larger sections via \begin{comment} ... \end{comment} 
% see:
% https://tex.stackexchange.com/questions/17816/commenting-out-large-sections
% https://tex.stackexchange.com/questions/11177/how-to-write-hidden-notes-in-a-latex-file/73418


\usepackage{color}               % colored text
\usepackage{listings}            % source code formatting 
%\lstset{language=python}
\definecolor{mygreen}{rgb}{0,0.6,0}
\definecolor{mygray}{rgb}{0.5,0.5,0.5}
\definecolor{mymauve}{rgb}{0.58,0,0.82}
\lstset{ %
  backgroundcolor=\color{white},   
  %basicstyle=\footnotesize\ttfamily,  % the size of the fonts that are used for the code
  basicstyle=\ttfamily,               % the size of the fonts that are used for the code
  captionpos=none,                    % no captions (and no empty space either)
  commentstyle=\color{mygreen},       % comment style
  frame=single,	                      % adds a frame around the code
  keywordstyle=\color{blue},          % keyword style
  language=Python,
  stringstyle=\color{mymauve},        % string literal style
  columns=flexible,                   %
  keepspaces=true,                    % keeps spaces in text
  tabsize=4,
}


\usepackage{tikz}
%\usetikzlibrary{calc} % maybe later
\usetikzlibrary{positioning}


%\DeclareMathOperator{\d}{d}                  % exterior derivative
\DeclareMathOperator{\grad}{\mathbf{grad}}
\DeclareMathOperator{\curl}{\mathbf{curl}}
\DeclareMathOperator{\dive}{div}
\DeclareMathOperator{\atan2}{atan2}
\DeclareMathOperator{\rank}{rank}
\DeclareMathOperator{\im}{im}                 % image of a function/map
\DeclareMathOperator{\vectorize}{vec}
\DeclareMathOperator{\geo}{geo}               % geometric multiplicity
\DeclareMathOperator{\alg}{alg}               % algebraic multiplicity 
\DeclareMathOperator{\sign}{sign}    
%\DeclareMathOperator{\Eig}{Eig} 

\DeclareMathOperator{\e}{\mathrm{e}}          % for Euler's number - ToDo: use \e consistently!
%\newcommand{\e}{\operatorname{e}}            % ...alternative definition (possibly)

\usepackage{mathtools}                        % for "\DeclarePairedDelimiter" macro
\DeclarePairedDelimiter{\floor}{\lfloor}{\rfloor}

\newcommand{\norm}[1]{\left\lVert#1\right\rVert}

%\let\cleardoublepage\clearpage

% Maybe move the stuff up to here into a _Setup.tex file that can be included from
% _FullBook.tex and _SingleChapter.tex


\begin{document}
	
% formatting:
\parindent=0in
\parskip=0pt
\pagenumbering{roman}
		
% main text
\pagenumbering{arabic} \setcounter{page}{1}

\author{Robin Schmidt}

\title{Mathematical Recipes for Scientists, Engineers and Programmers}
\maketitle

\chapter{Set Theory}


\section{Naive and Axiomatic Set Theory}



% ToDo: 
% -give Cantor'S original defition of a set
% -explain the naive/unrestricted comprehension axiom

\subsection{Antinomies}
% Maybe rename to paradoxes



\paragraph{Russel's Antinomy}
Define the Russel set\footnote{I think, it's actually not a \emph{set} but a \emph{proper class}? Verify!} $R$ as the set of all sets that do not contain themselves as an element. That means $R = \{x: x \notin x\}$. That means $x \in R \Leftrightarrow x \notin x$, i.e. $x$ is an element of $R$ if and only if $x$ is not an element of $x$. Here, $x$ can be \emph{any} set - for any set $x$ whatsoever, we can ask the question whether or not it is an element of $R$. Now ask what happens when we let $x = R$, i.e. we ask whether or not $R$ is an element of $R$. We immediately get the contradiction $R \in R \Leftrightarrow R \notin R$.

%Now ask: Is $R$ itself an element of $R$ or not? There are two possible answers to this question: either $R \in R$ or $R \notin R$. Let's assume that $R \notin R$, i.e. R does not contain itself as element. Then, by definition of $R$, $R$ should be an element of $R$. If, on the other hand, we assume that $R \in R$, then $R$ should not be an element of $R$.

%Let $R$ be the set of all sets that does not contain itself as element. Now ask: "is $R$ and element of $R$ or not?

...TBC...ToDo: explain how the paradox follows from unrestricted (naive) comprehension and how it can be fixed (I think, we need the axiom of foundation which (together with the axiom of pairing) forbids that sets can be elements of themselves)


% https://en.wikipedia.org/wiki/Zermelo%E2%80%93Fraenkel_set_theory#2._Axiom_of_regularity_(also_called_the_axiom_of_foundation)
 
% https://en.wikipedia.org/wiki/Axiom_schema_of_specification#Unrestricted_comprehension 
% https://aleph1.info/?call=Puc&permalink=mengenlehre1_1_1_Z10
% https://aleph1.info/?call=Puc&permalink=mengenlehre1_1_13

% Naive comprehension: if P(x) is a predicate, then we can build the set S = {x:  P(x)}, i.e. the set of all objects that sastify our predicate

% I think, we need the Fundierungsaxiom and need to disallow sets that can contain themselves
% https://de.wikipedia.org/wiki/Fundierungsaxiom
% Es gibt somit auch keine Menge, die sich selbst als Elemente enthält


% R = (...) - whatever in R is, R does not appear in its list of elements - by definition

% Maybe use also R-complement as the set of all sets that contain themselves

% https://de.wikipedia.org/wiki/Russellsche_Antinomie
% https://en.wikipedia.org/wiki/Russell%27s_paradox


% Note: the Russel set is actually not a set but a class, I think.

\subsection{Systems of Axioms}


\subsubsection{Zermelo, Fraenkel, Choice}
The most commonly uses system of axioms that mathematicians use today is the one proposed by Zermelo and Frankel together with the so called axiom of choice. This system of axioms is usually called the ZFC system (for Zermelo, Fraenkel, Choice). ...TBC...

% https://en.wikipedia.org/wiki/Zermelo%E2%80%93Fraenkel_set_theory
% https://de.wikipedia.org/wiki/Zermelo-Fraenkel-Mengenlehre

% https://de.wikipedia.org/wiki/Zermelo-Mengenlehre

\paragraph{The Axiom of Choice}

%The Axiom of Choice | Epic Math Time
%https://www.youtube.com/watch?v=Nnt4hyJYfGA
%-Shows how axiom of choice is used to show that every surjection f has a right inverse g
% such that f(g(y)) = y for all y in Y without explicitly specifying what g does. The proof leaves
% the task of picking an x in X for each y in Y (such that g(y) = x, f(x) = y) to the reader. Take 
% as example f(x) = x^2, compare the proof to the simpler proof for the bijective f(x) = x^3


\subsubsection{Neumann, Bernays, Gödel}



\section{Numbers as Sets}
In set theory, sets are the only thing that exists. Every mathematical object must somehow be viewed as some sort of set. To do all the cool math things that we love (or hate) so much, we obviously need numbers. So, we somehow need to build numbers from sets. Set theorists will talk about things like subsets of a number. This will at first sound totally nonsensical - what the heck is a subset of $10$ or $3/7$ or $\pi$ supposed to mean? But in set theory, numbers indeed \emph{are} sets (and therefore can have subsets) and we need to get used to this point of view.

%because we usually do not envision a number as a set. 
...TBC... 

\subsubsection{Ordinal Numbers}

\subsubsection{Cardinal Numbers}



\begin{comment}

ToDo:
-Infinite Sets 
 -Construction of natural, integer, rational and real numbers
 -Cardinal Numbers
 -Ordinal Numbers (as generalization of cardinal numbers)
 -Surreal Numbers

Cardilality:
-If there exists an injective function f: A to B  then  |A| <= |B|

Set Containment:
-If for all x in A, x in B, then A \subseteq B

Lebesgue-Measure:
 https://www.youtube.com/watch?v=0VD3BWDLmU0

...set theory sometimes appears like a "write-only-language

% See:
% https://aleph1.info/?call=Puc&permalink=mengenlehre1



Das Zermelo-Fraenkel-Axiomensystem der Mengenlehre (ZF)
https://www.youtube.com/watch?v=U10UYyXv5gM&list=PLb0zKSynM2PAuxxtMK1bxYPV_bUoPtpTB&index=1&t=0s

Bourbaki:
-Collective of authors writing a series of books in the early 1900s that aimed to build up all of math axiomatically. The first volume was about set theory

Was sind Kardinalzahlen? Was besagt die Kontinuumshypothese?
https://www.youtube.com/watch?v=qijXa3U4Nag&list=PLb0zKSynM2PCrgebQsfrzEsUIuA0I_wdG&index=1&t=0s




\end{comment}



% Maybe move into a file Biblio.tex
\begin{thebibliography}{9}
	
\bibitem{ABoAA} A Book of Abstract Algebra, \textit{Charles C. Pinter}  \newline
\href{http://www2.math.umd.edu/~jcohen/402/Pinter%20Algebra.pdf}
     {www2.math.umd.edu/~jcohen/402/Pinter Algebra.pdf}

\bibitem{ACRS} Algebraic Curves and Riemann Surfaces for Undergraduates, 
\textit{AnilNerode, Noam Greenberg}  \newline
\href{https://link.springer.com/book/10.1007/978-3-031-11616-2}
     {https://link.springer.com/book/10.1007/978-3-031-11616-2}

\bibitem{LinSysOverUnder} Solving over- and under-determined sets of equations, 
\textit{Berthold K.P. Horn}  \newline
\href{https://people.csail.mit.edu/bkph/articles/Pseudo_Inverse.pdf}
     {https://people.csail.mit.edu/bkph/articles/Pseudo\_Inverse.pdf}

\bibitem{SageCell} SageCell, \textit{Online SageMath Evaluator}  \newline
\href{https://sagecell.sagemath.org/}
     {sagecell.sagemath.org/}

\bibitem{VPDE_GiererMeinhardt} Gierer-Meinhardt Model, 
\textit{Online Interactive PDE Simulator}  \newline
\href{https://visualpde.com/mathematical-biology/gierer-meinhardt.html}
     {visualpde.com/mathematical-biology/gierer-meinhardt.html}

\bibitem{WK_GlasserMaster} Glasser's Master Theorem, \textit{Wikipedia Article} \newline
\href{https://en.wikipedia.org/wiki/Glasser%27s_master_theorem}
     {https://en.wikipedia.org/wiki/Glasser\%27s\_master\_theorem}

\bibitem{WK_ProjExtReals} Projectively extended real line, \textit{Wikipedia Article} \newline
\href{https://en.wikipedia.org/wiki/Projectively_extended_real_line}
     {https://en.wikipedia.org/wiki/Projectively\_extended\_real\_line}

\bibitem{YT_PowerSetRing} A classic example -- how the power set forms a ring. 
\textit{Michael Penn, on YouTube}
\href{https://www.youtube.com/watch?v=fvMnVKq3UtU}
     {https://www.youtube.com/watch?v=fvMnVKq3UtU}
     
\bibitem{YT_LeastAction} The Closest We Have to a Theory of Everything,
\textit{Sabine Hossenfelder, on YouTube}
\href{https://www.youtube.com/watch?v=A0da8TEeaeE}
     {https://www.youtube.com/watch?v=A0da8TEeaeE}     


\end{thebibliography}


\end{document}


