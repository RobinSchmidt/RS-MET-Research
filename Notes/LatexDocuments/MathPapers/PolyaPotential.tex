%\documentclass[12pt, twocolumn]{article}
\documentclass[12pt]{article}
%\documentclass[12pt]{scrartcl}  % to make \subtitle work
%\usepackage{fullpage}           % makes all margins 1 inch?
\topmargin=-1.0cm
\textheight=23cm
\evensidemargin=-1.0cm
\oddsidemargin=-1.0cm
\textwidth=19cm
\setcounter{secnumdepth}{-1}     % suppress numbering of sections
\usepackage{amsmath}
\usepackage{amssymb}             % for mathbb
%\usepackage[latin1]{inputenc}   % why?
\usepackage{graphicx}
\usepackage{float}               % for positioning graphics via [h!]
\usepackage{pstricks}
%\usepackage{pst-plot}
\usepackage{hyperref}            % for hyperlinks
%\usepackage{booktabs}
%\usepackage{fancyhdr}           % for headers and footers

%\pagestyle{fancy} 
%\rhead{}
%\rfoot{article available at: \htmladdnormallink{www.rs-met.com}{http://www.rs-met.com}}

% set the footer:
%\rfoot{http://www.rs-met.com}

\begin{document}

% formatting:
\parindent=0in
\parskip=0pt
\pagenumbering{roman}

% main text
\pagenumbering{arabic} \setcounter{page}{1}

\title{The Polya Potential of Complex Functions [DRAFT]\\ {\Large A New(?) Tool for Visualization, Analysis and Intuition Building}}
%\title{The Size of Infinity \ }
%\subtitle{(Subtitle)}
\author{Robin Schmidt}
\maketitle

\section{Preview}
In this paper, I propose a way to boil down the 4D information content of analytic functions into a 3D object without any loss of information. The key to this is the observation that analytic functions satisfy the Cauchy-Riemann equations which implies that their so called Polya vector fields are potential fields. A potential field is a vector field that can be obtained from a scalar field, called its potential, by way of taking partial derivatives. It turns out that the Cauchy-Riemann equations imply that the Polya vector field of an analytic function is indeed such a special potential field. We can therefore construct the potential of the Polya vector field which I will call the "Polya potential". Such a Polya potential is a function $P: \mathbb{R}^2 \rightarrow \mathbb{R}$, i.e. a function with two real inputs and one real output - all in all an object that lives in 3D space rather than in 4D as the original complex function does. It is therefore simpler than the original 4D function and easier to visualize. Yet, it contains the full information. We can always reconstruct our original complex function from it. I propose to make that Polya potential an object of investigation and I hope that, besides helping in visualizing complex functions, it may also suggest new ways of analyzing them. I'm not sure, if that idea is any new - but if it isn't, then information about that is not so easy to find on the internet (or my search skills are lacking). The Polya vector fields seem to be moderately well known - but their potentials seem to be, so far, largely ignored. I propose that we should take a closer look at them. To facilitate this, I will derive expressions for the Polya potentials of many important elementary functions, state some rules for how to derive such expressions for more complicated functions when they are expressed as power series and I'll explain how to interpret plots of such Polya potentials. 

\section{Analytic Functions}
Complex functions $f$ are functions that map a complex number $z$ to another complex number $w$. We will write this as $w = f(z)$. Both, input $z$ and output $w$, are two-dimensional entities such that plotting the graph of such a complex function $f$ would require four dimensions. That makes visualization of complex functions challenging for us lowly creatures that inhabit a 3D space. However, most complex functions of interest have a special property that allows us to boil these 4D objects down to 3D objects without any loss of information. That property is called "analyticity" also known as "holomorphy". These two terms are actually defined differently but in the context of complex analysis they imply each other, so they are often used interchangably [VERIFY!]. If we consider only functions that are analytic (or holomorphic), we are dealing with an important subset of all possible complex functions $f: \mathbb{C} \rightarrow \mathbb{C}$. That subset is so rigidly constrained that it is actually possible to construct a function $P: \mathbb{R}^2 \rightarrow \mathbb{R}$ that captures the full information content of our original $f$. This "rigidity" of analytic functions is captured in the  Cauchy-Riemann equations which relate real and imaginary part of a complex function in a way that severely constrains, what the analytic function can do. It is this rigidity that we exploit to boil down a 4D object into a 3D object without information loss.

\section{Polya Vector Fields}
When we interpret the complex conjugate of a complex function $f$ as a vector field, we obtain the so called Polya vector field that is associated with the complex function. Due to the analyticity of $f$, this vector field will be potential field. That means, there exists a scalar field, called a potential, such that the vector field can be obtained from it by taking partial derivatives. More formally, the Polya vector field of a complex function $f = f(z): \mathbb{C} \rightarrow \mathbb{C}$, is a 2D vector field, i.e. a function $(u(x,y), v(x,y)): \mathbb{R}^2 \rightarrow \mathbb{R}^2$ that is obtained from the complex function $f$ by using $u(x,y) = \Re(f), v(x,y) = -\Im(f)$ where $x = \Re(z), y = \Im(z)$. Negating the imaginary part of $f$ in the definition $v$ has the effect of turning the Cauchy-Riemann conditions for analytic complex functions into the condition that makes the Polya vector field curl-free (by virtue of $u_y = v_x$) and divergence-free (by virtue of $u_x = - v_y$). To avoid confusion, note that in my notation here, I use $v(x,y) = -\Im(f)$ whereas in a typical complex analysis textbook, you're are more likely to see things like $f(z) = f(x + \i y) = u(x,y) + \i v(x,y)$, i.e. $v(x,y) = \Im(f)$. Without the negation, the Cauchy-Riemann equations would look like $u_x = v_y, u_y = -v_x$ and that's how you'll see them in textbooks. But for our purposes here, where we are dealing with the Polya vector field, it is more convenient to have the negation already baked into $v$. So, the Cauchy-Riemann equations look like $u_x = -v_y, u_y = v_x$ with our notation here. The curl-free condition $u_y = v_x$ makes the Jacobian of the Polya vector field symmetric and in turn, a symmetric Jacobian is a necessary and sufficient condition for the vector field to be a gradient field, i.e. a vector field that can be obtained from a scalar field by the operation of taking its gradient. Symmetry of the Jacobian is in itself already enough to make the vector field a gradient field. But we have a further constraint: the divergence-free property $u_x = -v_y$ lets our Jacobian have a trace (sum of diagonal elements) of zero. Our vector field is not just any arbitrary gradient field but a very special kind of gradient field - namely, one that is divergence-free. I think, this divergence-free property is the reason why all stationary points of the Polya potential are saddles (at least, those I have seen so far). [Figure out! Do sources in the vector field correspond to maxima in the potential and sinks to minima? I think so.]

\section{Polya Potentials}
In general, a potential $P(x,y)$ for a given 2D vector field $u(x,y), v(x,y)$ is a scalar field whose partial derivatives with respect to $x$ and $y$ are equal to our two given bivariate functions $u$ and $v$ that define our vector field. That means $P$ must be such that $u = P_x = \partial P / \partial x$ and $v = P_y = \partial P / \partial v$. Our intention here is, of course, to let $u,v$ be the real and negated imaginary part of a complex function $f(z) = f(x + \i y)$, i.e. the Polya vector field of $f$. The potential for the Polya vector field of a complex function $f(z)$ is what I call the "Polya potential" of the complex function $f$. So, given $u$ and $v$ and assuming that $u,v$ indeed define a potential field, how do we find a potential $P(x,y)$ for our pair of bivariate functions? The equation $u = P_x$ suggests that to  find $P$, we might integrate $u$ with respect to $x$. In the following, "with respect to" will be abbreviated as "wrt". Let's call the integral of $u$ wrt $x$ by the letter $U = U(x,y)$. Also, the equation $v = P_y$ suggests that we might integrate $v$ wrt $y$. Let's call the result of that $V = V(x,y)$. So - does that mean we can choose either of the two ways and both will give us the same result, i.e. our desired potential $P(x,y)$ is equal to $U(x,y) = V(x,y)$? In some cases, it does indeed work out like that. But there's a little complication: In both integrations, there could be integration constants and these "constants" are actually constant only wrt our integration variable but can depend on the other variable. To take this into account, we may evaluate both of these integrals. If we are lucky, all of their terms that depend on both $x$ and $y$ match but we may get extra terms that depend only on $x$ when integrating $u$ wrt $x$ and other extra terms that depend only on $y$ when integrating $v$ wrt $y$. These "extra terms" are the desired integration "constants". What we have to do is to collect all the common terms that depend on $x$ and $y$ and add to the that the extra terms that depend only on $x$ or only on $y$ from both results to get our final expression for $P(x,y)$. In some cases, these extra terms turn out to be zero but at other times they are not. As a general rule, such extra terms pop up in $U$ whenever $u$ has terms that are independent of $y$, i.e. constant or dependent only on $x$ and they pop up in $V$, whenever $v$ has terms independent of $x$, i.e. constant or dependent only on $y$. This strategy seems to work fine for functions like $z^n, \exp(z), \sin(z), etc.$ but not so much for $\sqrt{z}, \log(z), \arcsin(z), etc.$. In the case of $\log$, the two integrals $U$ and $V$ look totally different and it is not quite clear how to combine them into a potential $P$. In the case of $\sqrt{z}$, the expression for $U$ looks unwieldy and I couldn't even find one for $V$ using SageMath. But I suppose, these problems are more of a technical rather than conceptual nature. We'll see. This will become clearer when we see examples where, for the time being, we will set that problem aside and focus on those cases, where analytic expressions for the the potentials can be found easily.

\section{Polya Vector Fields and Potentials of Basic Functions}
Now we want to build a little inventory of expressions for Polya vector fields and their potentials for some important elementary functions. The way I want to do this is by using a little snippet of SageMath code into which we can plug in our function of interest $f(z)$ and it spits out expressions for the components $u,v$ of the associated Polya vector field and the integrals $U,V$ of $u$ and $v$ with respect to $x$ or $y$ from which we can read off our Polya potential $P(x,y)$ by combining the results of these two integrals appropriately. You can execute that snippet yourself in SageCell \cite{SageCell}. In the example code, we consider the function $f(z) = z^2$.
\begin{verbatim}
var("x y")
assume(x, "real")
assume(y, "real")
z = x + I*y
w = z^2                 # function of interest
u =  w.real() 
v = -w.imag()
U = integral(u, x)
V = integral(v, y)
u, v, U, V	
\end{verbatim}
This produces the result:
\begin{verbatim}
(x^2 - y^2, -2*x*y, 1/3*x^3 - x*y^2, -x*y^2)
\end{verbatim}
In this example, we indeed see that $U$, the integral of $u$ wrt $x$, does indeed contain a term that depends only on $x$ but not on $y$, namlely $x^3/3$. That is the function of $x$ that would have to be added as integration "constant" to $V$ to give the full potential. The term that depends on both $x$ and $y$, namely $-x y^2$, is the same in $U$ and $V$, as it should be. So, we found the desired expressions for our Polya vector field and its potential for the complex function $f(z) = z^2$. They are:
\begin{equation}
f(z) = z^2: \quad 
u(x,y) = x^2 - y^2, \; 
v(x,y) = -2 x y, \quad 
P(x,y) = \frac{x^3}{3} -x y^2
\end{equation}
Replacing the line \texttt{w = z\textasciicircum2 } by some other function, say, \texttt{w = exp(z)}, we can get the expressions for other fucntions of interest. For $f(z) = e^z$, the result is:
\begin{verbatim}
(cos(y)*e^x, -e^x*sin(y), cos(y)*e^x, cos(y)*e^x)
\end{verbatim}
so we have:
\begin{equation}
f(z) = e^z: \quad 
u(x,y) =  e^x \cos(y), \; 
v(x,y) = -e^x \sin(y), \quad 
P(x,y) =  e^x \cos(y)
\end{equation}
In this case, $U$ and $V$ were indeed exactly equal and we didn't have to collect any extra terms for the integration constants, i.e. the correct integration constants turned out to be zero. Of course, we could still add true constants that depend neither on $x$ nor on $y$ to our final expression $P(x,y)$. That is a consequence of the fact that a potential for a given vector field is generally uniquely determined only up to an additive constant. This issue might become important, if you are dealing with different representations of the same function. For example, the Riemann zeta function can be expressed in multiple different ways. In such cases, it may make sense to pin down the integration constant by prescribing a specific function value for $P(x,y)$ at a specific point $(x,y)$, for example by requiring $P(0,0) = 0$. For the time being, we'll just take that additive constant to be zero, though. Continuing like this, we can build the following table:
\begin{center}
\begin{tabular}{ |p{1cm}|p{3cm}|p{3cm}|p{6cm}|  }
\hline
$f(z)$     & $u(x,y)$           & $v(x,y)$            & $P(x,y)$  \\
\hline
$a + \i b$ & $a$                & $-b$                & $a x - b y $                    \\
$z$        & $x$                & $-y$                & $(x^2 - y^2)/2 $                \\
$z^2$      & $x^2 - y^2$        & $-2xy$              & $x^3/3 - xy^2 $                 \\
$z^3$      & $x^3 - 3 x y^2$    & $-3 x^2 y + y^3$    & $x^4/4 - 3 x^2 y^2 / 2 + y^4/4$ \\
$1/z$      & $x/(x^2 + y^2)$    & $y/(x^2 + y^2)$     & $\ln(x^2+y^2)/2$                \\
$e^z$      & $e^x \cos(y)$      & $-e^x \sin(y)$      & $e^x \cos(y) $                  \\
$\sin(z)$  & $\sin(x) \cosh(y)$ & $-\cos(x) \sinh(y)$ & $-\cos(x) \cosh(y)$             \\
$\cos(z)$  & $\cos(x) \cosh(y)$ & $\sin(x) \sinh(y)$  & $ \sin(x) \cosh(y)$             \\
$\tan(z)$  & $ \frac{\sin(2x)}{\cos(2x) + \cosh(2y)}$              
           & $\frac{-\sinh(2y)}{\cos(2x) + \cosh(2y)}$               
           & $(-1/2) \log(\cos(2x) + \cosh(2y))$             \\
%$...$      & $...$              & $...$               & $...$             \\
\hline
\end{tabular}
\end{center}
% tan(z):
% sin(2*x)/(cos(2*x) + cosh(2*y)), -sinh(2*y)/(cos(2*x) + cosh(2*y)),
% -1/2*log(cos(2*x) + cosh(2*y)), -1/2*log(cos(2*x) + cosh(2*y))

...TBC...Verify formulas (also numerically). Maybe add sinh, cosh, tanh, Moebius-trafo \newline
% Make tables of:
% -small powers of z including -1,0,1,2,3, include f(z) = c = a + i b = const
% -f(z) = a z + b for complex a,b
% -f(z) = (a z + b) / (c z + d) for complex a,b,c,d
% -exponential: exp, pow, exp(i*z) -> important for Fourier series
% -trig: sin, cos, tan, 1/tan = cos/sin, 1/sin, 1/cos
% -hyp: sinh, cosh, tanh, ...
% -inverses: log, atan, asin %  ...for log it looks complicated, see .txt file
% -If the table grows large, maybe split into powers, trig-funcs, exp/sinh/cosh/log

\subsection{Positive Integer Powers}
We now want to derive expressions for the Polya vector field and potential a general positive integer power, i.e. for functions of the form $f(z) = z^n$. Setting $z = x + \i y$ and expanding $z^n$ according to the binomial theorem, we get:
\begin{equation}
	z^n = (x + \i y)^n = \sum_{k=0}^n \binom{n}{k}  x^k (\i y)^{n-k}
\end{equation}
To see what that means in terms of real and imaginary part of the result of $z^n$, we'll again use a little snippet of Sage code where we can tweak \texttt{n} and inspect the results:
\begin{verbatim}
n = 5	               # You may tweak this parameter
var("x y")
assume(x, "real")
assume(y, "real")
z = x + I*y
w = z^n
w.real(), w.imag()
\end{verbatim}
This code gives the following results for \texttt{n = 0..7}:
\begin{verbatim}
n  Real                                      Imag
0: 1,                                        0 
1: x,                                        y
2: x^2 - y^2,                                2*x*y
3: x^3 - 3*x*y^2,                            3*x^2*y - y^3
4: x^4 - 6*x^2*y^2  + y^4,                   4*x^3*y - 4*x*y^3
5: x^5 - 10*x^3*y^2 + 5*x*y^4,               5*x^4*y - 10*x^2*y^3 + y^5
6: x^6 - 15*x^4*y^2 + 15*x^2*y^4 - y^6,      6*x^5*y - 20*x^3*y^3 + 6*x*y^5
7: x^7 - 21*x^5*y^2 + 35*x^3*y^4 - 7*x*y^6,  7*x^6*y - 35*x^4*y^3 + 21*x^2*y^5 - y^7
\end{verbatim}
Note that here, we have not yet negated the imaginary parts, so we'll have to do that later. We always take binomial coeffs from the $n$-th line of Pascal's triangle. The coeffs of each line go alternatingly into the real and imaginary part. Within these parts, there's also a sign alternation. They multiply terms of the form $x^k y^{n-k}$ where $k$ starts at $n$ in the real part and at $n-1$ in the imaginary part and decrements by $2$ from term to term. We can make general formulas from these observations (these formulas aren't supposed to be obvious and took me quite a while to figure out):
\begin{equation}
\label{Eq:PolyaFieldPositivePowers}
\boxed{		
	u_n(x,y) = \sum_{k=0}^{n/2} (-1)^k \binom{n}{2 k} x^{n-2k} y^{2k}
	,\qquad
	v_n(x,y) = -\sum_{k=0}^{(n-1)/2} (-1)^k \binom{n}{2k+1} x^{n-(2k+1)} y^{2k+1}
}
\end{equation}
So, these are the $u_n(x,y), v_n(x,y)$ parts of the Polya vector field of a single term $z^n = (x + \i y)^n$ for $n \geq 0$. The upper summation limits are understood to be meant in the sense of integer division, i.e. if the result of the division by 2 is a non-integer, it will be rounded down towards zero. There is a little complication for the edge case of $n=0$ for $v_n$, i.e. for $v_0(x,y)$. In this case, the sum would formally run from $0$ to $-1/2$. Due to the integer division behavior, it actually runs from $0$ to $0$, i.e. produces exactly one term for $k=0$ which according to the formula turns out to be $v_0(x,y) = 0 x^{-1} y^1$ if we assume the 0-choose-1 binomial coefficient to be zero, which is the common convention. This result is actually kinda correct mathematically - we just want $v_0(x,y)$ to be the zero function. But when coding this stuff up, we need to be careful to treat this edge case correctly because it doesn't really fit the pattern of an $x^n y^m$ term for $n,m \geq 0$. In a coding context that works with bivariate polynomials, we might probably want to express the zero function as $v_0(x,y) = 0 x^0 y^0$ instead.

\medskip
To find the potential for the $z^n$ terms, we'll need to take (\ref{Eq:PolyaFieldPositivePowers}) and integrate $u_n(x,y)$ wrt $x$ and $v_n(x,y)$ wrt to $y$. We'll call the integration results $U_n, V_n$ respectively. The functions $u,v$ are just simple bivariate polynomials, so integrating wrt to $x$ involves incrementing the $x$-exponents in all terms by one and dividing the terms by their new exponents. Same story for $y$. This gives:
\begin{equation}
U_n(x,y) = \sum_{k=0}^{n/2} \binom{n}{2 k} \frac{(-1)^k x^{n-2k+1} y^{2k}}{n-2k+1}
,\qquad
V_n(x,y) = -\sum_{k=0}^{(n-1)/2} \binom{n}{2k+1} \frac{(-1)^k x^{n-(2k+1)} y^{2k+2}}{2k+2}
\end{equation}
When doing these integrations, all the terms that involve both $x$ and $y$ match in both of these integrals. But if $u$ contains terms that don't depend on $y$, i.e. a constant term and/or terms that only have $x$ in them but no $y$, then the integral of $u$ wrt $x$ may contain some etxra terms that only depend on $x$ and don't appear in the integral of $v$ wrt $y$. Likewise, if $v$ has terms that don't depend on $x$, they will give rise to terms that only depend on $y$ in the integral of $v$ wrt to $y$ which do not show up in the integral of $u$ wrt $x$. Such additional terms in either of these integrals can be regarded as integration constants - they are constant with respect to our integration variable but may depend on the other variable.

\medskip
We can in general just always use $V_n(x,y)$ and add to it a $x^{n+1}/(n+1)$ term as integration constant because all the $u_n$ have exactly one term that is independent of $y$ and that term is always $x^n$ (see the Sage output) which, when integrated, gives $x^{n+1}/(n+1)$. That gives:
\begin{equation}
\label{Eq:PolyaPotentialPositivePowers}	
\boxed{	
	P_n(x,y) = \frac{x^{n+1}}{n+1} 
	-\sum_{k=0}^{(n-1)/2} \binom{n}{2k+1} \frac{(-1)^k x^{n-(2k+1)} y^{2k+2}}{2k+2}
}
\end{equation}

\subsection{Negative Integer Powers}
Let's now derive expressions for the Polya vector field and potential a general negative integer power, i.e. for functions of the form $f(z) = 1/z^n = z^{-n}$.

  

We will again try a couple of examples and try to spot the patterns. We'll use this script:
\begin{verbatim}
n = 3                     # exponent without the minus sign
var("x y")
assume(x, "real")
assume(y, "real")
z = x + I*y
w = 1 / z^n               # function of interest
u =  w.real() 
v = -w.imag()
U = integral(u, x)
V = integral(v, y)
u, v, U, V
\end{verbatim}
This code produces the result:
\begin{verbatim}
u = (x^3 - 3*x*y^2)/((x^3 - 3*x*y^2)^2 + (3*x^2*y - y^3)^2)
v = (3*x^2*y - y^3)/((x^3 - 3*x*y^2)^2 + (3*x^2*y - y^3)^2)
U = V = -1/2*(x^2 - y^2)/(x^4 + 2*x^2*y^2 + y^4)
\end{verbatim}
We observe that for $u$ and $v$, we get rational functions with equal denominators. Let's produce more results for \texttt{n = 2..6} and look at the numerators of $u$ and $v$ as well as at the common denominators. Numerators first:
\begin{verbatim}
n  Numerator of u                                  Numerator of v
2  x^2 - y^2                                       2*x*y                         
3  x^3 - 3*x*y^2                                   3*x^2*y - y^3                  
4  x^4 - 6*x^2*y^2  + y^4                          4*x^3*y - 4*x*y^3
5  x^5 - 10*x^3*y^2 + 5*x*y^4                      5*x^4*y - 10*x^2*y^3 + y^5
6  x^6 - 15*x^4*y^2 + 15*x^2*y^4 - y^6             6*x^5*y - 20*x^3*y^3 + 6*x*y^5
\end{verbatim}
The denominators are:
\begin{verbatim}
n  Denominator
2  (x^2 - y^2)^2                           +  4*x^2*y^2
3  (x^3 - 3*x*y^2)^2                       + (3*x^2*y - y^3)^2
4  (x^4 - 6*x^2*y^2 + y^4)^2               + 16*(x^3*y - x*y^3)^2
5  (x^5 - 10*x^3*y^2 + 5*x*y^4)^2          + (5*x^4*y - 10*x^2*y^3 + y^5)^2
6  (x^6 - 15*x^4*y^2 + 15*x^2*y^4 - y^6)^2 + 4*(3*x^5*y - 10*x^3*y^3 + 3*x*y^5)^2
\end{verbatim}
And finally, the potentials:
\begin{verbatim}
n  Potential
2  -x / (x^2 + y^2)
3  -1/2*(x^2 - y^2) / (x^4 + 2*x^2*y^2 + y^4)
4  -1/3*(x^3 - 3*x*y^2) / (x^6 + 3*x^4*y^2 + 3*x^2*y^4 + y^6)
5  -1/4*(x^4 - 6*x^2*y^2 + y^4) / (x^8 + 4*x^6*y^2 + 6*x^4*y^4 + 4*x^2*y^6 + y^8)
6  -1/5*(x^5 - 10*x^3*y^2 + 5*x*y^4)  / 
        (x^10 + 5*x^8*y^2 + 10*x^6*y^4 + 10*x^4*y^6 + 5*x^2*y^8 + y^10)
\end{verbatim}
The numerators of real and imaginary parts are exactly as in the corresponding positive powers of $z$ and the denominator is always $u^2 + v^2$. In the potential, the denominator is $(x^2 + y^2)^{n-1}$. And the numerator of the $n$-th line is the same as the real part of the $(n-1)$-th line and the whole thing is divided by $n-1$?  So, the general formulas for the Polya vector field and potential for $f(z) = 1/z^n$ are:
\begin{equation}
\label{Eq:PolyaFieldNegativePowers}
\boxed
{		
u_{-n}(x,y) = \frac{u_n}{u_n^2 + v_n^2} ,\;
v_{-n}(x,y) = \frac{v_n}{u_n^2 + v_n^2} ,\quad
P_{-n}(x,y) = \frac{-u_{n-1}}{(n-1)(x^2+y^2)^{n-1}}
}
\end{equation}
where $u_n, v_n$ are the expressions for the corresponding positive power. It turns out that the case for $n=1$ doesn't follow the pattern, so we treat it separately. The result is actually already included in the table further above.

[Maybe move the Sage derivations into an appendix and keep only the results in the main text. Refer to the implementation of the formulas]

\section{Plotting the Polya Potentials}
One of the main goals of this investigation is to provide a new way of visualizing complex functions, so we shall now look at some plots...TBC...

[ToDo: Give some 3D plots and/or contour maps of Polya potentials. Explain how these plots can be interpreted. Example functions: $z, a z, z^2, z^3, 1/z, exp(z), sin(z), ...$]




\section{Rules for Complicated Functions}
Having tables for the Polya potentials of some basic elementary functions is well and good but often we have to deal with more complicated functions. For a given function $f$, we could just plug it into the little Sage script and see, what it produces. What we will discover is that things get messy really quick. Of course they do - after all, we are dealing with integrals here. Maybe you can get something more reasonable with other CAS systems like Mathematica or manually with some integration wizardry if you are a Richard Feynman type of person. You could perhaps deal with products using integration by parts or deal with composite functions using substitutions. But I want to go down some other route here. I will assume that our function $f(z)$ is given as some sort of combination (product, quotient or composition) of two other functions $g(z), h(z)$ for which we know a series expansion. I want to give formulas or algorithms to obtain a series expansion for $f$. This may actually be applicable in completely different contexts as well. From the so found series expansion of $f$, we can then find the corresponding Polya potential term-wise by using our tabulated expressions for simple functions like $z^n$. The setting is as follows:
\begin{equation}
g(z) = \sum_k a_k z^k, \quad
h(z) = \sum_m b_m z^m, \qquad
f(z) = \sum_n c_n z^n
\end{equation}
where the $a_k, b_m$ are known and we set out to figure out formulas or algorithms to compute the $c_n$ in terms of the known $a_k, b_m$. The function $f(z)$ is assumed to be given in terms of $g,h$ as a weighted sum $f(z) = a g(z) + b h(z)$, as product $f(z) = g(z) h(z)$, as quotient $f(z) = g(z) / h(z)$ or as composition $f(z) = g(h(z))$. I have deliberately kept the ranges of $k$ and $m$ ambiguous. They could go from zero to some finite number in which case we would be dealing with polynomials, from zero to infinity in which case we would be dealing with power series, from minus infinity to plus infinity in which case we would be dealing with Laurent series, etc.. The question is now: what would the coefficients $c_n$ of a series expansion of $f$ look like.

\subsection{Weighted Sums and Series}
The simplest case of that is when our function of interest $f$ is given by a weighted sum of functions whose potentials we already know, i.e. $f(z) = a g(z) + b h(z)$ for some constants $a,b$. In this case, the full potential is just the weighted sum of the individual potentials with the same weights as in $f$. This even extends to infinite weighted sums which is useful when we have a series expansion of our $f$ like, for example, a Taylor-, Laurent- or Fourier series. 

\subsection{Products}
Let's turn to products. The solution to that problem is actually well known: the sequence $c_n$ is given by the convolution of the sequences $a_k$ and $b_m$. Such a product of two given series expansions is also known as the Cauchy-Product of these series [VERIFY!]. So, to find a Polya potential of a product of two functions $g,h$ given as series expansions, we could construct a series expansion of $f$ using the Cauchy product, i.e. convolution, and then use the formulas for the Polya potentials of $z^n$ from our tables. 

\subsection{Quotients}
For a quotient $f(z) = g(z) / h(z)$, we would need to do a deconvolution.......TBC...

\subsection{Composite Functions}
For a function $f$ that results from composing (aka nesting or chaining) two functions such that $f(z) = g(h(z))$, an algorithm for finding the $c_n$ could look as follows ....TBC...link to the DefiniteIntegration... paper on my website - it has such an algorithm. Maybe the stuff about combining series should be factored out into a separate paper 

%A benign looking function such as \texttt{exp(z)*sin(z\textasciicircum2)} containing a multiplication and a composition, will produce very complicated expressions that even contain the \texttt{erf} erf function. But 

%that are formed from the basic functions by mathematical operations like multiplication, division and nesting. Addition and subtraction ins unproblematic because in this case, we can also just add or subtract the corresponding Polya potentials due to linearity.

% Give product, quotient and chain rule for power series (conv, deconv, polyNest)

% exp(z)*sin(z^2)                 long expresson, contains erf
% exp(z) / (1+ z^2)               even worse
% exp(z) / (1+ sin(z^2))          sage gives up
% Maybe some other CAS like Mathematica or an integration wizard like Richard Feynman could handle it....

% mention building Polya potententials from Talyor sereis, Laurent series, and Fourier series
% partial fraction expansions for rational functions

\subsection{Numerical Evaluation}
If all of our attempts to derive an analytic expression for the Polya potential of some function of interest fail, we may have no other choice than to turn to a numerical approximation. Inside my research codebase \cite{GitHub}, I written have a little function that takes as input a vector field represented as a pair of 2D arrays for $u(x,y)$ and $v(x,y)$ and produces as output a 2D array that represents the scalar field $P(x,y)$ whose numerical partial derivatives, computed via a central difference formula, match the given vector field data $u,v$. I explain the algorithm in \cite{PotentialNumerical}. Well, there is actually nothing computed via the central difference formula, but instead I use the central difference formula as an ansatz. I imagine that our given 2D arrays  $u,v$ have been computed by such a formula from an unknown scalar filed $P$ (also represented as 2D array) and then I ask, what the scalar field $P$ must look like to produce such numerical partial derivatives. That leads to an overdetermined linear system of equations for the elements of $P$ which I solve using a least squares approach. The systems grow big really fast, so it's mandatory to use a sparse matrix representation and solve the system by iterative methods. To this end, I have implemented SOR (successive over relaxation). It's sometimes horribly slow and sometimes even fails to converge. So, so far, this stuff is still in a rather preliminary shape. Also, the numerical error is quite visible. When doing contour plots of numerically computed potentials, the contour lines look very "noisy" compared to the ideal, mathematically correct contour lines. So, it's not a silver bullet and if we can, we should really try to figure out an analytic expression. Turning to the numerical algorithm should be the last resort.




% Old text snippet - obsolete?:
%In this context, the scalar field whose gradient our vector field is, is called the (scalar) potential of the vector field. In fact, the Jacobian of the vector field is just the Hessian of the potential. The potential for the Polya vector field of a complex function $f(z)$ is what I call the "Polya potential" of the complex function and it's an object that I have not yet seen as being used in complex analysis. Maybe taking a closer look at it can reveal some new insights. The goal is to derive expressions for the Polya potential of the zeta function and then analyze this Polya potential instead of the zeta function itself. I hope that it will turn out to be simpler to understand because its just a simple scalar field over $\mathbb{R}^2$, i.e. a 3D thing rather than a 4D one - but it contains nonetheless the same amount of information. We have taken advantage of the analyticity of zeta to achieve a dimensionality reduction of the problem. The all important zeros of the zeta function will be mapped to the stationary points of its Polya potential. Stationary points of a scalar field are points, where the gradient vanishes. In general, there are three types of such stationary poinrs: minima, maxima and saddle points. Experiments suggest that in the case of Polya potentials of complex functions, only saddles occur [I think, this might be a consequence of the additional divergence-free property - figure out!]. The Polya potential is a surface in 3D space, so it is also amenable to the tools of differential geometry like figuring out main curvature directions at the stationary points, figuring out the geodesics between the stationary points, etc.. Polya potentials give us a new angle of attack to analyze complex functions and, of course, to visualize them to hopefully build some intuition.

\begin{thebibliography}{9}
	
\bibitem{SageCell} SageCell, \textit{Online SageMath Evaluator}
\href{https://sagecell.sagemath.org/}{sagecell.sagemath.org/}

\bibitem{GitHub} RS-MET-Research, \textit{Repository on GitHub}
\href{https://github.com/RobinSchmidt/RS-MET-Research}{github.com/RobinSchmidt/RS-MET-Research}

\bibitem{PotentialNumerical} PotentialNumerical.txt, \textit{File in my research repo}
\href{https://github.com/RobinSchmidt/RS-MET-Research/blob/master/Notes/PotentialNumerical.txt}{Notes/PotentialNumerical.txt}

%http://rs-met.com/documents/dsp/DefiniteIntegrationOfPolynomialsWithPolynomialsAsLimits.pdf

\end{thebibliography}
 
 
\end{document}



\begin{comment}
	
Higher order saddles:	
https://www.researchgate.net/publication/256808897_Monkey_Starfish_and_Octopus_Saddles	
	
Notes:
-Using a capital P to denote the potential is consistent with the one used on  wikipedia.
	
ToDo:
-Write a "Conclusion" section that hints at the implementation for numerically computing 
 potentials. Use that implementation to produce plots of the Polya potentials of sqrt and log.

Resources:
https://en.wikipedia.org/wiki/Scalar_potential	
	
-This paper is supposed to factor out all the general Polya-potential stuff from the paper
 about the Polya potnetial of the Riemann zeta function such that it can eventually be removed
 from there.
 
-Give plots for Polya potentials for some simple functions
 f(z) = c, f(z) = z, f(z) = z + c, f(z) = z^2

-What about viewing Polya potentials throgh the lens of algebraic geometry by considering the
 curves p(x,y) = c for some constant c?
 
 
 


\end{comment}
